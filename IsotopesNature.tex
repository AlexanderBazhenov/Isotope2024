%%%%%%%%%%%%%%%%%%%%%%%%%%%%%%%%%%%%%%%%%
% Beamer Presentation
% LaTeX Template
% Version 1.0 (10/11/12)
%
% This template has been downloaded from:
% http://www.LaTeXTemplates.com
%
% License:
% CC BY-NC-SA 3.0 (http://creativecommons.org/licenses/by-nc-sa/3.0/)
%
%%%%%%%%%%%%%%%%%%%%%%%%%%%%%%%%%%%%%%%%%

%----------------------------------------------------------------------------------------
%	PACKAGES AND THEMES
%----------------------------------------------------------------------------------------

\documentclass{beamer}

%\usepackage{newtxtext,newtxmath,amsmath}


\usepackage{mathptmx}
\usepackage{amsmath}
\def\permille{\ensuremath{{}^\text{o}\mkern-5mu/\mkern-3mu_\text{oo}}}

\newcommand{\mbf}[1]{\protect\text{\boldmath$#1$}}
\newcommand{\mbb}{\mathbb}
\newcommand{\mrm}{\mathrm}
\newcommand{\mcl}{\mathcal}
\newcommand{\msf}{\mathsf}
\newcommand{\eus}{\EuScript}
\newcommand{\ov}{\overline}
\newcommand{\un}{\underline}
\newcommand{\m}{\mathrm{mid}\;}
\newcommand{\w}{\mathrm{wid}\;}
\newcommand{\Uni}{\mathrm{Uni}\,}
\newcommand{\Tol}{\mathrm{Tol}\,} 
\newcommand{\Uss}{\mathrm{Uss}\,} 
\newcommand{\Ab}{(\mbf{A}, \mbf{b})}
\newcommand{\sgn}{\mathrm{sgn}\;} 
\newcommand{\ran}{\mathrm{ran}\,} 
\newcommand{\pro}{\mathrm{pro}\,} 
\newcommand{\dom}{\mathrm{dom}\,} 
\newcommand{\IVE}{\mathrm{IVE}\,} 
\newcommand{\IED}{\mathrm{IED}\,} 
\newcommand{\calX}{\mathrsfs{X}} 
\newcommand{\cond}{\mathrm{cond}} 
\newcommand{\dual}{\mathrm{dual}\,} 
\newcommand{\dist}{\mathrm{dist}\,} 
\newcommand{\Dist}{\mathrm{Dist}\,} 
%\newcommand{\const}{\mathrm{const}} 
\newcommand{\USS}{\varXi_{\hspace{-0.5pt}uni}} 
\newcommand{\TSS}{\varXi_{\hspace{-0.5pt}tol}} 
\newcommand{\NExt}{_{\scalebox{0.57}{$\natural$}}}
%\newcommand{\mode}{\mathrm{mode}\,} 


%\renewcommand{\r}{\matrhm{rad}\;} 
\newcommand{\rd}{\mathrm{rad}\;} 
\newcommand{\ttM}{\tt METAPOST }
\mode<presentation> {

% The Beamer class comes with a number of default slide themes
% which change the colors and layouts of slides. Below this is a list
% of all the themes, uncomment each in turn to see what they look like.

%\usetheme{default}
%\usetheme{AnnArbor}
%\usetheme{Antibes}
%\usetheme{Bergen}
%\usetheme{Berkeley}
%\usetheme{Berlin}
%\usetheme{Boadilla}
%\usetheme{CambridgeUS}
%\usetheme{Copenhagen}
%\usetheme{Darmstadt}
%\usetheme{Dresden}
%\usetheme{Frankfurt}
%\usetheme{Goettingen}
%\usetheme{Hannover}
%\usetheme{Ilmenau}
%\usetheme{JuanLesPins}
%\usetheme{Luebeck}
\usetheme{Madrid}
%\usetheme{Malmoe}
%\usetheme{Marburg}
%\usetheme{Montpellier}
%\usetheme{PaloAlto}
%\usetheme{Pittsburgh}
%\usetheme{Rochester}
%\usetheme{Singapore}
%\usetheme{Szeged}
%\usetheme{Warsaw}

% As well as themes, the Beamer class has a number of color themes
% for any slide theme. Uncomment each of these in turn to see how it
% changes the colors of your current slide theme.

%\usecolortheme{albatross}
%\usecolortheme{beaver}
%\usecolortheme{beetle}
%\usecolortheme{crane}
%\usecolortheme{dolphin}
%\usecolortheme{dove}
%\usecolortheme{fly}
%\usecolortheme{lily}
%\usecolortheme{orchid}
%\usecolortheme{rose}
%\usecolortheme{seagull}
%\usecolortheme{seahorse}
%\usecolortheme{whale}
%\usecolortheme{wolverine}

%\setbeamertemplate{footline} % To remove the footer line in all slides uncomment this line
%\setbeamertemplate{footline}[page number] % To replace the footer line in all slides with a simple slide count uncomment this line

%\setbeamertemplate{navigation symbols}{} % To remove the navigation symbols from the bottom of all slides uncomment this line
}

\usepackage{cmap}					% поиск в PDF
\usepackage[T2A]{fontenc}			% кодировка
\usepackage[utf8]{inputenc}			% кодировка исходного текста
\usepackage[english,russian]{babel}	% локализация и переносы
%\usepackage{clrscode3e}            
%\usepackage{slashbox}            
\usepackage{graphicx} % Allows including images
\usepackage{booktabs} % Allows the use of \toprule, \midrule and \bottomrule in tables

%----------------------------------------------------------------------------------------
%	TITLE PAGE
%----------------------------------------------------------------------------------------

\title{Стабильные изотопы в природе} % The short title appears at the bottom of every slide, the full title is only on the title page

\author{А.Н.~Баженов}
\institute{%\large 
ФТИ им.А.Ф.Иоффе РАН\\
СПбПУ Петра Великого, Физмех\\
\bigskip	
ДОПОЛНИТЕЛЬНАЯ ПРОФЕССИОНАЛЬНАЯ ПРОГРАММА ПОВЫШЕНИЯ КВАЛИФИКАЦИИ\\
<<Новейшие достижения физики и математического моделирования>> 
}


\date{29.10.2024} % Date, can be changed to a custom date

%\date{25.11.2021}

\begin{document}

\begin{frame}
\titlepage % Print the title page as the first slide
\end{frame}



	




\input{Intro}

%%%%%%%%%%%%%%%%%%%%%%%%%%%%%%%%%%%%%%%%%%%%%%%%%%%%%%%%%%%%%%%%%%%%%%%%%%%%%%%%%%%%%%%%%%%%%%%%%%%%%%%%%%%%%%%%%%

\begin{frame}
\frametitle{Периодическая таблица Менделеева, элементы и изотопы}

\begin{center}
ПЕРИОДИЧЕСКАЯ ТАБЛИЦА МЕНДЕЛЕЕВА, \\
ЭЛЕМЕНТЫ И ИЗОТОПЫ
\end{center}

\end{frame}

\begin{frame}
	\frametitle{Периодическая таблица Менделеева}
1869 год. <<Соотношение свойств с атомным весом элементов>> \cite{Mendeleev1869ru}
\begin{figure}[ht] 
	\centering\small
	\unitlength=1mm
	\includegraphics[width=70mm]{Figures/1869Mendeleev.png} 
	\caption{Периодическая таблица 1869 \cite{Mendeleev1869}} 
	\label{f:1869Mendeleev}
\end{figure}	
\end{frame}

\begin{frame}
	\frametitle{Периодическая таблица Менделеева}
https://www.chemistryworld.com/opinion/is-this-the-worlds-oldest-classroom-periodic-table/3009960.article
	\begin{figure}[ht] 
		\centering\small
		\unitlength=1mm
		\includegraphics[width=100mm]{Figures/140538_PerTab3.jpg} 
		\caption{Периодическая таблица Менделеева, XIX век} 
		\label{f:MendeleevTableXIX}
	\end{figure}	
\end{frame}

\begin{frame}
\frametitle{Историческая справка --- литература}

\begin{itemize}
	\item Трифонов Д.Н., Кривомазов А.Н., Лисневский Ю.И. \\ Учение о периодичности и учение о радиоактивности (комментированная хронология важнейших событий). М., Атомиздат, 1974, 248 с.
	\item E. Scerri. \\
	The Periodic Table. Its Story and Its Significance. 
	2nd edition.  New York, NY : Oxford University Press, 2019 
	\item Бекман, И. Н.  Атомная и ядерная физика: радиоактивность и ионизирующие излучения : учебник для вузов / И. Н. Бекман. — 2-е изд., испр. и доп. — Москва : Издательство Юрайт, 2022. — 493 с. — (Высшее образование). — ISBN 978-5-534-08692-8. 
\end{itemize}



	
\end{frame}

\begin{frame}
\frametitle{Развитие учения о периодичности}
{\small	
	%{\tiny 
	\begin{tabular}{l|l|l}
		Год & Автор	& Проблема \\
		\hline
		1885 & Ридберг & {\color{red}Атомные веса не могут рассматриваться} \\
		~ & ~ & {\color{red} в качестве независимой переменной} \\	
		~ & Балмер & Формула для спектральных линий водорода \\	
		1886 & Крукс & {\color{red}Атомные веса не одинаковы для всех атомов } \\
		1888 & ~ &   {\color{red}элемента, а существует распределение }\\
		1890 & Ридберг & Обобщение формулы Бальмера на разные элементы \\	
		1895 & Рентген & Открытие X-лучей \\
		~ & Стрэтт, Рамзай & Аргон --- новая составляющая часть атмосферы \\
		1896 & Беккерель & Радиоактивность урановых соединений  \\
		~ & ~ &  и металлического урана\\
		1897 & Томсон & Катодные лучи --- носитель отрицательного заряда  \\
		~ & ~ &  для всех веществ, в 1800 раз легче водорода\\	
		~ & Ридберг & {\color{red}Атомный вес элементов $M = N+D$,} \\	
		~ & ~ &  {\color{red}$N$ --- целое, $D$ --- малая периодическая функция}\\			
		\hline
	\end{tabular}
}	
\end{frame}



\begin{frame}
	\frametitle{Исследования атома и ядра}
{\small	
%{\tiny 
	\begin{tabular}{l|l|l}
		Год & Автор	& Открытие \\
		\hline
		1913 & Дж.Дж.Томсон & Открытие {\color{red}изотопов} неона с массой 20 и 22\\
		~ & А.Ван-ден-Брук & {\color{red}Порядковый номер элемента в Периодической}  \\
		~ & ~ &  {\color{red}системе равен заряду ядра его 	атомов} \\
		~ & Ф.Содди & Понятие изотопа у радиоактивных
		элементов   \\
		1914, & Н.Бор & Свойства элементов периодической системы  \\
		1921 & ~ & суть  функции зарядов ядер их атомов \\
		1914 & Г.Мозли & Зависимость  частоты характеристического	  \\
		~ & ~ & излучения от порядкового номера элементов \\
		1916 & У.Д.Харкинс & {\color{red}Правило большей распространенности элементов}\\
		~ & ~ & {\color{red}с четными порядковыми номерами} \\
%		1917 & Ф.Содди  & Изотопы высшего порядка --- ядерные изомеры\\
%		1918 & Дж.Дж.Томсон  &  Доказательство существования изотопов среди\\
%		~ & ~ & продуктов радиоактивного распада \\
		1918 & А.У.Стюарт  &  Открытие {\color{red}изобар}\\
		1919 &Э.Резерфорд &   Открытие протона; доказательство  \\
		~ & ~ & наличия в ядрах элементов протонов; \\
		~ & ~ & первая искусственная ядерная реакция \\
%		~ & ~ & --- превращение азота в кислород \\
%		1920 & Г.Хевеши & Явление изотопного обмена\\
%		1921 & Н.Бор & Строение атомов --- связь периодичности  \\
%		~ & ~ & их химических и спектральных свойств с \\
%		~ & ~ &  характером формирования электронных конфигураций \\
%		~ & ~ &  по мере роста заряда ядра \\
%		1921 & О.Ган  & Открытие изомера урана\\
		1921 & Ф.Астон  & {\color{red}212 природных изотопов различных элементов;} \\
		~ & ~ &   {\color{red}Массы изотопов --- целые числа} \\
		\hline
	\end{tabular}
}	
\end{frame}


\begin{frame}
\frametitle{Накопление данных об изотопах на Земле --- до ядерных проектов}

\begin{center}
НАКОПЛЕНИЕ ДАННЫХ ОБ ИЗОТОПАХ ЭЛЕМЕНТОВ	
\end{center}


\end{frame}


\begin{frame}
	\frametitle{Накопление данных об изотопах на Земле --- до ядерных проектов}
1919 Астон и Демпстер --- создание масс-спектрографа (спектрометра) \\
~\\
Aстон --- 212 изотопов\\
~\\
К 1935 году была составлена почти полная таблица изотопных составов всех известных
к тому времени химических элементов.\\
~\\
Примерно к 1940 изотопный анализ был осуществлен для всех существующих на Земле элементов. \\
~\\
к 1950 были выявлены и
идентифицированы практически все стабильные и долгоживущие радиоактивные изотопы природных
элементов. 

\end{frame}

\begin{frame}
	\frametitle{Распространенность изотопов. Астон --- 1924}
	
	\begin{figure}[ht] 
		\centering\small
		\unitlength=1mm
		\includegraphics[width=100mm]{Figures/1925AstonElementAbudances.png} 
		%	\caption{$N-Z$ диаграмма  атомных ядер} 
		\label{f:Aston1925Ab}
	\end{figure}	
	
\end{frame}


\begin{frame}
\frametitle{Таблица элементов и изотопов Астона-Демстера --- 1925}

\begin{figure}[ht] 
	\centering\small
	\unitlength=1mm
	\includegraphics[width=120mm]{Figures/1925AstonTableElemIsotopes2.png} 
%	\caption{$N-Z$ диаграмма  атомных ядер} 
	\label{f:Aston1925}
\end{figure}	

\end{frame}




\begin{frame}
\frametitle{Накопление данных об изотопах --- ядерные проекты}


Ядерные проекты --- изучение деления атмов, ядер и радиоактивных нуклидов
\begin{itemize}
	\item Количественная теория атомного ядра
	\item Технологии разделения и производства изотопов
	\item Детальное изучение свойств ядер,элементарных частиц и реакций на реакторах и ускорителях
	\item Численные методы расчёта атомов --- квантовая химия
	\item Численные методы расчёта ядерных реакций ($10^4$ уравнений)
\end{itemize}

\end{frame}


\begin{frame}
\frametitle{Вторая половина XX в. --- накопление данных об изотопах на Земле и во Вселенной}
На Земле
\begin{itemize}
	\item Науки о Земле
	\item Науки о живом
\end{itemize}

~\\
С началом космической эры --- изучение изотопов атмосферы Земли и Солнечной системы, космоса
\begin{itemize}
	\item Солнечная система
	\item Космос
\end{itemize}	

\end{frame}
%%%%%%%%%%%%%%%%%%%%%%%%%%%%%%%%%%%%%%%%%%%%%%%%%%%%%%%%%%%%%%%%%%%%%%%%%%%%%%%%%%%%%%%%%%%%%%%%%%%%%%%%%%%%%%%%%%


\begin{frame}
\frametitle{ИЗОТОПЫ. ПРОИСХОЖДЕНИЕ и СВОЙСТВА.}

\begin{center}
ПРОИСХОЖДЕНИЕ и СВОЙСТВА ИЗОТОПОВ.\\
~\\
	НЕКОТОРЫЕ СЛЕДСТВИЯ ОСОБЕННОСТИ \\ СТРОЕНИЯ АТОМНЫХ ЯДЕР
\end{center}

\end{frame}

\begin{frame}
	\frametitle{Изотопы нуклидов во Вселенной и на Земле}
Элементы и их изотопы.	$N\!-\!Z$ диаграмма %  атомных ядер.
https://www.nndc.bnl.gov/nudat3/, моды распадов
\begin{figure}[ht] 
	\centering\small
	\unitlength=1mm
	\includegraphics[width=100mm]{Figures/NZdiagramDecayMode.png} 
%	\caption{$N-Z$ диаграмма  атомных ядер} 
	\label{f:NZdiagram}
\end{figure}	

\end{frame}

\begin{frame}
\frametitle{Доминирование элементов с чётными порядковыми номерами}

\begin{figure}[ht] 
	\centering\small
	\unitlength=1mm
	\includegraphics[width=110mm]{Figures/stable_isotopes_by_element_stairs.png} 
	\caption{Число стабильных изотопов у элементов} 
	\label{f:stable_isotopes_by_element}
\end{figure}	
	1916 --- У.Д.Харкинс --- {\color{red}Правило большей распространенности элементов с чётными порядковыми номерами} 
\end{frame}

\begin{frame}
\frametitle{Отсутствие стабильных изотопов технеция}

1934 Правило запрета Маттауха — Щукарева: {\color{red}в природе не могут существовать два стабильных изобара, заряды ядра которых отличаются на единицу. }

Если у какого-либо химического элемента есть устойчивый изотоп, то его ближайшие соседи по таблице не могут иметь устойчивых изотопов с тем же массовым числом.  
Пример ---$_{43}Tc^{55}$.
\begin{figure}[ht] 
	\centering\small
	\unitlength=1mm
	\includegraphics[height=45mm, width=20mm]{Figures/stable_isotopes_Tc.png} 
%	\caption{Число стабильных изотопов у элементов} 
	\label{f:stable_isotopes_by_elementTc}
\end{figure}	


 
\end{frame}

\begin{frame}
\frametitle{Отсутствие стабильных изотопов технеция}
Правило запрета Маттауха — Щукарева объясняет, в частности, {\color{red}отсутствие стабильных изотопов у технеция} %несмотря на то, что он находится в таблице Менделеева задолго до свинца
: соседние с ним молибден и рутений имеют стабильные изотопы с массовыми числами 92, 94, 95, 96, 97, 98, 100 и 96, 98, 99, 100, 101, 102, 104, соответственно. 

\begin{figure}[ht] 
	\centering\small
	\unitlength=1mm
	\includegraphics[height=50mm]{Figures/MoAbudanceRuAbudance.png} 
	%	\caption{Число стабильных изотопов у элементов} 
\end{figure}	
\end{frame}

\begin{frame}
\frametitle{Отсутствие стабильных изотопов технеция}

https://www.nndc.bnl.gov/nudat3/
\begin{figure}[ht] 
	\centering\small
	\unitlength=1mm
	\includegraphics[width=90mm]{Figures/MoTcRuNZ.png} 
	%	\caption{Число стабильных изотопов у элементов} 
\end{figure}	
Запрет по Маттауху — Щукарёву изотопа $_{43}Tc^{55}$
\end{frame}


\begin{frame}
\frametitle{ИЗОТОПЫ. ПРОИСХОЖДЕНИЕ.}

\begin{center}
{\large
ЯДЕРНЫЙ НУКЛЕОСИНТЕЗ}
\end{center}

Литература
\begin{itemize}
    \item  В.П.\,Чечев, А.В.\,Иванчик, Д.А.\,Варшалович «Синтез элементов во Вселенной: От Большого взрыва до наших дней». 2020. 304 с. ISBN 978-5-9710-7626-1.
    \item Б.C. Ишханов, И.М. Капитонов, И.А. Тутынь. Нуклеосинтез во вселенной.
		М., Изд-во Московского университета. 1998.\\
		http://nuclphys.sinp.msu.ru/nuclsynt/index.html
  \item Бедняков В. А. О происхождении химических элементов. Э. Ч. А. Я., Том 33 (2002), Часть 4 стр.914-963.
\end{itemize}
\end{frame}

\begin{frame}
\frametitle{Происхождение элементов}

	\begin{figure}[ht] 
	\centering\small
	%	\unitlength=1mm
	\includegraphics[width=0.9\textwidth]{Figures/1920px-Elements-origin-ru.png}
	%	\includegraphics[width=30mm]{Figures\Oxygen.png}
	\caption{Происхождение элементов} https://en.wikipedia.org/wiki/Нуклеосинтез 
	\label{f:Nucleosynthesis_periodic_table}
\end{figure}
\end{frame}

\begin{frame}
\frametitle{Реакции, важные для нуклеосинтеза}

Michael S. Smith.	Nuclear data resources and initiatives for nuclear astrophysics 	November 2023Frontiers in Astronomy and Space Sciences 10 	DOI: 10.3389/fspas.2023.1243615
	\begin{figure}[ht] 
	\centering\small
	%	\unitlength=1mm
	\includegraphics[width=0.7\textwidth]{Figures/Reactions-of-importance-for-nuclear-astrophysics-shown-on-the-N-Z-plane-for-stable-nuclei.png}
	%	\includegraphics[width=30mm]{Figures\Oxygen.png}
	\caption{Реакции, важные для нуклеосинтеза} 
	% Michael S. Smith.	Nuclear data resources and initiatives for nuclear astrophysics 	November 2023. Frontiers in Astronomy and Space Sciences 10 	DOI: 10.3389/fspas.2023.1243615	
	\label{f:Reactions-of-importance-for-nuclear-astrophysics}
\end{figure}

\end{frame}

\begin{frame}
\frametitle{Относительная распространенность нуклидов}
Бедняков В. А. О происхождении химических элементов \ldots
	\begin{figure}[ht] 
		\centering\small
		\unitlength=1mm
		\includegraphics[width=110mm]{Figures/NuclidsN.png} 
		\caption{Относительная распространенность нуклидов от атомной массы \cite{ElementsOrigin}}
			Обозначения указывают на различные процессы синтеза 
		\label{f:NuclidsN}
	\end{figure}

\end{frame}

\begin{frame}
\frametitle{Схематическое изображение распространенности }
Бедняков В. А. О происхождении химических элементов \ldots
\begin{figure}[ht] 
		\centering\small
		\unitlength=1mm
		\includegraphics[width=65mm, height=55 mm]{Figures/NuclidsNSchematics.png} 
		\caption{Схематическое изображение распространенности нуклидов}
%			Обозначения указывают на различные процессы синтеза } 
		\label{f:NuclidsNSchematics}
	\end{figure}
\end{frame}

\begin{frame}
\frametitle{Иллюстрация хода нуклеосинтеза}
{\footnotesize
H. Schatz, A.D.B. Reyes, A Best et al. Horizons: nuclear astrophysics in the 2020s and beyond. \ldots  Vol 49, No 11, November 2022 pp.1-78. DOI:10.1088/1361-6471/ac8890}
\begin{figure}[h] 
	\centering\small
	\unitlength=1mm
	\includegraphics[width=100mm]{Figures/ChartNucleosysnthesis.png} 
%	\caption{Иллюстрация хода $s$- и $r$-процессов} 
	\label{f:ChartNucleosysnthesis}
\end{figure}
\end{frame}

\begin{frame}
\frametitle{Области $(n,\gamma)$, $\beta^{-}$, EC}
	\begin{figure}[h] 
		\centering\small
		\unitlength=1mm
		\includegraphics[width=125mm]{Figures/(N,Gamma) Beta-decay Electron Conversion.png} 
		\caption{Области $(n,\gamma)$, $\beta^{-}$, EC} 
		\label{f:(N,Gamma) Beta-decay Electron Conversion}
	\end{figure}
\end{frame}



\begin{frame}
\frametitle{Иллюстрация хода $s$- и $r$-процессов}
\begin{figure}[h] 
	\centering\small
	\unitlength=1mm
	\includegraphics[width=100mm]{Figures/NucleosythesisTrajectory.png} 
	\caption{Иллюстрация хода $s$- и $r$-процессов} 
	\label{f:NucleosythesisTrajector}
\end{figure}

\end{frame}

\begin{frame}
\frametitle{Сечения захвата тепловых нейтронов}
\begin{figure}[h] 
	\centering\small
	\unitlength=1mm
	\begin{picture}(100, 70)
	\put(0,0){\includegraphics[height=70mm]{Figures/NZngamma.png}}
	\end{picture}
	\caption{Сечения захвата тепловых нейтронов} 
	\label{f:NZngamma}
\end{figure}

\end{frame}

\begin{frame}
\frametitle{Обозначения  процессов нуклеосинтеза}
	{\small 
\begin{tabular}{lcl}
U & --- & космологический синтез до образования звёзд \\
H & --- & горение водорода \\
CNO & --- & горение водорода при высоких температурах (CNO-цикл) \\
He & --- & взрывное горение гелия \\	
С & --- & взрывное горение углерода \\
O & --- & взрывное горение кислорода \\
Si & --- & взрывное горение кремния \\
NSi & --- & обогащённое нейтронами горение кремния \\
E & --- & статическое ядерное равновесие \\
s & --- & s-процесс. Продукты медленного захвата нейтронов\\
r & --- & r-процесс. Продукты быстрого захвата нейтронов\\
p & --- & p-процесс. Процессы на обеднённой нейтронами стороне \\
 & & долины $\beta$-стабильности\\
X & --- & Дробление космическими лучами \\
\end{tabular}
}
\end{frame}

\begin{frame}
\frametitle{Графики данных по нуклеосинтезу}
\begin{figure}[ht] 
	\centering\small
	\unitlength=1mm
	\begin{picture}(120,60)
	\put(0,0){\includegraphics[width=120mm]{Figures/NucleosynthesisProcesses.png}}
	\end{picture}
%	\caption{Нуклеосинтез изотопов Таблицы Менделеева} 
	\label{f:NucleosynthesisProcesses}
\end{figure}
\end{frame}

\begin{frame}
\frametitle{Легкие элементы и группа стабильности}
\begin{figure}[ht] 
	\centering\small
	\unitlength=1mm
	\begin{picture}(120,60)
	\put(20,0){\includegraphics[height=60mm]{Figures/NucleosynthesisProcessesLightEQ.png}}
	\end{picture}
%	\caption{Нуклеосинтез изотопов лёгких элементов и группы стабильности} 
	\label{f:NucleosynthesisProcessesLightEQ}
 \end{figure}
\end{frame}

\begin{frame}
\frametitle{Редкоземельные элементы}
\begin{figure}[ht] 
	\centering\small
	\unitlength=1mm
	\begin{picture}(120,70)
	\put(10,0){\includegraphics[height=70mm]{Figures/NucleosynthesisProcessesREE.png}}
	\end{picture}
%	\caption{Нуклеосинтез изотопов редкоземельных элементов Таблицы Менделеева} 
	\label{f:NucleosynthesisProcessesREE}
\end{figure}
\end{frame}

\begin{frame}
\frametitle{Распространённость изотопов химических элементов в коре Земли --- происхождение}
\begin{figure}[ht] 
	\centering\small
	\unitlength=1mm
	\begin{picture}(120,65)
	\put(-10,0){\includegraphics[width=135mm]{Figures/Element_Abudance_Earth_Crust NucleosynthesisProcesses.png}}
	\end{picture}
%	\caption{Распространённость изотопов химических элементов в коре Земли} По данным https://www.nndc.bnl.gov/nudat3 
	\label{f:Element_Abudance_Earth_Crust NucleosynthesisProcesses}
\end{figure}
\end{frame}

\begin{frame}
\frametitle{Распространённость изотопов химических элементов в коре Земли --- изотопы}
\begin{figure}[ht] 
	\centering\small
	\unitlength=1mm
	\begin{picture}(120,65)
	\put(-10,0){\includegraphics[width=135mm]{Figures/Element_Abudance_Earth_Crust Isotopes.png}}
	\end{picture}
%	\caption{Распространённость изотопов химических элементов в коре Земли} По данным https://www.nndc.bnl.gov/nudat3
	\label{f:Element_Abudance_Earth_Crust Isotopes}
\end{figure}
\end{frame}

\begin{frame}
\frametitle{<<Отсутствующие>> элементы}

 В первой половине XX в. предложено эмпирическое правило Щукарева—Маттауха о невозможности одновременного существования стабильных изобар, заряды ядер которых отличаются на единицу.\\
 ~\\
На Земле очень мало технеция и прометия

\begin{figure}[ht] 
	\centering\small
	\unitlength=1mm
	\includegraphics[width=40mm]{Figures/IUPAC Table Mo Tc Ru.png} 
%	\caption{Изотопы Tc, Mo и Ru} --- https://applets.kcvs.ca/IPTEI/IPTEI.html/
	\label{f:IUPAC Table Mo Tc Ru}
\end{figure}

\begin{figure}[ht] 
	\centering\small
	\unitlength=1mm
	\includegraphics[width=40mm]{Figures/IUPAC Table Nd Pm Sm.png} 
%	\caption{Изотопы Nd, Pm  и Sm} --- hhttps://applets.kcvs.ca/IPTEI/IPTEI.html/
	\label{f:IUPAC Table  Nd Pm Sm}
\end{figure}
\end{frame}

\begin{frame}
\frametitle{Случай технеция}

По правилу Щукарева-Маттауха, так как молибден и рутений имеют стабильные изотопы с массовыми числами $92, 94, 95, 96, 97, 98, 100$ и $96, 98, 99, 100, 101, 102, 104$, стабильный технеций (и его изотопы), не могут существовать.

\begin{figure}[ht] 
	\centering\small
	\unitlength=1mm
	\includegraphics[width=100mm]{Figures/MoTcRuNZmarked.png} 
%	\caption{Стабильные изотопы Mo и Ru --- чёрные квадраты} --- https://www.nndc.bnl.gov/nudat3 
	\label{f:T half time Mo Tc Ru.png}
\end{figure}




\end{frame}

\begin{frame}
\frametitle{Случай прометия}

Для всех изотопов технеция положительны энергии как $Q_{\beta}$ бета-распада, так и электронной конверсии $Q_{EC}$. 
В то же время у неодима и самария есть стабильные изотопы в диапазоне $A$ от 142 до 150.
\begin{figure}[ht] 
	\centering\small
	\unitlength=1mm
	\includegraphics[width=110mm]{Figures/T half time Nd Pm Sm.png} 
%	\caption{Стабильные изотопы Nd, Pm  и Sm --- чёрные квадраты} --- https://www.nndc.bnl.gov/nudat3
	\label{f:T half time MNd Pm Sm.png}
\end{figure}


\end{frame}

\begin{frame}
\frametitle{Случай хлора}
{\footnotesize
Ещё в XIX веке было известно, что атомная масса хлора заметно отличается от целочисленной. В частности, в аннотации пионерской публикации Д.И.\,Менделеева 1869 г.
<<Система химических элементов согласно их атомным весам и химическим свойствам>> 
был указан атомный вес, равный 35.5, что никак нельзя объяснить неточностью измерений}
\begin{figure}[ht] 
	\centering\small
	\unitlength=1mm
	\includegraphics[height=60mm]{Figures/T half time S Cl Ar.png} 
%	\caption{Стабильные изотопы S Cl Ar --- чёрные квадраты} --- https://www.nndc.bnl.gov/nudat3 
	\label{f:T half time S Cl Ar}
\end{figure}


\end{frame}

\begin{frame}
\frametitle{Случай хлора}
\begin{figure}[ht] 
	\centering\small
	\unitlength=1mm
	\includegraphics[width=70mm]{Figures/Cl with neighbors1d.png} 
	\caption{Графики энергий бета-распада и электронной конверсии для изотопов Cl, S  и Ar}% --- https://www.nndc.bnl.gov/nudat3/
	\label{f:Cl with neighbors1d}
\end{figure}
Таким образом, изотоп $Cl^{36}$ переходит путём бета-распада в изотоп  $Ar^{36}$ или путём электронного захвата в изотоп  $S^{36}$.

\end{frame}

\begin{frame}
\frametitle{Случай олова}
Три и более изотопа имеют почти все изотопы с чётным числом нуклонов.  
~\\
Особенно большое количество изотопов имеют элементы Pd, Cd, Sn, Te, Xe, Ba с электрическим зарядом  46, 48, 50, 52, 54, 56:
от 6 до 10.\\
~\\
Олово --- абсолютный рекордсмен по числу стабильных изотопов. Их всего 10!

\begin{figure}[ht] 
	\centering\small
	\unitlength=1mm
	\includegraphics[width=120mm]{Figures/T half time Sn In Sb.png} 
%	\caption{Стабильные изотопы Sn In Sb --- чёрные квадраты} --- https://www.nndc.bnl.gov/nudat3 
	\label{f:T half time Sn In Sb}
\end{figure}
\end{frame}

\begin{frame}
\frametitle{Случай олова}
\begin{figure}[ht] 
	\centering\small
	\unitlength=1mm
	\includegraphics[width=80mm]{Figures/Sn with neighbors1d.png} 
	\caption{Графики энергий бета-распада и электронной конверсии для изотопов Sn, In  и Sb} %--- https://www.nndc.bnl.gov/nudat3/
	\label{f:Sn with neighbors1d}
\end{figure}
\end{frame}

\begin{frame}
\frametitle{Примеры нуклеосинтеза, отличного от солнечной системы}

\begin{itemize}
    \item Карбоновые (углеродные) звёзды
    \item Технециевые звёзды
\end{itemize}



\end{frame}

\begin{frame}
\frametitle{Различие между  Солнцем и планетами}
{\footnotesize
 на Земле присутствуют элементы-потомки весьма разных предков, так что элементный состав Земли, во-первых богаче, чем продукция нуклеосинтеза Солнца, поскольку Солнечная система --- результат взрыва \emph{сверхновой звезды}. 	Во-вторых, между планетами и другими объектами Солнечной системы могут быть различия в распространённости элементов и их изотопном составе.}
	\begin{figure}[ht] 
	\centering\small
	%	\unitlength=1mm
	\includegraphics[width=0.7\textwidth]{Figures/1920px-Elements-origin-ru.png}
	%	\includegraphics[width=30mm]{Figures\Oxygen.png}
%	\caption{Происхождение элементов} %https://en.wikipedia.org/wiki/Нуклеосинтез 
	\label{f:Nucleosynthesis_periodic_table}
\end{figure}

\end{frame}

%%%%%%%%%%%%%%%%%%%%%%%%%%%%%%%%%%%%%%%%%%%%%%%%%%%%%%%%%%%%%%%%%%%%%%%%%%%%%%%%%%%%%%%%%%%%%%%%%%%%%%%%%%%%%%%%%%
\begin{frame}
\frametitle{IUPAC и CIAAW}

Накопление данных, поддержка табличных данных по изотопам в природе\\
~\\

Организации
\begin{itemize}
	\item
	Международный союз теоретической и прикладной химии IUPAC (International Union of Pure and Applied Chemistry) 
	\item
	Комиссия по изотопному и атомному весу (Commission on Isotopic Abundances and Atomic Weights, CIAAW) 
\end{itemize}

\end{frame}

%\begin{frame}
%\frametitle{Определение содержания изотопов в образцах}


%\end{frame}

%%%%%%%%%%%%%%%%%%%%%%%%%%%%%%%%%%%%%%%%%%%%%%%%%%%%%%%%%%%%%%%%%%%%%%%%%%%%%%%%%%%%%%%%%%%%%%%%%%%%%%%%%%%%%%%%%%%
\begin{frame}
\frametitle{Описание изменчивости изотопного состава}
\emph{Атомная масса} $\mbf{m_a(^iE)}$ несвязанного нейтрального атома нуклида $^iE$ элемента $E$ с массовым числом $i$  определяется как <<масса покоя атома в его основном состоянии>>. 

{\bf Атомный вес}  или {\bf относительная атомная масса}, $A_r(^iE)$, {\bf атома} (нейтрального нуклида в свободном состоянии) $^iE $ элемента $E$ определяется как <<отношение массы атома к универсальной атомной единице массы>>. 
\medskip
Атомная массовая постоянная $m_u$ равна дальтону (Да) или универсальной атомной единице массы $u$ и определяется через массу атома углерода-12:
\begin{equation}
m_u = 1 u = 1 Da = m_a(^{12}C)/12.
\end{equation}
Таким образом, атомный вес есть безразмерная величина:
\begin{equation} \label{ArE}
A_r(^iE) = m_a(^{i}E)/ \left[ m_a(^{12}C)/12 \right] 
\end{equation}


\end{frame}

\begin{frame}
\frametitle{Описание изменчивости изотопного состава}

{\bf Атомный вес элемента} $E$ в веществе $P$, $A_r(E, P)$ это средневзвешенное значение атомных весов
$A_r(^iE)$ изотопов (нуклидов) $^iE$ этого элемента в веществе $P$: \index{атомный вес элемента $E$ в веществе $P$, $A_r(E, P)$}
\begin{equation}  \label{ArEP}
A_r(E, P) = \sum \chi(^iE, P)A_r(E)
\end{equation}
Здесь $\chi(^iE, P)$ — количественная доля изотопа $^iE$ в веществе $P$ (также называемая изотопным составом), 
а суммирование проводится по всем стабильным изотопам и радиоактивным изотопам, имеющим характерные земные \emph{изотопные подписи} и они перечислены в Таблице изотопных составов элементов. 

\medskip
Атомный вес элемента в данном веществе можно определить, зная атомные массы изотопов и соответствующие количественные доли изотопов этого элемента в этом конкретном веществе.

\end{frame}

\begin{frame}
\frametitle{Описание изменчивости изотопного состава}

{\bf Cтандартный атомный вес элемента}, $A_{r}{\circ}(E)$, представляет собой <<рекомендуемое значение атомного веса (относительно атомная масса) элемента, пересматриваемого каждые два года комиссией CIAAW и применимого к элементам в любом обычном материале 
с высоким уровнем достоверности>>. 

\medskip
Он состоит либо из интервала (в настоящее время используется для 14 элементов), либо из базового значения и неопределенности (стандартная неопределенность атомного веса), которые в настоящее время используются для 71 элемента. 

\medskip
Стандартный атомный вес определяется на основе оценки рецензируемых научных
публикации. 
\end{frame}

\begin{frame}
\frametitle{Измерения величины $\delta$ изотопов.}
Обычно измерения изотопной дельты являются основой для определения атомного веса.

\medskip
Величина $\delta$ изотопа получается из отношения числа изотопов $R(^{i/j}E)$ в веществе $P$:
\begin{equation} \label{REP}
R(^{i/j}E, P) =  \frac{N(^iE, P)}{N(^jE, P)}  % (3)
\end{equation}
где $N(^iE, P)$ и $N(^jE, P)$ — число атомов каждого изотопа, а $^iE$ в общем случае обозначает наибольшее
(верхний индекс $i$) и $^jE$ наименьшее (верхний индекс $j$) атомные массовые числа изотопов химического элемента $E$ в вещество $P$.

\medskip
 $^jE$ представляет эталонный изотоп. 

\end{frame}

\begin{frame}
\frametitle{Измерения величины $\delta$ изотопов.}
Дельта-значение изотопов (символ $\delta$), также называемое разностью относительных изотопных отношений, представляет собой дифференциальное измерение, полученное из соотношения изотопов вещества $P$ и шкалы, представленной опорным материалом. 
\begin{equation} \label{DeltaIsotopesDef} % (4)
\delta_{\tt Ref}(^{i/j}E, P)  = \frac{R(^{i/j}E, P)}{R(^{i/j}E, {\tt Ref})} - 1.
\end{equation}

\medskip
Дельта-значения изотопов являются небольшими числами и поэтому часто представляются кратными $10^{-3}$ или промилле (символ $\permille$). 


\end{frame}


\begin{frame}
\frametitle{Измерения величины $\delta$ изотопов --- пример $^{13}C$.}

Чтобы согласовать дельта-шкалу изотопов элемента со шкалой количеств изотопов, необходимо вещество (образец),
содержание изотопов и дельта-значения изотопов которых также хорошо известны.\\
~\\

Например, для изотопа углерода $^{13}C$ шкала содержания согласуется посредством измерения изотопного эталона
материала NBS 19 (карбонат кальция).

%Например, для углерода $^{13}C$ шкала содержания согласуется с  

\medskip

$\delta_{\tt VPDB}(^{13/12}C)$ посредством измерения изотопного эталона
материал NBS 19 (карбонат кальция), которому было присвоено согласованное значение $\delta_{\tt VPDB}(^{13/12}C, {\tt NBS 19}) = +1.95 \permille$. 

\medskip
Отношение числа изотопов углерода для NBS 19 составляет
$R(^{13/12}C, {\tt NBS 19}) = 0.011 202 \pm 0.000 028$. 
%Это измерение служит <<наилучшим измерением одного наземного источника>>. Белемнит Vienna Peedee (VPDB) является нулевой точкой на шкале дельта-изотопов углерода и, следовательно,$  \delta_{\tt VPDB}(^{13/12}C, {\tt VPDB}) = 0$. 
%Поскольку $1 \permille = 0.001$, отсюда следует: %\begin{equation} % (5) R(^{13/12}C, {\tt VPDB}) = 0.011 202/(1 + 1.95 \times 0.001) = 0.011 180 \end{equation}

%Таким образом, без учета неопределенности соотношение между значениями дельты изотопов углерода ($\delta$) и $^{13}C$ составляет доли ($\chi$) материала P
%\begin{equation} %(6) \chi(^{13}C, P) = 1/ \left[  1 + 1/ \left\lbrace  R (^{13/12}C, {\tt VPDB} ) \times \left[  1 + \delta_{\tt VPDB}(^{13}C, P) \right]  \right\rbrace \right]  \end{equation}
\end{frame}



%%%%%%%%%%%%%%%%%%%%%%%%%%%%%%%%%%%%%%%%%%%%%%%%%%%%%%%%%%%%%%%%%%%%%%%%%%%%%%%%%%%%%%%%%%%%%%%%%%%%%%%%%%%%%%%%%%%



\begin{frame}
\frametitle{Периодическая таблица элементов и изотопов}


\begin{center}
	ПЕРИОДИЧЕСКАЯ ТАБЛИЦА ЭЛЕМЕНТОВ И ИЗОТОПОВ
\end{center}

\end{frame}


\begin{frame}
	\frametitle{Переход от элементов к изотопам}
	
В отчете 1969 г. Комиссия CIAAW впервые признала, что:\\
«Открытие того, что большинство химических элементов существует в природе в виде изотопных смесей,
которые, как известно, различаются по составу, делает необходимым изменить историческое
понятие атомных весов как констант природы. 
Несмотря на то, что для некоторых элементов (для 21) [стабильные] изотопы не найдены в природе, представляется более логичным считать, что \footnote{выделение автора презентации}
\begin{center}
{\color{red}	изотопные смеси представляют собой нормальное, а не исключительное состояние элемента. }
\end{center} 
\ldots Комиссия будет использовать процедуры определения весов, так чтобы значения были оптимизированы для
материалов в мировой науке, химической технологии и торговле, а не представлять расчетное среднее геохимическое значение».
	
\end{frame}

\begin{frame}
	\frametitle{Таблица стандартных атомных весов 2021}
Часть таблицы
\begin{table}[h!]
	{\scriptsize 
		\begin{tabular}{ccccccc}
			Символ  & Атомный &  Вес  & Неопреде- & Комм-й &  Вес  & Неопреде-  \\
			~ & номер &  ~  & лённость &  ~  &  ~  & лённость  \\
			\hline 
			2 & 3 & 4 & 5  & 6 & 7 & 8\\
			\hline 
			~ & ~ & ~ & ~ & ~  & ~ & ~\\
			H & 1 & [1.000784, 1.00811] & ~ & m & 1.0080 & 0.0002\\ [1mm]
			He & 2 & 4.002602 & 0.00002 &  g r & 4.0026 & 0.0001 \\ [1mm]
			Li & 3 & [6.938, 6.997] & ~ & m  & 6.94 & 0.06 \\ [1mm]
%			Be & 4 & 9.0121831 & 0.0000005 & ~  & 9.0122 & 0.0001 \\ [1mm]
			B & 5 & [10.806, 10.821] & ~ & m  & 10.81 & 0.02 \\ [1mm]
			C & 6 & [12.0096, 12.0116] & ~ & ~  & 12.011 & 0.002 \\ [1mm]
			N & 7 & [14.00643 14.00728] & ~ & m & 14.007 & 0.001 \\ [1mm]
			O & 8 & [15.99903, 15.99977] & ~ & m  & 15.999 & 0.001 \\ [1mm]
%			F & 9 & 18.998403162 & 0.000000005 & ~  & 18.998 & 0.0001 \\ [1mm]
%			Ne & 10 & 20.1797 & 0.0006 & g m  & 20.180 & 0.0001 \\ [1mm]
%			Na & 11 & 22.98976928 & 0.00000002 & ~  & 22.990 & 0.0001 \\ [1mm]
%			Mg & 12 & [24.304, 24.307] & ~ & ~  & 24.305 & 0.002 \\ [1mm]
%			Al & 13 & 26.9815384 & 0.0000003 & ~  & 26.982 & 0.0001 \\ [1mm]
%			Si & 14 & [28.084, 28.086] & ~ & ~  & 28.085 & 0.001 \\ [1mm]
%			P & 15 & 30.973761998 & 0.000000005 & ~  & 30.974 & 0.0001 \\ [1mm]
			S & 16 & [32.059, 32.076] & ~ & ~  & 32.06 & 0.02 \\ [1mm]
			Cl & 17 & [35.446, 35.457] & ~ & m  & 35.45 & 0.01 \\ [1mm]
			Ar & 18 & [39.792, 39.963] & ~ & ~  & 39.95 & 0.16 \\ [1mm]
%			K & 19 & 39.0983 & 0.001 & g  & 39.098 & 0.001 \\ [1mm]
			Br & 35 & [79.901, 79.907] & ~ & g  & 79.904 & 0.003\\ [1mm]
			Tl & 81 & [204.382, 204.385] & ~ & ~  & 204.38 & 0.01\\ [1mm]
			Pb & 82 & [206.14, 207.94] & ~ & ~  & 207.2 & 1.1\\ [1mm]
%			$\ldots$ & $\ldots$ & $\ldots$ & $\ldots$ & $\ldots$  & $\ldots$ & $\ldots$\\ [1mm]
			\hline 
		\end{tabular}
	}
%	\caption{Таблица стандартных атомных весов 2021}
%	\label{t:TSAW2021}
\end{table} 
	
	

	
	
\end{frame}

\begin{frame}
	\frametitle{Периодическая таблица элементов и изотопов}
	
https://applets.kcvs.ca/IPTEI/IPTEI.html

\begin{figure}[ht] 
	\centering\small
	\unitlength=1mm
	\includegraphics[width=120mm]{Figures/PeriodicTable2021all.png}
	\caption{Интерактивная таблица Менделеева элементов и изотопов, https://applets.kcvs.ca/IPTEI/IPTEI.html} 
	\label{f:PeriodicTable}
\end{figure}	
	
\end{frame}

\begin{frame}
\frametitle{Периодическая таблица элементов и изотопов}

Цветовая легенда

\begin{figure}[ht] 
	\centering\small
	\unitlength=1mm
	\includegraphics[width=60mm]{Figures/IPTEIabcd.png}
	\caption{Интерактивная таблица Менделеева элементов и изотопов, легенда} 
	\label{f:IPTEIabcd}
\end{figure}	

\end{frame}

\begin{frame}
\frametitle{Периодическая таблица элементов и изотопов}

<<Розовая>> информация

\begin{figure}[ht] 
	\centering\small
	\unitlength=1mm
	\includegraphics[width=100mm]{Figures/IPTEI2021pink.png}
	\caption{Интерактивная таблица Менделеева элементов и изотопов, случай интревального представления} 
	\label{f:IPTEI2021pink}
\end{figure}	

\end{frame}

\begin{frame}
	\frametitle{<<НЕЗЕМНЫЕ>> ЭЛЕМЕНТЫ}
	
	\begin{center}
		{\large
			<<НЕЗЕМНЫЕ>> ЭЛЕМЕНТЫ}
	\end{center}
	
В документах  IUPAC-2021 особое место уделено двум элементам, которые имеют огромные по отношению к другим элементам, вариации изотопного состава.

\begin{itemize}
	\item Аргон
	\item Свинец
\end{itemize}	
\end{frame}


\begin{frame}
	\frametitle{Аргон --- тяжёлая невидимка атмосферы}


Аргон является уникальным веществом.


 Удивительно, что газ, который составляет около 1\% атмосферы по массе, был открыт только в конце XIX в. 
 
 Причиной такого <<равнодушия>> является химическая инертность этого газа. Собственно, это первый \emph{благородный} газ, открытый на Земле.

\begin{figure}[ht] 
	\centering\small
	%	\unitlength=1mm
	\includegraphics[width=0.7\textwidth]{Figures/ArgonIsotopes.png}
	\caption{Период полураспада изотопов аргона и его стабильные изотопы} 
	%--- https://applets.kcvs.ca/IPTEI/IPTEI.html
	\label{f:ArgonIsotopes}
\end{figure}
		
\end{frame}

\begin{frame}
	\frametitle{Изотопы аргона и калия}
	
Аргон представлен в земной атмосфере тремя стабильными изотопами: $^{36}Ar$ (0.337 \%), $^{38}Ar$ (0.063\%), $^{40}Ar$ (99.600 \%). При этом в солнечной фотосфере и в атмосферах планет-гигантов изотопное содержание $^{40}Ar$ составляет лишь $\approx 0.01$ \%.
	
\begin{figure}[ht] 
	\centering\small
	%	\unitlength=1mm
	\includegraphics[width=0.6\textwidth]{Figures/40K40Ar.png}
	\caption{Распространённость изотопов калия и аргона \un{на Земле}.} 
%	--- по данным https://www.nndc.bnl.gov/nudat3/
	\label{f:40K40Ar}
\end{figure}

Образование аргона $^{40}Ar$ идёт путём захвата орбитального электрона 
\begin{equation}\label{40K40Ar}
	^{40}_{19}K + e^{-} \longrightarrow \ ^{40}_{18}Ar + \ov{\nu}_e.
\end{equation}

\end{frame}

\begin{frame}
	\frametitle{Вариации атомного веса  аргона}
	
\begin{figure}[ht] 
	\centering\small
	%	\unitlength=1mm
	\includegraphics[width=0.9\textwidth]{Figures/36ArgonIUPAC2021.png}
	\caption{Вариации атомного веса (чёрные линии) аргона, $A_r(Ar)$, и значения долей	(розовые линии) изотопа $^{36}Ar$, $\chi(36Ar)$, для некоторых веществ}
	\label{f:ArgonVar}
\end{figure}
	
\end{frame}


\begin{frame}
	\frametitle{Представление аргона в таблице Менделеева}
	
\begin{figure}[ht] 
	\centering\small
	%	\unitlength=1mm
	\includegraphics[width=0.4\textwidth]{Figures/Argon.png}
	\caption{Представление аргона в таблице Менделеева} 
	\label{f:Argon}
\end{figure}
\end{frame}




\begin{frame}
	\frametitle{Свинец --- не солнечные изотопы}
	
	\begin{figure}[ht] 
		\centering\small
		%	\unitlength=1mm
		\includegraphics[width=0.25\textwidth]{Figures/Lead.png}
		\caption{Представление свинца в таблице Менделеева} 
		\label{f:Lead}
	\end{figure}
	
{\small
	Изотопный портрет свинца весьма богат. При этом плотность свинца в различных материалах имеет рекордную изменчивость. Она настолько велика, что \un{превышает единицу величины атомной массы}. Эта разница была обнаружена даже при использовании весьма скромных по точности средств измерения. }
	
	
\end{frame}

\begin{frame}
	\frametitle{Происхождение изотопов свинца}
	
Изотопный состав и атомный вес свинца в земных материалах различны, поскольку три его самых тяжелых стабильных изотопа являются стабильными конечными продуктами радиоактивного распада различных изотопов урана 
\begin{align*}
	& ^{238}U \longrightarrow \ldots \longrightarrow  ^{206}Pb,  \\ 
	& ^{235}U \longrightarrow \ldots \longrightarrow  ^{207}Pb, 
\end{align*}
и тория 
\begin{equation*} 
	^{232}Th \longrightarrow \ldots \longrightarrow ^{208}Pb.
\end{equation*}
\end{frame}

\begin{frame}
	\frametitle{Ряды тория, радия и актиния}
	
\begin{figure}[ht] 
	\centering\small
	\unitlength=1mm
	\begin{picture}(130,58)
		\put(-10,0){\includegraphics[width=130mm]{Figures/Lead206207208SeriesAZ.png}}
	\end{picture}
	\caption{Ряды тория, радия и актиния --- основные ветви} %	--- по данным https://www.nndc.bnl.gov/nudat3/
	\label{f:Lead206207208SeriesAZ}
\end{figure}
	
\end{frame}

\begin{frame}
%	\frametitle{Изотопы свинца на Земле}
	
\begin{figure}[ht] 
	\centering\small
	%	\unitlength=1mm
	\includegraphics[width=70mm]{Figures/Pb204.png}
%	\caption{Вариации изотопа $^{204}Pb$ в различных объектах} 
	\label{f:204LeadVariation}
\end{figure}

	
	
\end{frame}

\begin{frame}
		\frametitle{Вариации изотопа $^{204}Pb$ в осадочных породах}

\begin{table}[h!]
	\begin{center}
		{\small
			\begin{tabular}{ccc}
				\hline
				Категория & Нижняя & Верхняя \\
				~ & граница & граница \\ 
				\hline
				Доломит	& 0.0115 & 0.0136 \\
				Известняк	& 0.0078 & 0.0160 \\
				Мергель 	& 0.0118 & 0.0137 \\
				Алевриты 	& 0.0113 & 0.0131 \\
				Фосфориты & 0.0096 & 0.0140 \\
				\hline
			\end{tabular}
		}
		\caption{Вариации изотопа $^{204}Pb$ в осадочных породах,  $ \chi ^{204}Pb$.}
		\label{t:204LeadVariation}
	\end{center}
\end{table}

\end{frame}

\begin{frame}
\frametitle{ИЗОТОПЫ В ПРИРОДЕ}

\begin{center}
{\large
ИЗОТОПЫ В ПРИРОДЕ}
\end{center}

Изотопные ниши, ландшафты, подписи:\\
 А.Н.~Баженов.	Естественнонаучные и технические применения интервального анализа: учебное пособие. Санкт-Петербург, 2022.\\
 https://elib.spbstu.ru/dl/5/tr/2021/tr21-169.pdf/info\\
~\\
 А.Н.~Баженов, А.Ю.~Тельнова.	Изотопы и таблица Менделеева: учебное пособие. Санкт-Петербург, 2024.\\
 https://elib.spbstu.ru/dl/5/tr/2024/tr24-29.pdf/info\\
 ~\\
 А.Н.~Баженов.	Интервальные арифметики и прослеживаемость изотопной подписи: учебное пособие. Санкт-Петербург, 2023.\\
 https://elib.spbstu.ru/dl/5/tr/2023/tr23-167.pdf/info
 
\end{frame}


\begin{frame}
\frametitle{Изотопная подпись}
{\small
Согласно Википедии, <<Изотопная подпись (иначе, изотопная сигнатура) --- специфическое соотношение нерадиоактивных <<стабильных изотопов>> или относительно стабильных радиоактивных изотопов или неустойчивых радиоактивных изотопов определённых химических элементов в исследуемом материале>>. } %--- https://ru.wikipedia.org/wiki/Изотопная\_подпись

\begin{figure}
	\centering\small
	\unitlength=1mm
	\includegraphics[width=100mm]{Figures/Elements-and-isotopes-in-bones-also-teeth-and-their-applications-for-palaeoenvironemtal.png} 
%	\caption{Изотопы в костях и зубах \cite{Tutken2011}.} 
	\label{f:Bones}
\end{figure}

\end{frame}


\begin{frame}
\frametitle{Внедрение изотопных методов в контроле пищевой продукции.}
{\small
Изотопный анализ является эффективным методом осуществления контроля качества и выявления
фальсифицированной пищевой продукции. Традиционно применяемым в области изотопного анализа пищевой
продукции и регламентированным в соответствующих нормативных документах методом является метод изотопной масс-спектрометрии с элементным анализатором (далее --- EA-IRMS).\\
~\\
В настоящий момент наибольший интерес в рамках изотопного анализа пищевой продукции представляет стремительно развивающийся и обладающий рядом достоинств метод измерений отношения изотопов углерода --- метод спектроскопии внутрирезонаторного затухания с модулем сжигания (далее --- CM–CRDS). 
} 
\end{frame}

\begin{frame}
\frametitle{Внедрение изотопных методов в контроле пищевой продукции.}
{\small
	В публикации\\ \textsc{Чубченко~Я.К.} Методика измерений отношения изотопов углерода в ванилине методом CM–CRDS с расширенной неопределенностью менее 0,1 \%. Эталоны. Стандартные образцы. 2023;19(3):129-144. https://doi.org/10.20915/2077-1177-2023-19-3-129-144 \\
	продемонстрировано, что
	<<Результаты международных сличений СCQM-K167 подтвердили возможность измерений отношения изотопов
	углерода в ванилине методом CM–CRDS по разработанной методике \ldots соответствует наилучшим измерениям, выполняемым методом EA-IRMS.
	Достигнутый результат обладает практической значимостью, потому что подтверждает возможность применения
	метода CM–CRDS для осуществления контроля качества и выявления фальсифицированного ванилина.>>
}
\end{frame}

\begin{frame}
\frametitle{Аттестованные методики измерений}
{\footnotesize
	ФР.1.31.2012.13424 Методика измерений отношения изотопов 13С/12С этанола в пиве и пивных напитках методом изотопной масс-спектрометрии // Федеральный информационный фонд по обеспечению единства измерений : официальный сайт.
	URL: https://fgis.gost.ru/fundmetrology/registry/16/items/282517.	\\
	~\\
	ФР.1.31.2013.15529 Методика устанавливает процедуру определения отношения изотопов 18О/16О экзогенной и эндогенной воды в винах и суслах // Федеральный информационный фонд по обеспечению единства измерений : официальный сайт.
	URL: https://fgis.gost.ru/fundmetrology/registry/16/items/281364.	\\
	~\\
	ФР.1.31.2014.17273 Методика измерений отношения изотопов углерода 13С/12С в спиртных напитках виноградного происхожде	ния методом изотопной масс-спектрометрии // Федеральный информационный фонд по обеспечению единства измерений :
	официальный сайт. URL: https://fgis.gost.ru/fundmetrology/registry/16/items/280000 	\\
	~\\
	ФР.1.31.2016.24753 Методика измерений отношения изотопов кислорода, 18O/16О экзогенной и эндогенной воды в винах и суслах методом изотопной масс-спектрометрии // Федеральный информационный фонд по обеспечению единства измерений : офи-
	циальный сайт. URL: https://fgis.gost.ru/fundmetrology/registry/16/items/298716  	\\
	~\\
	ФР.1.31.2016.24962 Методика измерений отношений изотопов этанола в коньяках и коньячных дистиллятах методом изотопной масс-спектрометрии // Федеральный информационный фонд по обеспечению единства измерений : официальный сайт.
	URL: https://fgis.gost.ru/fundmetrology/registry/16/items/298746  \\
	~\\
	ФР.1.31.2017.28360 Методика измерений отношений изотопов углерода, кислорода, водорода этанола для выявления присутствия синтетического спирта в алкогольной продукции, а также в спиртосодержащих пищевых ароматизаторах методом изотопной масс-спектрометрии // Федеральный информационный фонд по обеспечению единства измерений : официальный
	сайт. URL: https://fgis.gost.ru/fundmetrology/registry/16/items/299163  \\
	~\\
	ФР.1.31.2018.31997 Методика измерений отношения изотопов кислорода 18O/16O водной компоненты сидров и пуаре методом изотопной масс-спектрометрии // Федеральный информационный фонд по обеспечению единства измерений : официальный
	сайт. URL: https://fgis.gost.ru/fundmetrology/registry/16/items/495958 
} 
\end{frame}
  

\begin{frame}
\frametitle{Изотопная планетология}
Изотопные отношения для ряда планет Солнечной системы и марсианского метеорита Allan Hills 84001, умноженные на $10^3$\\
~\\
\begin{table}[h!]
	\begin{center}	
		%		{\small
		\begin{tabular}{|c|cccc|}
			\hline 
			Изотоп. отн.  & Земля &  Марс  & Венера & ALH84001   \\
			\hline 
			$^{15}N / ^{14}N$ & $3.66 \pm 0.01$ & $5.8 \pm 0.4$ & $3.7 \pm 0.7$ & 3.875\\ [1mm]
			$^{13}C / ^{12}C$ & $11.23 \pm 0.05$ & $11.75 \pm 0.04$ & $12 \pm 2$ &  $11.75 \pm 0.09$ \\ [1mm]
			\hline 
		\end{tabular}
%		\caption{Изотопные отношения для ряда планет} 
%		{\small Солнечной системы и марсианского метеорита Allan Hills 84001, умноженные на $10^3$}
		\label{t:PlanetsIsotope}
		%	}
	\end{center}
\end{table} 
~\\
Как видно из данных таблицы, имеются значимые различия в изотопном составе планет и метеоритов. 
\end{frame}


\begin{frame}
\frametitle{Изотопологи}

\emph{Изотопологи} --- молекулы, различающиеся только по изотопному составу атомов, из которых они состоят. Изотополог имеет в своём составе, по крайней мере, один атом определенного химического элемента, отличающийся по количеству нейтронов от остальных.

\end{frame}


\begin{frame}
\frametitle{Изотопологи воды}

\emph{Изотопологи} --- молекулы, различающиеся только по изотопному составу атомов, из которых они состоят. Изотополог имеет в своём составе, по крайней мере, один атом определенного химического элемента, отличающийся по количеству нейтронов от остальных.

Комбинации 5 стабильных изотопов водорода и кислорода дают набор 9 молекул-изотопологов воды. 
Природная вода представляет собой многокомпонентную смесь изотопологов. Содержание самого лёгкого изотополога в ней значительно превосходит концентрацию всех остальных вместе взятых. Содержание тяжёлых изотопов водорода и кислорода в природных водах определяется двумя международными стандартами, введенными Международным агентством по атомной энергии (МАГАТЭ):
\begin{itemize}
	\item Стандарт VSMOW (Vienna Standard Mean Ocean Water) определяет изотопный состав глубинной воды Мирового океана \cite{VSMOW} \index{VSMOW, Vienna Standard Mean Ocean Water}
	\item Стандарт SLAP (Standard Light Antarctic Precipitation) определяет изотопный состав природной воды из Антарктики \cite{SLAP} \index{SLAP, Standard Light Antarctic Precipitation}
\end{itemize}

Стандарт SLAP характеризует самую лёгкую природную воду на Земле, VSMOW --- глубинную океаническую воду. 

\end{frame}


\begin{frame}
\frametitle{Изотопологи воды}
Рассчитанные весовые количества изотопологов в природной воде, соответствующие международным стандартам SMOW (средняя молекулярная масса = 18,01528873) и SLAP (средняя молекулярная масса = 18,01491202) 
\begin{table}[h!]
	\begin{center}
		{\footnotesize
			\begin{tabular}{cccc}
				\hline
				Изотополог воды	& Молекулярная масса &	Содержание, г/кг & ~  \\
				~	& ~ &	SMOW & SLAP \\
				\hline
				1H216O &	18,01056470	& 997,032536356	& 997,317982662 \\
				1HD16O &	19,01684144	& 0,328000097&	0,187668379 \\
				D216O &	20,02311819	& 0,000026900	& 0,000008804\\
				1H217O	&19,01478127	& 0,411509070 &	0,388988825\\
				1HD17O &	20,02105801	& 0,000134998	& 0,000072993\\
				D217O &	21,02733476 &	0,000000011	& 0,000000003\\
				1H218O &	20,01481037	& 2,227063738	& 2,104884332\\
				1HD18O &	21,02108711	&0,000728769	&0,000393984\\
				D218O &	22,02736386	&0,000000059&	0,000000018	\\			
				\hline
			\end{tabular}
		}
	\end{center}
\end{table}
В природной воде весовая концентрация тяжёлых изотопологов может достигать 2.97 г/кг, что является значимой величиной, сопоставимой, например, с содержанием минеральных солей.

\end{frame}


\begin{frame}
\frametitle{Изотопологи воды}
Природная вода, близкая по содержанию изотополога 1H216O к стандарту SLAP, а также специально очищенная с существенно увеличенной долей этого изотополога по сравнению со стандартом SLAP, определяется как особо чистая лёгкая вода (менее строгое определение, которое применимо в реальной жизни).

В лёгкой воде доля самого лёгкого изотополога составляет (мол.\%):

\begin{equation*}
99.76 \leq \ ^1H_2\,^{16}O \ \leq 100.
\end{equation*}

Если из воды, отвечающей стандарту SMOW, удалить все тяжёлые молекулы, массовое содержание которых составляет 2,97 г/кг и заменить их на 1H216O, то масса 1 л такой лёгкой и изотопно чистой воды уменьшится на 250 мг. \\
Таким образом, параметры лёгкой воды, в первую очередь, её <<лёгкость>> и изотопный состав поддаются измерению с помощью таких методов, как масс-спектрометрия, гравиметрия, лазерная абсорбционная спектроскопия, ЯМР.

\end{frame}


\begin{frame}
\frametitle{Изотопомеры}

Если молекула содержит не один атом какого-либо вещества, то на её свойства влияет и то место, где расположен изотоп. Молекулы или ионы, отличающиеся расположением изотопов, называют \emph{изотопомеры} --- см. ГОСТ 58567.

\begin{figure}[ht] 
	\centering\small
	%	\unitlength=1mm
	\includegraphics[width=0.8\textwidth]{Figures/Isotopocule.png}
	%	\includegraphics[width=30mm]{Figures\Oxygen.png}
%	\caption{Isotopocule} 
	\label{f:Isotopocule}
\end{figure}

Например, исследование распределения изотопомеров в аспаргиновой кислоте (одна из 20 протеиногенных аминокислот организма). 



\end{frame}





\begin{frame}
\frametitle{Изотопомеры}
Изотопомеры нашли применение в определении параметров молекулярных взаимодействий.
Если молекула обладает высокой пространственной симметрией, часть переходов на различные состояния запрещена, так что по ним невозможно получить информацию.\\
~\\

Если какой-то из атомов замещен изотопом, симметрия уменьшается. Таким образом, анализ спектров высокого разрешения изотопологов молекул является хорошим дополнительным источником информации при определении внутренней динамики молекул. 

\end{frame}


\begin{frame}
\frametitle{Изотопомеры}
На Рис. приведён пример распределения масс изотопомер в органических кислотах  
\begin{figure}[ht] 
	\centering\small
	%	\unitlength=1mm
	\includegraphics[width=0.8\textwidth]{Figures/Mass isotopomer distribution in organic acids.png}
	%	\includegraphics[width=30mm]{Figures\Oxygen.png}
%	\caption{Распределения масс изотопомер в органических кислотах \cite{MassIsotopomers2020}} 
	\label{f:MassIsotopomerOrganicAcids}
\end{figure}

В англоязычной литературе в качестве общего обозначения для изотопологов и изотопомеров иногда используется термин \emph{isotopocule}
\end{frame}

\begin{frame}
\frametitle{Изотопологи в оптических спектрах}

{\footnotesize
Изотопологи изучают не только в масс-спектрах. Оптическая (в широком смысле) спектроскопия является мощным средством исследования органических соединений. В случае пространственно недоступных объектов, например, внеземных, спектральная информация несёт основную долю информации.

Спектроскопия высокого разрешения является динамично развивающейся в теоретическом и экспеиментальном аспектах современной физики. Наличие в сложных молекулах различных степеней свободы порождает очень богаты спектры. Даже для относительно простых молекул, таких как этилен,
спектры весьма сложны.

На Рис. представлены экспериментально зарегистрированные спектры высокого разрешения в диапазоне нижних фундаментальных полос для молекулы этилена и её изотопологов.

Громова, Ольга Васильевна. Спектроскопия высокого разрешения молекул типа асимметричного волчка: C2H4, SO2, H2S, ClO2, NH3 и их изотопологи : диссертация в виде научного доклада на соискание ученой степени д.ф.-м.н. \ldots — Томск: 2022.

Для различных термов имеются тысячи возможных переходов, а общее количество спектральных линий в сложных молекулах составляет сотни тысяч. В многочисленных публикациях, приводятся результаты расчётов и экспериментальных данных, дающих изотопные
<<портреты>> изотопологов различных органических молекул.}
\end{frame}

\begin{frame}
\frametitle{Изотопологи в оптических спектрах}

Спектры высокого разрешения в диапазоне нижних фундаментальных полос для молекулы этилена и её изотопологов.

\begin{figure}[ht] 
	\centering\small
	%	\unitlength=1mm
	\includegraphics[width=0.4\textwidth]{Figures/IsotopologyC2H4.png}
\end{figure}

\end{frame}


\begin{frame}
\frametitle{Фракционирование изотопов в природе}
{\small
Фракционирование изотопов в природе, то есть, по массе, является проявлением очень общих закономерностей процессов, протекающих на планете. Поскольку химические свойства изотопов одного элемента очень близки друг к другу, то это предоставляет уникальную возможность исследования закономерностей эволюции планеты на больших масштабах времени.\\
~\\
Все \emph{биофильные элементы}, участвующие в биосинтезе микробных продуктов и их клеточных компонентов, за исключением фосфора, являются полиизотопными элементами. Стабильные изотопы этих элементов ($^1H, ^2H$ --- водород, $^{12}C, ^{13}C$ --- углерод, $^{14}N, ^{15}N$ --- азот, $^{16}O, ^{17}O, ^{18}O $ --- кислород, $^{32}S, ^{33}S, ^{34}S$ --- сера) имеют близкие химические свойства, но различаются по массам.
}

\end{frame}

\begin{frame}
\frametitle{Фракционирование изотопов в природе}

\begin{figure}[ht] 
	\centering\small
	\unitlength=1mm
	{\includegraphics[width=120mm]{Figures/IsotopeTracing.png}} 
	\caption{Периодическая таблица элементов Менделеева} с данными для отслеживания изотопных подписей
	\label{f:IsotopeTracers}
\end{figure}
\end{frame}

\begin{frame}
\frametitle{Изотопы углерода в нефтяной тематике}
В нефтяной науке изотопы углерода обычно связаны с
интерпретирующая информация, которую несут относительные, естественные распределения
двух видов стабильных изотопов углерода, $^{12}C$ и $^{13}C$, в различных органических и неорганических соединениях.

На Рис.~представлена схема формирования керогена из биологических отложений. \index{геополимеры}

\begin{figure}[h]
	\centering
	\includegraphics[width=0.6\linewidth]{Figures/BiotoGeoPolymers.png}
	\caption{Фомирование гетерогенных геополимеров (кероген)
		из исходных биополимеров захороненной органики.}
	\label{fig:BiotoGeoPolymers}
\end{figure}

При переходе углерода между фазами и веществами, например, из керогена 
в нефть, или $CO_2$ в $CH_4$, трансформации связаны с изотопными
эффектами, которые производят систематические и в целом предсказуемые
изотопное фракционирование. 
Таким образом, изотопы углерода частично
разделены таким образом, что одна углеродсодержащая фаза по сравнению с
другая в системе может быть относительно обогащена или обеднена
$^{12}C$ или $^{13}C$.

\end{frame}

\begin{frame}
\frametitle{Изотопы углерода в нефтяной тематике}


При переходе углерода между фазами и веществами, например, из керогена 
в нефть, или $CO_2$ в $CH_4$, трансформации связаны с изотопными
эффектами, которые производят систематические и в целом предсказуемые
изотопное фракционирование. \\
~\\
Таким образом, изотопы углерода частично
разделены таким образом, что одна углеродсодержащая фаза по сравнению с
другая в системе может быть относительно обогащена или обеднена
$^{12}C$ или $^{13}C$.

\end{frame}


\begin{frame}
\frametitle{Внутреннее строение Земли. Мантия Земли и изотопные распределения алмазов}



{\small

Согласно Википедии 
Мантия --- часть Земли (геосфера), расположенная непосредственно под корой и выше ядра. В ней находится большая часть вещества Земли. Мантия есть и на других планетах земной группы. Земная мантия находится в диапазоне от 30 до 2900 км от земной поверхности. Мантия занимает около 80\% объёма Земли.\\
~\\
Мантию изучают, в частности, по фрагментам мантийных пород, выносимые на поверхность мантийными же расплавами — кимберлитами, щелочными базальтами и др. Это ксенолиты, ксенокристы и алмазы. Алмазы занимают среди источников информации о мантии особое место. Именно в алмазах установлены самые глубинные минералы, которые, возможно, происходят даже из нижней мантии. В таком случае эти алмазы представляют собой самые глубокие фрагменты Земли, доступные непосредственному изучению.
}



\end{frame}

\begin{frame}
\frametitle{Внутреннее строение Земли. Мантия Земли и изотопные распределения алмазов}



\begin{figure}[h]
	\centering
	\includegraphics[width=0.7\linewidth]{Figures/13CDiamondsMantle.png}
	\caption{Вариация изотопа $\delta^{13}C$ в алмазах}
	\label{fig:13CDiamondsMantle}
\end{figure}

На Рис.~показаны вариации изотопа $\delta^{13}C$ в алмазах различных геосфер Земли.


\end{frame}

\begin{frame}
\frametitle{Изотопы углерода как геохимический трассер}


Диапазон значений изотопов углерода в обычных неорганических и
биологические резервуарах углерода приведён на Рис.~\ref{fig:GeoTracer}.
\begin{figure}[h]
	\centering
	\includegraphics[width=0.8\linewidth]{Figures/GeoTracer.png}
	\caption{Изотопы углерода как геохимический трассер}
	\label{fig:GeoTracer}
\end{figure}
Часть значений для некоторых объектов вышла за диапазон рисунка и показаны условно. При этом ряд других значений имеет весьма узкие интервалы.

\end{frame}


\begin{frame}
\frametitle{Изотопы углерода как геохимический трассер}

\begin{figure}[h]
	\centering
	\includegraphics[width=0.8\linewidth]{Figures/13CvsAge.png}
	\caption{Зависимость изотопной подписи значений $\delta^{13}C$  от времени  для органинического углерода (Corg) и углеродных карбонатов (Ccarb).}
	\label{fig:13CvsAge}
\end{figure}


\end{frame}

\begin{frame}
\frametitle{Изотопы углерода как геохимический трассер}
\begin{figure}[h]
	\centering
	\includegraphics[width=0.8\linewidth]{Figures/IdealizedFigure.png}
	\caption{Зависимость идеализированной изотопной подписи значений $\delta^{13}C$ в морских карбонатах от времени}
	\label{fig:IdealizedFigure}
\end{figure}

Как видно из рисунков, на некотрых этапах эволюции биосферы изотопная подпись была весьма изменчива.
\end{frame}

\begin{frame}
\frametitle{Изотопологи углекислого газа.}




Термометрия изотопов слипшихся карбонатов --- геохимический инструмент.
используется для определения температуры образования карбонатных минералов. В отличие от других подобных карбонатных
термометров, термометрия слипшихся изотопов не требует предположений об изотопном составе
жидкость, из которой выпал карбонат. \\
~\\

Путем измерения слипшегося изотопного состава ($\Delta_{47}$)
карбонатных минералов с известной температурой образования, можно построить эмпирическую калибровку для
термометр слипшихся изотопов, который необходимо преобразовать из значения $\Delta_{47}$ в температуру пласта.
\end{frame}

\begin{frame}
\frametitle{Изотопологи углекислого газа.}
<<Геотермометр>> на основе изотополога углекислого газа
\begin{figure}[h]
	\centering
	\includegraphics[width=0.45\linewidth]{Figures/CO2delta47isotop.png}
%	\caption{<<Геотермометр>> на основе изотополога углекислого газа\cite{Isotopology47}.}
	\label{fig:CO2delta47isotop}
\end{figure}



\end{frame}


\begin{frame}
\frametitle{Происхождение жизни и углеродные соединения.}

{\footnotesize
Важнейшим вопросом естествознания является проблема возникновения жизни. В связи с выходом человечества в Космос, он приобрёл новый импульс и получил основательную приборную базу для исследований. Развились новые дисциплины науки.\\
~\\
Астробиология (экзобиология) --- \index{астробиология, экзобиология}
научная дисциплина, рассматривающая возможность появления, эволюции и сохранения жизни на других планетах во Вселенной. \\
Возможная схема формирования глицина }

\begin{figure}[h]
	\centering
	\includegraphics[width=0.5\linewidth]{Figures/GlycineFormation.png}
%	\caption{Возможная схема формирования глицина \cite{glycine2021}}.
	\label{fig:GlycineFormation}
\end{figure}

\end{frame}


\begin{frame}
\frametitle{Возможная схема формирования глицина}

{\small
Для поиска жизни на других планетах необходимо определить исходные положения. Одним из важнейших предположений является то, что подавляющее большинство форм жизни в нашей Галактике основано на углеродной химии, как и все формы жизни на Земле. В молекулярных облаках найдено огромное количество углеродосодержащих молекул. \\
~\\
Большое внимание привлекает проблема обнаружения в космосе аминокислот. \index{глицин}
Наиболее вероятным кандидатом в настоящее время считается глицин, $NH_2CH_2COOH$, синтез которого из уже обнаруженных молекул вполне возможен.\\
~\\
На Рис. (выше) представлены возможные варианты формирования глицина из молекул льдов, богатых водой, на ранних стадиях формирования звёзд малой массы.
Среди вариантов обсуждаются реакции с изотопами углерода:
\begin{align*}
^{13}C^{18}O + NH_2CH_3 + O_2 + H, \\
^{13}C^{18}O + NH_2CH_3  + ^{18}O_2 + H,\\
CO + NH_2CH_3  + O_2 + D.
\end{align*}
}
\end{frame}

\begin{frame}
\frametitle{Изотопные ландшафты --- $^{15}N$ на Земле}

Изотопное отношение для $\delta^{15}N$ в растениях	

\begin{figure}[ht] 
	\centering\small
	\unitlength=1mm
	{\includegraphics[width=100mm]{Figures/Plant15N.png}} 
\end{figure}


\end{frame}

\begin{frame}
\frametitle{ Изотопная подпись  --- кислород  на Земле}
%Вариации атомного веса и изотопного состава ряда материалов, содержащих кислород на Земле.
\begin{figure}[ht] 
	\centering\small
	\unitlength=1mm
	\includegraphics[width=70mm]{Figures/OxygenNature.png} 
\end{figure}

\end{frame}


\begin{frame}
\frametitle{Пример фракционирования изотопов в природе}

Фракционирование изотопов водорода и кислорода при испарении-конденсации
\begin{figure}[ht] 
	\centering\small
	\unitlength=1mm
	\includegraphics[width=70mm]{Figures/EvaporationCondensation.png} 
\end{figure}

\end{frame}


\begin{frame}
\frametitle{Изотопная ниша }
Ареалы 12 видов водяных печников по изотопному составу перьев 
\begin{figure}[ht] 
	\centering\small
	\unitlength=1mm
	\includegraphics[width=45mm]{Figures/BirdsArealLeft.png}
\end{figure}
\end{frame}



%%%%%%%%%%%%%%%%%%%%%%%%%%%%%%%%%%%%%%%%%%%%%%%%%%%%%%%%%%%%%%%%%%%%%%%%%%%%%%%%%%%%%%%%%%%%%%%%%%%%%%%%%%%%%%%%%%%
\begin{frame}
	\frametitle{Манифест интервальной статистики}
	%А.Н.~Баженов, С.И. Жилин, С.И. Кумков, С.П. Шарый.
	%		Обработка и анализ интервальных данных. Издательство «ИКИ»		2024 г.
	\begin{figure}[ht]
		\centering
		\includegraphics[width=0.4\textwidth]{Figures/Обложка книги РХД.jpg}
		%	\includegraphics[width=0.8\textwidth]{7.mps}	
	\end{figure}
\end{frame}

\begin{frame}
	\frametitle{Всероссийский веб-семинар по интервальному анализу и его приложениям. 28.11.2022.}
А.Н.~Баженов.\\
~\\ 
Интервальная таблица Менделеева элементов и изотопов. \\
~\\ 
Всероссийский веб-семинар по интервальному анализу и его приложениям. 28.11.2022.\\
~\\
http://interval.ict.nsc.ru/WebSeminar/ANBazhenov-28.XI.2022.pdf
\end{frame}

\begin{frame}
	\frametitle{Трудности в использовании интервальной неопределённости}

\begin{center}	
ТРУДНОСТИ В ИСПОЛЬЗОВАНИИ \\ ИНТЕРВАЛЬНОЙ НЕОПРЕДЕЛЁННОСТИ
\end{center}	
	
\end{frame}


\begin{frame}
	\frametitle{Трудности в использовании интервальной неопределённости}
	
Общая проблема --- интерпретация интервала как равномерного распределения

\medskip
Наиболее актуальный документ по стандарным атомным весам
\medskip
T.Prohaska, J.Irrgeher, J.Benefield, et al. 

\medskip
Standard atomic weights of the elements 2021 (IUPAC Technical Report) 

Pure Appl. Chem. 2022; 94(5): 573–600

	
\end{frame}

\begin{frame}
	\frametitle{Интерпретация интервала как равномерного распределения}
{\footnotesize 
	The interval does not imply any statistical distribution of atomic weight values between the lower and upper bound (e.g. the arithmetic mean of a and b is not necessarily the most likely value). Similarly, the interval does not convey a simple statistical representation of uncertainty. The probability density function may differ case by case, due to varying sources and their proportions may need to be considered. 
	
	If no additional information is available or utilized, \un{the probability density function} associated with the standard atomic weights \un{can be considered as uniform (rectangular)}.
}
Формулировка, провоцирующая ложную интерпретацию:

Если нет дополнительной доступной или уже используемой информации, о функции плотности вероятности, связанной с атомными весами, то она может считаться равномерной (прямоугольной).
	
\end{frame}


\begin{frame}
	\frametitle{Непонимание интервальной арифметики}

Рассмотрим публикацию IUPAC.

\medskip	
A. Possolo, A.M.H. van der Veen, J. Meija and D.B. Hibbert
<<Interpreting and propagating the uncertainty of the standard atomic weights>> (IUPAC Technical Report). \\
Pure Appl. Chem. 2018; 90(2): 395–424 %\cite{IUPACUncertainty}




\end{frame}

\begin{frame}
	\frametitle{Непонимание интервальной арифметики}
Исходные данные
	\begin{align*}
A_r(H) = & \, [1.007 84, 1.008 11],\\
A_r(O) = & \,[15.999 03, 15.999 77]], \\
M_r(H_2O) = & \, 2 \cdot[1.007 84, 1.00811] + [15.99903,15.999 77] \\
 = & \,[18.01471, 18.015 99]
\end{align*}
Расчёт массовой доли кислорода в молекуле воды даёт разные результаты в зависимости от способа расчёта
\begin{align}
		w_{H_2O}(O) = & \frac{1}{2A_r(H)/A_r(O)+1} = &[0.888 \un{083}, \, 0.888 \un{114}] \label{H2O1} \\
		\ne & \frac{A_r(O)}{M_r(H_2O)} =& [0.888 \un{046}, \, 0.888\un{150}]. \label{H2O2}
\end{align}
Сравним ширины \eqref{H2O1}	и \eqref{H2O2}:
$wid (1) = 0.000\un{031} \ne wid (2) = 0.000\un{104} $.
	
\end{frame}


\begin{frame}
	\frametitle{Непонимание интервальной арифметики}


	
<<Ввиду очевидных практических трудностей, связанных с применением интервальной арифметики, и учитывая
желательность количественной оценки и распространения неопределенностей в расчетах, связанных с атомными весами, мы рекомендуем
вместо этого вероятностный подход к интерпретации интервалов, который мы считаем полностью последовательным
с мотивацией, которая побудила CIAAW принять такое обозначение. 
\medskip
В остальной части этого отчета объясняется основа и
общие процедуры такого подхода и иллюстрирует их применение на конкретных и содержательных примерах.>>
	
\end{frame}


\begin{frame}
	\frametitle{Непонимание интервальной арифметики}
	
	\begin{figure}[ht] 
		\centering\small
		\unitlength=1mm
		\includegraphics[width=100mm]{Figures/BoronTable1data.png}
		\caption{Данные для атомных весов образцов бора} 
		\label{f:BoronTable1data}
	\end{figure}	
	
	
\end{frame}

\begin{frame}
	\frametitle{Непонимание интервальной арифметики}
Множественные пики и впадины на синей кривой обусловлены тем, что соответствующее распределение представляет собой смесь 13 различных прямоугольных распределений.
\begin{figure}[ht] 
	\centering\small
	\unitlength=1mm
	\includegraphics[width=80mm]{Figures/BoronTable1.png}
	\caption{<<Плотность вероятности>> для атомных весов образцов бора и её аналитическая аппроксимация} 
	\label{f:BoronTable1}
\end{figure}	
	
	
\end{frame}

\begin{frame}
\frametitle{Проблемы описания интервальных весов}
T.B. Coplen, N.E. Holden, T. Ding, H.A.J. Meijer, J. Vogl, and X. Zhu, <<The Table of Standard Atomic Weights-An exercise in consensus>>, \\ Rapid Commun Mass Spectrom. 2022; 36:e8864. %\cite{IUPACCosensus}	

\begin{itemize}
	\item Элементы с 2 изотопами
	\item Элементы с асимметричнными распределениями изотопов
\end{itemize}

\end{frame}

\begin{frame}
	\frametitle{Элементы с 2 изотопами }
<<\ldots введения нового определения значений атомного веса в 1979 г.
\ldots неопределенности в значении атомного веса для элементов с двумя стабильными изотопами стали асимметричными; то есть, неопределенность в верхней части значения атомного веса не была
такой же, как и на нижней стороне. Это привело к серьезной проблеме
которая сохраняется в течение последних четырех десятилетий. Середина симметричной неопределенности может соответствовать значениям атомного веса, которое не относится ни к одному известному источнику этого элемента, обнаруженному в природе. Естественное решение этой проблемы
было бы введением асимметричных неопределенностей. К сожалению,
Комиссия последовательно отвергала это решение. В итоге,
это привело к введению интервалов атомного веса для
выбранных элементов \ldots >>
\end{frame}

\begin{frame}
		\frametitle{Элементы с асимметричнными распределениями изотопов }
\begin{figure}[ht] 
	\centering\small
	\unitlength=1mm
	\includegraphics[width=60mm]{Figures/LiExample.png}
	\caption{Пример распределения атомных масс для стандартного образца карбоната лития} 
	\label{f:LiExample}
\end{figure}
\end{frame}

\begin{frame}
	\frametitle{3 способа описания неопределённостей   }
	\begin{figure}[ht] 
		\centering\small
		\unitlength=1mm
		\includegraphics[width=60mm]{Figures/Uncertainty_cases.png}
		\caption{Типы представления неопределённостей} 
		\label{f:Uncertainty cases}
	\end{figure}
\end{frame}

\begin{frame}
	\frametitle{3 способа описания неопределённостей   }
A,\\ пример одного из 21 элемента, имеющего значение $A_r (E)$, определяемое одним
изотопом, Guide to the Expression of Uncertainty in
Measurement, оцененная распределением Гаусса с коэффициентом охвата 6.

B, \\пример одного из 49 элементов, имеющих стандартный атомный вес
значение и неопределенность, определяемые
консенсусом; наибольшее значение вероятности
функция плотности не обязательно должно совпадать с $A_r (E)$

C,\\
Пример одного из 14 элементов, имеющих
согласованные стандартные атомные веса, выраженные как
интервалы; функции плотности вероятности неизвестны. 

Распределение функции вероятностей
также неизвестны для 6 элементов, имеющих
метку «r» (гелий, никель, медь, цинк, селен и стронций)


\end{frame}

\begin{frame}
\frametitle{Обработка данных методами интервального анализа}


\begin{center}	
ОБРАБОТКА ДАННЫХ \\
МЕТОДАМИ ИНТЕРВАЛЬНОГО АНАЛИЗА \\
--- примеры
\end{center}

\end{frame}


\begin{frame}
	\frametitle{Обработка данных методами интервального анализа}
Обработка данных методами интервального анализа. Теория --- книга	\cite{BookIntStat}
\begin{itemize}
	\item Представление данных (диаграмма рассеяния)
	\item Интервальные оценки (мода)
	\item Одновременное вычисление внешних и внутренних оценок (твины)
	\item Минимумы по включению
	\item $\ldots$
\end{itemize}
	
\end{frame}


\begin{frame}
	\frametitle{Вариации углерода в органических объектах}
	
Рассмотрим часть данных, относящихся к углероду природных объектов. 

\begin{table}[h!]
	\begin{center}
		{\small
			\begin{tabular}{ccc}
				\hline
				Категория & Нижняя & Верхняя \\
				~ & граница & граница \\ 
				\hline
				Наземные растения (C3 метаболический процесс)	& -35 & -21 \\
				Наземные растения (C4 метаболический процесс)	& -16 & -9 \\
				Наземные растения  (CAM метаболический процесс)	& -34 & -10 \\
				\hline
				Морские организмы 	& -74.3 & -2 \\
				Морские отложения и соединения &	-130.3 & 7 \\
				\hline
				Уголь &	-30 & -19 \\
				Сырая нефть &	-44 & -16.8 \\
				Природный этанол & -32 &  -10.3 \\
				\hline
			\end{tabular}
		}
		\caption{Вариации углерода в органических объектах} в единицах $10^3 \cdot \delta ^{13}C_{VPDB}$
		\label{t:OrganicCarbonVariation}
	\end{center}
\end{table}
	
\end{frame}


\begin{frame}
	\frametitle{Вариации углерода в органических объектах}
	
\begin{align}
	10^3 \cdot \delta ^{13}C_{VPDB} = \{ \
	&	[-35, -21 ], [ -16, -9 ], [ -34, -10 ] \nonumber \\
	&	[ -74.3, -2 ], [ -130.31, +7 ], [ -30, -19], \nonumber \\
	& 	[ -44, -16.8], [ -32, -10.3] \	\}
\end{align}

Диаграмма рассеяния 
\begin{figure}[ht] 
	\centering\small
	\unitlength=1mm
	{\includegraphics[width=55mm]{Figures/CarbonOrganicDataIUPAC2016.png}} 
	\caption{Вариации углерода в органических объектах} в единицах $10^3 \cdot \delta ^{13}C_{VPDB}$
	\label{f:OrganicCarbonVariation}
\end{figure}

	
\end{frame}

\begin{frame}
	\frametitle{Вариации углерода в органических объектах}
	
\begin{figure}[ht] 
	\begin{center}
		\unitlength=1mm
		{\includegraphics[width=60mm]{Figures/CarbonOrganicDataIUPAC2016ModeArray.png}} 
		\caption{Интервальная мода 
			вариаций $10^3 \cdot \delta ^{13}C_{VPDB}$ в органических объектах} 
		\label{f:OrganicCarbonVariationMode}
	\end{center}	
\end{figure}

Интервальная мода
\begin{equation}
	{\tt mode}( 10^3 \cdot \delta ^{13}C_{VPDB} ) = [-30, -21].
\end{equation}	

\end{frame}

\begin{frame}
	\frametitle{Расчёт молекулярной массы молекулы метана в виде твина}

Рассмотрим  массу молекулы метана $CH_4$.

Пусть источником водорода и углерода является   морская фауна. 

Данные об изотопных вариациях берутся c сайта 

\medskip
https://www.sciencebase.gov/catalog/item/580e719ae4b0f497e794b7d8.

\medskip
Представим массы компонент молекулы в виде твинов в форме Нестерова.
\begin{equation*}
	\mbf{X}_{\subseteq} = [ \, \mbf{X}^{in} \, ,  \, \mbf{X}^{out} \, ]
\end{equation*}
$\mbf{X}^{in}, \mbf{X}^{out}$ --- интервалы внутренней и внешней оценки величины.

\medskip
\emph{Внешней оценкой} будет служит максимум по включению для атомов молекулы, \\
\emph{внутренней оценкой} --- мода выборки.

\end{frame}

% внешняя оценка Hydrogen :  '[1.00012, 1.00017]'
% мода Hydrogen :  '[1.00013, 1.00015]'
%внешняя оценка Carbon :  '[12.0096, 12.0111]'
%мода Carbon :  '[12.0102, 12.011]'
%T_h= ['[1.00013, 1.00015]', '[1.00012, 1.00017]']
%T_c= ['[12.0102, 12.011]', '[12.0096, 12.0111]']
%T=T_c+4*T_h= ['[16.0109, 16.0115]', '[16.0101, 16.0118]']

\begin{frame}
	\frametitle{Расчёт молекулярной массы молекулы метана в виде твина}
Твины атомных весов водорода и углерода
\begin{align}
	\mbf{M}(H) = &[ \, [1.00013, 1.00015] \, ,  \, [1.00012, 1.00017] \, ], \\
	\mbf{M}(C) = & [ \, [12.0102, 12.011] \, ,  \, [12.0096, 12.0111] \, ]
\end{align}	
Твин веса молекулы метана $CH_4$
	\begin{equation}
		\mbf{M} (CH_4) = [ \, [16.0109, 16.0115] \, ,  \, [16.0101, 16.0118] \, ].
	\end{equation}	
Графическое представление твина веса молекулы метана $\mbf{M} (CH_4)$
	\begin{figure}[ht] 
	\begin{center}
		\unitlength=1mm
  \begin{picture}(90,10)
	\put(10,0){\includegraphics[width=60mm]{Figures/TwinCH4.png}}
	\put(45,5){\mbox{\small $X^{in}$}} 
	\put(35,4){\vector(1,0){28}}
	\put(57,4){\vector(-1,0){28}}
	\put(10,-1){\vector(1,0){60}}
	\put(60,-1){\vector(-1,0){50}}
	\put(30,-5){\mbox{\small $X^{out}$}}
\end{picture}	%	{\includegraphics[width=60mm]{Figures/TwinCH4.png}} 
%		\caption{Твин молекулы метана $CH_4$} 
		\label{f:TwinCH4}
	\end{center}	
\end{figure}	
\end{frame}

\begin{frame}
\frametitle{Мультиинтревалы --- пример технеция }
Мультиинтревалы стабильных изтопов молибдена и рутения и наиболее долгоживущий изотоп технеция

\begin{equation*}
\begin{aligned}
\mbf{M}_{Mo} & = [92, [94,& ~{\color{red}98}]&, 100 \ ], \\
\mbf{M}_{Tc} &=  &[ {\color{red}98} &  ], \\
\mbf{M}_{Ru} &= [ 96,   & [{\color{red}98}&, 102], 104 \ ].
\end{aligned}
\end{equation*}
Минимум по включению мультиинтревалов стабильных изтопов молибдена и рутения содержит значение наилучшего кандидата для технеция $_{43}Tc^{55}$
\begin{equation*}
\mbf{M}_{Mo} \cap \mbf{M}_{Ru} = [96, {\color{red}98}] \supseteq \mbf{M}_{Tc}.
\end{equation*}

\end{frame}

\begin{frame}
\frametitle{Мультиинтревалы --- пример технеция }
\begin{figure}[ht] 
	\begin{center}
		\unitlength=1mm
		{\includegraphics[width=100mm]{Figures/MoTcRuNZmarked.png}}
	\end{center}	
\end{figure}	

\end{frame}


\begin{frame}
\frametitle{Программное обеспечение для анализа данных с интервальной неопределённостью}

С.И.Жилин \\
Matlab/Octave/Scilab:
https://github.com/szhilin/octave-interval-examples\\
https://github.com/szhilin/kinterval \\
Julia
https://github.com/szhilin/julia-interval-examples



\medskip
Python: \\
базовая библиотека\\
{\tt intvalpy} --- А.Андросов https://github.com/AndrosovAS/intvalpy, \\
арифметика твинов Нестерова\\
{\tt twin} --- А.Жаворонкова https://github.com/Zhavoronkova-Alina/twin\\ 
Вычисления с изотопами\\
{\tt MendeleevTwin} --- Т.Яворук https://github.com/Tatiana655/MendeleevTwin



\end{frame}



%%%%%%%%%%%%%%%%%%%%%%%%%%%%%%%%%%%%%%%%%%%%%%%%%%%%%%%%%%%%%%%%%%%%%%%%%%%%%%%%%%%%%%%%%%%%%%%%%%%%%%%%%%%%%%%%%

\begin{frame}
\frametitle{Литература}
\begin{thebibliography}{9}

{\footnotesize
\bibitem{Mendeleev1869ru}
\textsc{Менделеев, Д.} (1869). “Соотношение свойств с атомным весом элементов”. Журнал Русского Химического Общества. 1: 60—77.

\bibitem{Mendeleev1869}
\textsc{Mendeleev, Dmitri} (1869). “Versuche eines Systems der Elemente nach ihren Atomgewichten und chemischen Functionen”. Journal für Praktische Chemie. 106: 251.

\bibitem{BookIntStat} \textsc{А.Н.\,Баженов, С.И.\,Жилин, С.И.\,Кумков, С.П.\,Шарый.} Обработка и анализ данных с интервальной неопределённостью. РХД. Cерия <<Интервальный анализ и его приложения>>. Ижевск. 2024. с.356.	
}


%\bibitem{IntApplication2021} 
%\textsc{Баженов А.Н.} Естественнонаучные и технические применения интервального анализа: учебное пособие. - Санкт-Петербургский политехнический университет Петра Великого -- Санкт-Петербург, 2021 – 83~с.  Доступно на \url{https://elib.spbstu.ru/dl/5/tr/2021/tr21-169.pdf}

\end{thebibliography}	
\end{frame}

\end{document} 









