\begin{frame}
\frametitle{ИЗОТОПЫ. ПРОИСХОЖДЕНИЕ.}

\begin{center}
{\large
ЯДЕРНЫЙ НУКЛЕОСИНТЕЗ}
\end{center}

Литература
\begin{itemize}
    \item  В.П.\,Чечев, А.В.\,Иванчик, Д.А.\,Варшалович «Синтез элементов во Вселенной: От Большого взрыва до наших дней». 2020. 304 с. ISBN 978-5-9710-7626-1.
    \item Б.C. Ишханов, И.М. Капитонов, И.А. Тутынь. Нуклеосинтез во вселенной.
		М., Изд-во Московского университета. 1998.\\
		http://nuclphys.sinp.msu.ru/nuclsynt/index.html
  \item Бедняков В. А. О происхождении химических элементов. Э. Ч. А. Я., Том 33 (2002), Часть 4 стр.914-963.
\end{itemize}
\end{frame}

\begin{frame}
\frametitle{Происхождение элементов}

	\begin{figure}[ht] 
	\centering\small
	%	\unitlength=1mm
	\includegraphics[width=0.9\textwidth]{Figures/1920px-Elements-origin-ru.png}
	%	\includegraphics[width=30mm]{Figures\Oxygen.png}
	\caption{Происхождение элементов} https://en.wikipedia.org/wiki/Нуклеосинтез 
	\label{f:Nucleosynthesis_periodic_table}
\end{figure}
\end{frame}

\begin{frame}
\frametitle{Реакции, важные для нуклеосинтеза}

Michael S. Smith.	Nuclear data resources and initiatives for nuclear astrophysics 	November 2023Frontiers in Astronomy and Space Sciences 10 	DOI: 10.3389/fspas.2023.1243615
	\begin{figure}[ht] 
	\centering\small
	%	\unitlength=1mm
	\includegraphics[width=0.7\textwidth]{Figures/Reactions-of-importance-for-nuclear-astrophysics-shown-on-the-N-Z-plane-for-stable-nuclei.png}
	%	\includegraphics[width=30mm]{Figures\Oxygen.png}
	\caption{Реакции, важные для нуклеосинтеза} 
	% Michael S. Smith.	Nuclear data resources and initiatives for nuclear astrophysics 	November 2023. Frontiers in Astronomy and Space Sciences 10 	DOI: 10.3389/fspas.2023.1243615	
	\label{f:Reactions-of-importance-for-nuclear-astrophysics}
\end{figure}

\end{frame}

\begin{frame}
\frametitle{Относительная распространенность нуклидов}
Бедняков В. А. О происхождении химических элементов \ldots
	\begin{figure}[ht] 
		\centering\small
		\unitlength=1mm
		\includegraphics[width=110mm]{Figures/NuclidsN.png} 
		\caption{Относительная распространенность нуклидов от атомной массы \cite{ElementsOrigin}}
			Обозначения указывают на различные процессы синтеза 
		\label{f:NuclidsN}
	\end{figure}

\end{frame}

\begin{frame}
\frametitle{Схематическое изображение распространенности }
Бедняков В. А. О происхождении химических элементов \ldots
\begin{figure}[ht] 
		\centering\small
		\unitlength=1mm
		\includegraphics[width=65mm, height=55 mm]{Figures/NuclidsNSchematics.png} 
		\caption{Схематическое изображение распространенности нуклидов}
%			Обозначения указывают на различные процессы синтеза } 
		\label{f:NuclidsNSchematics}
	\end{figure}
\end{frame}

\begin{frame}
\frametitle{Иллюстрация хода нуклеосинтеза}
{\footnotesize
H. Schatz, A.D.B. Reyes, A Best et al. Horizons: nuclear astrophysics in the 2020s and beyond. \ldots  Vol 49, No 11, November 2022 pp.1-78. DOI:10.1088/1361-6471/ac8890}
\begin{figure}[h] 
	\centering\small
	\unitlength=1mm
	\includegraphics[width=100mm]{Figures/ChartNucleosysnthesis.png} 
%	\caption{Иллюстрация хода $s$- и $r$-процессов} 
	\label{f:ChartNucleosysnthesis}
\end{figure}
\end{frame}

\begin{frame}
\frametitle{Области $(n,\gamma)$, $\beta^{-}$, EC}
	\begin{figure}[h] 
		\centering\small
		\unitlength=1mm
		\includegraphics[width=125mm]{Figures/(N,Gamma) Beta-decay Electron Conversion.png} 
		\caption{Области $(n,\gamma)$, $\beta^{-}$, EC} 
		\label{f:(N,Gamma) Beta-decay Electron Conversion}
	\end{figure}
\end{frame}



\begin{frame}
\frametitle{Иллюстрация хода $s$- и $r$-процессов}
\begin{figure}[h] 
	\centering\small
	\unitlength=1mm
	\includegraphics[width=100mm]{Figures/NucleosythesisTrajectory.png} 
	\caption{Иллюстрация хода $s$- и $r$-процессов} 
	\label{f:NucleosythesisTrajector}
\end{figure}

\end{frame}

\begin{frame}
\frametitle{Сечения захвата тепловых нейтронов}
\begin{figure}[h] 
	\centering\small
	\unitlength=1mm
	\begin{picture}(100, 70)
	\put(0,0){\includegraphics[height=70mm]{Figures/NZngamma.png}}
	\end{picture}
	\caption{Сечения захвата тепловых нейтронов} 
	\label{f:NZngamma}
\end{figure}

\end{frame}

\begin{frame}
\frametitle{Обозначения  процессов нуклеосинтеза}
	{\small 
\begin{tabular}{lcl}
U & --- & космологический синтез до образования звёзд \\
H & --- & горение водорода \\
CNO & --- & горение водорода при высоких температурах (CNO-цикл) \\
He & --- & взрывное горение гелия \\	
С & --- & взрывное горение углерода \\
O & --- & взрывное горение кислорода \\
Si & --- & взрывное горение кремния \\
NSi & --- & обогащённое нейтронами горение кремния \\
E & --- & статическое ядерное равновесие \\
s & --- & s-процесс. Продукты медленного захвата нейтронов\\
r & --- & r-процесс. Продукты быстрого захвата нейтронов\\
p & --- & p-процесс. Процессы на обеднённой нейтронами стороне \\
 & & долины $\beta$-стабильности\\
X & --- & Дробление космическими лучами \\
\end{tabular}
}
\end{frame}

\begin{frame}
\frametitle{Графики данных по нуклеосинтезу}
\begin{figure}[ht] 
	\centering\small
	\unitlength=1mm
	\begin{picture}(120,60)
	\put(0,0){\includegraphics[width=120mm]{Figures/NucleosynthesisProcesses.png}}
	\end{picture}
%	\caption{Нуклеосинтез изотопов Таблицы Менделеева} 
	\label{f:NucleosynthesisProcesses}
\end{figure}
\end{frame}

\begin{frame}
\frametitle{Легкие элементы и группа стабильности}
\begin{figure}[ht] 
	\centering\small
	\unitlength=1mm
	\begin{picture}(120,60)
	\put(20,0){\includegraphics[height=60mm]{Figures/NucleosynthesisProcessesLightEQ.png}}
	\end{picture}
%	\caption{Нуклеосинтез изотопов лёгких элементов и группы стабильности} 
	\label{f:NucleosynthesisProcessesLightEQ}
 \end{figure}
\end{frame}

\begin{frame}
\frametitle{Редкоземельные элементы}
\begin{figure}[ht] 
	\centering\small
	\unitlength=1mm
	\begin{picture}(120,70)
	\put(10,0){\includegraphics[height=70mm]{Figures/NucleosynthesisProcessesREE.png}}
	\end{picture}
%	\caption{Нуклеосинтез изотопов редкоземельных элементов Таблицы Менделеева} 
	\label{f:NucleosynthesisProcessesREE}
\end{figure}
\end{frame}

\begin{frame}
\frametitle{Распространённость изотопов химических элементов в коре Земли --- происхождение}
\begin{figure}[ht] 
	\centering\small
	\unitlength=1mm
	\begin{picture}(120,65)
	\put(-10,0){\includegraphics[width=135mm]{Figures/Element_Abudance_Earth_Crust NucleosynthesisProcesses.png}}
	\end{picture}
%	\caption{Распространённость изотопов химических элементов в коре Земли} По данным https://www.nndc.bnl.gov/nudat3 
	\label{f:Element_Abudance_Earth_Crust NucleosynthesisProcesses}
\end{figure}
\end{frame}

\begin{frame}
\frametitle{Распространённость изотопов химических элементов в коре Земли --- изотопы}
\begin{figure}[ht] 
	\centering\small
	\unitlength=1mm
	\begin{picture}(120,65)
	\put(-10,0){\includegraphics[width=135mm]{Figures/Element_Abudance_Earth_Crust Isotopes.png}}
	\end{picture}
%	\caption{Распространённость изотопов химических элементов в коре Земли} По данным https://www.nndc.bnl.gov/nudat3
	\label{f:Element_Abudance_Earth_Crust Isotopes}
\end{figure}
\end{frame}

\begin{frame}
\frametitle{<<Отсутствующие>> элементы}

 В первой половине XX в. предложено эмпирическое правило Щукарева—Маттауха о невозможности одновременного существования стабильных изобар, заряды ядер которых отличаются на единицу.\\
 ~\\
На Земле очень мало технеция и прометия

\begin{figure}[ht] 
	\centering\small
	\unitlength=1mm
	\includegraphics[width=40mm]{Figures/IUPAC Table Mo Tc Ru.png} 
%	\caption{Изотопы Tc, Mo и Ru} --- https://applets.kcvs.ca/IPTEI/IPTEI.html/
	\label{f:IUPAC Table Mo Tc Ru}
\end{figure}

\begin{figure}[ht] 
	\centering\small
	\unitlength=1mm
	\includegraphics[width=40mm]{Figures/IUPAC Table Nd Pm Sm.png} 
%	\caption{Изотопы Nd, Pm  и Sm} --- hhttps://applets.kcvs.ca/IPTEI/IPTEI.html/
	\label{f:IUPAC Table  Nd Pm Sm}
\end{figure}
\end{frame}

\begin{frame}
\frametitle{Случай технеция}

По правилу Щукарева-Маттауха, так как молибден и рутений имеют стабильные изотопы с массовыми числами $92, 94, 95, 96, 97, 98, 100$ и $96, 98, 99, 100, 101, 102, 104$, стабильный технеций (и его изотопы), не могут существовать.

\begin{figure}[ht] 
	\centering\small
	\unitlength=1mm
	\includegraphics[width=100mm]{Figures/MoTcRuNZmarked.png} 
%	\caption{Стабильные изотопы Mo и Ru --- чёрные квадраты} --- https://www.nndc.bnl.gov/nudat3 
	\label{f:T half time Mo Tc Ru.png}
\end{figure}




\end{frame}

\begin{frame}
\frametitle{Случай прометия}

Для всех изотопов технеция положительны энергии как $Q_{\beta}$ бета-распада, так и электронной конверсии $Q_{EC}$. 
В то же время у неодима и самария есть стабильные изотопы в диапазоне $A$ от 142 до 150.
\begin{figure}[ht] 
	\centering\small
	\unitlength=1mm
	\includegraphics[width=110mm]{Figures/T half time Nd Pm Sm.png} 
%	\caption{Стабильные изотопы Nd, Pm  и Sm --- чёрные квадраты} --- https://www.nndc.bnl.gov/nudat3
	\label{f:T half time MNd Pm Sm.png}
\end{figure}


\end{frame}

\begin{frame}
\frametitle{Случай хлора}
{\footnotesize
Ещё в XIX веке было известно, что атомная масса хлора заметно отличается от целочисленной. В частности, в аннотации пионерской публикации Д.И.\,Менделеева 1869 г.
<<Система химических элементов согласно их атомным весам и химическим свойствам>> 
был указан атомный вес, равный 35.5, что никак нельзя объяснить неточностью измерений}
\begin{figure}[ht] 
	\centering\small
	\unitlength=1mm
	\includegraphics[height=60mm]{Figures/T half time S Cl Ar.png} 
%	\caption{Стабильные изотопы S Cl Ar --- чёрные квадраты} --- https://www.nndc.bnl.gov/nudat3 
	\label{f:T half time S Cl Ar}
\end{figure}


\end{frame}

\begin{frame}
\frametitle{Случай хлора}
\begin{figure}[ht] 
	\centering\small
	\unitlength=1mm
	\includegraphics[width=70mm]{Figures/Cl with neighbors1d.png} 
	\caption{Графики энергий бета-распада и электронной конверсии для изотопов Cl, S  и Ar}% --- https://www.nndc.bnl.gov/nudat3/
	\label{f:Cl with neighbors1d}
\end{figure}
Таким образом, изотоп $Cl^{36}$ переходит путём бета-распада в изотоп  $Ar^{36}$ или путём электронного захвата в изотоп  $S^{36}$.

\end{frame}

\begin{frame}
\frametitle{Случай олова}
Три и более изотопа имеют почти все изотопы с чётным числом нуклонов.  
~\\
Особенно большое количество изотопов имеют элементы Pd, Cd, Sn, Te, Xe, Ba с электрическим зарядом  46, 48, 50, 52, 54, 56:
от 6 до 10.\\
~\\
Олово --- абсолютный рекордсмен по числу стабильных изотопов. Их всего 10!

\begin{figure}[ht] 
	\centering\small
	\unitlength=1mm
	\includegraphics[width=120mm]{Figures/T half time Sn In Sb.png} 
%	\caption{Стабильные изотопы Sn In Sb --- чёрные квадраты} --- https://www.nndc.bnl.gov/nudat3 
	\label{f:T half time Sn In Sb}
\end{figure}
\end{frame}

\begin{frame}
\frametitle{Случай олова}
\begin{figure}[ht] 
	\centering\small
	\unitlength=1mm
	\includegraphics[width=80mm]{Figures/Sn with neighbors1d.png} 
	\caption{Графики энергий бета-распада и электронной конверсии для изотопов Sn, In  и Sb} %--- https://www.nndc.bnl.gov/nudat3/
	\label{f:Sn with neighbors1d}
\end{figure}
\end{frame}

\begin{frame}
\frametitle{Примеры нуклеосинтеза, отличного от солнечной системы}

\begin{itemize}
    \item Карбоновые (углеродные) звёзды
    \item Технециевые звёзды
\end{itemize}



\end{frame}

\begin{frame}
\frametitle{Различие между  Солнцем и планетами}
{\footnotesize
 на Земле присутствуют элементы-потомки весьма разных предков, так что элементный состав Земли, во-первых богаче, чем продукция нуклеосинтеза Солнца, поскольку Солнечная система --- результат взрыва \emph{сверхновой звезды}. 	Во-вторых, между планетами и другими объектами Солнечной системы могут быть различия в распространённости элементов и их изотопном составе.}
	\begin{figure}[ht] 
	\centering\small
	%	\unitlength=1mm
	\includegraphics[width=0.7\textwidth]{Figures/1920px-Elements-origin-ru.png}
	%	\includegraphics[width=30mm]{Figures\Oxygen.png}
%	\caption{Происхождение элементов} %https://en.wikipedia.org/wiki/Нуклеосинтез 
	\label{f:Nucleosynthesis_periodic_table}
\end{figure}

\end{frame}