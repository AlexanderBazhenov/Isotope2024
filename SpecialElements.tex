\begin{frame}
	\frametitle{<<НЕЗЕМНЫЕ>> ЭЛЕМЕНТЫ}
	
	\begin{center}
		{\large
			<<НЕЗЕМНЫЕ>> ЭЛЕМЕНТЫ}
	\end{center}
	
В документах  IUPAC-2021 особое место уделено двум элементам, которые имеют огромные по отношению к другим элементам, вариации изотопного состава.

\begin{itemize}
	\item Аргон
	\item Свинец
\end{itemize}	
\end{frame}


\begin{frame}
	\frametitle{Аргон --- тяжёлая невидимка атмосферы}


Аргон является уникальным веществом.


 Удивительно, что газ, который составляет около 1\% атмосферы по массе, был открыт только в конце XIX в. 
 
 Причиной такого <<равнодушия>> является химическая инертность этого газа. Собственно, это первый \emph{благородный} газ, открытый на Земле.

\begin{figure}[ht] 
	\centering\small
	%	\unitlength=1mm
	\includegraphics[width=0.7\textwidth]{Figures/ArgonIsotopes.png}
	\caption{Период полураспада изотопов аргона и его стабильные изотопы} 
	%--- https://applets.kcvs.ca/IPTEI/IPTEI.html
	\label{f:ArgonIsotopes}
\end{figure}
		
\end{frame}

\begin{frame}
	\frametitle{Изотопы аргона и калия}
	
Аргон представлен в земной атмосфере тремя стабильными изотопами: $^{36}Ar$ (0.337 \%), $^{38}Ar$ (0.063\%), $^{40}Ar$ (99.600 \%). При этом в солнечной фотосфере и в атмосферах планет-гигантов изотопное содержание $^{40}Ar$ составляет лишь $\approx 0.01$ \%.
	
\begin{figure}[ht] 
	\centering\small
	%	\unitlength=1mm
	\includegraphics[width=0.6\textwidth]{Figures/40K40Ar.png}
	\caption{Распространённость изотопов калия и аргона \un{на Земле}.} 
%	--- по данным https://www.nndc.bnl.gov/nudat3/
	\label{f:40K40Ar}
\end{figure}

Образование аргона $^{40}Ar$ идёт путём захвата орбитального электрона 
\begin{equation}\label{40K40Ar}
	^{40}_{19}K + e^{-} \longrightarrow \ ^{40}_{18}Ar + \ov{\nu}_e.
\end{equation}

\end{frame}

\begin{frame}
	\frametitle{Вариации атомного веса  аргона}
	
\begin{figure}[ht] 
	\centering\small
	%	\unitlength=1mm
	\includegraphics[width=0.9\textwidth]{Figures/36ArgonIUPAC2021.png}
	\caption{Вариации атомного веса (чёрные линии) аргона, $A_r(Ar)$, и значения долей	(розовые линии) изотопа $^{36}Ar$, $\chi(36Ar)$, для некоторых веществ}
	\label{f:ArgonVar}
\end{figure}
	
\end{frame}


\begin{frame}
	\frametitle{Представление аргона в таблице Менделеева}
	
\begin{figure}[ht] 
	\centering\small
	%	\unitlength=1mm
	\includegraphics[width=0.4\textwidth]{Figures/Argon.png}
	\caption{Представление аргона в таблице Менделеева} 
	\label{f:Argon}
\end{figure}
\end{frame}




\begin{frame}
	\frametitle{Свинец --- не солнечные изотопы}
	
	\begin{figure}[ht] 
		\centering\small
		%	\unitlength=1mm
		\includegraphics[width=0.25\textwidth]{Figures/Lead.png}
		\caption{Представление свинца в таблице Менделеева} 
		\label{f:Lead}
	\end{figure}
	
{\small
	Изотопный портрет свинца весьма богат. При этом плотность свинца в различных материалах имеет рекордную изменчивость. Она настолько велика, что \un{превышает единицу величины атомной массы}. Эта разница была обнаружена даже при использовании весьма скромных по точности средств измерения. }
	
	
\end{frame}

\begin{frame}
	\frametitle{Происхождение изотопов свинца}
	
Изотопный состав и атомный вес свинца в земных материалах различны, поскольку три его самых тяжелых стабильных изотопа являются стабильными конечными продуктами радиоактивного распада различных изотопов урана 
\begin{align*}
	& ^{238}U \longrightarrow \ldots \longrightarrow  ^{206}Pb,  \\ 
	& ^{235}U \longrightarrow \ldots \longrightarrow  ^{207}Pb, 
\end{align*}
и тория 
\begin{equation*} 
	^{232}Th \longrightarrow \ldots \longrightarrow ^{208}Pb.
\end{equation*}
\end{frame}

\begin{frame}
	\frametitle{Ряды тория, радия и актиния}
	
\begin{figure}[ht] 
	\centering\small
	\unitlength=1mm
	\begin{picture}(130,58)
		\put(-10,0){\includegraphics[width=130mm]{Figures/Lead206207208SeriesAZ.png}}
	\end{picture}
	\caption{Ряды тория, радия и актиния --- основные ветви} %	--- по данным https://www.nndc.bnl.gov/nudat3/
	\label{f:Lead206207208SeriesAZ}
\end{figure}
	
\end{frame}

\begin{frame}
%	\frametitle{Изотопы свинца на Земле}
	
\begin{figure}[ht] 
	\centering\small
	%	\unitlength=1mm
	\includegraphics[width=70mm]{Figures/Pb204.png}
%	\caption{Вариации изотопа $^{204}Pb$ в различных объектах} 
	\label{f:204LeadVariation}
\end{figure}

	
	
\end{frame}

\begin{frame}
		\frametitle{Вариации изотопа $^{204}Pb$ в осадочных породах}

\begin{table}[h!]
	\begin{center}
		{\small
			\begin{tabular}{ccc}
				\hline
				Категория & Нижняя & Верхняя \\
				~ & граница & граница \\ 
				\hline
				Доломит	& 0.0115 & 0.0136 \\
				Известняк	& 0.0078 & 0.0160 \\
				Мергель 	& 0.0118 & 0.0137 \\
				Алевриты 	& 0.0113 & 0.0131 \\
				Фосфориты & 0.0096 & 0.0140 \\
				\hline
			\end{tabular}
		}
		\caption{Вариации изотопа $^{204}Pb$ в осадочных породах,  $ \chi ^{204}Pb$.}
		\label{t:204LeadVariation}
	\end{center}
\end{table}

\end{frame}