\begin{frame}
\frametitle{ИЗОТОПЫ В ПРИРОДЕ}

\begin{center}
{\large
ИЗОТОПЫ В ПРИРОДЕ}
\end{center}

Изотопные ниши, ландшафты, подписи:\\
 А.Н.~Баженов.	Естественнонаучные и технические применения интервального анализа: учебное пособие. Санкт-Петербург, 2022.\\
 https://elib.spbstu.ru/dl/5/tr/2021/tr21-169.pdf/info\\
~\\
 А.Н.~Баженов, А.Ю.~Тельнова.	Изотопы и таблица Менделеева: учебное пособие. Санкт-Петербург, 2024.\\
 https://elib.spbstu.ru/dl/5/tr/2024/tr24-29.pdf/info\\
 ~\\
 А.Н.~Баженов.	Интервальные арифметики и прослеживаемость изотопной подписи: учебное пособие. Санкт-Петербург, 2023.\\
 https://elib.spbstu.ru/dl/5/tr/2023/tr23-167.pdf/info
 
\end{frame}


\begin{frame}
\frametitle{Изотопная подпись}
{\small
Согласно Википедии, <<Изотопная подпись (иначе, изотопная сигнатура) --- специфическое соотношение нерадиоактивных <<стабильных изотопов>> или относительно стабильных радиоактивных изотопов или неустойчивых радиоактивных изотопов определённых химических элементов в исследуемом материале>>. } %--- https://ru.wikipedia.org/wiki/Изотопная\_подпись

\begin{figure}
	\centering\small
	\unitlength=1mm
	\includegraphics[width=100mm]{Figures/Elements-and-isotopes-in-bones-also-teeth-and-their-applications-for-palaeoenvironemtal.png} 
%	\caption{Изотопы в костях и зубах \cite{Tutken2011}.} 
	\label{f:Bones}
\end{figure}

\end{frame}


\begin{frame}
\frametitle{Внедрение изотопных методов в контроле пищевой продукции.}
{\small
Изотопный анализ является эффективным методом осуществления контроля качества и выявления
фальсифицированной пищевой продукции. Традиционно применяемым в области изотопного анализа пищевой
продукции и регламентированным в соответствующих нормативных документах методом является метод изотопной масс-спектрометрии с элементным анализатором (далее --- EA-IRMS).\\
~\\
В настоящий момент наибольший интерес в рамках изотопного анализа пищевой продукции представляет стремительно развивающийся и обладающий рядом достоинств метод измерений отношения изотопов углерода --- метод спектроскопии внутрирезонаторного затухания с модулем сжигания (далее --- CM–CRDS). 
} 
\end{frame}

\begin{frame}
\frametitle{Внедрение изотопных методов в контроле пищевой продукции.}
{\small
	В публикации\\ \textsc{Чубченко~Я.К.} Методика измерений отношения изотопов углерода в ванилине методом CM–CRDS с расширенной неопределенностью менее 0,1 \%. Эталоны. Стандартные образцы. 2023;19(3):129-144. https://doi.org/10.20915/2077-1177-2023-19-3-129-144 \\
	продемонстрировано, что
	<<Результаты международных сличений СCQM-K167 подтвердили возможность измерений отношения изотопов
	углерода в ванилине методом CM–CRDS по разработанной методике \ldots соответствует наилучшим измерениям, выполняемым методом EA-IRMS.
	Достигнутый результат обладает практической значимостью, потому что подтверждает возможность применения
	метода CM–CRDS для осуществления контроля качества и выявления фальсифицированного ванилина.>>
}
\end{frame}

\begin{frame}
\frametitle{Аттестованные методики измерений}
{\footnotesize
	ФР.1.31.2012.13424 Методика измерений отношения изотопов 13С/12С этанола в пиве и пивных напитках методом изотопной масс-спектрометрии // Федеральный информационный фонд по обеспечению единства измерений : официальный сайт.
	URL: https://fgis.gost.ru/fundmetrology/registry/16/items/282517.	\\
	~\\
	ФР.1.31.2013.15529 Методика устанавливает процедуру определения отношения изотопов 18О/16О экзогенной и эндогенной воды в винах и суслах // Федеральный информационный фонд по обеспечению единства измерений : официальный сайт.
	URL: https://fgis.gost.ru/fundmetrology/registry/16/items/281364.	\\
	~\\
	ФР.1.31.2014.17273 Методика измерений отношения изотопов углерода 13С/12С в спиртных напитках виноградного происхожде	ния методом изотопной масс-спектрометрии // Федеральный информационный фонд по обеспечению единства измерений :
	официальный сайт. URL: https://fgis.gost.ru/fundmetrology/registry/16/items/280000 	\\
	~\\
	ФР.1.31.2016.24753 Методика измерений отношения изотопов кислорода, 18O/16О экзогенной и эндогенной воды в винах и суслах методом изотопной масс-спектрометрии // Федеральный информационный фонд по обеспечению единства измерений : офи-
	циальный сайт. URL: https://fgis.gost.ru/fundmetrology/registry/16/items/298716  	\\
	~\\
	ФР.1.31.2016.24962 Методика измерений отношений изотопов этанола в коньяках и коньячных дистиллятах методом изотопной масс-спектрометрии // Федеральный информационный фонд по обеспечению единства измерений : официальный сайт.
	URL: https://fgis.gost.ru/fundmetrology/registry/16/items/298746  \\
	~\\
	ФР.1.31.2017.28360 Методика измерений отношений изотопов углерода, кислорода, водорода этанола для выявления присутствия синтетического спирта в алкогольной продукции, а также в спиртосодержащих пищевых ароматизаторах методом изотопной масс-спектрометрии // Федеральный информационный фонд по обеспечению единства измерений : официальный
	сайт. URL: https://fgis.gost.ru/fundmetrology/registry/16/items/299163  \\
	~\\
	ФР.1.31.2018.31997 Методика измерений отношения изотопов кислорода 18O/16O водной компоненты сидров и пуаре методом изотопной масс-спектрометрии // Федеральный информационный фонд по обеспечению единства измерений : официальный
	сайт. URL: https://fgis.gost.ru/fundmetrology/registry/16/items/495958 
} 
\end{frame}
  

\begin{frame}
\frametitle{Изотопная планетология}
Изотопные отношения для ряда планет Солнечной системы и марсианского метеорита Allan Hills 84001, умноженные на $10^3$\\
~\\
\begin{table}[h!]
	\begin{center}	
		%		{\small
		\begin{tabular}{|c|cccc|}
			\hline 
			Изотоп. отн.  & Земля &  Марс  & Венера & ALH84001   \\
			\hline 
			$^{15}N / ^{14}N$ & $3.66 \pm 0.01$ & $5.8 \pm 0.4$ & $3.7 \pm 0.7$ & 3.875\\ [1mm]
			$^{13}C / ^{12}C$ & $11.23 \pm 0.05$ & $11.75 \pm 0.04$ & $12 \pm 2$ &  $11.75 \pm 0.09$ \\ [1mm]
			\hline 
		\end{tabular}
%		\caption{Изотопные отношения для ряда планет} 
%		{\small Солнечной системы и марсианского метеорита Allan Hills 84001, умноженные на $10^3$}
		\label{t:PlanetsIsotope}
		%	}
	\end{center}
\end{table} 
~\\
Как видно из данных таблицы, имеются значимые различия в изотопном составе планет и метеоритов. 
\end{frame}


\begin{frame}
\frametitle{Изотопологи}

\emph{Изотопологи} --- молекулы, различающиеся только по изотопному составу атомов, из которых они состоят. Изотополог имеет в своём составе, по крайней мере, один атом определенного химического элемента, отличающийся по количеству нейтронов от остальных.

\end{frame}


\begin{frame}
\frametitle{Изотопологи воды}

\emph{Изотопологи} --- молекулы, различающиеся только по изотопному составу атомов, из которых они состоят. Изотополог имеет в своём составе, по крайней мере, один атом определенного химического элемента, отличающийся по количеству нейтронов от остальных.

Комбинации 5 стабильных изотопов водорода и кислорода дают набор 9 молекул-изотопологов воды. 
Природная вода представляет собой многокомпонентную смесь изотопологов. Содержание самого лёгкого изотополога в ней значительно превосходит концентрацию всех остальных вместе взятых. Содержание тяжёлых изотопов водорода и кислорода в природных водах определяется двумя международными стандартами, введенными Международным агентством по атомной энергии (МАГАТЭ):
\begin{itemize}
	\item Стандарт VSMOW (Vienna Standard Mean Ocean Water) определяет изотопный состав глубинной воды Мирового океана \cite{VSMOW} \index{VSMOW, Vienna Standard Mean Ocean Water}
	\item Стандарт SLAP (Standard Light Antarctic Precipitation) определяет изотопный состав природной воды из Антарктики \cite{SLAP} \index{SLAP, Standard Light Antarctic Precipitation}
\end{itemize}

Стандарт SLAP характеризует самую лёгкую природную воду на Земле, VSMOW --- глубинную океаническую воду. 

\end{frame}


\begin{frame}
\frametitle{Изотопологи воды}
Рассчитанные весовые количества изотопологов в природной воде, соответствующие международным стандартам SMOW (средняя молекулярная масса = 18,01528873) и SLAP (средняя молекулярная масса = 18,01491202) 
\begin{table}[h!]
	\begin{center}
		{\footnotesize
			\begin{tabular}{cccc}
				\hline
				Изотополог воды	& Молекулярная масса &	Содержание, г/кг & ~  \\
				~	& ~ &	SMOW & SLAP \\
				\hline
				1H216O &	18,01056470	& 997,032536356	& 997,317982662 \\
				1HD16O &	19,01684144	& 0,328000097&	0,187668379 \\
				D216O &	20,02311819	& 0,000026900	& 0,000008804\\
				1H217O	&19,01478127	& 0,411509070 &	0,388988825\\
				1HD17O &	20,02105801	& 0,000134998	& 0,000072993\\
				D217O &	21,02733476 &	0,000000011	& 0,000000003\\
				1H218O &	20,01481037	& 2,227063738	& 2,104884332\\
				1HD18O &	21,02108711	&0,000728769	&0,000393984\\
				D218O &	22,02736386	&0,000000059&	0,000000018	\\			
				\hline
			\end{tabular}
		}
	\end{center}
\end{table}
В природной воде весовая концентрация тяжёлых изотопологов может достигать 2.97 г/кг, что является значимой величиной, сопоставимой, например, с содержанием минеральных солей.

\end{frame}


\begin{frame}
\frametitle{Изотопологи воды}
Природная вода, близкая по содержанию изотополога 1H216O к стандарту SLAP, а также специально очищенная с существенно увеличенной долей этого изотополога по сравнению со стандартом SLAP, определяется как особо чистая лёгкая вода (менее строгое определение, которое применимо в реальной жизни).

В лёгкой воде доля самого лёгкого изотополога составляет (мол.\%):

\begin{equation*}
99.76 \leq \ ^1H_2\,^{16}O \ \leq 100.
\end{equation*}

Если из воды, отвечающей стандарту SMOW, удалить все тяжёлые молекулы, массовое содержание которых составляет 2,97 г/кг и заменить их на 1H216O, то масса 1 л такой лёгкой и изотопно чистой воды уменьшится на 250 мг. \\
Таким образом, параметры лёгкой воды, в первую очередь, её <<лёгкость>> и изотопный состав поддаются измерению с помощью таких методов, как масс-спектрометрия, гравиметрия, лазерная абсорбционная спектроскопия, ЯМР.

\end{frame}


\begin{frame}
\frametitle{Изотопомеры}

Если молекула содержит не один атом какого-либо вещества, то на её свойства влияет и то место, где расположен изотоп. Молекулы или ионы, отличающиеся расположением изотопов, называют \emph{изотопомеры} --- см. ГОСТ 58567.

\begin{figure}[ht] 
	\centering\small
	%	\unitlength=1mm
	\includegraphics[width=0.8\textwidth]{Figures/Isotopocule.png}
	%	\includegraphics[width=30mm]{Figures\Oxygen.png}
%	\caption{Isotopocule} 
	\label{f:Isotopocule}
\end{figure}

Например, исследование распределения изотопомеров в аспаргиновой кислоте (одна из 20 протеиногенных аминокислот организма). 



\end{frame}





\begin{frame}
\frametitle{Изотопомеры}
Изотопомеры нашли применение в определении параметров молекулярных взаимодействий.
Если молекула обладает высокой пространственной симметрией, часть переходов на различные состояния запрещена, так что по ним невозможно получить информацию.\\
~\\

Если какой-то из атомов замещен изотопом, симметрия уменьшается. Таким образом, анализ спектров высокого разрешения изотопологов молекул является хорошим дополнительным источником информации при определении внутренней динамики молекул. 

\end{frame}


\begin{frame}
\frametitle{Изотопомеры}
На Рис. приведён пример распределения масс изотопомер в органических кислотах  
\begin{figure}[ht] 
	\centering\small
	%	\unitlength=1mm
	\includegraphics[width=0.8\textwidth]{Figures/Mass isotopomer distribution in organic acids.png}
	%	\includegraphics[width=30mm]{Figures\Oxygen.png}
%	\caption{Распределения масс изотопомер в органических кислотах \cite{MassIsotopomers2020}} 
	\label{f:MassIsotopomerOrganicAcids}
\end{figure}

В англоязычной литературе в качестве общего обозначения для изотопологов и изотопомеров иногда используется термин \emph{isotopocule}
\end{frame}

\begin{frame}
\frametitle{Изотопологи в оптических спектрах}

{\footnotesize
Изотопологи изучают не только в масс-спектрах. Оптическая (в широком смысле) спектроскопия является мощным средством исследования органических соединений. В случае пространственно недоступных объектов, например, внеземных, спектральная информация несёт основную долю информации.

Спектроскопия высокого разрешения является динамично развивающейся в теоретическом и экспеиментальном аспектах современной физики. Наличие в сложных молекулах различных степеней свободы порождает очень богаты спектры. Даже для относительно простых молекул, таких как этилен,
спектры весьма сложны.

На Рис. представлены экспериментально зарегистрированные спектры высокого разрешения в диапазоне нижних фундаментальных полос для молекулы этилена и её изотопологов.

Громова, Ольга Васильевна. Спектроскопия высокого разрешения молекул типа асимметричного волчка: C2H4, SO2, H2S, ClO2, NH3 и их изотопологи : диссертация в виде научного доклада на соискание ученой степени д.ф.-м.н. \ldots — Томск: 2022.

Для различных термов имеются тысячи возможных переходов, а общее количество спектральных линий в сложных молекулах составляет сотни тысяч. В многочисленных публикациях, приводятся результаты расчётов и экспериментальных данных, дающих изотопные
<<портреты>> изотопологов различных органических молекул.}
\end{frame}

\begin{frame}
\frametitle{Изотопологи в оптических спектрах}

Спектры высокого разрешения в диапазоне нижних фундаментальных полос для молекулы этилена и её изотопологов.

\begin{figure}[ht] 
	\centering\small
	%	\unitlength=1mm
	\includegraphics[width=0.4\textwidth]{Figures/IsotopologyC2H4.png}
\end{figure}

\end{frame}


\begin{frame}
\frametitle{Фракционирование изотопов в природе}
{\small
Фракционирование изотопов в природе, то есть, по массе, является проявлением очень общих закономерностей процессов, протекающих на планете. Поскольку химические свойства изотопов одного элемента очень близки друг к другу, то это предоставляет уникальную возможность исследования закономерностей эволюции планеты на больших масштабах времени.\\
~\\
Все \emph{биофильные элементы}, участвующие в биосинтезе микробных продуктов и их клеточных компонентов, за исключением фосфора, являются полиизотопными элементами. Стабильные изотопы этих элементов ($^1H, ^2H$ --- водород, $^{12}C, ^{13}C$ --- углерод, $^{14}N, ^{15}N$ --- азот, $^{16}O, ^{17}O, ^{18}O $ --- кислород, $^{32}S, ^{33}S, ^{34}S$ --- сера) имеют близкие химические свойства, но различаются по массам.
}

\end{frame}

\begin{frame}
\frametitle{Фракционирование изотопов в природе}

\begin{figure}[ht] 
	\centering\small
	\unitlength=1mm
	{\includegraphics[width=120mm]{Figures/IsotopeTracing.png}} 
	\caption{Периодическая таблица элементов Менделеева} с данными для отслеживания изотопных подписей
	\label{f:IsotopeTracers}
\end{figure}
\end{frame}

\begin{frame}
\frametitle{Изотопы углерода в нефтяной тематике}
В нефтяной науке изотопы углерода обычно связаны с
интерпретирующая информация, которую несут относительные, естественные распределения
двух видов стабильных изотопов углерода, $^{12}C$ и $^{13}C$, в различных органических и неорганических соединениях.

На Рис.~представлена схема формирования керогена из биологических отложений. \index{геополимеры}

\begin{figure}[h]
	\centering
	\includegraphics[width=0.6\linewidth]{Figures/BiotoGeoPolymers.png}
	\caption{Фомирование гетерогенных геополимеров (кероген)
		из исходных биополимеров захороненной органики.}
	\label{fig:BiotoGeoPolymers}
\end{figure}

При переходе углерода между фазами и веществами, например, из керогена 
в нефть, или $CO_2$ в $CH_4$, трансформации связаны с изотопными
эффектами, которые производят систематические и в целом предсказуемые
изотопное фракционирование. 
Таким образом, изотопы углерода частично
разделены таким образом, что одна углеродсодержащая фаза по сравнению с
другая в системе может быть относительно обогащена или обеднена
$^{12}C$ или $^{13}C$.

\end{frame}

\begin{frame}
\frametitle{Изотопы углерода в нефтяной тематике}


При переходе углерода между фазами и веществами, например, из керогена 
в нефть, или $CO_2$ в $CH_4$, трансформации связаны с изотопными
эффектами, которые производят систематические и в целом предсказуемые
изотопное фракционирование. \\
~\\
Таким образом, изотопы углерода частично
разделены таким образом, что одна углеродсодержащая фаза по сравнению с
другая в системе может быть относительно обогащена или обеднена
$^{12}C$ или $^{13}C$.

\end{frame}


\begin{frame}
\frametitle{Внутреннее строение Земли. Мантия Земли и изотопные распределения алмазов}



{\small

Согласно Википедии 
Мантия --- часть Земли (геосфера), расположенная непосредственно под корой и выше ядра. В ней находится большая часть вещества Земли. Мантия есть и на других планетах земной группы. Земная мантия находится в диапазоне от 30 до 2900 км от земной поверхности. Мантия занимает около 80\% объёма Земли.\\
~\\
Мантию изучают, в частности, по фрагментам мантийных пород, выносимые на поверхность мантийными же расплавами — кимберлитами, щелочными базальтами и др. Это ксенолиты, ксенокристы и алмазы. Алмазы занимают среди источников информации о мантии особое место. Именно в алмазах установлены самые глубинные минералы, которые, возможно, происходят даже из нижней мантии. В таком случае эти алмазы представляют собой самые глубокие фрагменты Земли, доступные непосредственному изучению.
}



\end{frame}

\begin{frame}
\frametitle{Внутреннее строение Земли. Мантия Земли и изотопные распределения алмазов}



\begin{figure}[h]
	\centering
	\includegraphics[width=0.7\linewidth]{Figures/13CDiamondsMantle.png}
	\caption{Вариация изотопа $\delta^{13}C$ в алмазах}
	\label{fig:13CDiamondsMantle}
\end{figure}

На Рис.~показаны вариации изотопа $\delta^{13}C$ в алмазах различных геосфер Земли.


\end{frame}

\begin{frame}
\frametitle{Изотопы углерода как геохимический трассер}


Диапазон значений изотопов углерода в обычных неорганических и
биологические резервуарах углерода приведён на Рис.~\ref{fig:GeoTracer}.
\begin{figure}[h]
	\centering
	\includegraphics[width=0.8\linewidth]{Figures/GeoTracer.png}
	\caption{Изотопы углерода как геохимический трассер}
	\label{fig:GeoTracer}
\end{figure}
Часть значений для некоторых объектов вышла за диапазон рисунка и показаны условно. При этом ряд других значений имеет весьма узкие интервалы.

\end{frame}


\begin{frame}
\frametitle{Изотопы углерода как геохимический трассер}

\begin{figure}[h]
	\centering
	\includegraphics[width=0.8\linewidth]{Figures/13CvsAge.png}
	\caption{Зависимость изотопной подписи значений $\delta^{13}C$  от времени  для органинического углерода (Corg) и углеродных карбонатов (Ccarb).}
	\label{fig:13CvsAge}
\end{figure}


\end{frame}

\begin{frame}
\frametitle{Изотопы углерода как геохимический трассер}
\begin{figure}[h]
	\centering
	\includegraphics[width=0.8\linewidth]{Figures/IdealizedFigure.png}
	\caption{Зависимость идеализированной изотопной подписи значений $\delta^{13}C$ в морских карбонатах от времени}
	\label{fig:IdealizedFigure}
\end{figure}

Как видно из рисунков, на некотрых этапах эволюции биосферы изотопная подпись была весьма изменчива.
\end{frame}

\begin{frame}
\frametitle{Изотопологи углекислого газа.}




Термометрия изотопов слипшихся карбонатов --- геохимический инструмент.
используется для определения температуры образования карбонатных минералов. В отличие от других подобных карбонатных
термометров, термометрия слипшихся изотопов не требует предположений об изотопном составе
жидкость, из которой выпал карбонат. \\
~\\

Путем измерения слипшегося изотопного состава ($\Delta_{47}$)
карбонатных минералов с известной температурой образования, можно построить эмпирическую калибровку для
термометр слипшихся изотопов, который необходимо преобразовать из значения $\Delta_{47}$ в температуру пласта.
\end{frame}

\begin{frame}
\frametitle{Изотопологи углекислого газа.}
<<Геотермометр>> на основе изотополога углекислого газа
\begin{figure}[h]
	\centering
	\includegraphics[width=0.45\linewidth]{Figures/CO2delta47isotop.png}
%	\caption{<<Геотермометр>> на основе изотополога углекислого газа\cite{Isotopology47}.}
	\label{fig:CO2delta47isotop}
\end{figure}



\end{frame}


\begin{frame}
\frametitle{Происхождение жизни и углеродные соединения.}

{\footnotesize
Важнейшим вопросом естествознания является проблема возникновения жизни. В связи с выходом человечества в Космос, он приобрёл новый импульс и получил основательную приборную базу для исследований. Развились новые дисциплины науки.\\
~\\
Астробиология (экзобиология) --- \index{астробиология, экзобиология}
научная дисциплина, рассматривающая возможность появления, эволюции и сохранения жизни на других планетах во Вселенной. \\
Возможная схема формирования глицина }

\begin{figure}[h]
	\centering
	\includegraphics[width=0.5\linewidth]{Figures/GlycineFormation.png}
%	\caption{Возможная схема формирования глицина \cite{glycine2021}}.
	\label{fig:GlycineFormation}
\end{figure}

\end{frame}


\begin{frame}
\frametitle{Возможная схема формирования глицина}

{\small
Для поиска жизни на других планетах необходимо определить исходные положения. Одним из важнейших предположений является то, что подавляющее большинство форм жизни в нашей Галактике основано на углеродной химии, как и все формы жизни на Земле. В молекулярных облаках найдено огромное количество углеродосодержащих молекул. \\
~\\
Большое внимание привлекает проблема обнаружения в космосе аминокислот. \index{глицин}
Наиболее вероятным кандидатом в настоящее время считается глицин, $NH_2CH_2COOH$, синтез которого из уже обнаруженных молекул вполне возможен.\\
~\\
На Рис. (выше) представлены возможные варианты формирования глицина из молекул льдов, богатых водой, на ранних стадиях формирования звёзд малой массы.
Среди вариантов обсуждаются реакции с изотопами углерода:
\begin{align*}
^{13}C^{18}O + NH_2CH_3 + O_2 + H, \\
^{13}C^{18}O + NH_2CH_3  + ^{18}O_2 + H,\\
CO + NH_2CH_3  + O_2 + D.
\end{align*}
}
\end{frame}

\begin{frame}
\frametitle{Изотопные ландшафты --- $^{15}N$ на Земле}

Изотопное отношение для $\delta^{15}N$ в растениях	

\begin{figure}[ht] 
	\centering\small
	\unitlength=1mm
	{\includegraphics[width=100mm]{Figures/Plant15N.png}} 
\end{figure}


\end{frame}

\begin{frame}
\frametitle{ Изотопная подпись  --- кислород  на Земле}
%Вариации атомного веса и изотопного состава ряда материалов, содержащих кислород на Земле.
\begin{figure}[ht] 
	\centering\small
	\unitlength=1mm
	\includegraphics[width=70mm]{Figures/OxygenNature.png} 
\end{figure}

\end{frame}


\begin{frame}
\frametitle{Пример фракционирования изотопов в природе}

Фракционирование изотопов водорода и кислорода при испарении-конденсации
\begin{figure}[ht] 
	\centering\small
	\unitlength=1mm
	\includegraphics[width=70mm]{Figures/EvaporationCondensation.png} 
\end{figure}

\end{frame}


\begin{frame}
\frametitle{Изотопная ниша }
Ареалы 12 видов водяных печников по изотопному составу перьев 
\begin{figure}[ht] 
	\centering\small
	\unitlength=1mm
	\includegraphics[width=45mm]{Figures/BirdsArealLeft.png}
\end{figure}
\end{frame}

