%%%%%%%%%%%%%%%%%%%%%%%%%%%%%%%%%%%%%%%%%%%%%%%%%%%%%%%%%%%%%%%%%%%%%%%%%%%%%%%%%%%%%%%%%%%%%%%%%%%%%%%%%%%%%%%%%%
\begin{frame}
	\frametitle{Манифест интервальной статистики}
	%А.Н.~Баженов, С.И. Жилин, С.И. Кумков, С.П. Шарый.
	%		Обработка и анализ интервальных данных. Издательство «ИКИ»		2024 г.
	\begin{figure}[ht]
		\centering
		\includegraphics[width=0.4\textwidth]{Figures/Обложка книги РХД.jpg}
		%	\includegraphics[width=0.8\textwidth]{7.mps}	
	\end{figure}
\end{frame}

\begin{frame}
	\frametitle{Всероссийский веб-семинар по интервальному анализу и его приложениям. 28.11.2022.}
А.Н.~Баженов.\\
~\\ 
Интервальная таблица Менделеева элементов и изотопов. \\
~\\ 
Всероссийский веб-семинар по интервальному анализу и его приложениям. 28.11.2022.\\
~\\
http://interval.ict.nsc.ru/WebSeminar/ANBazhenov-28.XI.2022.pdf
\end{frame}

\begin{frame}
	\frametitle{Трудности в использовании интервальной неопределённости}

\begin{center}	
ТРУДНОСТИ В ИСПОЛЬЗОВАНИИ \\ ИНТЕРВАЛЬНОЙ НЕОПРЕДЕЛЁННОСТИ
\end{center}	
	
\end{frame}


\begin{frame}
	\frametitle{Трудности в использовании интервальной неопределённости}
	
Общая проблема --- интерпретация интервала как равномерного распределения

\medskip
Наиболее актуальный документ по стандарным атомным весам
\medskip
T.Prohaska, J.Irrgeher, J.Benefield, et al. 

\medskip
Standard atomic weights of the elements 2021 (IUPAC Technical Report) 

Pure Appl. Chem. 2022; 94(5): 573–600

	
\end{frame}

\begin{frame}
	\frametitle{Интерпретация интервала как равномерного распределения}
{\footnotesize 
	The interval does not imply any statistical distribution of atomic weight values between the lower and upper bound (e.g. the arithmetic mean of a and b is not necessarily the most likely value). Similarly, the interval does not convey a simple statistical representation of uncertainty. The probability density function may differ case by case, due to varying sources and their proportions may need to be considered. 
	
	If no additional information is available or utilized, \un{the probability density function} associated with the standard atomic weights \un{can be considered as uniform (rectangular)}.
}
Формулировка, провоцирующая ложную интерпретацию:

Если нет дополнительной доступной или уже используемой информации, о функции плотности вероятности, связанной с атомными весами, то она может считаться равномерной (прямоугольной).
	
\end{frame}


\begin{frame}
	\frametitle{Непонимание интервальной арифметики}

Рассмотрим публикацию IUPAC.

\medskip	
A. Possolo, A.M.H. van der Veen, J. Meija and D.B. Hibbert
<<Interpreting and propagating the uncertainty of the standard atomic weights>> (IUPAC Technical Report). \\
Pure Appl. Chem. 2018; 90(2): 395–424 %\cite{IUPACUncertainty}




\end{frame}

\begin{frame}
	\frametitle{Непонимание интервальной арифметики}
Исходные данные
	\begin{align*}
A_r(H) = & \, [1.007 84, 1.008 11],\\
A_r(O) = & \,[15.999 03, 15.999 77]], \\
M_r(H_2O) = & \, 2 \cdot[1.007 84, 1.00811] + [15.99903,15.999 77] \\
 = & \,[18.01471, 18.015 99]
\end{align*}
Расчёт массовой доли кислорода в молекуле воды даёт разные результаты в зависимости от способа расчёта
\begin{align}
		w_{H_2O}(O) = & \frac{1}{2A_r(H)/A_r(O)+1} = &[0.888 \un{083}, \, 0.888 \un{114}] \label{H2O1} \\
		\ne & \frac{A_r(O)}{M_r(H_2O)} =& [0.888 \un{046}, \, 0.888\un{150}]. \label{H2O2}
\end{align}
Сравним ширины \eqref{H2O1}	и \eqref{H2O2}:
$wid (1) = 0.000\un{031} \ne wid (2) = 0.000\un{104} $.
	
\end{frame}


\begin{frame}
	\frametitle{Непонимание интервальной арифметики}


	
<<Ввиду очевидных практических трудностей, связанных с применением интервальной арифметики, и учитывая
желательность количественной оценки и распространения неопределенностей в расчетах, связанных с атомными весами, мы рекомендуем
вместо этого вероятностный подход к интерпретации интервалов, который мы считаем полностью последовательным
с мотивацией, которая побудила CIAAW принять такое обозначение. 
\medskip
В остальной части этого отчета объясняется основа и
общие процедуры такого подхода и иллюстрирует их применение на конкретных и содержательных примерах.>>
	
\end{frame}


\begin{frame}
	\frametitle{Непонимание интервальной арифметики}
	
	\begin{figure}[ht] 
		\centering\small
		\unitlength=1mm
		\includegraphics[width=100mm]{Figures/BoronTable1data.png}
		\caption{Данные для атомных весов образцов бора} 
		\label{f:BoronTable1data}
	\end{figure}	
	
	
\end{frame}

\begin{frame}
	\frametitle{Непонимание интервальной арифметики}
Множественные пики и впадины на синей кривой обусловлены тем, что соответствующее распределение представляет собой смесь 13 различных прямоугольных распределений.
\begin{figure}[ht] 
	\centering\small
	\unitlength=1mm
	\includegraphics[width=80mm]{Figures/BoronTable1.png}
	\caption{<<Плотность вероятности>> для атомных весов образцов бора и её аналитическая аппроксимация} 
	\label{f:BoronTable1}
\end{figure}	
	
	
\end{frame}

\begin{frame}
\frametitle{Проблемы описания интервальных весов}
T.B. Coplen, N.E. Holden, T. Ding, H.A.J. Meijer, J. Vogl, and X. Zhu, <<The Table of Standard Atomic Weights-An exercise in consensus>>, \\ Rapid Commun Mass Spectrom. 2022; 36:e8864. %\cite{IUPACCosensus}	

\begin{itemize}
	\item Элементы с 2 изотопами
	\item Элементы с асимметричнными распределениями изотопов
\end{itemize}

\end{frame}

\begin{frame}
	\frametitle{Элементы с 2 изотопами }
<<\ldots введения нового определения значений атомного веса в 1979 г.
\ldots неопределенности в значении атомного веса для элементов с двумя стабильными изотопами стали асимметричными; то есть, неопределенность в верхней части значения атомного веса не была
такой же, как и на нижней стороне. Это привело к серьезной проблеме
которая сохраняется в течение последних четырех десятилетий. Середина симметричной неопределенности может соответствовать значениям атомного веса, которое не относится ни к одному известному источнику этого элемента, обнаруженному в природе. Естественное решение этой проблемы
было бы введением асимметричных неопределенностей. К сожалению,
Комиссия последовательно отвергала это решение. В итоге,
это привело к введению интервалов атомного веса для
выбранных элементов \ldots >>
\end{frame}

\begin{frame}
		\frametitle{Элементы с асимметричнными распределениями изотопов }
\begin{figure}[ht] 
	\centering\small
	\unitlength=1mm
	\includegraphics[width=60mm]{Figures/LiExample.png}
	\caption{Пример распределения атомных масс для стандартного образца карбоната лития} 
	\label{f:LiExample}
\end{figure}
\end{frame}

\begin{frame}
	\frametitle{3 способа описания неопределённостей   }
	\begin{figure}[ht] 
		\centering\small
		\unitlength=1mm
		\includegraphics[width=60mm]{Figures/Uncertainty_cases.png}
		\caption{Типы представления неопределённостей} 
		\label{f:Uncertainty cases}
	\end{figure}
\end{frame}

\begin{frame}
	\frametitle{3 способа описания неопределённостей   }
A,\\ пример одного из 21 элемента, имеющего значение $A_r (E)$, определяемое одним
изотопом, Guide to the Expression of Uncertainty in
Measurement, оцененная распределением Гаусса с коэффициентом охвата 6.

B, \\пример одного из 49 элементов, имеющих стандартный атомный вес
значение и неопределенность, определяемые
консенсусом; наибольшее значение вероятности
функция плотности не обязательно должно совпадать с $A_r (E)$

C,\\
Пример одного из 14 элементов, имеющих
согласованные стандартные атомные веса, выраженные как
интервалы; функции плотности вероятности неизвестны. 

Распределение функции вероятностей
также неизвестны для 6 элементов, имеющих
метку «r» (гелий, никель, медь, цинк, селен и стронций)


\end{frame}

\begin{frame}
\frametitle{Обработка данных методами интервального анализа}


\begin{center}	
ОБРАБОТКА ДАННЫХ \\
МЕТОДАМИ ИНТЕРВАЛЬНОГО АНАЛИЗА \\
--- примеры
\end{center}

\end{frame}


\begin{frame}
	\frametitle{Обработка данных методами интервального анализа}
Обработка данных методами интервального анализа. Теория --- книга	\cite{BookIntStat}
\begin{itemize}
	\item Представление данных (диаграмма рассеяния)
	\item Интервальные оценки (мода)
	\item Одновременное вычисление внешних и внутренних оценок (твины)
	\item Минимумы по включению
	\item $\ldots$
\end{itemize}
	
\end{frame}


\begin{frame}
	\frametitle{Вариации углерода в органических объектах}
	
Рассмотрим часть данных, относящихся к углероду природных объектов. 

\begin{table}[h!]
	\begin{center}
		{\small
			\begin{tabular}{ccc}
				\hline
				Категория & Нижняя & Верхняя \\
				~ & граница & граница \\ 
				\hline
				Наземные растения (C3 метаболический процесс)	& -35 & -21 \\
				Наземные растения (C4 метаболический процесс)	& -16 & -9 \\
				Наземные растения  (CAM метаболический процесс)	& -34 & -10 \\
				\hline
				Морские организмы 	& -74.3 & -2 \\
				Морские отложения и соединения &	-130.3 & 7 \\
				\hline
				Уголь &	-30 & -19 \\
				Сырая нефть &	-44 & -16.8 \\
				Природный этанол & -32 &  -10.3 \\
				\hline
			\end{tabular}
		}
		\caption{Вариации углерода в органических объектах} в единицах $10^3 \cdot \delta ^{13}C_{VPDB}$
		\label{t:OrganicCarbonVariation}
	\end{center}
\end{table}
	
\end{frame}


\begin{frame}
	\frametitle{Вариации углерода в органических объектах}
	
\begin{align}
	10^3 \cdot \delta ^{13}C_{VPDB} = \{ \
	&	[-35, -21 ], [ -16, -9 ], [ -34, -10 ] \nonumber \\
	&	[ -74.3, -2 ], [ -130.31, +7 ], [ -30, -19], \nonumber \\
	& 	[ -44, -16.8], [ -32, -10.3] \	\}
\end{align}

Диаграмма рассеяния 
\begin{figure}[ht] 
	\centering\small
	\unitlength=1mm
	{\includegraphics[width=55mm]{Figures/CarbonOrganicDataIUPAC2016.png}} 
	\caption{Вариации углерода в органических объектах} в единицах $10^3 \cdot \delta ^{13}C_{VPDB}$
	\label{f:OrganicCarbonVariation}
\end{figure}

	
\end{frame}

\begin{frame}
	\frametitle{Вариации углерода в органических объектах}
	
\begin{figure}[ht] 
	\begin{center}
		\unitlength=1mm
		{\includegraphics[width=60mm]{Figures/CarbonOrganicDataIUPAC2016ModeArray.png}} 
		\caption{Интервальная мода 
			вариаций $10^3 \cdot \delta ^{13}C_{VPDB}$ в органических объектах} 
		\label{f:OrganicCarbonVariationMode}
	\end{center}	
\end{figure}

Интервальная мода
\begin{equation}
	{\tt mode}( 10^3 \cdot \delta ^{13}C_{VPDB} ) = [-30, -21].
\end{equation}	

\end{frame}

\begin{frame}
	\frametitle{Расчёт молекулярной массы молекулы метана в виде твина}

Рассмотрим  массу молекулы метана $CH_4$.

Пусть источником водорода и углерода является   морская фауна. 

Данные об изотопных вариациях берутся c сайта 

\medskip
https://www.sciencebase.gov/catalog/item/580e719ae4b0f497e794b7d8.

\medskip
Представим массы компонент молекулы в виде твинов в форме Нестерова.
\begin{equation*}
	\mbf{X}_{\subseteq} = [ \, \mbf{X}^{in} \, ,  \, \mbf{X}^{out} \, ]
\end{equation*}
$\mbf{X}^{in}, \mbf{X}^{out}$ --- интервалы внутренней и внешней оценки величины.

\medskip
\emph{Внешней оценкой} будет служит максимум по включению для атомов молекулы, \\
\emph{внутренней оценкой} --- мода выборки.

\end{frame}

% внешняя оценка Hydrogen :  '[1.00012, 1.00017]'
% мода Hydrogen :  '[1.00013, 1.00015]'
%внешняя оценка Carbon :  '[12.0096, 12.0111]'
%мода Carbon :  '[12.0102, 12.011]'
%T_h= ['[1.00013, 1.00015]', '[1.00012, 1.00017]']
%T_c= ['[12.0102, 12.011]', '[12.0096, 12.0111]']
%T=T_c+4*T_h= ['[16.0109, 16.0115]', '[16.0101, 16.0118]']

\begin{frame}
	\frametitle{Расчёт молекулярной массы молекулы метана в виде твина}
Твины атомных весов водорода и углерода
\begin{align}
	\mbf{M}(H) = &[ \, [1.00013, 1.00015] \, ,  \, [1.00012, 1.00017] \, ], \\
	\mbf{M}(C) = & [ \, [12.0102, 12.011] \, ,  \, [12.0096, 12.0111] \, ]
\end{align}	
Твин веса молекулы метана $CH_4$
	\begin{equation}
		\mbf{M} (CH_4) = [ \, [16.0109, 16.0115] \, ,  \, [16.0101, 16.0118] \, ].
	\end{equation}	
Графическое представление твина веса молекулы метана $\mbf{M} (CH_4)$
	\begin{figure}[ht] 
	\begin{center}
		\unitlength=1mm
  \begin{picture}(90,10)
	\put(10,0){\includegraphics[width=60mm]{Figures/TwinCH4.png}}
	\put(45,5){\mbox{\small $X^{in}$}} 
	\put(35,4){\vector(1,0){28}}
	\put(57,4){\vector(-1,0){28}}
	\put(10,-1){\vector(1,0){60}}
	\put(60,-1){\vector(-1,0){50}}
	\put(30,-5){\mbox{\small $X^{out}$}}
\end{picture}	%	{\includegraphics[width=60mm]{Figures/TwinCH4.png}} 
%		\caption{Твин молекулы метана $CH_4$} 
		\label{f:TwinCH4}
	\end{center}	
\end{figure}	
\end{frame}

\begin{frame}
\frametitle{Мультиинтревалы --- пример технеция }
Мультиинтревалы стабильных изтопов молибдена и рутения и наиболее долгоживущий изотоп технеция

\begin{equation*}
\begin{aligned}
\mbf{M}_{Mo} & = [92, [94,& ~{\color{red}98}]&, 100 \ ], \\
\mbf{M}_{Tc} &=  &[ {\color{red}98} &  ], \\
\mbf{M}_{Ru} &= [ 96,   & [{\color{red}98}&, 102], 104 \ ].
\end{aligned}
\end{equation*}
Минимум по включению мультиинтревалов стабильных изтопов молибдена и рутения содержит значение наилучшего кандидата для технеция $_{43}Tc^{55}$
\begin{equation*}
\mbf{M}_{Mo} \cap \mbf{M}_{Ru} = [96, {\color{red}98}] \supseteq \mbf{M}_{Tc}.
\end{equation*}

\end{frame}

\begin{frame}
\frametitle{Мультиинтревалы --- пример технеция }
\begin{figure}[ht] 
	\begin{center}
		\unitlength=1mm
		{\includegraphics[width=100mm]{Figures/MoTcRuNZmarked.png}}
	\end{center}	
\end{figure}	

\end{frame}


\begin{frame}
\frametitle{Программное обеспечение для анализа данных с интервальной неопределённостью}

С.И.Жилин \\
Matlab/Octave/Scilab:
https://github.com/szhilin/octave-interval-examples\\
https://github.com/szhilin/kinterval \\
Julia
https://github.com/szhilin/julia-interval-examples



\medskip
Python: \\
базовая библиотека\\
{\tt intvalpy} --- А.Андросов https://github.com/AndrosovAS/intvalpy, \\
арифметика твинов Нестерова\\
{\tt twin} --- А.Жаворонкова https://github.com/Zhavoronkova-Alina/twin\\ 
Вычисления с изотопами\\
{\tt MendeleevTwin} --- Т.Яворук https://github.com/Tatiana655/MendeleevTwin



\end{frame}

