\documentclass[a5paper,openany]{book}


%\input{BookStyle.tex}
\usepackage{cmap}  
\usepackage[utf8]{inputenc}
\usepackage[T2A]{fontenc} 
\usepackage[russian]{babel} 
\usepackage{amsmath,amssymb} 
\usepackage{euscript,upref}  
\usepackage{array,longtable}
\usepackage{indentfirst} 
\usepackage{graphicx} 
\usepackage{stmaryrd} 
\usepackage[justification=centering]{caption}
\usepackage{calrsfs} 
\usepackage{url}
%\usepackage{index}
\usepackage{imakeidx} 
\usepackage{multirow,makecell,array}
%\usepackage{setspace} 
%\usepackage{calligra}
\usepackage{pgf,tikz}
\usepackage{pgfplots}
\usepackage{pgfplotstable}
\usepackage{subcaption}
\usepackage{ifthen}
\usepackage{subfiles}
%\usepackage{hyperref}

\usetikzlibrary{arrows.meta}
%\pgfplotsset{compat=1.7}
\pgfplotsset{compat=newest}

%%%%%%%%%%%%%%%%%%%%%%%%%%%%%%%%%%%%%%%%%%%%%%%%%%%%%%%%%%%%%%%%%%%%%%%%%%%%%%%%%%%%%%%%  
% URL проекта - https://ru.overleaf.com/project/5e954c887ac0ac0001d54ece   
%%%%%%%%%%%%%%%%%%%%%%%%%%%%%%%%%%%%%%%%%%%%%%%%%%%%%%%%%%%%%%%%%%%%%%%%%%%%%%%%%%%%%%%%
%234567890123456789012345678901234567890123456789012345678901234567890123456789012345678
%%%%%%%%%%%%%%%%%%%%%%%%%%%%%%%%%%%%%%%%%%%%%%%%%%%%%%%%%%%%%%%%%%%%%%%%%%%%%%%%%%%%%%%%

\textwidth=114truemm
\textheight=165truemm
\oddsidemargin=-1cm
\evensidemargin=\oddsidemargin
\topmargin=-1cm
\sloppy

\pagestyle{plain}
%\mathsurround=1pt
%\tolerance=400
%\hfuzz=2pt
\makeindex

\captionsetup{font=small,labelsep=period,margin=7mm} 

%%%%%%%%%%%%%%%%%%%%%%%%%%%%%%%%%%%%%%%%%%%%%%%%%%%%%%%%%%%%%%%%%%%%%%%%%%%%%%%%%%%%%%%%
% Закомментировать следующую строку, если требуется откомпилировать печатный,
% а не электронный вариант книги (в электронном варианте в библиографии добавляются DOI)
\newcommand{\electronicbook}{}%

%%%%%%%%%%%%%%%%%%%%%%%%%%%%%%%%%%%%%%%%%%%%%%%%%%%%%%%%%%%%%%%%%%%%%%%%%%%%%%%%%%%%%%%%
%
%           Определения новых команд и макросов
%    
%\DeclareMathAlphabet{\mathcalligra}{T1}{calligra}{m}{n}
%\DeclareFontShape{T1}{calligra}{m}{n}{<->s*[1.8]callig15}{}
\newcommand{\mbf}[1]{\protect\text{\boldmath$#1$}}
\newcommand{\mbb}{\mathbb}
\newcommand{\mrm}{\mathrm}
\newcommand{\mcl}{\mathcal}
\newcommand{\msf}{\mathsf}
\newcommand{\eus}{\EuScript}
\newcommand{\ov}{\overline}
\newcommand{\un}{\underline}
\newcommand{\m}{\mathrm{mid}\;}
\newcommand{\w}{\mathrm{wid}\;} 
\newcommand{\Uni}{\mathrm{Uni}\,} 
\newcommand{\Tol}{\mathrm{Tol}\,} 
\newcommand{\Uss}{\mathrm{Uss}\,} 
\newcommand{\Ab}{(\mbf{A}, \mbf{b})}
\newcommand{\Arg}{\mathrm{Arg}\;} 
\newcommand{\sgn}{\mathrm{sgn}\;} 
\newcommand{\ran}{\mathrm{ran}\,} 
\newcommand{\rer}{\mathrm{rer}\:} 
\newcommand{\pro}{\mathrm{pro}\,} 
\newcommand{\dom}{\mathrm{dom}\,} 
\newcommand{\SEV}{\mathrm{SEV}\,} 
\newcommand{\WEV}{\mathrm{WEV}\,} 
\newcommand{\Rsv}{\mathrm{Rsv}\,} 
\newcommand{\calX}{\mathrsfs{X}} 
\newcommand{\cond}{\mathrm{cond}} 
\newcommand{\mode}{\mathrm{mode}\,} 
\newcommand{\dual}{\mathrm{dual}\,} 
\newcommand{\dist}{\mathrm{dist}\,} 
\newcommand{\Dist}{\mathrm{Dist}\,} 
\newcommand{\Ji}{\textsl{Ji}\,}
\newcommand{\const}{\mathrm{const}} 
\newcommand{\USS}{\varXi_\mathit{\hspace{-0.5pt}uni}} 
\newcommand{\TSS}{\varXi_\mathit{\hspace{-0.5pt}tol}} 
\newcommand{\NExt}{_{\scalebox{0.57}{$\natural$}}} 
%\newcommand{\ih}{\scalebox{0.67}[0.87]{$\Box$\hspace*{1pt}}} 
\newcommand{\ih}{\scalebox{0.7}[1.0]{$\oblong$}} 
\newcommand{\doi}[1]{
	\ifdefined\electronicbook
	%DOI:\href{http://doi.org/#1}{#1}
	DOI:#1
	\fi}%

\renewcommand{\r}{\mathrm{rad}\;} 
\renewcommand{\vert}{\mathrm{vert}\,} 

%%%%%%%%%%%%%%%%%%%%%%%%%%%%%%%%%%%%%%%%%%%%%%%%%%%%%%%%%%%%%%%%%%%%%%%%%%%%%%%%%%%%%%%%

\renewcommand{\textfraction}{0}
\renewcommand{\topfraction}{1}
\renewcommand{\bottomfraction}{1} 
\renewcommand{\indexname}{Предметный указатель}
\newcommand{\permil}{\ensuremath{{}^\text{o}\mkern-5mu/\mkern-3mu_\text{oo}}}
%%%%%%%%%%%%%%%%%%%%%%%%%%%%%%%%%%%%%%%%%%%%%%%%%%%%%%%%%%%%%%%%%%%%%%%%%%%%%%%%%%%%%%%%
%
%           Определение счётчиков 
%  
\newcounter{DefNum}[section]
\newcounter{ExmpNum}[section]
\renewcommand{\theExmpNum}{\thesection.\arabic{ExmpNum}}
\newcounter{IncluDefi}
\newcounter{IreneExmp} 
\newcounter{BazhenovExmp} 
\newcounter{RadarExmp} 
% \newcounter{ConstExmp} 
% \newcounter{VarExmp} 

%%%%%%%%%%%%%%%%%%%%%%%%%%%%%%%%%%%%%%%%%%%%%%%%%%%%%%%%%%%%%%%%%%%%%%%%%%%%%%%%%%%%%%%%
%
%           Определение необходимых окружений          
%
\newtheorem{definition}{Определение}[section] 
% \newenvironment{example}% 
%   {\par\addvspace{\medskipamount}\addtocounter{ExmpNum}{1} 
%   \noindent\textbf{Пример {\thesection}.\arabic{ExmpNum}.}}% 
%   {\hfill$\blacksquare$\par\medskip} 
\newenvironment{example}% 
{\refstepcounter{ExmpNum}%
	\par\addvspace{\medskipamount} 
	\noindent\textbf{Пример {\theExmpNum}.}
}% 
{\hfill$\blacksquare$\par\medskip} 

%%%%%%%%%%%%%%%%%%%%%%%%%%%%%%%%%%%%%%%%%%%%%%%%%%%%%%%%%%%%%%%%%%%%%%%%%%%%%%%%
%
%                Определения новых цветов
%
\definecolor{MyRed}{rgb}{0.6,0.3,0.1}
\definecolor{MyGreen}{rgb}{0.2,0.6,0.3}
\definecolor{MyBlue}{rgb}{0.3,0.5,0.85}
\definecolor{Blau}{rgb}{0.5,0.5,0.9}
\definecolor{Gray1}{rgb}{0.6,0.6,0.65}
\definecolor{Gray2}{rgb}{0.5,0.55,0.5}
\definecolor{Gray3}{rgb}{0.6,0.55,0.55}



%%%%%%%%%%%%%%%%%%%%%%%%%%%%%%%%%%%%%%%%%%%%%%%%%%%%%%%%%%%%%%%%%%%%%%%%%%%%%%%%%%%%%%%%  
%For contents 
%\renewcommand{\l@section}{\@dottedtocline{1}{0.5em}{1.5em}}
%\renewcommand{\l@subsection}{\@dottedtocline{1}{2.5em}{2.0em}}
%\makeatother
%\setlength{\marginparwidth}{2cm}

%%%%%%%%%%%%%%%%%%%%%%%%%%%%%%%%%%%%%%%%%%%%%%%%%%%%%%%%%%%%%%%%%%%%%%%%%%%%%%%%%%%%%%%%

%\title{Интервальная таблица Менделеева\\* 
%	элементов и изотопов}
%\title{Изотопы и 
	%Ядерная составляющая \\
	%Интервальная 
%	таблица Менделеева%\\* 
%	}

%\author{А.Н.\,Баженов, А.Ю.\,Тельнова}

%%%%%%%%%%%%%%%%%%%%%%%%%%%%%%%%%%%%%%%%%%%%%%%%%%%%%%%%%%%%%%%%%%%%%%%%%%%%%%%%%%%%%%%%

\begin{document}

\begin{center}
	\hfill \break
	Министерство науки и высшего образования  Российской Федерации\\
	%	\hfill \break
	$\ov{~~~~~~~~~~~~~}$\\
	\normalsize{	САНКТ-ПЕТЕРБУРГСКИЙ \\
		ПОЛИТЕХНИЧЕСКИЙ УНИВЕРСИТЕТ ПЕТРА ВЕЛИКОГО}\\ 
	$\ov{~~~~~~~~~~~~~~~~~~~~~~~~~~~~~~~~~~~~~~~~~~~~~~~~~~~~~~~~~~~~~~~~~~~~~~~~~~~~~~~~~~~~~~~~~~~~~~}$\\	
	{Физико-механический институт}\\
	{\small Высшая школа прикладной математики и вычислительной физики}\\
	\hfill \break 		\hfill \break		\hfill \break	
	\Large{\it А.Н.\,Баженов, А.Ю.\,Тельнова\\
		\hfill \break		\hfill \break			\hfill \break		\hfill \break		}
	{\Large	ИЗОТОПЫ И ТАБЛИЦА МЕНДЕЛЕЕВА\\}
	\hfill \break 	\hfill \break	
	\Large{	Учебное пособие	
	}\\
\end{center}
		\hfill \break		\hfill \break	
\begin{center}\Large{Санкт-Петербург \\
		%		\hfill \break
		2024} \end{center}
\thispagestyle{empty} % выключаем отображение номера для этой страницы	
%	\maketitle 
	\newpage
	УДК 544.116.027,  54.027, 54.021, 54.081, 544.58, 539.1, 539.183.2, 546.02%,  519.6%, {\color{red} XXX\ldots}
	%54.027  - Изотопы в целом
	%54.021 -  	Общий состав и формулы. Молекулярные формулы, символы и т.д.
	% 54.081 - Химические единицы и константы
	% 554.58 - Радиохимия. Ядерная химия. Химия радиоактивных нуклидов
	% 539.1 - Ядерная, атомная, молекулярная физика
	% 539.183.2 Масса атома. Изотопы
	% 546.02 Состав. Структура. Изотопы
	% 544.116.027	Общие представления об изотопах
	%  519.6 Вычислительная математика, численный анализ и программирование (машинная математика)
	\\
	
	Аннотация \\
	
	\begin{quote}
		В пособии рассмотрена современная версия таблицы Менделеева с интервальными значениями атомных весов, разработанная Международным союзом теоретической и прикладной химии IUPAC, 
		%Обсуждается широкий круг вопросов, связанных с изотопами. %Даны исторические сведения об атомных весах и изотопах, ядерной физики, науках о Земле. 
		приводятся терминология и изотопные данные.
		
		Дана общая картина <<изотопного океана>> в природе и на Земле. 
		%Предлагается взгляд на элементы таблицы Менделеева с точки зрения изотопов атомных ядер.
		Элементы таблицы Менделеева обсуждаются с точки зрения наличия у них стабильных изотопов и их количества.
		На качественном уровне приводятся пояснения наличия в природе стабильных изотопов. 
		Кратко приведены некоторые сведения из ядерного нуклеосинтеза.
		
		Для различных диапазонов масс атомных элементов (легкие, средние и тяжёлые) очерчены характерные особенности свойств изотопов.
		Обсуждаются мало освещенные в популярном изложении случаи <<отсутствующих на Земле>> изотопов элементов технеция и прометия,
		поясняются случаи существования элементов с одним изотопом и закономерности относительной распространённости элементов с двумя изотопами.
		Для радиогенных изотопов аргона и свинца даны пояснения их генезиса и распространённости на Земле.
		
		Для математического описания и практических вычислений представлены понятия, методы и инструменты анализа данных с интервальной неопределённостью применительно к тематике изотопов.
		Обсуждается цель ближайших исследований --- выяснение условий прослеживаемости изотопных подписей в природных и синтезированных образцах веществ.
		
		Пособие адресовано всем, кто интересуется современным естествознанием в различных областях и применению математики к решению практических задач. В пособии содержится большое количество табличных данных, на основании которых можно проводить вычисления.
	\end{quote}
	
	
	
	
	
	
	%%%%%%%%%%%%%%%%%%%%%%%%%%%%%%%%%%%%%%%%%%%%%%%%%%%%%%%%%%%%%%%%%%%%%%%%%%%%%%%%%%%%%%%%
	
	\tableofcontents      %  Содержание  
	
	\listoffigures
	
	\listoftables
	
	%%%%%%%%%%%%%%%%%%%%%%%%%%%%%%%%%%%%%%%%%%%%%%%%%%%%%%%%%%%%%%%%%%%%%%%%%%%%%%%%%%%%%%%%
	%%%%%%%%%%%%%%%%%%%%%%%%%%%%%%%%%%%%%%%%%%%%%%%%%%%%%%%%%%%%%%%%%%%%%%%%%%%%%%%%%%%%%%%%
	
	\chapter*{Введение}
	\addcontentsline{toc}{chapter}{Введение}   
	
%	\paragraph{Мотивация публикации.}
	
	В пособии рассмотрена современная версия таблицы Менделеева, разработанная Международным союзом теоретической и прикладной химии IUPAC. В настоящее время  атомные веса многих элементов в ней заданы интервальными значениями. Причиной этого является наличие у химических элементов различных изотопов, которые неравномерно распределены в природе вообще, и конкретно на Земле.
		
	%Рассмотрение вопроса проведено с разных позиций. 
	Исходно таблица Менделеева (ТМ) носила <<химический>> характер. Она описывала свойства элементов с точки зрения их способности формировать молекулы, составляющие природные и синтезированные человеком соединения. Классификация элементов по группам и периодам сообразно атомным массам позволила правильно расположить известные элементы в периодическом порядке, уточнить атомные веса, предсказать существование новых ранее неизвестных элементов. \index{ТМ, таблица Менделеева}
	Со временем в таблицу были добавлены отсутствующие элементы и целые группы элементов: благородные газы, редкоземельные элементы и актиноиды. 
	
	С открытием в конце XIX в. явления радиоактивности началось развитие атомной и ядерной физики, что со временем привело к существенной трансформации представлений о природе. Были созданы и развиты модели атома, квантовая механика, теория атомного ядра, современные представления о Вселенной. Атомные массы в ТМ были заменены на электрические заряды ядер. Было выяснено, что атомные массы элементов не есть их базовая характеристика. Напротив, то что казалось фундаментом материи, оказалось группами островков и отдельными островами стабильных изотопов в море возможных ядерных конфигураций.  Изотопные композиции оказались преимущественным способом существования элементов в природе, а атомные веса --- производными величинами.
	
	С выходом человечества в Космос и развитием новых методов анализа состава вещества, их применения в науках о Земле и биологии, созданием планетной и звездной космохимии и астробиологии, изотопной планетологии, изучения элементного и изотопного состава звёзд, ситуация с атомными весами в таблице Менделеева ещё раз изменилась.  
	
	Теперь следует говорить о  таблице Менделеева как способе описания элементов на конкретной планете Солнечной системы Земле с её специфической геологической и биологической историей, с взаимодействием с Космосом в конкретном месте Галактики и в текущее время планетной эволюции.
	На других планетах и иных космических объектах, на других этапах их геологического развития и этапах существования специфических форм жизни, наличие и распространённость изотопов различных элементов другие. 
	
	Вследствие различия распространённости изотопов, различаются и атомные веса элементов. %Такое положение вещей необходимо осознавать, изучать и применять. 
	Для первого знакомства с предметом в пособии приводятся необходимые сведения по  ядерной физике и нуклеосинтезу, об изотопных распределениях в живой и неживой природе на Земле.
	
	С другой стороны, в фундаментальных и прикладных применениях изотопных данных уже давно существует потребность в их удобном представлении и обработке. Специфика изотопных данных такова, что популярные методы, основанные на теоретико-вероятностном подходе, далеко не всегда адекватны. В пособии представлен подход с интервальным описанием данных в виде специальных объектов --- твинов. Формулируются вопросы по способам представления данных и выбору способов их обоработки с целью сохранения изотопных особенностей, важных для предметных областей.
	
	
%	\paragraph{Про таблицу Менделеева.}
	Обратимся к истории атомных весов.
	В середине XIX века многие исследователи пытались найти закономерности химических свойств элементов. Наиболее удачной оказалась система Д.И.Менделеева. Его пионерская статья <<Соотношение свойств с атомным весом элементов>> была опубликована в Журнале Русского Химического Общества в 1869 г. \cite{Mendeleev1869ru}. В то же время таблица была аннотирована в Германии \cite{Mendeleev1869} и стала достоянием научной общественности. В дальнейшем Менделеев развил свои идеи и дал название предложенной системе  \emph{Периодический закон} \cite{Mendeleev1870}. \index{Менделеев} \index{Периодический закон}
	
	Истории создания Периодической таблицы элементов посвящена обширная библиография. В книгах \cite{Lisnevsky1984, Trifonov1974, Scerri2019} дано краткое изложение основных вех этого процесса, а также освещены экспериментальные исследования и создание теории электронного строения атомов и ядер химических элементов, которые привели к созданию науки об изотопах. \index{Периодическая таблица элементов} В настоящее время идёт процесс более подробного представления этой системы, которая теперь называется \emph{Периодическая таблица элементов и изотопов}.
	\index{Периодическая таблица элементов и изотопов} Тем самым привычная таблица Менделеева дополняется ещё одним измерением --- ядерным. \index{изотопы} 
	
	Это измерение, привычное для ядерных физиков, а также для специалистов в ряде дисциплин геологии, палеонтологии и биологии, ещё  не является общеизвестным для более широкой научной общественности. Есть потребность в разнообразной литературе для популяризации терминологии, понятийной базе, методологии и развитии прикладных программных средств обработки данных, без чего немыслима работа современного исследователя.
	
	В пособии дана общая картина <<изотопного океана>> в природе и на Земле. 
	%Предлагается взгляд на элементы таблицы Менделеева с точки зрения изотопов атомных ядер.
	Элементы таблицы Менделеева обсуждаются с точки зрения наличия у них стабильных изотопов и их количества.
	Для радиогенных изотопов аргона и свинца даны пояснения их генезиса и распространённости на Земле.
	Приводится большое количество иллюстраций, как заимствованных, так и построенных на основе изотопных данных, находящихся с открытом доступе.
	
%	\paragraph{Состав пособия.}	
	Кратко обсудим состав пособия. Оно состоит из 5 содержательных глав, главы с табличными данными и Заключения.
	
	В первой главе даются сведения из ядерной физики, необходимые для понимания изотопной тематики. Обсуждается представление о изотопах химических элементах как представителям некоторого нуклонного сообщества (<<океан изотопов>>).
	
	Во второй главе приводятся сведения об изотопных распределениях на Земле.
	На качественном уровне приводятся пояснения наличия в природе стабильных изотопов. Для различных диапазонов масс атомных элементов (легкие, средние и тяжёлые) очерчены характерные свойства свойств изотопов.
	Обсуждаются мало освещенные в популярном изложении случаи <<отсутствующих на Земле>> изотопов элементов технеция и прометия,
	поясняются случаи существования элементов с одним изотопом и закономерности относительной распространённости элементов с двумя изотопами.
	
	В третьей главе
	рассмотрена современная версия таблицы Менделеева с интервальными значениями атомных весов, разработанная Международным союзом теоретической и прикладной химии IUPAC и таблица стандартных атомных весов.
	
	В четвертой главе обсуждаются данные данные IUPAC-2021, оносящиеся к <<особым элементам>> на Земле, аргону и свинцу. Подавляющая часть изотопов этих элементов не присутствовала на нашей планете в период её формирования, а возникла в ходе ядернофизических процессов в течение жизни нашей планеты.
	
%	Третья группа глав относится к анализу данных.
	В пятой главе кратко обсуждаются понятия, методы и инструменты анализа данных с интервальной неопределённостью применительно к тематике изотопов.	Приводятся примеры интервальных вычислений со специальными математическими объектами --- твинами, ставится вопрос, в каких видах вычислений, и с какого рода данными возможно наиболее содержательное использование изотопной информации.
%	В последней главе приводятся справочные данные IUPAC, которые еще не получили окончательного оформления. 
%Также обсуждается арифметика твинов.
	Шестая и седьмая главы содержат табличные данные. 
	
	В Заключении кратко освещается место публикации в ряду других материалов и ставятся задачи на следующие публикации. 
	
	\chapter{Изотопы элементов в ядерной физике} \label{IsotopesNature}
	
	
	
	В настоящей главе даётся информация, необходимая для понимания природы изотопов.
	Сначала приводятся краткие сведения по ядерной физике.
	Далее рассматривается вопрос стабильности атомных ядер по отношению к их различным превращениям.
	Очень кратко обсуждаются некоторые вопросы нуклеосинтеза ядер в ходе эволюции Вселенной, предшествовавшему образованию Земли.
	% об изотопных распределениях в живой и неживой природе на Земле, о том как изотопы по-разному фракционируются в различных геологических и биологических процессах. 
	
	
	\section{Сведения из ядерной физики} \label{s:NuclPhys}
	
	\emph{Атомное ядро} \index{ядро атомное}  --- центральная часть атома, в которой сосредоточена основная его масса. Ядро заряжено положительно, заряд ядра определяет \index{химический элемент} \emph{химический элемент}, к которому относят атом. Размеры ядер атомов более чем в 10000 раз меньше размеров самого атома. Атомные ядра изучает \index{ядерная физика} \emph{ядерная физика}.
	
	В этом разделе приводятся сведения из ядерной физики \cite{NuclPhys}, необходимые для понимания специфики изотопных данных.	
	Атомное ядро состоит из \index{нуклон} \emph{нуклонов} --- положительно заряженных протонов и не имеющих заряд нейтронов, которые связаны между собой при помощи \index{сильные взаимодействия}  \emph{сильного взаимодействия}.
	Сильное ядерное взаимодействие ---  одно из четырёх фундаментальных взаимодействий в физике. В сильном взаимодействии участвуют \index{кварки} \emph{кварки} и \emph{глюоны} и составленные из них частицы, называемые адронами.
	
	Кварк --- бесструктурная элементарная частица и фундаментальная составляющая материи. Всё обычно наблюдаемое вещество состоит из двух видов кварков (u и d) \index{u-кварк} \index{d-кварк}  и электронов. Кварки имеют дробный электрический заряд. Протон состоит из двух u-кварков и одного d-кварка, а нейтрон --- из одного u-кварка и двух d-кварков. 
	На рис.~\ref{f:N_P} представлена кварковая структура нейтрона и протона.
	\begin{figure}[ht] 
		\centering\small
		\unitlength=1mm
		\begin{picture}(120,22)
			\put(30,0){\includegraphics[width=25mm]{Figures/Neutron_quark_structure.png}}
			\put(60,0){\includegraphics[width=25mm]{Figures/Proton_quark_structure.png}}
		\end{picture}
		\caption{Кварковая структура нейтрона  (udd) и протона (uud)} 
		\label{f:N_P}
	\end{figure}
	
	Глюон --- элементарная безмассовая частица, квант векторного поля, переносчик сильного взаимодействия.  Кварки и нуклоны имеют полуцелый спин и по статистическим свойствам относятся к \index{фермионы} \emph{фермионам}. Глюоны, подобно фотонам, имеют целый спин и являются \index{бозоны} \emph{бозонами}.
	
	Нейтрон и протон могут переходить друг в друга посредством обмена $\pi$-\emph{мезонами}. Мезоны \index{мезоны} ---  составные элементарные частицы, состоящие из равного числа кварков и антикварков.
	\begin{figure}[ht] 
		\centering\small
		\unitlength=1mm
		\begin{picture}(120,30)
		\put(2,5){\includegraphics[width=50mm]{Figures/Pn_scatter_pi0.png}}
		\put(60,0){\includegraphics[width=50mm]{Figures/Pn_Scatter_Quarks.png}}
		\end{picture}
		\caption{Рассеяние нейтрона и протона с обменом $\pi$-мезоном} 
		\label{f:Pn_scatter_pi0}
	\end{figure}	
	
	На рис.~\ref{f:Pn_scatter_pi0} приведена \index{диаграмма Фейнмана} \emph{диаграмма Фейнмана}, иллюстрирующая рассеяние нейтрона и протона с обменом $\pi$-мезоном. 
	
	С точки зрения сильного взаимодействия, протон и нейтрон являются одинаковыми частицами, а многие другие свойства у них также близки. Поэтому была разработана модель, по которой любой нуклон обладает \index{изотопический спин} \emph{изотопическим спином}, равным 1/2, у которого есть две возможные «проекции» в особом изотопическом пространстве. Когда проекция изотопического спина $I_z$ равна +1/2, то нуклон становится протоном, а когда -1/2 --- нейтроном.
	
	Для не слишком тяжёлых ядер, когда кулоновское отталкивание протонов относительно невелико, атомное ядро можно считать коллективным состоянием одинаковых частиц, отличающихся только изотопическим спином.
	
	\subsection{Экспериментальные факты про ядра, нуклоны, изотопы} \label{ExpNuclData}
	
	Актуальным источником данных является база данных МАГАТЭ NUBASE2020 \cite{NUBASE2020}.
	Интерактивные модели различных свойств данных размещены на сайте МАГАТЭ  \cite{IAEA} и на других ресурсах, например,  https://www.nndc.bnl.gov/ensdf/ и https://www.nndc.bnl.gov/nudat3/.	
	В статье \cite{Smith2023} подробно описаны источники данных для ядерной физики и ядерного нуклеосинтеза. 
	% Evaluated Nuclear Structure Data File
 Описание формата данных {\tt ENSDF} \cite{ENSDF}. % http://cdfe.sinp.msu.ru/services/ensdfr/ensdfhelp\_ru/ENSDF.html
	% Автор перевода: Шуляк Георгий Иванович. 	Издательство: ПИЯФ РАН, Гатчина, 2006.
	
	Далее описан способ получения информации о ядерных данных.	
	
	\paragraph{$NZ$-диаграммы.} Атомные ядра состоят из протонов и нейтронов. Принято обозначать количество протонов в ядре как $Z$, это также соотвествует номеру химического элемента в периодической таблице. Число нейтронов обозначают как $N$. Для представления свойств атомных ядер элементов и изотопов используют так называемые $N-Z$ диаграммы (диаграммы Сегре), на которых по осям \index{NZ-диаграммы, диаграммы Сегре} отложены $Z$ и $N$, а интересующая величина отображается в первом ортанте. 
	
	Пример $N-Z$ диаграммы приведен на рис.~\ref{f:NZHalf-time}. На данном графике строки  соответствуют ядрам с одинаковым числом протонов, а столбцы – ядрам с одинаковым числом нейтронов.	
	
	%Таблица свойств ядер --- https://www-nds.iaea.org/amdc/ame2020/mass_1.mas20.txt
	
	
	
	\begin{figure}[ht] 
		\centering\small
		\unitlength=1mm
		\includegraphics[width=100mm]{Figures/NZHalf-time.png} 
		\caption{$N-Z$ диаграмма  атомных ядер} По данным https://www.nndc.bnl.gov/nudat3/ \cite{NUDAT3}
		\label{f:NZHalf-time}
	\end{figure}
	Черным цветом на рис.~\ref{f:NZHalf-time} выделены стабильные ядра, так называемая долина стабильности атомных ядер.  \index{долина стабильности атомных ядер}
	Справа от долины стабильности располагаются ядра, испытывающие $\beta^{-}$-распад, слева --- ядра, испытывающие $\beta^{+}$-распад и $e^{-}$-захват. В области больших $A = N+Z$ находятся ядра, испытывающие $\alpha$-распад, и спонтанно делящиеся ядра. 
	
	Линия $B_p = 0$ (proton drip-line) ограничивает	область существования атомных ядер слева,
	то есть невозможно существование ядер с б\'{о}льшим числом протонов.
	Линия, ограничивающее число нейтронов, $B_n = 0$ (neutron drip-line) находится справа.
	
	На Рис.~\ref{f:NZHalf-time} большинство элементов лежит ниже прямой $N=Z$. Это соответствует тому, что протоны испытывают электростатическое отталкивание, а нейтроны --- нет.
	Также на Рис.~\ref{f:NZHalf-time} выделены ядра содержащие 2, 8, 20, 28, 50, 82, 126 нуклонов. Экспериментальные исследования показали, что данные ядра являются наиболее устойчивыми, за что они были названы «магическими». \index{ядра магические} 	\index{ядра чётно-чётные} 	\index{ядра чётно-нечётные} 	\index{ядра нечётно-нечётные}
	На устойчивость ядер влияет и четное или нечетное количество нуклонов. Так, было установлено, что чётно-чётные ядра (имеющие чётное количество как протонов, так и нейтронов) являются более устойчивыми, чем чётно-нечётные и нечётно-нечётные ядра.  

	\paragraph{$N-A$-диаграммы.}	\index{NA-диаграммы}
Представление свойств атомных ядер в координатах $(N,Z)$ не единственно. В тех случаях, когда есть привязка к атомному весу изотопа, удобнее представлять данные в в координатах $(N,A)$. 
Дополнительным преимуществом  $(N,A)$ координат является  тот факт, что превращения ядер, связанные с излучением или захватом электрона, в этих координатах происходит c изменением $Z$ и без изменения значения $A$. В то же время, в координатах  $(N,Z)$ при этом изменяется также и значение $N$. Например, см. рис.\ref{f:ThLead} в \S\ref{PbIsotopesOrigin} <<Происхождение изотопов свинца>>.

 Для того, чтобы использовать данные для проведения расчётов и построения графиков, можно эти данные выгрузить с ресурса https://www.nndc.bnl.gov/nudat3/. Он  предоставляет возможность выгрузки данных в csv-формате. 
	
	Например, для выгруки величин моды распада, распространённости изотопов и энергий, высвобождаемых в процессах бета-распада 
	и электронного  захвата, надо выбрать в интерфейсе поля базы данных Abudаnce, Deacay Mode, Q$\beta$-, QEC,  как это показано на рис.~\ref{f:Nudata3csvExport}
	\begin{figure}[ht] 
		\centering\small
			\unitlength=1mm
			\begin{picture}(120,55)
			\put(15,0){\includegraphics[height=55mm]{Figures/Nudata3csvExportMenu.png}}
			\put(55,0){\includegraphics[height=55mm]{Figures/Nudata3csvExport.png}}
			\end{picture}
			\caption{Выгрузка данных в формате csv с сайта https://www.nndc.bnl.gov/nudat3/} 
			\label{f:Nudata3csvExport}
	\end{figure}
	
При этом данные экспортируются в csv-файл. В момент экспорта можно выбрать различные возмможности, и в результате формируется текстовый файл. В первой строчке содержится легенда данных. Ниже приведён фрагмент файла.	
	
	\begin{verbatim}
z,n,name,decayModes,abundance,betaMinus(keV),electronCapture(keV)
0,1,Neutron,B- = 100.00,,782.347,
1,0,1H,,99.9885%,,-782.347
1,1,2H,,0.0115%,,
1,2,3H,B- = 100.00,,18.59202,
1,3,4H,N = 100.00,,22200,
1,4,5H,2N = 100.00,,21660,
1,5,6H,N = 100.00,,24300,
1,6,7H,2N?  ,,23100,
...
118,177,295Og,,,
	\end{verbatim}

Всего в выгружаемых файлах содержится 6658 строк, включая строку легенды. Это значение отвечает общему количеству мод распада и превышает число стабильных изотопов.  

Рассмотрим для иллюстрации моды и вероятности распадов лития.  В интерфейсе программы на https://www.nndc.bnl.gov/nudat3/ эти данные выглядят как показано на рис.~\ref{f:LiDecay}.

\begin{figure}[ht] 
	\centering\small
	\unitlength=1mm
		\begin{picture}(120,15)
		\put(0,0){\includegraphics[width=120mm]{Figures/LiDecay.png}}
		\end{picture}
	\caption{Моды распада изотопов лития} 
	\label{f:LiDecay}
\end{figure}

Для изотопа $^5Li$ при выгрузки из базы данных его распаду \index{моды распада изотопов лития}
\begin{equation}\label{5Li}
_3^5Li \longrightarrow _2^4He + _0^1p
\end{equation}
отвечают 2 строки
\begin{verbatim}
	3,2,5Li,A = 100.00,,-25500,450
	3,2,5Li,P = 100.00,,-25500,450
\end{verbatim}		
Обе эти записи отвечают реакции \eqref{5Li}.

Если изотоп имеет более одной моды распада, как например в случае изотопа висмута-212
\begin{align}\
_{83}^{212}Bi & \longrightarrow _{84}^{212}Po + \beta^{-}, \label{212BiPo}\\
_{83}^{212}Bi & \longrightarrow _{81}^{208}Tl + \alpha. \label{212BiTl}
\end{align}
Распадам \eqref{212BiPo} и \eqref{212BiTl} отвечают записи  \index{моды распада изотопа висмута-212}
\begin{verbatim}
83,130,213Bi,B- = 97.80,,1422,-2028
83,130,213Bi,A = 2.20,,1422,-2028
\end{verbatim}		
Вероятность процесса \eqref{212BiPo} составляет  97.8\%, а процесса \eqref{212BiTl} --- 2.2\%.
Реакция \eqref{212BiPo}  будет участвовать как промежуточный агент в ряде тория в \S\ref{PbIsotopesOrigin}. \index{ряд тория}
	
	\subsection{Модели ядра} \label{NucleiModels}
	
	Cвойства ядер весьма сложны. Протоны и нейтроны участвуют во всех видах известных взаимодействий, сильных, слабых, электромагнитных и гравитационных. Помимо того, что взаимодействия нуклонов сложное, их количество в ядрах достаточно велико и имеет место проблема многочастичной системы, которую невозможно описать точно. В то же самое время число нуклонов не так велико, чтобы можно было применять статистические методы.  
	
	Поэтому нет такой теоретической модели ядер, которая бы успешно количественно объясняла все их наблюдаемые свойства.
	Для начального представления о моделях следует упомянуть капельную и оболочечную модели ядра.
	
	Дадим очень краткое изложение основных \index{модели атомных ядер} моделей атомных ядер. 
	
	\paragraph{Капельная модель ядер.}
	Капельную модель ядер Н.Бор предложил в 1936. В данной теории ядро представляется сферической каплей некоторой заряженной несжимаемой сверхплотной жидкости, называемой ядерной. Помимо выше перечисленных свойств, к свойства ядра-капли можно добавить поверхностное натяжение, разделение капли на более мелкие (деление ядер), а также слияние небольших капель в капли побольше (синтез ядер). Основным достижением данной модели стала полуэмпирическая формула Вейцзеккера для энергии связи ядра{Weizsacker}:
	\begin{equation}\label{Weizsacker}
	E={\alpha}A-{\beta}A^{2/3}-{\gamma}\frac{Z^2} {A^{1/2}}-{\epsilon}\frac{(A-2Z)^2} {A} +{\delta}A^{-3/4},
	\end{equation}
	где $\delta$=+|$\delta$|, для четно-четных ядер,  $\delta$=-|$\delta$|, для ядер с нечетным $A$ и $\delta$=0 для нечетно-нечетных ядер. $\alpha$,  $\beta$, $\gamma$, $\delta$ - эмпирические коэффициенты. \index{капельная модель ядер}

	\begin{figure}[ht] 
	\centering\small
	\unitlength=1mm
	\includegraphics[width=80mm]{Figures/Liquid_drop_model.png} 
	\caption{Рисунок, иллюстрирующий ядро как каплю несжимаемой жидкости} с грубым учётом набюдаемых вариаций энергии связи --- https://en.wikipedia.org/wiki/Atomic\_nucleus
	\label{f:Liquid_drop_model}
\end{figure}
	
	
	Удельная энергия связи ядра в зависимости от массового числа $A$ имеет следующий вид:
	\begin{figure}[ht] 
		\centering\small
		\unitlength=1mm
		\includegraphics[width=100mm]{Figures/Weizsacker.png} 
		\caption{Удельная энергия связи ядра, расчитанная по формуле Вейцзеккера} 
		\label{f:Weizsacker}
	\end{figure}

	Рассмотрим подробнее смысл всех слагаемых формулы \eqref{Weizsacker}. Собственно капельной  аналогии соотвествуют первые три члена формулы. Первое слагаемое, пропорциональное массовому числу $A$,  описывает постоянство удельной энергии связи ядер. Второй член учитывает, поправку на уменьшение полной энергии связи, обусловленную тем, что часть нуклонов находится у поверхности ядра, т. е. нуклоны, находящиеся на поверхности имеют меньше связей, чем частицы внутри него. Это поправка на поверхностное натяжение. Третий член определяет взаимное кулоновское расталкивание протонов. В модели предполагается, что электрический заряд протонов равномерно распределен внутри сферы ядра. Данный коэффициент может быть вычислен на основании представления о равномерном распределении электрического заряда по объёму сферы с заданным радиусом. 

	
	Два последних слагаемых  уже не следуют из модели жидкой капли. 
	Четвертое слагаемое учитывает симметрию ядра и отражает тенденцию к стабильности ядер с $N=Z$. Этот член --- поправка на энергию симметрии, отражает наблюдаемую в природе тенденцию к симметрии в строении ядер. Пятое слагаемое --- энергия спаривания, которая учитывает повышенную стабильность основных состояний ядер с чётным числом протонов и/или  нейтронов. Этот член отражает распространенность стабильных элементов и учитывает эффект спаривания одинаковых нуклонов. 
	
	В целом график на рис.~\ref{f:Weizsacker} удовлетворительно отображает общий ход зависимости 
	удельной энергия связи ядер в зависимости от массового числа $A$.
	
	\paragraph{Оболочечная модель ядра.} \label{ShellNucleiModel}
	Однако капельная модель ядра оказалась не универсальной. Накопленный экспериментальный опыт свидетельствовал, что стабильность ядер зависит от количества нуклонов, находящихся в них. \index{оболочечная модель ядра}
	
	Экспериментальные исследования атомных ядер выявили некоторую периодичность в изменении
	индивидуальных характеристик (энергии связи, спины, магнитные моменты, четности, некоторые особенности $\alpha$- и $\beta$-распадов) основных и возбужденных состояний атомных ядер. \index{$\alpha$- и $\beta$-распады}
	Обнаруженная периодичность подобна периодичности свойств электронных оболочек атома и определяется так называемыми \emph{магическими} числами нейтронов и протонов. 
	
	В частности, было обнаружено, что наибольшую энергию связи имеют ядра с \emph{магическими} числами нейтронов и протонов, равными \index{магические числа нейтронов и протонов}
	\begin{align}
	N & \quad 2, 8, 20, 28, 50, 82, 126, 184(?) \label{MagicN} \\
	Z & \quad 2, 8, 20, 28, 50, 82, 114(?) \label{MagicP}
	\end{align}
	
		Особой стабильностью характеризуются так называемые дважды магические ядра, в которых количества и протонов, и нейтронов составляют магические числа. В природе существуют следующие дважды магические ядра: \index{дважды магические ядра}
	\begin{equation*}
	^A_Z Element ^N :  \ ^4_2He^2, \ ^{16}_8O^8, \ ^{40}_{20}Ca^{20}, \ ^{48}_{20}Ca^{28}, \ ^{208}_{82}Pb^{126}.
	\end{equation*}
	
	% Так, оказалось, ядра, содержащие 2, 8, 20, 28, 50, 82, 126 нуклонов являются наиболее устойчивыми, за что были названы «магические». 
	Для объяснения данного факта, капельной модели было уже недостаточно. Точно также капельная модель не могла объяснить, почему, почему при делении ядра разваливаются на два осколка разной массы. Для объяснения этих феноменов была придумана оболочечная модель ядра.
	
	Результатом работы по систематизации и обобщения огромного
	количества экспериментальных данных было создание в середине XX века
	модели оболочек атомных ядер. \index{модель оболочек атомных ядер}  
	Эта модель позволяет не только объяснить многие закономерности свойств ядер, но и предсказывать возможные пути синтеза новых элементов.

 Число нуклонов на ядерных оболочках равно соответственно
\begin{align}
& 2 & = 2&,  \\
& 8 & = 2&+6, \\
& 20 & =2&+6+12, \\
& 28 & =2&+6+12+8,\\
& 50 & = 2&+6+12+8+22 \label{Magic50}\\
& 82 & = 2&+6+12+8+22+32, \label{Magic82}\\
& 126 & =2&+6+12+8+22+32+44,\\
& 184 &= 2&+6+12+8+22+32+44+58.
\end{align}

	\begin{figure}[ht] 
	\centering\small
	\unitlength=1mm
	\includegraphics[width=120mm]{Figures/Shell_gap.png} 
	\caption{Эмпирические протонные и нейтронные уровни оболочек}, полученные из наблюдаемых энергий связи \cite{AME2016}
	\label{f:Shell_gap}
\end{figure}
% Wang Meng et al 2017 Chinese Phys. C 41 030003 The AME2016 atomic mass evaluation (II). Tables, graphs and references. DOI 10.1088/1674-1137/41/3/030003 

На рис.~\ref{f:Shell_gap} приведены эмпирические протонные и нейтронные уровни оболочек, полученные из наблюдаемых энергий связи \cite{AME2016}, а также прямые $N=Z$. Расчёты соответствуют формулам
\begin{align}
\Delta_{2p}(N,Z) = E_{N,Z+2} + E_{N,Z-2} -2E_{N,Z},  \\
\Delta_{2n}(N,Z) = E_{N+2,Z} + E_{N-2,Z} -2E_{N,Z}.
\end{align}	
	
	
	Оболочечная модель ядра является аналогией к модели электронных оболочек атома. Согласно этой модели, нуклоны в ядре расположены на оболочках, причём на каждой из них может находиться лишь определенное число нуклонов. Каждый энергетический уровень заполняется нуклонами независимо друг от друга. Каждый нуклон в ядре находится в определённом  квантовом состоянии в соответствии с принципом Паули. \index{принцип Паули} Тогда ядра атомов с полностью заполненными оболочками, будут иметь повышенную устойчивость. Таким образом модель оболочек позволяет объяснить существование магических ядер. \index{ядра магические} 
	Тоже самое касается и деления ядер на неравные осколки: ядро стремится разрушиться так, чтобы ядра-осколки оказались с заполненными оболочками.

	\begin{figure}[ht] 
	\centering\small
	\unitlength=1mm
	\includegraphics[width=90mm]{Figures/NZstateE1.png} 
	\caption{Энергия первого возбуждённого  состояния ядра в координатах (N, Z)} --- https://www.nndc.bnl.gov/nudat3/
	\label{f:NZstateE1}
\end{figure}

На рис.~\ref{f:NZstateE1} показаны энергии первого возбуждённого  состояния ядер в координатах (N, Z). График показывает, что в области магических чисел величина $E_{1ext}$ резко возрастает.


\begin{figure}[ht] 
	\centering\small
	\unitlength=1mm
	\begin{picture}(130,60)
	\put(0,0){\includegraphics[width=120mm]{Figures/E1ext Z=40-92 A=100-218.png} }
		\end{picture}
	\caption{Энергия первого возбуждённого  состояния ядра Sn---Pb} --- по данным https://www.nndc.bnl.gov/nudat3/
	\label{f:NZstateE1SnPb}
\end{figure}

На рис.~\ref{f:NZstateE1SnPb} показаны энергии первого возбуждённого состояния ядер в координатах (N, Z) для диапазонов A = 100---218 и Z = 40---92.
Резкие пики имеют место для магических ядер $_{50}^{132}Sn$ и $_{82}^{208}Pb$.

\begin{figure}[ht] 
	\centering\small
	\unitlength=1mm
	\begin{picture}(130,45)
	\put(10,0){\includegraphics[width=100mm]{Figures/E1ext (keV) Z=50-82 A=100-218.png} }
	\end{picture}
	\caption{Энергия первого возбуждённого  состояния ядра} --- по данным https://www.nndc.bnl.gov/nudat3/
	\label{f:NZstateE1SnPb1d}
\end{figure}
	
На одномерных графиках рис.~\ref{f:NZstateE1SnPb1d}  помимо острых пиков при Z = 50 и Z = 82
хорошо видна тенденция к более выгодным энергетическим состояниям для чётных значений A.	
	
	\paragraph{Обобщённая модель ядра.} Ряд ядер и их возбуждёных состояний не описываются удовлетворительно ни капельной, ни оболочечной моделями ядра. \index{обобщённая модель ядра}
	Обобщённая модель ядра учитывает и коллективные эффекты движений многих нуклонов, и одночастичные движения отдельных нуклонов в общем поле, создаваемом другими нуклонами. Таким образом удалось описать несферичность ядер и колебания формы ядра вокруг равновесной. 
	
	В целом спектроскопия возбуждённых состояний ядер и другие эксперименты свидетельствуют о большом количестве сложных эффектов в ядрах.
	
	
	
	\subsection{Изобары, изотопы} \label{IsobarsIsotopes}
	Атомные ядра, содержащие одинаковое число протонов и различное число нейтронов (то есть отличаются по массе, но не по заряду) называют \emph{изотопы}. Существуют как устойчивые (стабильные), так и радиоактивные изотопы. Ядра с различным числом протонов и нейтронов, но с одинаковым суммарным числом нуклонов называют \emph{изобарами}. Также, в ядерной физике применяется понятие \emph{изотоны} для обозначения ядер с одинаковым числом нейтронов и разным числом протонов.
	\index{изотопы}   \index{изотоны}
	\index{изобары}
	
	Что касается распространённости изотопов в природе, то имеется ряд закономерностей \cite{Bekman}.
	Оболочечная модель согласуется с эмпирическим правилом повышения устойчивости ядер,
	содержащих чётное число протонов и нейтронов --- Табл.~\ref{t:IsotopeOddEven}. Менее стабильны ядра с нечётным числом протонов, но
	чётным числом нейтронов (и наоборот). Неустойчивыми являются ядра, в которых число протонов и
	нейтронов нечётное. Последняя закономерность хорошо иллюстрируется распределением естественных
	известных природных изотопов по правилу чётности. 
	
	\begin{table}[!h]
		{\small 
			\begin{center}
				\begin{tabular}{|c|c|c|c|c|}
					\hline
					Число изотопов & N & Z & Примеры Табл.~\ref{t:IsotopeCounts}\\
					\hline
					166 & чётное & чётное & ~ \\
					55 & чётное & нечётное & 1 изотоп --- $_{9}F^{19} \ldots \, _{83}Bi^{109} $ \\
					~ & ~ & ~ &  2 изотопа --- $_{1}H^{1, 2}  \ldots \, _{81}Tl^{203, 205}   $ \\
					47 & нечётное & чётное & ~  \\
					5 & нечётное & нечётное & $_{1}H^{2},   \,  _{3}Li^{6},  \,  _{5}B^{10},   \, _{7}N^{14},  \,  _{47}Ag^{109} $   \\
					\hline
				\end{tabular}
			\end{center}	
		}	
		\caption{Распределение естественных изотопов по правилу чётности нуклонов}
		\label{t:IsotopeOddEven}
	\end{table}
	
	Для 24 элементов известно лишь по одному устойчивому изотопу. Такие элементы называют
	моноизотопными. Из них 23 элемента - с нечётными $Z$. 
	У других элементов, преимущественно с чётными порядковыми номерами, число устойчивых изотопов доходит до 10 --- Табл.~\ref{t:IsotopeCounts}. \index{моноизотопные элементы}
	
	В \cite{IsotopeGeoChem} в удобной приведены данные о количестве стабильных изотопов различных элементов на Земле. Приведём часть данных из этой публикации.
	\begin{table}
		{\scriptsize 
			\begin{tabular}{cp{6cm}cc}
				Число  & Элементы, $ \ _{Z}E^{N_1, N_2, \ldots, N_k}$, $z \ $ --- заряд ядра,  &  Число  & Общее  \\
				стабильных  & $N_j$ --- атомный вес изотопа $j$ &  элементов & число  \\
				изотопов, $k$ & ~ &  ~ &  изотопов \\
				\hline 
				~ & ~ &  ~ &  ~\\
				1 &  $_{4}Be^{9}, \, _{9}F^{19}, \, _{11}Na^{23}, \, _{13}Al^{27}, \, _{15}P^{31}, \,  _{21}Sc^{45},  \,  _{23}V^{51},  $ & 24 & 24 \\ [1mm]
				~ &  $_{25}Mn^{55}, \, _{27}Co^{59}, \, _{33}As^{75}, \, _{39}Y^{89}, \, _{41}Nb^{93}, \,  _{45}Rh^{103},  $ & ~  & ~ \\ [1mm]
				~ &  $ _{53}I^{127}, \, _{55}Cs^{133}, \, _{57}La^{139}, \, _{59}Pr^{141}, \, _{65}Tb^{159},  \, _{67}Ho^{165}, $ & ~  & ~ \\ [1mm]
				~ &  $ _{69}Tm^{169}, \, _{71}Lu^{175}, \, _{75}Ta^{181}, \, _{79}Au^{197}, \, _{83}Bi^{109}  $ & ~  & ~ \\ [3mm]
				%\hline
				%\vspace{3mm}
				2 &  $_{1}H^{1, 2},  \,  _{2}He^{3, 4},  \,  _{3}Li^{6, 7},  \,  _{5}B^{10, 11},  \, _{6}C^{12,13},  \, _{7}N^{14,15}, $ & 19 & 38 \\ [1mm]
				~ &  $_{17}Cl^{35, 37},  \,  _{19}K^{39, 41},  \,  _{29}Cu^{63, 65},  \,  _{31}Ga^{69, 71},  \, _{35}Br^{79,81},  $ & ~ & ~\\ [1mm]
				~ &  $_{37}Rb^{85, 87},  \,  _{47}Ag^{107, 109},  \,  _{49}In^{113, 115},  \,  _{63}Eu^{151, 153},    $ & ~ & ~\\ [1mm]
				~ &  $ _{76}Re^{185, 187}, \,  _{77}Ir^{191, 193},  \,  _{81}Tl^{203, 205} $ & ~ & ~ \\ [1mm]
				3 &  $_{8}O^{16, 17, 18}, \, _{10}Ne^{20, 21, 22}, \,  _{12}Mg^{24, 25, 26} $ & 6 & 18 \\ [1mm]
				~ &  $_{14}Si^{28, 29, 30}, \, _{16}Ar^{36, 38, 40}, \,  _{58}Ce^{138, 140, 142} $ & ~ & ~ \\ [1mm]
				4 &  $_{16}S^{32, 33, 34, 36}, \, _{24}Cr^{50, 52, 53, 54}, \,  _{28}Fe^{54, 56, 57, 58} $ & 5 & 20 \\ [1mm]
				~ &  $_{38}Sr^{84, 86, 87, 88}, \,  _{82}Pb^{204, 206, 207, 208} $ & ~ & ~ \\ [1mm]
				5 &  $_{22}Ti^{46-50}, \, _{28}Ni^{58, 60, 61, 62, 64}, \, _{30}Zn^{64, 66, 67, 68, 70} $ & 6 & 30 \\ [1mm]
				~ &  $_{32}Ge^{70, 72, 73, 74, 76}, \, _{40}Zr^{90, 91, 92, 94, 96}, $ & ~ & ~ \\ [1mm]
				~ &  $_{74}W^{180, 182, 183, 184, 186} $ & ~ & ~ \\ [1mm]
				6 &  $_{20}Ca^{40, 42, 43, 44, 46, 48}, \, _{34}Se^{74, 76, 77, 78, 80, 82} $ & 7 & 42 \\ [1mm]
				~ &  $_{36}Kr^{78, 80, 82, 83, 84, 86},  \,_{46}Pd^{102, 104, 105, 106, 108, 110} $ & ~ & ~ \\ [1mm]
				~ &  $_{68}Er^{162, 164, 166, 167, 168, 170},  \, _{72}Hf^{174, 176, 177, 178, 179, 180} $ & ~ & ~ \\ [1mm]
				~ &  $_{78}Pt^{190, 192, 194, 195, 196, 198} $ & ~ & ~ \\ [1mm]
				7 &  $_{42}Mo^{92, 94-98, 100}, \, _{44}Ru^{96, 98-102, 104}, \, $ & 10 & 70 \\ [1mm]
				~ &  $_{56}Ba^{130, 132, 134-138}, \, _{60}Nd^{142-146, 148, 150}, $ & ~ & ~ \\ [1mm]
				~ &  $_{62}Sm^{144, 147-150, 152, 154}, \, _{64}Gd^{152, 154-158, 160},$ & ~ & ~ \\ [1mm]
				~ &  $_{66}Dy^{156, 158, 160-164}, \, _{70}Yb^{168, 170-174, 176} $ & ~ & ~ \\ [1mm]
				~ &  $_{76}Os^{184, 186-190, 192}, \, _{80}Hg^{196, 198-202, 204} $ & ~ & ~ \\ [1mm]
				8 &  $_{48}Cd^{106, 108,  110-114}, \, _{52}Te^{120, 122, 123-126, 128, 130}$ & 2 & 16 \\ [1mm]
				9 &  $_{54}Xe^{124, 126, 128-132, 134, 136}$ & 1 & 9 \\ [1mm]
				10 &  $_{50}Sn^{112, 114-120, 122, 124}$ & 1 & 10 \\ [1mm]
				\hline
				~ & ~ &  ~ &  ~\\
				~ & ~ & 81 & 276
			\end{tabular}
		}
		\caption{Количество стабильных изотопов различных элементов на Земле}
		\label{t:IsotopeCounts}
	\end{table} 
	
	
	%\input{IsotopeCounts.tex} 
В табл.~\ref{t:IsotopeCounts} приведены данные о количестве стабильных элементов для различных химических элементов. Она организована по блокам следующим образом.
В 10 горизонтальных блоках сгруппированы элементы, имеющие одинаковое количество стабильных изотопов, от одного до 10.

Система обозначения следующая. Изотопы элемента  обозначены как $ \ _{Z}E^{N_1, N_2, \ldots, N_k}$, $z \ $ --- заряд ядра.
Величины $N_j=n_j+z$ указаны в верхнем индексе, электрический заряд $z$ --- в нижнем индексе.
Число нейтронов $n_j$ включено в $N_j$.
Массив $ \{ N_j \}$ дан или перечислением всех изотопов ($ _{82}Pb^{204, 206, 207, 208}$) или, в случае большого количества изотопов с непрерывно идущими номерами, с указанием диапазона атомных весов ($_{50}Sn^{112, 114-120, 122, 124}$).

На рис.~\ref{f:HIST Isotopes_per_Elements} приведёна гистограмма числа изотопов химических элементов в коре Земли. 

\begin{figure}[ht] 
	\centering\small
	\unitlength=1mm
	\begin{picture}(120,40)
	\put(30,0){\includegraphics[width=50mm]{Figures/HIST Isotopes_per_Elements.png}}
	\end{picture}
	\caption{Гистограмма числа изотопов химических элементов в коре Земли} По данным https://www.nndc.bnl.gov/nudat3
	\label{f:HIST Isotopes_per_Elements}
\end{figure}

На рис.~\ref{f:HIST Isotopes_per_Elements Even-Odd} приведёна гистограмма числа изотопов химических элементов в коре Земли с учётом чётности числа нуклонов. 

\begin{figure}[ht] 
	\centering\small
	\unitlength=1mm
	\begin{picture}(120,40)
	\put(30,0){\includegraphics[width=50mm]{Figures/HIST Isotopes_per_Elements Even-Odd.png}}
	\end{picture}
	\caption{Гистограмма числа изотопов химических элементов в коре Земли с чётным и нечётным количеством нуклонов} По данным https://www.nndc.bnl.gov/nudat3
	\label{f:HIST Isotopes_per_Elements Even-Odd}
\end{figure}

На рис.~\ref{f:Isotopes_per_Elements Even-Odd} приведён график числа изотопов химических элементов в коре Земли с учётом чётности числа нуклонов. 

\begin{figure}[ht] 
	\centering\small
	\unitlength=1mm
	\begin{picture}(120,60)
	\put(-3,0){\includegraphics[width=120mm]{Figures/Isotopes_per_Elements Even-Odd.png}}
	\end{picture}
	\caption{Число изотопов для химических элементов} По данным https://www.nndc.bnl.gov/nudat3
	\label{f:Isotopes_per_Elements Even-Odd}
\end{figure}

	\paragraph{Правило Щукарева-Маттауха.}


Для изобар справедливо \emph{правило Щукарева—Маттауха}, объясняющее, в частности, отсутствие стабильных изотопов у технеция. \index{правило Щукарева—Маттауха}
Суть правила заключается в том, что в природе не могут существовать два стабильных изобара, заряды ядра которых отличаются на единицу. Другими словами, если у какого-либо химического элемента есть устойчивый изотоп, то его ближайшие соседи по таблице не могут иметь устойчивых изотопов с тем же массовым числом. %\index{технеций} 

Рассмотрим этот вопрос в \S\ref{RareElementsMono} <<Редкие элементы>>.

	

  	\subsection{Распад атомных ядер} \label{NucleiDecay}
	
	Как видно из Рис.~\ref{f:NZHalf-time}, стабильность ядер --- весьма редкое явление, и как уже обсуждалось выше, стабильность - понятие условное.
	
	
	На  Рис.~\ref{f:NZDecayMode} приведена $N-Z$ диаграмма  атомных ядер с их модами распада. \index{моды распада атомных ядер} Следует отметить, что термин <<распад>> не вполне корректно отражает смысл процессов, происходящих с ядрами. В частности, при бета-распаде происходит увеличение электричского заряда и увеличение порядкового номера элемента. Больше подошло бы слово <<превращение>>, но мы будем придерживаться традиционной терминологии.	
	
	
	\begin{figure}[ht] 
		\centering\small
		\unitlength=1mm
		\includegraphics[width=100mm]{Figures/NZDecayMode.png} 
		\caption{$N-Z$ диаграмма  атомных ядер} Моды распада атомных ядер https://www.nndc.bnl.gov/nudat3 
		\label{f:NZDecayMode}
	\end{figure}
	Основными модами радиоактивного распада являются альфа- и бета-распады, гамма-распад (или изомерный переход). Также существует спонтанное деление ядер, протонный и кластерный распады. 
	
	
	\emph{Альфа-распад} --- вид распада, при котором ядро испускает ядро гелия, при этом заряд ядра уменьшается на два, а масса --- на 4. Альфа-распад присущ тяжелым элементам (тяжелее свинца), например распад урана-238:  	\index{альфа-распад}
	\begin{equation*}
	^{238}_{92}U \rightarrow ^{234}_{90}Th  + ^{4}_{2}He.
	\end{equation*}
	На  Рис.~\ref{f:NZDecayMode} область ядер, испытываюих альфа-распад, обозначена жёлтым цветом.
	
	
	\emph{Бета-распад} \index{бета-распад} --- вид распада, при котором, испускается (или поглощается) электрон и антинейтрино, или позитрон и нейтрино. Существуют несколько типов бета-распада:
	\begin{enumerate}
		\item Электронный (бета-минус) распад  (превращение нейтрона в протон); \index{электронный (бета-минус)  распад}
		\item Позитронный (бета-плюс) распад (превращение протона в нейтрон); \index{позитронный распад}
		\item Электронный захват. \index{электронный захват}
	\end{enumerate}
	
	На рис.~\ref{f:Beta_Negative_Decay} показана диаграмма Фейнмана для бета-минус распада нейтрона. Переносчиком слабого взаимодействия \index{слабые взаимодействия} является \index{$W^{-}$-бозон} $W^{-}$-бозон. \index{диаграмма Фейнмана}	
	\begin{figure}[ht] 
	\centering\small
\unitlength=1mm
\begin{picture}(120,40)
	\put(60,0){\includegraphics[width=40mm]{Figures/Beta_Negative_Decay.png}}
	\put(15,7){\includegraphics[width=25mm]{Figures/Beta_Negative_Decay0.png}}
\end{picture}	 
	\caption{Диаграмма Фейнмана для бета-минус распада нейтрона} 
	\label{f:Beta_Negative_Decay}
\end{figure}

	
	Бета-минус (электронный распад) распад характерен для изотопов с избытком нейтронов, например:
	\begin{equation*}
	^{14}_{6}C \rightarrow ^{14}_{7}N  + e^{-}+\tilde{\nu}.
	\end{equation*}
	Изотоп углерода-14 испытывает бета-минус распад, превращаясь при этом в стабильный азот с испусканием электрона и антинейтрино. На  Рис.~\ref{f:NZDecayMode} бета-минус распад изображен розовым цветом.
	Бета-плюс распад и электронный захват характерны для изотопов с недостатком нейтронов. Пример бета-плюс распада:
	\begin{equation*}
	^{11}_{6}C \rightarrow ^{11}_{5}B  + e^{+}+\nu.
	\end{equation*}
	Изотоп углерода-11 испытывает бета-плюс распад, превращаясь в стабильный бор, при этом испускается позитрон и нейтрино. 
	
	
	При электронном захвате ядро захватывает электрон атома с собственной электронной оболочки. Электронный захват существенен для тяжелых ядер, у которых внутренняя оболочка расположена близко к ядру. Примером легкого ядра, захватывающего элетрон является бериллий-7, захватывающий электрон и превращающийся в литий-7:
	\begin{equation*}
	^{7}_{4}Be+e^{-} \rightarrow ^{7}_{3}Li.
	\end{equation*}
	Бета-плюс распад и электронный захват на Рис.~\ref{f:NZDecayMode} изображены синим цветом.
	
	Зеленым цветом на Рис.~\ref{f:NZDecayMode} обозначены тяжелые ядра, которым присуще спонтанное деление. Также на рисунке, в области легких ядер, присутствуют нейтроноизбыточные (фиолетовые) и протоноизбыточные ядра (оранжевые). Такие изотопы распадаются испуская нейтрон (несколько нейтронов) или протон соответственно.


	\section{Стабильность атомных ядер} \label{StabilityNuclei}

Важнейшей характеристикой изотопов химических элементов является их стабильность по отношению к различным модам превращений.
Изложенный в \S\ref{NucleiDecay} материал носит качественный характер. Для получения более точных представлений об изотопах в природе нужны уже количественные данные о принципиальной возможности различных процессов и их вероятности.

	\subsection{Стабильность атомных ядер в целом} \label{StabilityNucleiGeneral}

Для атомных ядер эти данные содержатся в виде численных зависимостей различных возбуждений ядра.
Стабильность атомных ядер возможна, если не одно из возбуждений не способно перевести конкретное атомное ядро в энергетически более выгодное состояние. 
Если ядро устойчиво ко всем модам распада, \S\ref{NucleiDecay}, оно стабильно.

\begin{figure}[ht] 
	\centering\small
	\unitlength=1mm
	\begin{picture}(120,60)	
		\put(0,0){\includegraphics[width=120mm]{Figures/DecayPositiveALL.png}}
	\end{picture}	
	\caption{График энергии всех видов распада} 
	\label{f:DecayPositiveALL}
\end{figure}

На рис.~\ref{f:DecayPositiveALL} приведен график энергии всех видов распада.
 На рис.~\ref{f:DecayPositiveALL} стабильным ядрам отвечает неокрашенная полоска внутри графика.
Следует заметить, что рост энергии, высвобождаемой при распаде, очень быстрый по мере удаления от дорожки стабильности. На периферии <<океана>> изотопов значения достигают более 20 МэВ, что более чем в два раза превышает энергию связи на нуклон.

\subsection[Стабильность атомных ядер по отношению\\ к различным видам распадо]
{Стабильность атомных ядер по отношению к различным видам распадов} \label{StabilityNucleiChannel}

Рассмотрим теперь стабильность атомных ядер по отношению к различным видам распадов.

Периферию <<океана>> изотопов ограничивают распады с вылетом одной или нескольких составляющих ядра, протонов и нейтронов.
На рис.~\ref{f:DecayPositivePN} приведен график энергии протонных и нейтронных распадов.

\begin{figure}[ht] 
	\centering\small
	\unitlength=1mm
	\begin{picture}(120,60)	
		\put(0,0){\includegraphics[width=120mm]{Figures/DecayPositivePN.png}}
	\end{picture}	
	\caption{График энергии протонных и нейтронных распадов} 
	\label{f:DecayPositivePN}
\end{figure}

Наиболее ярко испускание протонов и нейтронов демонстрируют самые лёгкие ядра.
Рассмотрим водород. Его изотопы с массой 4---7 распадаются с испусканием нейтронов.
\begin{align}
^4H & \ \longrightarrow ^3H + n, \label{4Hdecay} \\
^5H & \  \longrightarrow ^3H + 2n, \\
^6H & \ \longrightarrow ^3H + 3n, \\
^6H & \ \longrightarrow ^2H + 4n, \\
^7H & \ \longrightarrow ^3H + 4n. \label{7Hdecay} 
\end{align}
Время протекания процессов \eqref{4Hdecay}---\eqref{7Hdecay} составляет $10^{-22}$ с. Такие времена характерны для \index{сильные взаимодействия} \emph{сильного взаимодействия}.

Рассмотрим теперь литий. Его изотопы с массой 3---5 распадаются с испусканием нейтронов.
\begin{align}
	^3Li & \ \longrightarrow ^2He + p, \label{3Lidecay} \\
	^4Li & \  \longrightarrow ^3He + p, \\
	^5Li & \ \longrightarrow ^4He + p. \label{5Lidecay} 
\end{align}
Время протекания процессов \eqref{3Lidecay}---\eqref{5Lidecay} также имеет порядок $10^{-22}$ с.
Следует заметить, что распад изотопа $^5Li$ приводит к отсутствию ядер с атомным весом, равным 5.

Альфа-распад имеет место для нескольких изотопов самых лёгких и является доминирующим каналом распада для ядер тяжелее свинца.

На рис.~\ref{f:DecayPositiveAlpha} приведен график энергии альфа распада
\begin{figure}[ht] 
	\centering\small
	\unitlength=1mm
	\begin{picture}(120,60)	
		\put(0,0){\includegraphics[width=120mm]{Figures/DecayPositiveAlpha.png}}
	\end{picture}	
	\caption{График энергии альфа-распада} 
	\label{f:DecayPositiveAlpha}
\end{figure}

Рассмотрим случай бериллия.
\begin{align}
	^8Be & \ \longrightarrow ^4\!He + \alpha. \label{8Bedecay} 	
\end{align}	
Эта реакция приводит к отсутствию ядер с атомным весом, равным 8, аналогично случаю \eqref{5Lidecay} для веса 5.
Время протекания распада \eqref{8Bedecay} имеет порядок $10^{-18}$ с.

Что касается тяжелых ядер, то наличие у них большого электрического заряда служит причиной их нестабильности. При этом как варианты распада, так и времена периодов полураспада разнятся колоссально, от $10^{-15}$ с, до возраста Вселенной.

Изотопы урана и тория, в силу колоссальных времён полураспада, по распространённости превышают редкие элементы, и присутствуют в земной коре в количестве порядка г/т. 
В \S\ref{s:Lead} подробно рассмотрены цепочки ядерных реакций, имеющих начало в изотопах урана и тория и приводящих к образованию изотопов свинца.

Перейдём к рассмотрению <<обычных>> изотопов элементов.

\begin{figure}[ht] 
	\centering\small
	\unitlength=1mm
\begin{picture}(120,60)	
	\put(0,0){\includegraphics[width=120mm]{Figures/DecayPositiveBetaEC.png}}
\end{picture}	
	\caption{График энергии бета-распада и электронной конверсии} 
	\label{f:QECQbeta}
\end{figure}

На рис.~\ref{f:QECQbeta} представлен совместный график энергии бета-распада $Q_{\beta}$ и электронной конверсии $Q_{EC}$ в координатах (заряд, массовое число). Данные приведены для положительных значений энергий возбуждений, то есть таких, которые переводят ядро в энергетически более выгодное состояние. 
Эти процессы являются в определённой мере антагонистичными, увеличивают или уменьшают заряд ядра и наиболее ярко проявляются для нейтро-избыточных и нейтрон-дефицитных конфигураций.
В ряде случаев, они могут идти для одного и того же ядра. 

Бета-распад  и электронная конверсия (также позитронный распад) определяют в большинстве случаев, какие из изотопов химических элементов являются стабильными. По типу взаимодействия они относятся к \index{слабые взаимодействия} \emph{слабым взаимодействиям}. 
В зависимости от устройства ядер изотопов, разброс периодов полураспада составялет  $10^{-15}$ с, до возраста Вселенной.


В \S\ref{RareElementsMono}	будут рассмотрены конкретные примеры для <<несуществующих>> на Земле элементов технеция и прометия.
В \S\ref{StabitityNucleiReview} дана картина типологий устойчивости ядер для различного количества составляющих их нуклонов.
	
	\section{Ядерный нуклеосинтез} \label{Nucleosynthesis}
	
	Для понимания распределений изотопов элементов в природе вообще, и на Земле в частности, необходимо иметь представление о том, как они образовывались. Наиболее  полное современное издание по этому вопросу --- книга   В.П.\,Чечев, А.В.\,Иванчик, Д.А.\,Варшалович «Синтез элементов во Вселенной: От Большого взрыва до наших дней» \cite{Nucleosynthesis}, см. также учебник \cite{MSU98}. Краткое изложение вопроса содержится, например, в статье \cite{ElementsOrigin}. Будем далее использовать термин \index{обилие, распространённость изотопов} \emph{распространённость изотопов}, ему синонимичен термин \emph{обилие}.
	
	По современным представлениям, Вселенная развивается согласно \emph{стандартной космологической модели} (Большого взрыва). Согласно этой модели, более 15 млрд. лет назад началось и продолжается поныне расширение Вселенной из некоторого сверсжатого и сверхгорячего состояния. \index{стандартная космологическая модель}

На рис.~\ref{f:Nucleosynthesis_periodic_table} показано происхождение элементов из периодической таблицы Менделеева.
	\begin{figure}[ht] 
	\centering\small
	%	\unitlength=1mm
	\includegraphics[width=0.9\textwidth]{Figures/1920px-Elements-origin-ru.png}
	%	\includegraphics[width=30mm]{Figures\Oxygen.png}
	\caption{Происхождение элементов} https://en.wikipedia.org/wiki/Нуклеосинтез 
	%	https://ru.wikipedia.org/wiki/%D0%9D%D1%83%D0%BA%D0%BB%D0%B5%D0%BE%D1%81%D0%B8%D0%BD%D1%82%D0%B5%D0%B7
	\label{f:Nucleosynthesis_periodic_table}
\end{figure}

Как видно из  рис.~\ref{f:Nucleosynthesis_periodic_table},  синтез элементов весьма различен, и кроме того, одни и те же элементы могут иметь разное происхождение.
На рис.~\ref{f:Reactions-of-importance-for-nuclear-astrophysics} показаны реакции, которые могут идти в звёздах  \cite{Smith2023}. Участниками реакций явялются атомные ядра, простые ядра (дейтоны, альфа-частицы) и элементарные частицы: протоны, нейтроны, гамма-кванты. В скобках на первом месте указывается частица, налетающая на атомное ядро, на втором месте --- частица, являющаяся продуктом реакции, см. например \eqref{neutroncapture}.
	\begin{figure}[ht] 
	\centering\small
	%	\unitlength=1mm
	\includegraphics[width=0.6\textwidth]{Figures/Reactions-of-importance-for-nuclear-astrophysics-shown-on-the-N-Z-plane-for-stable-nuclei.png}
	%	\includegraphics[width=30mm]{Figures\Oxygen.png}
	\caption{Реакции, важные для нуклеосинтеза \cite{Smith2023}} 
	% Michael S. Smith.	Nuclear data resources and initiatives for nuclear astrophysics 	November 2023Frontiers in Astronomy and Space Sciences 10 	DOI: 10.3389/fspas.2023.1243615	
	\label{f:Reactions-of-importance-for-nuclear-astrophysics}
\end{figure}


	Рассмотрим пунктирно происхождение некоторых групп элементов.
	Образование разнообразия элементов происходит на разных этапах развития Вселенной. Самые лёгкие элементы --- водород и гелий --- рождаются на начальном этапе, вскоре после Большого взрыва. Первичной базовой реакцией является слияние протона и нейтрона --- образование легчайшего составного ядра  дейтерия \index{процесс образования ядра  дейтерия}
	\begin{equation}\label{npdgamma}
	n + p \longrightarrow d + \gamma.
	\end{equation}

	Энергия связи дейтона равна 2.22 МэВ, и весьма мала по меркам ядерной физики. Поэтому в горячей Вселенной дейтон быстро разрушается под действием гамма-квантов. По мере охлаждения Вселенной при её расширении, обратная к \eqref{npdgamma} реакция \index{процесс фоторасщепления ядра  дейтерия} \index{фоторасщепление ядер}
	\begin{equation}\label{gammad}
	\gamma +	d  \longrightarrow n + p 
	\end{equation}
	становится менее вероятной, и появляется возможность старта процесса \emph{нуклеосинтеза}. \index{нуклеосинтез}
	
	В ещё горячей Вселенной, а затем в звёдах лёгкие элементы и далее элементы средней массы вплоть до железа образуются за счёт \index{термоядерные реакции} \emph{термоядерных реакций}. Энергия связи нуклонов в ядре железа максимальна --- см. рис.~\ref{f:Weizsacker}. Дальнейший синтез тяжёлых элементов невыгоден из-за того, что энергия связи становится меньше по мере утяжеления элементов.
	
	Синтез более тяжёлых элементов идёт уже в реакциях захвата нейтронов ядрами. При этом образуются более тяжелые изотопы существующих ядер, в основном в процессах радиационного захвата нейтронов \index{процесс радиационного захвата нейтронов}
	\begin{equation}\label{neutroncapture}
	^N_ZA(n, \gamma)^{N+1}_ZA
	\end{equation}
	а также ядра с б\'{о}льшим зарядом за счёт $\beta$-распада  \index{процесс $\beta$-распада}
	\begin{equation}\label{betadecay}
	^N_zA \longrightarrow ^N_{Z+1}B+e^{-}.
	\end{equation}
	При этом возможны и процессы распада получившихся ядер в зависимости от их стабильности по отношению к различным распадам, о чём говорилось в \S\ref{StabilityNuclei}.
	
	\begin{figure}[ht] 
		\centering\small
		\unitlength=1mm
		\includegraphics[width=120mm]{Figures/NuclidsN.png} 
		\caption{Относительная распространенность нуклидов от атомной массы \cite{ElementsOrigin}}
			Обозначения указывают на различные процессы синтеза 
		\label{f:NuclidsN}
	\end{figure}

	На Рис.~\ref{f:NuclidsN} представлена относительная распространенность нуклидов от атомной массы.
Обозначения указывают на различные процессы синтеза, наиболее значимые из них мы рассмотрим далее.

	
	Необходимо связать информацию о распространенности нуклидов из рис.~\ref{f:NuclidsN} с закономерностями строения ядра \S\ref{s:NuclPhys}. Рис.~\ref{f:NuclidsN} отражает много информации. Для анализа общих закономерностей рассмотрим менее подробное представление распространенности нуклидов, показанное на Рис.~\ref{f:NuclidsNSchematics}.
	\begin{figure}[ht] 
		\centering\small
		\unitlength=1mm
		\includegraphics[width=60mm, height=50 mm]{Figures/NuclidsNSchematics.png} 
		\caption{Схематическое изображение распространенности нуклидов от атомной массы \cite{ElementsOrigin}}
%			Обозначения указывают на различные процессы синтеза } 
		\label{f:NuclidsNSchematics}
	\end{figure}
	
	 Распространённость элементов тяжелее гелия в целом экспоненциально снижается с ростом атомной массы $A$ до 100 и после $A >120$ практически постоянна. Наблюдается локальный максимум в области группы железа $Fe-Co-Ni$. 
	Он отвечает максимальной энергии связи нуклонов в этих ядрах --- см. рис.~\ref{f:Weizsacker} из \S\ref{NucleiModels}. 
	
	В дальнейшем ходе зависимости имеется 3 локальных двойных максимумов. Одни из элементов пар лежат в области магических нейтронных чисел $N = 50, 80, 126$ --- см. ряд из формулы \eqref{MagicN}. Массы ядер при этом равны $A = 90, 138, 208$. Другие максимумы сдвинуты примерно на 10-15 массовых числел. Эти особенности помечены  буквами $s$ и $r$, они будут рассмотрены в \S\ref{s:sProcess} и \S\ref{s:rProcess}.
	
	
\begin{figure}[h] 
	\centering\small
	\unitlength=1mm
	\includegraphics[width=100mm]{Figures/ChartNucleosysnthesis.png} 
	\caption{Иллюстрация хода $s$- и $r$-процессов \cite{NuclearAstrophysics2022}} 
	\label{f:ChartNucleosysnthesis}
\end{figure}
	
	С точки зрения конечного состава планетного вещества важно понимать пути нуклеосинтеза тяжелых элементов и равновесное состояние на настоящее время. Общая схема образования элементов от железа до актинидов начинается с ядра железа.

Возможны различные траектории (схемы) нуклеосинтеза. Наиболее интересна на первом шаге область стабильных элементов. При сопоставлении $N-Z$ диаграммы атомных ядер закладки <<Периоды полураспада>>,  рис.~\ref{f:NZHalf-time}, с другими табличными данными, видно, что в области стабильных ядер максимальны и сечения захвата нейтронов. 
%На рис.~\ref{f:ChartNucleosysnthesis} представлены различные процессы в координатах (N, Z).
	
	\begin{figure}[h] 
		\centering\small
		\unitlength=1mm
		\includegraphics[width=120mm]{Figures/(N,Gamma) Beta-decay Electron Conversion.png} 
		\caption{Области $(n,\gamma)$, $\beta^{-}$, EC} 
		\label{f:(N,Gamma) Beta-decay Electron Conversion}
	\end{figure}

 На рис.~\ref{f:(N,Gamma) Beta-decay Electron Conversion} представлены различные процессы, важные для анализа путей нуклеосинтеза, в координатах (A, Z). Жёлтым цветом даны изотопы, у которых имеется ненулевое сечение захвата нейтронов. Выше них показаны нестабильные изотопы, распадающие с захватом электрона, а ниже --- 
нестабильные изотопы, распадающие с вылетом электрона.

При наличии потока нейтронов, в <<жёлтых>> изотопах происходит увеличение атомного веса $A$ без изменения заряда $Z$ и происходит движение на графике вправо. При электронном захвате уменьшается $Z$ при постоянном значении $A$ и происходит движение на графике вниз. При бета-распаде увеличивается $Z$ при постоянном значении $A$ и происходит движение на графике вверх.
Тем самым идёт движение от малых величин $A$ к большим за счёт захвата нейтронов, а процессы с электронами возвращают процесс в область стабильных ядер.  

Таким образом, в первом приближении для понимания формирования ядер химических элементов, образующих планетное вещество, следует рассмотреть процесс нейтронного захвата.
	
	\subsection{Процесс медленного нейтронного захвата} \index{процесс медленного нейтронного захвата ($s$-процесс)} \label{s:sProcess}

	Основными схемами нуклеосинтеза являются реакции  медленного нейтронного захвата ($s$-процесс) и быстрого нейтронного захвата ($r$-процесс). Рассмотрим кратко эти схемы, следуя \cite{Nucleosynthesis, ElementsOrigin}.

\begin{figure}[h] 
	\centering\small
	\unitlength=1mm
	\includegraphics[width=100mm]{Figures/NucleosythesisTrajectory.png} 
	\caption{Иллюстрация хода $s$- и $r$-процессов \cite{ElementsOrigin}} 
	\label{f:NucleosythesisTrajector}
\end{figure}


По современным представлениям тяжёлые элементы образуются в реакциях захвата нейтронов. 
Обычно вероятность захвата нейтронов ядром имеет энергетическую зависимость 
\begin{equation}\label{sigmaEn}
\sigma(E_n) \propto 1/\sqrt(E_n).
\end{equation}

При более высоких энергиях сечение может иметь более сложную зависимость, как показано на рис.~\ref{f:NeutronCaptureE}.
\begin{figure}[ht] 
	\centering\small
	\unitlength=1mm
	\begin{picture}(60,50)
	\put(0,0){\includegraphics[width=70mm]{Figures/NeutronCaptureE.png}}
	\end{picture}
	\caption{Сечение захвата нейтрона изотопами бора  --- Википедия} 
	\label{f:NeutronCaptureE}
\end{figure}


В связи с типом зависимости \eqref{sigmaEn}, справочные данные обычно приводят для \emph{тепловой энергии} \index{тепловая энергия нейтронов} нейтронов $T=300$ К. 
Результатом захвата является изотоп с б\'{о}льшим количеством нейтронов:
\begin{equation} \label{NeutronCapture}
(A, Z) + n \longrightarrow (A+1, Z) + \gamma.
\end{equation}

Как правило, процесс \eqref{NeutronCapture} сопровождается испусканием  \index{процесс радиационного захвата нейтронов} $\gamma$-кванта, как в в базовой ядерной реакции захвата нейтрона протоном \eqref{npdgamma}.

На рис.~\ref{f:NZngamma} показаны табличные значения \emph{сечений захвата} тепловых нейтронов \index{сечения захвата тепловых нейтронов} в координатах (N, Z).  Термин сечение используется в ядерной физике для характеризации вероятности ядерных реакций. Единицей измерений служит 1 \index{барн} \emph{барн}, который имеет размерность площади и равен 
\begin{equation*}
1 \text{ барн} = 10^{-24} \text{ см}^2 = 100 \text{ фм}^2
\end{equation*}
--- примерному размеру атомного ядра.

\begin{figure}[h] 
	\centering\small
	\unitlength=1mm
	\begin{picture}(100, 70)
	\put(0,0){\includegraphics[height=70mm]{Figures/NZngamma.png}}
	\end{picture}
	\caption{Сечения захвата тепловых нейтронов} 
	\label{f:NZngamma}
\end{figure}


Стабильность дочернего изотопа $(A+1, Z)$ определяется энергией по отношению к бета-распаду или захвату электрона, рассмотренной в \S\ref{StabilityNucleiChannel}.

Быстрый (r)- и медленный (s)-процессы захвата нейтронов (от английских слов rapid и slow) различаются отношением скорости захвата нейтронов (реакция $(n, \gamma)$) к скорости бета-распада. При условии 
\begin{equation}\label{sprocess}
\frac{\tau_{\beta}} {\tau_{(n, \gamma)}}  \ll  1 
\end{equation}
в цепочку процессов образования тяжелых элементов будут вовлечены только стабильные и бета-радиоактивные ядра с большими периодами полураспада. То есть образование элементов будет происходить вдоль долины бета-стабильности. Нейтроны добавляются к ядрам последовательно. При этом могут образоваться только сравнительно устойчивые ядра. Ядра с малыми периодами полураспада исчезают раньше, чем они успевают захватить следующий нейтрон. Поэтому ясно, что образование тяжелых элементов должно заканчиваться свинцом и висмутом.

Теоретические оценки показывают, что для протекания \index{s-процесс} s-процесса достаточно плотности нейтронов $10^{10}$ н/см3. В качестве исходных ядер, из которых в результате последовательного захвата нейтронов будут образовываться тяжелые элементы, можно выбрать ядра <<железного пика>>. При плотности нейтронов
$10^{10}$ н/см3 полное время облучения, необходимое для образования свинца из железа, составляет около $10^{3}$  лет.

Рассмотрим последовательный процесс захвата нейтронов  \cite{MSU98}
\begin{equation}
\ldots \longrightarrow (A-1, Z) \longrightarrow (A, Z) \longrightarrow (A+1, Z) \longrightarrow \ldots
\end{equation}

Изменение числа $n(A)$ ядер с массовым числом A описывается уравнением:
\begin{equation}\label{dnAdt}
\frac{dn(A)}{dt} = kn(A-1)\sigma_{n,\gamma}(A-1) - kn(A)\sigma_{n,\gamma}(A),
\end{equation}
где $k$--- поток нейтронов. Если процесс стационарный, то $\frac{dn(A)}{dt} =0$. В таком случае из \eqref{dnAdt} получаем:
\begin{equation} \label{Nconst}
n(A-1)\sigma_{n,\gamma}(A-1) = n(A)\sigma_{n,\gamma}(A) = const.
\end{equation}
Из соотношения \eqref{Nconst} следует, что чем меньше сечение радиационного захвата нейтронов, тем больше должна быть распространенность элемента, образующегося в s-процессе. В частности, это объясняет почему ядра с магическими числами N и Z встречаются чаще. Связано это с тем, что для магических ядер величина сечения радиационного захвата нейтронов падает на порядок по сравнению с соседними немагическими. 

 
На рис.~\ref{f:SProcessFeZr} представлен график s-процессов после группы железа \cite{s-process2007}.
\begin{figure}[h] 
	\centering\small
	\unitlength=1mm
%	\begin{picture}(100, 70)
%	\put(0,0){
		\includegraphics[width=90mm]{Figures/SProcessFeZr.png} %}
%	\end{picture}
	\caption{s-процесс после группы железа \cite{s-process2007}} 
	\label{f:SProcessFeZr}
\end{figure}
На нем присутствуют как процессы бета-распада, так и электронного захвата. 


Данные, представленные на рис.~\ref{f:NZngamma}, дают понимание того факта, что ядерный нуклеосинтез в <<обычных>> условиях завершается изотопами $^{208}Pb$ и $^{209}Bi$. 
Сечение захвата нейтрона ядром  $^{208}Pb$ на 3 порядка меньше, чем аналогичное сечение $^{207}Pb$. Также сечение процесса с  $^{209}Bi$ в 10 раз меньше, чем у $^{207}Pb$.
Даже если захват произойдет, следующие за висмутом элементы нестабильны по отношению к бета- и альфа-распаду, а сечения захвата нейтронов очень малы.


\begin{figure}[h] 
	\centering\small
	\unitlength=1mm
		\includegraphics[width=50mm]{Figures/Sigma NGamma Z=81-85 A=203-211.png}
	\caption{Сечения захвата тепловых нейтронов для изотопов свинца и висмута} 
	\label{f:NZngammaPb}
\end{figure}

Завершение s-процесса в области изотопов свинца и висмута представлено на рис.~\ref{f:sProcessesTermination}
\begin{figure}[h] 
	\centering\small
	\unitlength=1mm
		\includegraphics[width=60mm]{Figures/sProcessesTermination.png}
	\caption{Завершение s-процесса в области изотопов свинца и висмута \cite{s-processTermination2004}} 
	\label{f:sProcessesTermination}
\end{figure}
%  U. Ratzel, C. Arlandini, F. Käppeler, A. Couture, M. Wiescher, R. Reifarth, R. Gallino, A. Mengoni, and C. Travaglio. Nucleosynthesis at the termination point of the s process Phys. Rev. C 70, 065803 – Published 10 December 2004
% L.-S. The, Mounib F. El Eid and B.S. Meyer. s-PROCESS NUCLEOSYNTHESIS IN ADVANCED BURNING PHASES OF MASSIVE STARS. The Astrophysical Journal, 655:1058Y1078, 2007 February 1

\subsection{Процесс быстрого нейтронного захвата} \index{процесс быстрого нейтронного захвата ($r$-процесс)} \label{s:rProcess}

В настоящее время общепризнанно, что многие ядра тяжелее железа, включая все ядра тяжелее $^{209}Bi$, образуются в $r$-процессе путем \index{r-процесс} быстрого последовательного захвата большого количества нейтронов. Главное условие такого сценария: скорость захвата нейтронов должна быть больше скорости бета-распада. Захват нейтронов происходит до тех пор, пока скорость реакции $(n, \gamma)$ не станет меньше скорости распада изотопа
\begin{equation}\label{rprocess}
\frac{\tau_{\beta}} {\tau_{(n, \gamma)}}  >  1 
\end{equation}


Образующееся ядро распадается затем в результате бета-распада и вновь начинается последовательный захват нейтронов.

\begin{figure}[ht] 
	\centering\small
	\unitlength=1mm
	\includegraphics[width=90mm]{Figures/srProcessesStableIsotopes.png} 
	\caption{Способы синтеза изотопов с указанием преобладания s- или r-процесса \cite{r-process2007}} 
	\label{f:srProcessesStableIsotopes}
\end{figure}


Линия, вдоль которой происходит образование ядер в $r$-процессе, смещена от дорожки стабильности (трека $s$-процесса) в направлении нейтроноизбыточных изотопов --- Рис.~\ref{f:NuclidsNSchematics}.


%На рис.~\ref{f:NZsrProcessesStableIsotopes} показаны способы синтеза изотопов с участием s- и r-процессов, и границы протонной и нейтронной устойчивости
%\begin{figure}[ht] 
%	\centering\small
%	\unitlength=1mm
%	\includegraphics[width=90mm]{Figures/NZsrProcessesStableIsotopes.png} 
%	\caption{Способы синтеза изотопов с указанием преобладания s- или r-процессов, и границы протонной и нейтронной устойчивости \cite{r-process2007}} 
%	\label{f:NZsrProcessesStableIsotopes}
%\end{figure}



На рис.~\ref{f:srProcessesStableIsotopes} показаны способы синтеза изотопов с указанием преобладания s- или r-процесса.
Наиболее сильные пики распространённости изотопов, образовавшихся по r-процессу, приходятся на значения атомных весов s A = 90 (Se, Br и Kr), A = 130 (Te, I и Xe) and A = 195 (Os, Ir и Pt).

%\begin{figure}[ht] 
%	\centering\small
%	\unitlength=1mm
	%\includegraphics[width=90mm]{Figures/2015-r process.png} 
	%\caption{Иллюстрация равновесия $(n, \gamma)$---$(\gamma, n)$ в ходе $r$-процессов %\cite{rprocess2015}} 
%	\label{f:2015-r process}
%\end{figure}
% M. Eichler et al. THE ROLE OF FISSION IN NEUTRON STAR MERGERS AND ITS IMPACT ON THE r-PROCESS PEAKS. The Astrophysical Journal, 808:30 (13pp), 2015 July 20. doi:10.1088/0004-637X/808/1/30

%На рис.~\ref{f:2015-r process} приведёна иллюстрация качества расчёта.Линия стабильности (черные кружки) и путь r-процесса, возникающий в среде слияния нейтронных звезд \cite{rprocess2015}. Траектория следует из химического равновесия между захватом нейтронов и фоторасщеплением  \index{фоторасщепление ядер} \index{процесс радиационного захвата нейтронов} в каждой изотопной цепочке $(n, \gamma)$---$(\gamma, n)$. Настоящие расчеты были выполнены для системы, содержащей более 3000 ядерных реакций. На цветовой легенде показано, что ошибка расчёта не превышает 2 раз.	
	
При промежуточных между быстрыми (r)- и медленными (s)-процессами имеется трасса, которая называется \emph{промежуточными} (i-(intermediate) process) \index{промежуточные процессы, i-(intermediate) process} процессами. 
	
\section{Табличные данные по нуклеосинтезу}	 \label{s:TableNucleosynthesis}

Привёдём некоторые данные по нуклеосинтезу --- Табл.4.1 в \cite{Nucleosynthesis}. 
В дальнейшем будем сопоставлять эти данные с <<Таблицей стандартных атомных весов 2021>>
табл.~\ref{t:TSAW2021} \S\ref{s:TableStandardAtomicWeights}.

\begin{table}[h!]
	{\small 
		\begin{tabular}{ll}
			I & Элемент \\
			II & Массовое число A \\	
			III & Содержание нуклида в \% по отношению к общему количеству \\
					 & элемента в настоящее время \\
			IV & Процесс нуклеосинтеза  \\
			V & Распространённость нуклида (нормировано к Si = $10^6$)  \\
		& в эпоху образования Солнечной системы		
	\end{tabular}
	}
	\caption{Распространённость нуклидов в Солнечной системе --- Обозначения столбцов}
	\label{t:AbudancesSolarNotes}
\end{table}

Обозначения  процессов нуклеосинтеза:\\
	{\small 
\begin{tabular}{lcl}
U & --- & космологический синтез до образования звёзд \\
H & --- & горение водорода \\
CNO & --- & горение водорода при высоких температурах (CNO-цикл) \\
He & --- & взрывное горение гелия \\	
С & --- & взрывное горение углерода \\
O & --- & взрывное горение кислорода \\
Si & --- & взрывное горение кремния \\
NSi & --- & обогащённое нейтронами горение кремния \\
E & --- & статическое ядерное равновесие \\
s & --- & s-процесс. Продукты медленного захвата нейтронов\\
r & --- & r-процесс. Продукты быстрого захвата нейтронов\\
p & --- & p-процесс. Процесы на обеднённой нейтронами стороне \\
 & & долины $\beta$-стабильности\\
X & --- & Дробление космическими лучами \\
\end{tabular}
}

В табл.~\ref{t:AbudanceI} приведены данные о распространённости нуклидов первого периода Таблицы Менделеева, H и He.

\begin{table}[h!]
	{\footnotesize 
		\begin{tabular}{ccccc}
			Символ  & Атомный &  Содержание  & Процесс & Распростра   \\
			~ & Вес &  нуклида  &  нуклеосинтеза  & нённость  \\
			\hline 
			I & II &  III  & IV & V \\
			\hline 
			%			~ & ~ & ~ & ~ & ~  \\
			H & 1 & 99.9885 &  &  $2.59 \times 10^{10}$\\ [1mm]
			& 2 & 0.0115 & U & $5.03 \times 10^{15}$\\ [1mm]
			He & 3 & 0.000134 & U &  $1.03 \times 10^{6}$\\ [1mm]
			& 4 & 99.999866 & H, U & $2.51 \times 10^{9}$\\ [1mm]
			& 5 & &  & \\ [1mm]
			\hline 
		\end{tabular}
	}
	\caption{Pаспространённость нуклидов, H---He}
	\label{t:AbudanceI}
\end{table} 

Строка с атомным весом, равным 5, оставлена пустой.
~\\
~\\

В табл.~\ref{t:AbudancePb-U} приведены данные о распространённости последних стабильных нуклидов Таблицы Менделеева и трансурановых элементов.

\begin{table}[h!]
	{\footnotesize 
		\begin{tabular}{ccccc}
			Символ  & Атомный &  Содержание  & Процесс & Распростра   \\
			~ & Вес &  нуклида  &  нуклеосинтеза  & нённость  \\
			\hline 
			I & II &  III  & IV & V \\
			\hline 
			~ & ~ & ~ & ~ & ~  \\
			Pb & 204 & 1.4 & s &  0.066 \\ [1mm]
			& 206 & 24.1 & s, r & 0.614 \\ [1mm] 			
			& 207 & 22.1 & s, r & 0.680 \\ [1mm] 				
			& 208 & 52.4 & s, r & 1.946 \\ [1mm] 	
			Bi & 209 & 100 &  s, r &  0.1382 \\ [1mm]
			\hline
			Th & 232 & 100 &  r &  0.0440 \\ [1mm]	
			U & 234 & 0.0054 &  r &  0.00000049 \\ [1mm]
			& 235 & 0.7204 &  r &  0.00058 \\ [1mm]	
			& 238 & 99.2742 &  r &  0.0180  \\ [1mm]		
			\hline 
		\end{tabular}
	}
	\caption{Pаспространённость нуклидов в Солнечной системе, Pb--U}
	\label{t:AbudancePb-U}
\end{table}

~\\
~\\
~\\


\begin{table}[h!]
	{\footnotesize 
		\begin{tabular}{ccccc}
			Символ  & Атомный &  Содержание  & Процесс & Распростра   \\
			~ & Вес &  нуклида  &  нуклеосинтеза  & нённость  \\
			\hline 
			I & II &  III  & IV & V \\
			\hline 
			Li & 6 & 7.59 & X &  4.2 \\ [1mm]
			 & 7 & 92.41 & X, H, U & 51.4\\ [1mm]
			  			 & 8 & &  & \\ [1mm]
			 Be & 9 & 100 & X  &  0.612 \\ [1mm]
			B & 10 & 19.9 & X &  3.7 \\ [1mm]
			 & 11 & 80.1 & X & 15.1\\ [1mm]
 			\hline 			
 			C & 12 & 98.93 & He &  $7.11 \times 10^{6}$\ \\ [1mm]
			 & 13 & 1.07 & H & $7.99 \times 10^{4}$ \\ [1mm]
			N & 14 & 99.636 & H &  $2.12 \times 10^{6}$\ \\ [1mm]
			& 15 & 0.364 & H & $7.78 \times 10^{3}$ \\ [1mm] 
			O & 16 & 99.757 & He &  $2.12 \times 10^{6}$\ \\ [1mm]
			  & 17 & 0.038 & H & 5900 \\ [1mm] 			
			  & 18 & 0.205 & He, CNO & $3.15 \times 10^{4}$ \\ [1mm] 
			F & 19 & 100 & CNO &  804 \\ [1mm] 
			Ne & 20 & 90.48 & C &  $3.06 \times 10^{6}$\ \\ [1mm]
			& 21 & 0.27 & He, CNO & 7330 \\ [1mm] 			
			& 22 & 9.25 & He, CNO & $2.25 \times 10^{5}$ \\ [1mm] 
			Na & 23 & 100 & C &   $5.77 \times 10^{4}$ \\ [1mm] 
			Mg & 24 & 78.99 & C &  $8.10 \times 10^{5}$ \\ [1mm]
			& 25 & 10.00 & C & $1.03 \times 10^{5}$ \\ [1mm] 			
			& 26 & 11.01 & C & $1.13 \times 10^{5}$ \\ [1mm] 
			Al & 27 & 100 & C &   $8.46 \times 10^{4}$ \\ [1mm] 
			Si & 28 & 92.223 & O, Si &  $9.22 \times 10^{5}$ \\ [1mm]
			& 29 & 4.685 & O & $4.68 \times 10^{4}$ \\ [1mm] 			
			& 30 & 3.092 & O & $3.09 \times 10^{4}$ \\ [1mm] 
						P & 31 & 100 & O  &  8300 \\ [1mm]
			S & 32 & 94.99 & O, Si & 400258 \\ [1mm]
			& 33 & 0.75 & O, Si & 3160 \\ [1mm] 			
			& 34 & 4.25 & O, Si & 17800\\ [1mm] 
			& 36 & 0.01 & NSi (?) & 72 \\ [1mm]
			Cl & 35 & 75.76 & O, Si &  3920 \\ [1mm]
			& 37 & 24.24 & O, Si & 1250 \\ [1mm] 
			Ar & 36 & 0.3336 & O, Si &  78400 \\ [1mm]
			& 38 & 0.0629 & O, Si & 14300 \\ [1mm] 			
			& 40 & 99.6035 & s & 22\\ [1mm] 
			  \hline 
		\end{tabular}
	}
	\caption{Pаспространённость нуклидов, 2 период ТМ}
	\label{t:AbudanceII}
\end{table} 

\begin{table}[h!]
	{\footnotesize 
		\begin{tabular}{ccccc}
			Символ  & Атомный &  Содержание  & Процесс & Распростра   \\
			~ & Вес &  нуклида  &  нуклеосинтеза  & нённость  \\
			\hline 
			I & II &  III  & IV & V \\
			\hline 
			~ & ~ & ~ & ~ & ~  \\
			\hline
			K & 39 & 93.2581 & O, Si &  3500 \\ [1mm]
			& 40 & 0.0117 & O, Si & 6 \\ [1mm] 			
			& 41 & 6.7302 & O, Si  & 253\\ [1mm] 	
			Ca & 40 & 96.94 & O, Si &  58500 \\ [1mm]
			& 42 & 0.647 & Si, s & 391 \\ [1mm] 			
			& 43 & 0.135 & Si, s  & 82\\ [1mm] 
			& 44 & 2.09 & Si, s & 1260 \\ [1mm] 			
			& 46 & 0.004 & NSi (?)  & 2\\ [1mm] 	
			& 48 & 0.187 & NSi (?)  & 113\\ [1mm] 	 
			Sc & 45 & 100 & Si, E  &  34.4 \\ [1mm]	
			\hline			
						Ge & 70 & 20.57 & E, s &  24.3 \\ [1mm]
			& 72 & 27.45 & E, s & 31.7 \\ [1mm] 
			& 73 & 7.75 & E, s & 8.8 \\ [1mm] 
			& 74 & 36.50 & E, s & 41.2 \\ [1mm] 
			& 76 & 7.73 & E, s & 8.5 \\ [1mm] 
			As & 75 & 100 & s, r &  6.10 \\ [1mm]
			Se & 74 & 0.89 & p & 0.60 \\ [1mm]
			& 76 & 9.37 & s & 6.32 \\ [1mm] 
			& 77 &  7.63 & s, r & 5.15 \\ [1mm]  			
			& 78 & 23.77 & s & 16.04\\ [1mm] 
			& 80 & 49.61 & s, r & 33.48 \\ [1mm]			
			& 82 & 8.73 & r & 5.89 \\ [1mm]			
			Br & 79 & 50.69 & s, r  & 5.43 \\ [1mm]
			& 81 & 49.31 & s, r  & 5.28 \\ [1mm] 
			Kr & 78 & 0.335 & p & 0.20 \\ [1mm]
			& 80 & 2.286 & s, p & 1.30 \\ [1mm] 
			& 82 &  11.593 & s & 6.51 \\ [1mm]  			
			& 83 & 11.500 & s, r & 6.45 \\ [1mm] 
			& 84 & 59.987 & s, r & 31.78 \\ [1mm]			
			& 86 & 17.729 & r & 9.61 \\ [1mm]			
			\hline 
		\end{tabular}
	}
	\caption{Pаспространённость нуклидов, 3 период}
	\label{t:AbudanceIII}
\end{table}

\begin{table}[h!]
	{\footnotesize 
		\begin{tabular}{ccccc}
			Символ  & Атомный &  Содержание  & Процесс & Распростра   \\
			~ & Вес &  нуклида  &  нуклеосинтеза  & нённость  \\
			\hline 
			I & II &  III  & IV & V \\
			\hline 
%			~ & ~ & ~ & ~ & ~  \\
			Ti & 46 & 8.25 & E &  204 \\ [1mm]
& 47 & 7.44 & E & 184 \\ [1mm] 			
& 48 & 73.72 & E  & 1820 \\ [1mm] 
& 49 & 5.41 & E & 134 \\ [1mm] 			
& 50 & 5.18 & E, NSi (?) & 128 \\ [1mm] 
V & 50 & 0.250 & E &  0.7 \\ [1mm]
& 51 & 99.750 & E & 285.7 \\ [1mm] 	
			Cr & 50 & 4.35 & E & 569 \\ [1mm]
			& 52 & 83.789 & E & 11000 \\ [1mm] 			
			& 53 & 9.501 & E & 1240 \\ [1mm] 
			& 54 & 2.365 & E & 309 \\ [1mm]
			Mn & 55 & 75.76 & E &  9220 \\ [1mm]
			Fe & 54 & 5.845 & E &  49600 \\ [1mm]
			& 56 & 91.754 & E & $7.78 \times 10^5$ \\ [1mm] 			
			& 57 & 2.119 & E & 18000 \\ [1mm] 
			& 58 & 0.282 & E & 2390 \\ [1mm]			
			Co & 59 & 100 & E &  2350 \\ [1mm]
			Ni& 58 & 68.077 & E &  33400 \\ [1mm]
			& 60 & 26.233 & E & 12900 \\ [1mm] 			
			& 61 & 1.1399 & E & 559 \\ [1mm] 
			& 62 & 3.6346 & E & 1780 \\ [1mm]
			& 64 & 0.9255 & E & 454 \\ [1mm]
			Cu& 63 & 69.15 & E &  374 \\ [1mm]
			& 65 & 30.85 & E & 167 \\ [1mm]
			Zn & 64 & 49.17 & E &  630 \\ [1mm]
			& 66 & 27.13 & E, s & 362 \\ [1mm] 			
			& 67 & 4.04 & E, s & 53 \\ [1mm] 
			& 68 & 18.45 & E, s & 243 \\ [1mm] 
			& 70 & 0.61 & E, s & 8 \\ [1mm] 
			Ga & 69 & 60.108 & E, s &  22.0 \\ [1mm]
			& 71 & 39.892 & E, s & 14.6 \\ [1mm]
				\hline 
		\end{tabular}
	}
	\caption{Pаспространённость нуклидов в Солнечной системе, группа железа}
	\label{t:AbudanceFe}
\end{table}

\input{TableNucleosynthesisFigures}
	
	\chapter{Изотопы на Земле и других планетах} \label{IsotopesNatureSect}

Элементный состав небесного тела существенно различен для разных \emph{оболочек планеты}. \index{оболочки планеты}

В \emph{Солнечной системе} выделяют различные группы планет.
В частности, \emph{планеты земной группы} --- четыре планеты Солнечной системы: Меркурий, Венера, Земля и Марс. \index{Солнечная система} 

Различают три наружных оболочки Земли: \emph{литосферу}, \emph{гидросферу} и \emph{атмосферу}. 

\begin{figure}[ht] 
	\centering\small
	\unitlength=1mm
	\begin{picture}(60,25)
		\put(10,0){\includegraphics[width=35mm]{Figures/Earth-crust-cutaway-ru.png}}
	\end{picture}
	\caption{Строение планеты Земля --- Википедия} 
	\label{f:Earth-crust-cutaway-ru}
\end{figure}


Под литосферой понимают верхний твердый покров планеты, который служит ложем океана, а на материках совпадает с сушей. 

Гидросфера --- это все виды химически не связанной воды в твердом, жидком \index{гидросфера}
и газообразном состоянии. В состав гидросферы входят воды Мирового океана, подземные воды, воды рек, озер, болот, морей, ледников и других криогенных образований. Мировой океан покрывает 71\% поверхности Земли, его средняя глубина 3900 м \cite{2015Planetology}. 

Современная \emph{атмосфера Земли}, составляющая менее одной миллионной доли массы планеты, содержит примерно 78\% азота, 20\% кислорода, 1\% инертных газов, 0.03\% углекислого газа и незначительное количество других газов. \index{атмосфера Земли}

Все планеты земной группы Солнечной системы имеют следующее строение литосферы: \index{планеты земной группы} \index{литосфера}
\begin{itemize}
	\item В центре планеты находится \emph{ядро} из железа с примесью никеля. \index{ядро планеты}
	\item \emph{Мантия} --- промежуточный слой литосферы. Она образуется в результате отделения от первичного планетного вещества металлической части, которая уходит в ядро, и плавления, продукты которого формируют кору планеты. Мантия состоит из различных силикатов. \index{мантия планеты}
	\item \emph{Кора} --- внешняя оболочка планеты, образовавшаяся в результате частичного плавления мантии и состоящая также из силикатных пород, но обогащённая другими элементами. \index{кора планеты}
\end{itemize}	

Проблематика изотопной планетологии кратко рассмотрена в \S\ref{IsotopePlanetology}, см. также учебное пособие \cite{2015Planetology}.

\section[Общие сведения о распространённости элементов и\\ изотопов в земной коре]
{Общие сведения о распространённости элементов и изотопов в земной коре}\label{ElementsEarth}	

Рассмотрим \emph{земную кору} как наиболее важную для человечества оболочку планеты. 
 
Земную кору составляет сравнительно небольшое число элементов. Около 50\% массы земной коры приходится на кислород, более 25 \% — на кремний. Всего 18 элементов: O, Si, Al, Fe, Ca, Na, K, Mg, H, Ti, C, Cl, P, S, N, Mn, F, Ba — составляют 99,8 \% массы земной коры. \index{земная кора}


\begin{figure}[ht] 
	\centering\small
	\unitlength=1mm
	\begin{picture}(120,65)
	\put(-10,0){\includegraphics[width=135mm]{Figures/Element_Abudance_Earth_Crust Isotopes max.png}}
	\end{picture}
	\caption{Распространённость химических элементов в коре Земли} По данным https://www.nndc.bnl.gov/nudat3
	\label{f:Element_Abudance_Earth_Crust Isotopes max}
\end{figure}


На рис.~\ref{f:Element_Abudance_Earth_Crust Isotopes max} представлены данные по распространённости химических элементов в коре Земли по данным https://www.nndc.bnl.gov/nudat3.

Как видно из рис.~\ref{f:Element_Abudance_Earth_Crust Isotopes max}, имеет место колоссальная неравномерность распространённости химических элементов в коре Земли. Ввиду этого они представлены в логарифмическом масштабе.


\begin{figure}[ht] 
	\centering\small
	\unitlength=1mm
	\begin{picture}(120,60)
	\put(-10,0){\includegraphics[width=135mm]{Figures/Element_Abudance_Earth_Crust_10ppm.png}}
	\end{picture}
	\caption{Распространённость химических элементов в коре Земли, превышающая 10 ppm} По данным https://www.nndc.bnl.gov/nudat3
	\label{f:Element_Abudance_Earth_Crust_10ppm}
\end{figure}

Для лучшей иллюстрации На рис.~\ref{f:Element_Abudance_Earth_Crust_10ppm} представлены данные по распространённости химических элементов, превышающая 10 ppm (относительная величина $10^{-5}$ при распространённости кремния, равной $10^{5}$ ).

Принято условно выделять несколько групп элементов по их распространённости, а также исторически сложившиеся названия. 
\emph{Горные породы} формируют H, C, O, F, Na, Mg, Al, Si, P, S, Cl, K, Ca, Ti, Mn, Fe. \index{горные породы}
Как можно заметить, большая часть из них находится в первых периодах талицы Менделеева.

Противоположную позицию \emph{редких элементов} в основном занимают \emph{благородные металлы}: \index{благородные металлы} Ru, Rh, Pd, Os, Ir, Pt, Au, а также Te и Re.


Вопреки своему названию, \emph{редкоземельные элементы}: Sc, Y, La, Ce, Pr, Nd, Pm, Sm, Eu, Gd, Tb, Dy, Ho, Er, Tm, Yb, Lu, не являются столь уж редкими. 
\index{редкоземельные элементы}
Как правило, редкоземельные элементы встречаются в природе совместно. Суммарное содержание редкоземельных элементов составляет более 100 г/т. 

\begin{figure}[ht] 
	\centering\small
	\unitlength=1mm
	\begin{picture}(120,65)
	\put(-10,0){\includegraphics[width=135mm]{Figures/Element_Abudance_Earth_Crust Isotopes.png}}
	\end{picture}
	\caption{Распространённость изотопов химических элементов в коре Земли} По данным https://www.nndc.bnl.gov/nudat3
	\label{f:Element_Abudance_Earth_Crust Isotopes}
\end{figure}

Распределение изотопов  химических элементов в земной коре весьма разнообразно. Для конкретного химического элемента содержание различных изотопов может отличаться на несколько порядков.

На рис.~\ref{f:Element_Abudance_Earth_Crust Isotopes} представлены данные по распространённости изотопов химических элементов в коре Земли по данным https://www.nndc.bnl.gov/nudat3.

\begin{figure}[ht] 
	\centering\small
	\unitlength=1mm
	\begin{picture}(120,65)
	\put(-10,0){\includegraphics[width=135mm]{Figures/Element_ratio max min abudances_Earth_Crust Isotopes.png}}
		\end{picture}
		\caption{Отношение распространённости изотопов химических элементов в коре Земли} По данным https://www.nndc.bnl.gov/nudat3
		\label{f:Element_ratio max min abudances_Earth_Crust Isotopes}
\end{figure}

На рис.~\ref{f:Element_ratio max min abudances_Earth_Crust Isotopes} представлены данные по отношениям распространённости изотопов (максимальной к минимальной) химических элементов в коре Земли по данным https://www.nndc.bnl.gov/nudat3.
Отдельно выделены элементы-моноизотопы и элементы, отсутствующие в коре Земли.

Важную информацию о распространённости изотопов дают данные об их нуклеосинтезе --- \S\ref{s:TableNucleosynthesis}.
На рис.~\ref{f:Element_Abudance_Earth_Crust NucleosynthesisProcesses} представлены данные по распространённости изотопов химических элементов в коре Земли по данным https://www.nndc.bnl.gov/nudat3.
Также на легенде данных указаны типы происхождения всех изотопов.
Изучение данных рис.~\ref{f:Element_Abudance_Earth_Crust NucleosynthesisProcesses} даёт богатую информацию как об абсолютных величинах распространённости различных элементов, так и об соотношениях распространённости их изотопов.

Например, можно заметить, что изотопы, возникшие в результате \index{p-процесс} p-процессов, как правило имеют существенно более низкую распропространённость.

\begin{figure}[ht] 
	\centering\small
	\unitlength=1mm
	\begin{picture}(120,65)
	\put(-10,0){\includegraphics[width=135mm]{Figures/Element_Abudance_Earth_Crust NucleosynthesisProcesses.png}}
	\end{picture}
	\caption{Распространённость изотопов химических элементов в коре Земли} По данным https://www.nndc.bnl.gov/nudat3 
	\label{f:Element_Abudance_Earth_Crust NucleosynthesisProcesses}
\end{figure}



{\bf Лёгкие элементы.}
Рассмотрим три первых периода таблицы Менделеева, с зарядами от 1 (H) до 2 (He), с 3 (Li) до 10 (Ne) и с 11 (Na) до 18 (Ar).

На рис.~\ref{f:H_Ar_Abudance_3d} представлены относительные распространённости изотопов химических элементов $Z$=1---18 в коре Земли.

Максимальная высота столбика соответствует 100\% относительной распространённости изотопа.
Такую или практически такую высоту имеют изотопы $^1H$, $^4He$, $^9Be$, $^{19}F$, $^{40}Ar$. 

\begin{figure}[ht] 
	\centering\small
	\unitlength=1mm
	\begin{picture}(120,48)
		\put(0,0){\includegraphics[width=120mm]{Figures/H_Ar_Abudance_3d.png}}
		\put(10,38){H} \put(16,39){He}
		\put(78,45){Cl} \put(84,46){Ar}
	\end{picture}
	\caption{Относительные распространённости изотопов химических элементов $Z$=1-18 в коре Земли} 
	\label{f:H_Ar_Abudance_3d}
\end{figure}

{\bf Тяжёлые элементы.} На рис.~\ref{f:Cs_U_Abudance_3d} представлены относительные распространённости изотопов химических элементов $Z$=55---92 в коре Земли.

\begin{figure}[ht] 
	\centering\small
	\unitlength=1mm
	\begin{picture}(120,55)
		\put(0,0){\includegraphics[width=120mm]{Figures/Cs_U_Abudance_3d.png}}
		\put(25,25){Cs} \put(80,47){Pb}
		\put(92,53){Th} \put(97,55){U}
	\end{picture}
	\caption{Относительные распространённости изотопов химических элементов $Z$=55-92 в коре Земли} 	\label{f:Cs_U_Abudance_3d}
\end{figure}

Для тяжёлых элементов доминирование отдельных изотопов редко.

	\section{Редкие элементы} \label{RareElementsMono}	



В табл.~\ref{t:IsotopeCounts} приведены данные о количестве стабильных элементов для различных химических элементов.
В первом горизонтальном блоке сгруппированы элементы, имеющие только один стабильный изотоп.
Всего их 24, что составляет  весьма небольшую долю общего количества элементов.


Как показано на рис.~\ref{f:Isotopes_per_Elements Even-Odd}, моноизотопные элементы в ряде химических элементов не встречаются один за другим, и всегда соседствуют с немоноизотопными элементами.

\subsection{<<Отсутствующие>> элементы} \label{AbsentElements}

Интересно разобраться в деталях этой закономерности. Наиболее ярко эффект выражен для двух элементов, вообще не вошедших в табл.~\ref{t:IsotopeCounts}. Это технеций Tc и прометий Pm, все изотопы которых представлены на Земле только в следовых количествах. В связи с этим в таблице Менделеева не приводятся их массы.

\index{правило Щукарева—Маттауха}
\index{технеций}	
\index{прометий}	

В \S\ref{IsobarsIsotopes} приводится эмпирическое правило Щукарева—Маттауха о невозможности одновременного существования стабильных изобар, заряды ядер которых отличаются на единицу.
%Правило Щукарева-Маттауха не следует понимать буквально. 
Обсудим примеры, которые часто приводят для его иллюстрации. 


\paragraph{Случай технеция.}
Элемент технеций был открыт достаточно поздно (в 1937 году), в сравнении с окружающими его молибденом и рутением. 



Технеций был получен путем облучения молибдена нейтронами (технические подобности --- \cite{Mausolf2021}). В дальнейшем, технеций был обнаружен в спектрах звезд и осколках деления урана \cite{Technetium2017}. 

По правилу Щукарева-Маттауха, так как молибден и рутений имеют стабильные изотопы с массовыми числами $92, 94, 95, 96, 97, 98, 100$ и $96, 98, 99, 100, 101, 102, 104$, стабильный технеций (и его изотопы), не могут существовать.  На рис.~\ref{f:IUPAC Table Mo Tc Ru} представлен фрагмент таблицы Менделеева IUPAC.

\begin{figure}[ht] 
	\centering\small
	\unitlength=1mm
	\includegraphics[width=40mm]{Figures/IUPAC Table Mo Tc Ru.png} 
	\caption{Изотопы Tc, Mo и Ru} --- https://applets.kcvs.ca/IPTEI/IPTEI.html/
	\label{f:IUPAC Table Mo Tc Ru}
\end{figure}

\begin{figure}[ht] 
	\centering\small
	\unitlength=1mm
	\includegraphics[width=60mm]{Figures/MoTcRuNZmarked.png} 
	\caption{Стабильные изотопы Mo и Ru --- чёрные квадраты} --- https://www.nndc.bnl.gov/nudat3 
	\label{f:T half time Mo Tc Ru.png}
\end{figure}




Однако, справедливости ради стоит отметить, что <<стабильное ядро>> понятие условное, и стабильность ядра зависит от того, какой период полураспада мы будем считать достаточно длительным. На Рис.~\ref{f:NZHalf-time} стабильными считаются ядра с периодом полураспада $T_{1/2}>3 \cdot 10^{24} $лет. Период полураспада относительно долгоживущих изотопов технеция (A=97-99) составляет $ (0.2-4.2) \cdot 10^{6}$ лет, что действительно мало, в сравнении с временем жизни протона, но вполне достаточно для его подробного изучения. В настоящее время технеций в больших количествах образуется в реакторах, как продукт деления урана.

\begin{figure}[ht] 
	\centering\small
	\unitlength=1mm
	\includegraphics[width=100mm]{Figures/Tc with neighbors1d.png} 
	\caption{Графики энергий бета-распада и электронной конверсии для изотопов Tc, Mo и Ru} --- https://www.nndc.bnl.gov/nudat3/. 
	\label{f:Tc with neighbors}
\end{figure}


На рис.~\ref{f:Tc with neighbors} приведены графики энергий бета-распада и электронной конверсии для изотопов технеция с электрическим зарядом ядра $Z=43$ и его ближайших соседей, молибдена  $Z=42$ и рутения  $Z=44$.

Для изотопов каждого элемента приведены две ветви энергий: для  $Q_{\beta}$ бета-распада и электронной конверсии $Q_{EC}$ из данных https://www.nndc.bnl.gov/nudat3/. Они даны соотвественно значками $\triangledown$, что соответствует уменьшению заряда ядра изотопа, и $\triangle$ в случае увеличения заряда. Пунктирной линией показан уровень нуля энергии. В случае наличия стабильных изотопов, им формально присвоено значение энергии -1000 кэВ. 

На стыке ветвей $Q_{\beta}$  и $Q_{EC}$ возможно существование стабильных изотопов.
Наименьшие энергии распадов имеют изотопы технеция с числом нуклонов $A$, равным 97-99.
Однако для всех изотопов технеция положительны энергии или $Q_{\beta}$ бета-распада, или электронной конверсии $Q_{EC}$. 
Для изотопа $^{99}Tc$ энергия $Q_{\beta}$  отрицательна, но положительна $Q_{EC}$.
В то же время у  молибдена и рутения есть стабильные изотопы в диапазоне $A$ от 95 до 102. 

\begin{figure}[ht] 
	\centering\small
	\unitlength=1mm
	\includegraphics[width=50mm]{Figures/Tc half time.png} 
	\caption{Периоды полураспада изотопов Tc} --- https://www.nndc.bnl.gov/nudat3 
	\label{f:Tc half time}
\end{figure}

На рис.~\ref{f:Tc half time} приведены графики периодов полураспада изотопов технеция. Видно, что времена жизни изотопов с $A$, равным 97-99 весьма велики. Однако у молибдена и рутения они существенно больше, что при малом генезисе изотопов технеция приводит к их изьятию из коры Земли на геологических масштабах времени.

\begin{figure}[ht] 
	\centering\small
	\unitlength=1mm
	\includegraphics[width=50mm]{Figures/MoAbudanceRuAbudance.png} 
	\caption{Распространенность изотопов молибдена и рутения} --- данные https://www.nndc.bnl.gov/nudat3/
	\label{f:MoAbudanceRuAbudance}
\end{figure}

Возвращаясь к правилу Щукарева-Маттауха, можно предожить иллюстрацию рис.~\ref{f:MoAbudanceRuAbudance} <<покрытия>> изотопами молибдена и рутения всех вакантных мест для стабильных изотопов технеция. Получается, что в окрестности наиболее энергетически выгодных конфигураций изотопов технеция имеются еще более выгодные конфигурации элементов-соседей.
Подробности можно найти в публикации \cite{Technetium2017}.

\paragraph{Случай прометия.}

Другим примером <<отсутствующего в природе>> элемента является прометий. \index{прометий}
В отличие от технеция, этот элемент не только слабо представлен в земной коре, но и имеет весьма малые времена полураспада.

На рис.~\ref{f:IUPAC Table  Nd Pm Sm} представлен фрагмент таблицы Менделеева IUPAC.
Как и в случае технеция, элементы с электрическим зарядом $Z=Z(Pm) \pm 1$ имеют по нескольку стаблильных изотопов.

\begin{figure}[ht] 
	\centering\small
	\unitlength=1mm
	\includegraphics[width=40mm]{Figures/IUPAC Table Nd Pm Sm.png} 
	\caption{Изотопы Nd, Pm  и Sm} --- hhttps://applets.kcvs.ca/IPTEI/IPTEI.html/
	\label{f:IUPAC Table  Nd Pm Sm}
\end{figure}

На рис.~\ref{f:T half time MNd Pm Sm.png} представлен фрагмент $NZ$-диаграммы.
Черными квадратами показаны стабильные изотопы неодима  $Z=60$ и самария $Z=62$.

\begin{figure}[ht] 
	\centering\small
	\unitlength=1mm
	\includegraphics[width=60mm]{Figures/T half time Nd Pm Sm.png} 
	\caption{Стабильные изотопы Nd, Pm  и Sm --- чёрные квадраты} --- https://www.nndc.bnl.gov/nudat3
	\label{f:T half time MNd Pm Sm.png}
\end{figure}
Обратимся к численным данным.

На рис.~\ref{f:Pm with neighbors} приведены графики энергий бета-распада и электронной конверсии для изотопов прометия с электрическим зарядом ядра $Z=61$ и его ближайших соседей, неодима  $Z=60$ и самария $Z=62$.

\begin{figure}[ht] 
	\centering\small
	\unitlength=1mm
	\includegraphics[width=100mm]{Figures/Pm with neighbors1d.png} 
	\caption{Графики энергий бета-распада и электронной конверсии для изотопов Pm, Nd и Sm} --- https://www.nndc.bnl.gov/nudat3/
	\label{f:Pm with neighbors}
\end{figure}

Пунктирной линией показан уровень нуля энергии. В случае наличия стабильных изотопов, им формально присвоено значение энергии -1000 кэВ. 
На стыке ветвей $Q_{\beta}$  и $Q_{EC}$ возможно существование стабильных изотопов.
Наименьшие энергии распадов имеют изотопы прометия с числом нуклонов $A$, равным 145 и 147.
Однако для всех изотопов технеция положительны энергии как $Q_{\beta}$ бета-распада, иак и электронной конверсии $Q_{EC}$. 
В то же время у неодима и самария есть стабильные изотопы в диапазоне $A$ от 142 до 150. 

На рис.~\ref{f:Pm half time} приведён график периода полураспада изотопов Pm для его различных изотопов. 

\begin{figure}[ht] 
	\centering\small
	\unitlength=1mm
	\includegraphics[width=50mm]{Figures/Pm half time.png} 
	\caption{Периоды полураспада изотопов Pm} --- https://www.nndc.bnl.gov/nudat3/
	\label{f:Pm half time}
\end{figure}

\section{Элементы ---  моноизотопы} \label{ElementsMono}	

Рассмотрим теперь  элементы --- моноизотопы. В табл.~\ref{t:MonoIsotopeCounts} приведён первый блок
табл.~\ref{t:IsotopeCounts}. 

\begin{table}
	{\scriptsize 
		\begin{tabular}{cp{6cm}cc}
			\hline
			Число  & Элементы, $ \ _{Z}E^{N_1, N_2, \ldots, N_k}$, $z \ $ --- заряд ядра,  &  Число  & Общее  \\
			стабильных  & $N_j$ --- атомный вес изотопа $j$ &  элементов & число  \\
			изотопов, $k$ & ~ &  ~ &  изотопов \\
			\hline 
			~ & ~ &  ~ &  ~\\
			1 &  $_{4}Be^{9}, \, _{9}F^{19}, \, _{11}Na^{23}, \, _{13}Al^{27}, \, _{15}P^{31}, \,  _{21}Sc^{45},  \,  _{23}V^{51},  $ & 24 & 24 \\ [1mm]
			~ &  $_{25}Mn^{55}, \, _{27}Co^{59}, \, _{33}As^{75}, \, _{39}Y^{89}, \, _{41}Nb^{93}, \,  _{45}Rh^{103},  $ & ~  & ~ \\ [1mm]
			~ &  $ _{53}I^{127}, \, _{55}Cs^{133}, \, _{57}La^{139}, \, _{59}Pr^{141}, \, _{65}Tb^{159},  \, _{67}Ho^{165}, $ & ~  & ~ \\ [1mm]
			~ &  $ _{69}Tm^{169}, \, _{71}Lu^{175}, \, _{75}Ta^{181}, \, _{79}Au^{197}, \, _{83}Bi^{109}  $ & ~  & ~ \\ [3mm]
%			\vdots & \vdots &  \vdots &  \vdots \\		
			\hline
%			~ & ~ &  ~ &  ~\\
%			~ & ~ & 81 & 276
		\end{tabular}
	}
	\caption{Cтабильные  элементы --- моноизотопы. \\
		Часть табл.~\ref{t:IsotopeCounts}}
	\label{t:MonoIsotopeCounts}
\end{table} 

Всего имеется 24 элемента---моноизотопа. Рассмотрим несколько примеров, подобно тому как это сделано в \S\ref{RareElementsMono}. 

В \S\ref{PeriodicTableElemIsotopes} <<Периодическая таблица элементов и изотопов>>
на рис.~\ref{f:PeriodicTable} представлена таблица Менделеева элементов и изотопов \cite{IUPAC}

\subsection{Случай бериллия.} \label{LiBeB}

Бериллий  ---  наиболее легкий химический элемент с одним стабильным изотопом.
Известны 12 изотопов бериллия, от $Be^5$ и $Be^{16}$.

Единственным стабильным изотопом является $Be^9$, его природная изотопная распространённость равна 100 \%. Таким образом, бериллий практически моноизотопный элемент. Также в следовых количествах присутствуют $Be^7$ и $Be^{10}$, возникающие в атмосфере в результате ядерных реакций, индуцированных космическим излучением. Самым долгоживущим радиоизотопом является $Be^{10}$ с периодом полураспада 1.387 млн лет.

Ближайшими к бериллию элементами являются литий и бор. Они имеют по 2 стабильных изотопа. 

\subsection{Случай фтора.} \label{OFNe}

Фтор  --- второй после бериллия химический элемент с одним стабильным изотопом.

\begin{figure}[ht] 
	\centering\small
	\unitlength=1mm
	\includegraphics[width=70mm]{Figures/F with neighbors1d.png} 
	\caption{Графики энергий бета-распада и электронной конверсии для изотопов O, F и Ne} --- https://www.nndc.bnl.gov/nudat3/
	\label{f:F with neighbors}
\end{figure}

Известны изотопы фтора с массовыми числами от 14 до 31 и 2 ядерных изомера.
Единственным стабильным изотопом фтора является $F^{19}$, его природная изотопная распространённость равна 100 \%. Самым долгоживущим радиоизотопом является $F^{18}$ с периодом полураспада 110 минут.

Ближайшими к фтору элементами являются кислород и неон. Они имеют по 3 стабильных изотопа. 



На рис.~\ref{f:F with neighbors} приведены графики энергий бета-распада и электронной конверсии для изотопов O, F и Ne. Все изотопы фтора, кроме $F^{19}$, неустойчивы по отношению к бета-распаду или электронной конверсии.
При этом кислород и неон имеют стабильные изотопы с атомными массами 16-18 и 20-22 соответственно.

\section{Элементы с двумя стабильными изотопами} \label{ElementsDuo}	

Вернемся к рис.~\ref{f:Isotopes_per_Elements Even-Odd} и табл.~\ref{t:IsotopeCounts}.
Подробнее рассмотрим элементы с двумя стабильными изотопами.

На Земле имеются 19 элементов с двумя стабильными изотопами. Они приведены в табл.~\ref{t:IsotopeDuoCounts}, которая представляет второй блок табл.~\ref{t:IsotopeCounts}.

\begin{table}[h]
	{\scriptsize 
		\begin{tabular}{cp{6cm}cc}
			Число  & Элементы, $ \ _{Z}E^{N_1, N_2, \ldots, N_k}$, $z \ $ --- заряд ядра,  &  Число  & Общее  \\
			стабильных  & $N_j$ --- атомный вес изотопа $j$ &  элементов & число  \\
			изотопов, $k$ & ~ &  ~ &  изотопов \\
			\hline 
			~ & ~ &  ~ &  ~\\
			2 &  $_{1}H^{1, 2},  \,  _{2}He^{3, 4},  \,  _{3}Li^{6, 7},  \,  _{5}B^{10, 11},  \, _{6}C^{12,13},  \, _{7}N^{14,15}, $ & 19 & 38 \\ [1mm]
			~ &  $_{17}Cl^{35, 37},  \,  _{19}K^{39, 41},  \,  _{29}Cu^{63, 65},  \,  _{31}Ga^{69, 71},  \, _{35}Br^{79,81},  $ & ~ & ~\\ [1mm]
			~ &  $_{37}Rb^{85, 87},  \,  _{47}Ag^{107, 109},  \,  _{49}In^{113, 115},  \,  _{63}Eu^{151, 153},    $ & ~ & ~\\ [1mm]
			~ &  $ _{76}Re^{185, 187}, \,  _{77}Ir^{191, 193},  \,  _{81}Tl^{203, 205} $ & ~ & ~ \\ [1mm]
			\hline 
		\end{tabular}
	}
	\caption{Элементы с двумя стабильными изотопами. \\
		Часть табл.~\ref{t:IsotopeCounts}}
	\label{t:IsotopeDuoCounts}
\end{table} 

Эти элементы можно разделить на две группы. К первой относятся элементы с изотопами имеющими две смежные по порядку атомные массы:  $H^{1, 2}$,  $He^{3, 4}$,  $  Li^{6, 7}$,  $ B^{10, 11}$,  $ C^{12,13}$,  $N^{14,15}$, $K^{39, 41}$.

В табл.~\ref{t:IsotopeDuoGroup1} 
представлены элементы этой группы. 

\begin{table}
\begin{center}
	{\footnotesize 
\begin{tabular}{|c|c|c|}
	\hline
  Элемент &z,n  & Изотопная распространённость на Земле,\% \\
     \hline 
  Водород  & 1,1 & 99,98\\
 & 1,2 & 0,02 \\  
    \hline
Гелий & 2, 1& 0,00001 \\
 &2,2 & 99,99999\\
     \hline
Литий  & 3,3 & 7,9\\
 & 3,4 & 92,1\\
      \hline
Бор & 5,5& 18,8\\
 & 5,6& 81,2 \\
       \hline
 Углерод  & 6,6 & 98,9 \\  
 & 6,7 & 1,1 \\
        \hline
Азот  & 7,7 & 99,62\\
 & 7,8 & 0,38\\
 \hline
 \hline
 Калий & 19, 39 &  93,258 \\
 & 19, 41	&  6,730 \\
 \hline 
\end{tabular}
	}
\end{center}
\caption{Элементы с двумя стабильными изотопами. Элементы с существенно разной распространённостью изотопов}
\label{t:IsotopeDuoGroup1}
\end{table} 

В этой группе исключительное положение занимает калий. По группированию стабильных изотопов он принадлежит к другой группе, в табл.~\ref{t:IsotopeDuoGroup2}, поскольку атомный веса его стабильных изотопов отличается на 2. При этом  
минорный изотоп имеет имеет распространённость всего 6,73\%. 
Далее этот случай будет рассмотрен в параграфе про производство аргона \S\ref{s:Argon}.

\begin{figure}[ht] 
	\centering\small
	\unitlength=1mm
	\begin{picture}(120,42)
	\put(20,0){\includegraphics[width=60mm]{Figures/Elements Isotope Count =2Lo.png}}
	\end{picture}
	\caption{Элементы с двумя стабильными изотопами \\ Часть рис.~\ref{f:Isotopes_per_Elements Even-Odd}} 
	\label{f:Isotopes_per_Elements Even-Odd2}
\end{figure}

Во вторую группу входят элементы с массой изотопов, отличающимися на 2:
$Cl^{35, 37}$,  ($K^{39, 41}$),  $Cu^{63, 65}$,  $Ga^{69, 71}$,  $Br^{79,81}$,  
$Rb^{85, 87}$,  $ Ag^{107, 109}$, $In^{113, 115}$,   $Eu^{151, 153}$,    $Re^{185, 187}$, $ Ir^{191, 193}$,  $Tl^{203, 205}$. 

\begin{table}	
	\begin{center}
		{\footnotesize 
			\begin{tabular}{|c|c|c|}
				\hline
				Элемент &z, A & Изотопная распространённость на Земле,\% \\
				\hline 
				Хлор	& 17, 35 &  75,78\\
 &17, 37 & 24,22\\				
				 \hline
 	Медь &  29, 63 & 69,1\\
 & 29, 65 &  30,9 \\
 \hline
 	Галлий & 31,69 & 60,11 \\
& 31, 71 &  39,89 \\
\hline
Бром & 35,	79 & 50,56 \\
& 35, 81 & 49,44 \\
\hline
Сурьма	& 51, 121 & 57,36\\
& 51, 123	& 42,64 \\
\hline
	Европий	& 63, 151 & 52,2 \\
& 63, 153	& 47,8 \\
\hline
Иридий	& 77, 191 & 37,3 \\
& 77, 193	& 62,7 \\
\hline
	Таллий	& 81, 203 & 29,52 \\
& 81, 205	& 70,48 \\
\hline
			\end{tabular}
		}
	\end{center}
	\caption{Элементы с двумя стабильными изотопами. Элементы с сопоставимой распространённостью изотопов}
	\label{t:IsotopeDuoGroup2}
\end{table} 

\begin{figure}[ht] 
	\centering\small
	\unitlength=1mm
	\begin{picture}(120,65)
	\put(-7,0){\includegraphics[width=130mm]{Figures/Elements Isotope Count =2up.png}}
	\end{picture}
	\caption{Элементы с двумя стабильными изотопами. \\ Часть рис.~\ref{f:Isotopes_per_Elements Even-Odd}} 
	\label{f:Isotopes_per_Elements Even-Odd2up}
\end{figure}

Для первой группы характерно доминирование распространённости одного из изотопов. Во второй группе мажорный и минорный изотопы имеют сопоставимые величины распространённости.

\begin{figure}[ht] 
	\centering\small
	\unitlength=1mm
	\begin{picture}(120,65)
	\put(-10,0){\includegraphics[width=135mm]{Figures/Elements Isotope Count =2Ratio Abudances.png}}
	\end{picture}
	\caption{Отношение распространённости изотопов химических элементов в коре Земли} По данным https://www.nndc.bnl.gov/nudat3
	\label{f:Elements Isotope Count =2Ratio Abudances}
\end{figure}

На рис.~\ref{f:Elements Isotope Count =2Ratio Abudances} представлена часть рис.~\ref{f:Element_ratio max min abudances_Earth_Crust Isotopes} на котором выделены элементы, имеющих 2 изотопа.
Синим цветом выделены данные по элементам первой группы, состоящей из лёгких элементов. Их изотопы имеют последовательные значения атомных номеров (электрических зарядов) и имеют разную чётность. За исключением бора, различие распространённостей этих изотопов очень велико, до 6 порядков для изотопов гелия.
Красным цветом показаны данные по второй группе, состоящей из более тяжёлых элементов. Их изотопы имеют одинаковую чётность и в этом случае распространённости отличаются менее, чем на порядок.

Отметим исключения. Литий (A = 6, 7) и бор (A = 10, 11) имеют разную чётность стабильных изотопов, при этом отношение распространённостей изотопов этих элементов составляет 12 и 4. В то же время индий (A=113, 115) имеет отношение распространённостей изотопов, равное 22.

\subsection{Случай хлора.} \label{Сl3537}

Рассмотрим в качестве примера ситуацию с изотопами хлора. Это самый лёкий из элементов, имеющих 2 изотопа с сопоставимыми распространённостями. 

Ещё в XIX веке было известно, что атомная масса хлора заметно отличается от целочисленной. В частности, в аннотации пионерской публикации Д.И.\,Менделеева 1869 г.
<<Система химических элементов согласно их атомным весам и химическим свойствам>> \cite{Mendeleev1869}
рис.~\ref{f:1869Mendeleev}
был указан атомный вес, равный 35.5, что никак нельзя объяснить неточностью измерений.

\begin{figure}[ht] 
	\centering\small
	\unitlength=1mm
	\includegraphics[width=30mm]{Figures/T half time S Cl Ar.png} 
	\caption{Стабильные изотопы S Cl Ar --- чёрные квадраты} --- https://www.nndc.bnl.gov/nudat3 
	\label{f:T half time S Cl Ar}
\end{figure}

На рис.\ref{f:T half time S Cl Ar} чёрными квадратами представлены стабильные изотопы хлора и его соседей, серы и аргона.
Нестабильный изотоп $Cl^{36}$ имеет две моды превращений, бета-распад (98.1\%) и электронный захват (1.9\%).


\begin{figure}[ht] 
	\centering\small
	\unitlength=1mm
	\includegraphics[width=80mm]{Figures/Cl with neighbors1d.png} 
	\caption{Графики энергий бета-распада и электронной конверсии для изотопов Cl, S  и Ar} --- https://www.nndc.bnl.gov/nudat3/
	\label{f:Cl with neighbors1d}
\end{figure}

На рис.~\ref{f:Cl with neighbors1d} приведены графики энергий бета-распада и электронной конверсии для изотопов хлора, серы и аргона. Зелеными квадратами показаны стабильные изотопы хлора $Cl^{35}$, $Cl^{37}$. Синие квадраты соответствуют стабильным изотопам серы (32-34, 36), красные --- стабильным  изотопам аргона  (36, 38, 40).

Таким образом, изотоп $Cl^{36}$ переходит путём бета-распада в изотоп  $Ar^{36}$ или путём электронного захвата в изотоп  $S^{36}$ .

\subsection{Случай меди.} \label{Сu3537}

Помимо хлора, существенно нецелочисленные атомные массы были измерены в XIX в. и для ряда других элементов, в частности, для меди.

Конкретно, в \cite{Mendeleev1869}
рис.~\ref{f:1869Mendeleev}
был указан атомный вес, равный 63.4, чтотоже  никак  не объясняется неточностью измерений.

\begin{figure}[ht] 
	\centering\small
	\unitlength=1mm
	\includegraphics[width=30mm]{Figures/T half time Ni Cu Zn.png} 
	\caption{Стабильные изотопы Ni Cu Zn --- чёрные квадраты} --- https://www.nndc.bnl.gov/nudat3 
	\label{f:T half time  Ni Cu Zn}
\end{figure}

На рис.\ref{f:T half time  Ni Cu Zn} чёрными квадратами представлены стабильные изотопы хлора и его соседей, никеля и цинка.
Нестабильный изотоп $Cu^{64}$ имеет две моды превращений, бета-распад (38.5\%) и электронный захват (61.5\%).

\begin{figure}[ht] 
	\centering\small
	\unitlength=1mm
	\includegraphics[width=80mm]{Figures/Cu with neighbors1d.png} 
	\caption{Графики энергий бета-распада и электронной конверсии для изотопов Cu, Ni  и Zn} --- https://www.nndc.bnl.gov/nudat3/ 
	\label{f:Cu with neighbors1d}
\end{figure}

На рис.~\ref{f:Cu with neighbors1d} приведены графики энергий бета-распада и электронной конверсии для изотопов хлора, никеля и цинка. Зелеными квадратами показаны стабильные изотопы хлора $Cu^{63}$, $Cu^{65}$. Синие квадраты соответствуют стабильным изотопам никеля (58, 60-62, 64), красные --- стабильным  изотопам цинка  (64, 66-68, 70).

Таким образом, изотоп $Cu^{64}$ переходит путём бета-распада в изотоп  $Zn^{64}$ или путём электронного захвата в изотоп  $Ni^{64}$.

\section{Элементы с многими стабильными изотопами} \label{ElementMulti}	

Большое количество изотопов никак нельзя назвать редким свойством среди элементов.
Достаточно посмотреть на гистограмму распределения числа стабильных изотопов у химических элементов рис.~\ref{f:HIST Isotopes_per_Elements Even-Odd} в \S\ref{IsobarsIsotopes}.

\begin{figure}[ht] 
	\centering\small
	\unitlength=1mm
	\begin{picture}(120,35)
		\put(30,0){\includegraphics[width=50mm]{Figures/HIST Isotopes_per_Elements Even-Odd.png}}
	\end{picture}
	\caption{Гистограмма числа изотопов химических элементов в коре Земли} --- По данным https://www.nndc.bnl.gov/nudat3
	\label{f:HIST Isotopes_per_Elements Even-Odd1}
\end{figure}

На рис.~\ref{f:Isotopes_per_Elements Even-Odd} приведён график числа изотопов химических элементов в коре Земли с учётом чётности числа нуклонов. 

\begin{table}[!h]
	{\small 
		\begin{center}
			\begin{tabular}{|c|c|c|c|c|}
				\hline
				Число изотопов & N & Z & Примеры Табл.~\ref{t:IsotopeCounts}\\
				\hline
				166 & чётное & чётное & ~ \\
				55 & чётное & нечётное & 1 изотоп --- $_{9}F^{19} \ldots \, _{83}Bi^{109} $ \\
				~ & ~ & ~ &  2 изотопа --- $_{1}H^{1}  \ldots \, _{81}Tl^{203, 205}   $ \\
				47 & нечётное & чётное & ~  \\
				5 & нечётное & нечётное & $_{1}H^{2},   \,  _{3}Li^{6},  \,  _{5}B^{10},   \, _{7}N^{14},  \,  _{47}Ag^{109} $   \\
				\hline
			\end{tabular}
		\end{center}	
	}	
	\caption{Распределение естественных изотопов по правилу чётности нуклонов}
	\label{t:IsotopeOddEven1}
\end{table}

В \S\ref{IsobarsIsotopes} обсуждается 
распространённость изотопов в природе, то имеется ряд закономерностей \cite{Bekman}.
Оболочечная модель ядра (с. \pageref{ShellNucleiModel}) хорошо согласуется с эмпирическим правилом повышения устойчивости ядер,
содержащих чётное число протонов и нейтронов --- Табл.~\ref{t:IsotopeOddEven1}. Менее стабильны ядра с нечётным числом протонов, но
чётным числом нейтронов (и наоборот). Неустойчивыми являются ядра, в которых число протонов и
нейтронов нечётное. 




\begin{figure}[ht] 
	\centering\small
	\unitlength=1mm
	\begin{picture}(120,50)
	\put(5,0){\includegraphics[width=100mm]{Figures/Isotopes_per_Elements Even-Odd.png}}
	\end{picture}
	\caption{Число изотопов для химических элементов} --- По данным https://www.nndc.bnl.gov/nudat3
	\label{f:Isotopes_per_Elements Even-Odd1}
\end{figure}

Три и более изотопа имеют почти все изотопы с чётным числом нуклонов.  

Особенно большое количество изотопов имеют элементы Pd, Cd, Sn, Te, Xe, Ba с электрическим зарядом  46, 48, 50, 52, 54, 56:
от 6 до 10.

\subsection{Случай олова} \label{SnIsotopes}	
Олово --- абсолютный рекордсмен по числу стабильных изотопов: их целых 10!

\begin{figure}[ht] 
	\centering\small
	\unitlength=1mm
	\includegraphics[width=80mm]{Figures/T half time Sn In Sb.png} 
	\caption{Стабильные изотопы Sn In Sb --- чёрные квадраты} --- https://www.nndc.bnl.gov/nudat3 
	\label{f:T half time Sn In Sb}
\end{figure}

На рис.\ref{f:T half time  Sn In Sb} чёрными квадратами представлены стабильные изотопы олова и его соседей, индия и сурьмы.

\begin{figure}[ht] 
	\centering\small
	\unitlength=1mm
	\includegraphics[width=80mm]{Figures/Sn with neighbors1d.png} 
	\caption{Графики энергий бета-распада и электронной конверсии для изотопов Sn, In  и Sb} --- https://www.nndc.bnl.gov/nudat3/
	\label{f:Sn with neighbors1d}
\end{figure}

На рис.~\ref{f:Sn with neighbors1d} приведены графики энергий бета-распада и электронной конверсии для изотопов олова, индия и сурьмы. Зелеными квадратами показаны стабильные изотопы олова , $Sn^{112}$, $Sn^{114-120}$, $Sn^{122}$, $Sn^{124}$. Синие квадраты соответствуют стабильным изотопам индия (113, 115), красные --- стабильным  изотопам сурьмы (121, 123).

%\section{Устойчивость ядер химических элементов} \label{StabitityNucleiReview}

%\subsection{Лёгкие элементы} \label{LightStabitityNuclei}

%\subsection{Элементы средних масс} \label{MiddleStabitityNuclei}

%\subsection{Тяжёлые элементы} \label{HeavyStabitityNuclei}

		\section{Примеры нуклеосинтеза, отличного от солнечной системы}	
	
	Ввиду того, что звёзды различны по своей природе, распространённость элементов и изотопов на них различно.
	
	\paragraph{Карбоновые (углеродные) звёзды.}
	\index{звёзды карбоновые (углеродные)}
	Углеродная звезда содержит в атмосфере больше углерода, чем кислорода. Эти компоненты смешиваются в верхних слоях звезды, образуя монооксид углерода $CO$, который связывает весь кислород в атмосфере. Оставшийся углерод образует различные соединения, по которым эти звёзды и обнаруживаются. 	
	
	\paragraph{Технециевые звёзды.} \index{звёзды технециевые}
	На некоторых звёздах обнаружены линии поглощения технеция. Этот факт свидетельствует об относительно недавнем рождении этого элемента и его вполне макроскопическом количестве.
	
	
	Ядерный нуклеосинтез, рассмотренный в \S\ref{Nucleosynthesis}, в принципиальном плане определяет распространённость элементов и изотопов во Вселенной. 
	
	На Рис.~\ref{f:Nucleosynthesis_periodic_table} \S\ref{Nucleosynthesis} приведена версия таблицы Менделеева, на которой указаны цветами  механизмы происхождения различных элементов.
	
	
	Как следует из Рис.~\ref{f:Nucleosynthesis_periodic_table}, на Земле присутствуют элементы-потомки весьма разных предков, так что элементный состав Земли, во-первых богаче, чем продукция нуклеосинтеза Солнца, поскольку Солнечная система --- результат взрыва \emph{сверхновой звезды}. В таких случаях используется термин \emph{звёздное население I}. \index{звёздное население I} \index{звезда сверхновая}
	
	Во-вторых, между планетами и другими объектами Солнечной системы могут быть различия в распространённости элементов и их изотопном составе. Этот вопрос мы рассмотрим в \S\ref{IsotopePlanetology}.
	
	Изотопный состав планеты, вообще говоря, величина непостоянная. Причины для этого можно разделить на внутренние, связанные с радиоактивным распадом элементов планеты, и внешние, обязанные взаимодействию атмосферы и тела планеты с космическим излучением. Абсолютный вклад и тех, и других процессов на Земле невелик. Относительная малость этих эффектов не означает их незначимость. В частности, накопление радона как продукта распада урана, и образование тяжелых изотопов углерода и азота в атмосфере являются важными процессами.
	
	В доступном для неспециалиста изложении многочисленные примеры использования изотопных подписей приводятся в книге палеонтолога А.Ю.\,Журавлёва <<Сотворение Земли. Как живые организмы создали наш мир>> \cite{Zhuravlev2019}.	
	
	\subsection{Фракционирование изотопов в природе.}	\label{IsotopeFractioning}
	Фракционирование изотопов в природе, то есть, по массе, является проявлением очень общих закономерностей процессов, протекающих на планете. Поскольку химические свойства изотопов одного элемента очень близки друг к другу, то это предоставляет уникальную возможность исследования закономерностей эволюции планеты на больших масштабах времени. Изучая распространённость изотопов выбранного элемента в разных природных окружениях, на разных глубинах захоронения образцов, в разных по типу отложениях, в разных частях тел биологических останков, можно строить многомерные картины эволюции живого и минерального вещества Земли.
	
	%Далее, в \S\ref{EarthScience} и \S\ref{BioScience},  мы кратко осветим вопросы фракционирования изотопов в геологических структурах и в живой природе. 

Хорошее введение в вопрос даётся в презентациях А.К.\,Яковлева \cite{Iakovlev2013} и Д.В.\,Гричука \cite{Grichuk2022}. Первая из них относится к экологии, вторая --- к геологии.

Для заинтересованного читателя можно отметить книгу B.\,Fry <<Stable Isotope Ecology>> \cite{Fry2006} и сборник под редакцией K.A.\,Hobson и  L.I.\,Wassenaar  <<Tracking Animal Migration with Stable Isotopes>> \cite{HobsonWassenaar2018}. 


% Keith A. Hobson PhD (Editor), Leonard I. Wassenaar (Editor). Tracking Animal Migration with Stable Isotopes. Academic Press; 2nd edition (September 17, 2018), 268 pages

%  Brian Fry. Stable Isotope Ecology. 2006 Springer Science+Business Media, LLC. 390 pages. ISBN: 0387305130 ISBN-13(EAN): 9780387305134

	\subsection{Изотопная планетология} \label{IsotopePlanetology} \index{изотопная планетология}


Изотопная планетология --- новая ветвь науки, раздел планетологии. 
Согласно Википедии (https://ru.wikipedia.org/wiki/Планетология),
% https://ru.wikipedia.org/wiki/%D0%9F%D0%BB%D0%B0%D0%BD%D0%B5%D1%82%D0%BE%D0%BB%D0%BE%D0%B3%D0%B8%D1%8F
<<планетология --- это комплекс наук, изучающих планеты и их спутники, а также солнечную систему в целом и другие планетные системы с их экзопланетами. Планетология изучает физические свойства, химический состав, строение поверхности, внутренних и внешних оболочек планет и их спутников, а также условия их формирования и развития.>>

Планетология развилась из наук о Земле %(\S\ref{EarthScience}), 
и объединяет множество дисциплин, в частности, связанных с планетной геологией, космохимией и астробиологией.

Изучение других планет, в первую очередь Солнечной системы, позволяет лучше понять и нашу планету.
В частности, сравнение Земли и других внутренних планет Солнечной системы, Венеры и Марса, а также Луны, даёт информацию о влиянии различных факторов на формирование климата, тектоники и биосферы.

В частности, на Земле открыто около 5000 минералов, на Марсе --- почти на порядок меньше, а на Луне --- около 150 \cite{Zhuravlev2019}. При этом более половины минералов на Земле прямо или опосредовано связано с наличием на ней жизни. Архейские бактериальные сообщества начали преобразовывать протокору и постепенно запустили тектоническую активность. Живые существа в ходе своего метаболизма меняют изотопные балансы, в частности, углерода. По изотопным подписям различных геологических пород и включений в них, можно делать выводы о процессах, протекавших в разные геологические эпохи. На разных стадиях эволюции климата планет изотопная подпись разная. В бескислородные периоды живые организмы можно классифицировать по изотопным подписям углерода и серы. Например, метаногенез производится археями и бактериями. 

В качестве примера 	комплексного сравнительного исследования, привлекающего данные разных планет,
можно привести статью <<Изотопные следы активности раннего солнца>> \cite{IsotopeTraceEarlySun2022}, где приводятся данные по изотопным отншениям азота и углерода для ряда планет Солнечной системы и \emph{марсианского метеорита} \index{Allan Hills 84001 --- марсианский метеорит} Allan Hills 84001. 
Образование изотопа $^{15}N$ происходит в ядерных реакциях протонов и нейтронов c $^{16}O$. В атмосферах с малым содержанием кислорода $^{15}N$  образуется преимущественно при захвате нейтрона $^{14}N$. Изотоп $^{13}C$ образуется из $^{14}N$.	
\begin{table}[h!]
	\begin{center}	
		%		{\small
			\begin{tabular}{|c|cccc|}
				\hline 
				Изотоп. отн.  & Земля &  Марс  & Венера & ALH84001   \\
				\hline 
				$^{15}N / ^{14}N$ & $3.66 \pm 0.01$ & $5.8 \pm 0.4$ & $3.7 \pm 0.7$ & 3.875\\ [1mm]
				$^{13}C / ^{12}C$ & $11.23 \pm 0.05$ & $11.75 \pm 0.04$ & $12 \pm 2$ &  $11.75 \pm 0.09$ \\ [1mm]
				\hline 
			\end{tabular}
			\caption{Изотопные отношения для ряда планет} 
			{\small Солнечной системы и марсианского метеорита Allan Hills 84001, умноженные на $10^3$}
			\label{t:PlanetsIsotope}
			%	}
	\end{center}
\end{table} 

Как видно из данных Табл.~\ref{t:PlanetsIsotope}, имеются значимые различия в изотопном составе планет и метеоритов. 

Одним из важнейших факторов формирования климата планеты является \emph{карбонатно-силикатный геохимический цикл}, \index{карбонатно-силикатный геохимический цикл} который отвечает примерно за 80\% обмена углекислым газом между планетой и ее атмосферой \cite{2015Planetology}. 
<<Цикл начинается с образования угольной кислоты ($H_2CO_3$) в атмосфере, когда содержащийся в ней углекислый газ растворяется в капельках воды, выпадающей затем в виде дождевых осадков на земную поверхность. Угольная кислота вступает в реакцию с кальциево-силикатными минералами (соединения кальция, кремния и кислорода), находящимися на поверхности, и высвобождает ионы кальция и бикарбоната ($Ca^{++}$ и $HCO_3^{-}$ ), которые поступают в грунтовые и поверхностные воды и в конечном счете попадают в океан.

В океанах и морях ионы кальция и бикарбоната извлекаются морскими организмами для строительства своих скелетов, покровов, оболочек, а затем, после отмирания организма, на дне водоемов формируются карбонатные осадочные породы.

Согласно теории литосферных плит, объясняющей движение отдельных блоков литосферы, рано или поздно осадочные породы океанического дна должны погрузиться в недра Земли. Здесь в условиях высокой температуры и давления происходит \emph{карбонатный метаморфизм} \index{карбонатный метаморфизм} --- карбонат кальция соединяется с кремнием (кварцем), образуя силикатные породы и выделяя газообразный $СО_2$, возвращающийся в атмосферу через вулканы, срединно-океанические хребты, рифтовые зоны.

Карбонатно-силикатный цикл будет происходить до тех пор, пока в недрах Земли выделяется тепло. Если Земля остынет настолько, что движение плит прекратится, то это станет началом конца жизни на Земле.

Поскольку живые организмы играют важную роль в круговороте углекислого газа, так же как и важнейшим звеном карбонатно-силикатного цикла являются морские организмы, можно предположить, что биота несет главную ответственность за изменения климата Земли. Уменьшение содержания $СО_2$ в атмосфере в течение всей геологической истории Земли было прямым следствием биологического «вмешательства», без живых организмов развитие земного климата, возможно, могло пойти по другому пути.>>



%	2004 The Isotope Geochemistry and Cosmochemistry  of Magnesium.pdf\\2009 The distribution of short-lived radioisotopes in the early solar system and the chronology of asteroid accretion, differentiation, and secondary mineralization.pdf\\2010 Evdokimova Исследование взаимодействия атмосферы \ldots Марса.pdf\\2015 Тебиева Д.И. Планетология учебное пособие.pdf\\ 2015 Krasnopolsky Variations of the HDO-H2O ratio in the martian atmosphere.pdf\\ 2016 The MetNet vehicle A lander to deploy environmenta.pdf\\ 2017 DH ratios of the inner Solar System.pdf\\ 2017 ISOTOPE GEOCHEMISTRY FOR COMPARATIVE PLANETOLOGY OF EXOPLANETS.pdf\\ 2019 Arevalo - Mass spectrometry and planetary exploration  A brief review and future Journal of Mass Spectrometry .pdf\\ 2019 High-precision measurements of krypton and xenon isotopes with a new static-mode Quadrupole Ion Trap Mass Spectrometer.pdf\\ 2019 Thiemens Isotope Planetology Thesis.pdf\\ 2020 Multiple early-formed water reservoirs in the interior of Mars.pdf\\ 2021 Tellurium isotope cosmochemistry.pdf\\ 2022 Noble Gases and Stable Isotopes Track the Origin and Early Evolution of the Venus Atmosphere.pdf\\ 2022 Noble Gas Mass Spectrometry in Earth and Planetary.pdf\\ 2022 The end of the isotopic evolution of atmospheric xenon.pdf\\ 2022 Venus, the Planet Introduction to the Evolution of Earth's Sister Planet.pdf\\

	\section{Масштабы времени природных явлений} \label{TimeScales}
	
	В этом разделе мы рассмотрим масштабы времени для различных классов явлений.
	Такое рассмотрение необходимо для того, чтобы сопоставлять время жизни конкретных изотопов и других процессов на Земле, связанных как с планетой в её геологическом развитии, так и с жизнью и эволюцией биосферы.
	

	
	\subsection{Ядерные масштабы времени} \label{NuclearTimeScales}

	Природные процессы протекают с различной скоростью. В связи с этим, важно разобраться, какие изотопы могут о них <<рассказать>>.  Фракционирование изотопов по массе, скорости течения химических реакций и т.д. приводит к обогащению одних объектов и обеднению другими. Для надёжного обнаружения этого факта необходимо, чтобы <<время жизни>> изотопа, его период полураспада был сопоставим или превышал интересующий нас в конкретном случае интервал времени.
	
	Ядерные масштабы времени имеют колоссальные различия. Рисунок \ref{f:NZHalf-time} из \S\ref{ExpNuclData} не даёт полного представления степени этой изменчивости. Из легенды к графику на сайте https://www.nndc.bnl.gov/nudat3/ можно узнать, что цветовая легенда имеет размах
		\begin{equation} \label{NuclTimeScale}
		T \backsim 10^{-15} - 10^{15}   \text{ с}.
	\end{equation}

	Разброс времён жизни ядер по этой шкале составляет 30 порядков! Реально эта величина ещё больше, поскольку заметное количество изотопов имеет время жизни более $10^{25}$ с.
	
	Меньшая величина рисунке \ref{f:NZHalf-time} на соответствует периоду колебания видимого света --- см. материал И.П.Иванова <<Фемтосекунды: 1. Мир электронов и света>> https://elementy.ru/time/femto/femto-1.html. 
	\begin{equation} \label{NuclTimeMin}
		T = \frac{\lambda}{c}\approx \frac{0.4 \cdot 10^{-6} \text{ м}}{3 \cdot 10^{8} \text{ м/c} } \approx 10^{-15} \text{ с}.
	\end{equation}
	
	Также время между столкновениями электронов в металлах имеет порядок \emph{cкорости Ферми} $v_F \approx 10^6$ м/c. \index{скорость Ферми}
	(Квантовая теория (Л.К. Мартинсон, Е.В. Смирнов) --- http://fn.bmstu.ru/data-physics/library/physbook/tom5/ch6/texthtml/ch6\_5.htm) также имеет тот же порядок.
		\begin{equation} \label{NuclTimeMax}
		T = \frac{r_A}{v_F}\approx \frac{10^{-8} \text{ м}}{10^{6} \text{ м/c} } \approx 10^{-14} \text{ с}.
	\end{equation}
	
	Б\'{о}льшая величина времени жизни ядер $10^{15} \text{ с} \approx 10^{12} \text{ лет}$  намного превышает \emph{время жизни Вселенной} $\approx 14 \cdot 10^{9} \text{ лет}$. \index{время жизни Вселенной} 
	
	Для абсолюного любого  природного процесса существуют атомные ядра, время жизни которых существенно меньше, сопоставимы или намного больше временного масштаба этого явления.
	Таким образом, всегда найдутся изотопы, изучение отношения которых даст информацию об исследуемом процессе.
	
	Например, в \S\ref{IsotopePlanetology} 	в таблице \ref{t:PlanetsIsotope}		<<Изотопные отношения \ldots  для ряда планет Солнечной системы и марсианского метеорита Allan Hills 84001>>  приводятся отношения стабильных изотопов азота и углерода. Эти величины сформировались на масштабах жизни Солнечной системы.  

	Важным обстоятельством, помимо времён жизни изотопов ядер, является тот факт, что большинство элементов имеет два и более стабильных изотопа --- см. таблицу \ref{t:IsotopeCounts} <<Количество стабильных изотопов различных элементов на Земле>> \S\ref{IsobarsIsotopes}.
	Таким образом, даже на основании анализа распространённостей изотопов даже одного элемента во многих случаях можно получить важную информацию о природных явлениях.  

	\begin{figure}[ht] 
		\centering\small
	\unitlength=1mm
	\begin{picture}(110, 55)
	\put(10,0){\includegraphics[width=80mm]{Figures/PbThalf.png}}
		\put(5,25){\mbox{$lg T_{1/2}$, с}} 
	\put(50,0){\mbox{$A$, а.е.м.}} 
	\end{picture}	
	\caption{Время жизни изотопов свинца} 
	\label{f:PbThalf}
\end{figure}

	Рисунок \ref{f:PbThalf} иллюстрирует разнообразие времён жизни изотопов элемента на примере свинца. Электрический заряд $Z =82$, а число нейтронов $N= 96 \ldots 138$. Изотопы свинца с атомным весом 204, 206-208 стабильны. Другие изотопы имеют время жизни от $10^{-5}$ от $10^{15}$ с. Вариативность атомных весов свинца на Земле очень велика, как в различных породах, так и географически, см. \S\ref{s:Lead}.


	\subsection{Различные масштабы времени в эволюции Земли} \label{EarthTimeScales}


	Для нашей планеты геологические развитие и эволюция биосферы очень тесно связаны.  Геологическая активность, континенты, океаны и атмосфера,  климат и биота сосуществуют и влияют друг на друга. Подробно эта связь обсуждается в книге А.Ю.Журавлева <<Сотворение Земли. Как живые организмы создали наш мир>> \cite{Zhuravlev2019} .
	Поэтому приводимое далее разделение на геологические и биологические процессы в значительной степени условно. В этом смысле Земля уникальна,  а следовательно, уникальны и её временные циклы и, соответсвенно, изотопные портреты в разные эпохи развития.
	
	При этом Земля, по образному выражению Александра Чижевского, находится <<в объятиях Солнца>> \cite{Chizevsky2004}.
	На Землю влияет и Солнце, и в более широком смысле, вся Солнечная система.
% Чижевский А. Л. Земля в объятиях Солнца. — М.: Эксмо, 2004. — 928 с.	
	
	\subsubsection{Астрономические масштабы времени} \label{AstroTimeScales}

Солнечная система возникла около 4.5 млрд. лет назад.
Возраст Земли составляет 4.54 миллиарда лет по данным  радиоизотопной датировке одного из типов метеоритов ---   \index{хондриты} \emph{хондритов}.  \emph{Метеоритом} называют любой объект \index{метеорит} космического происхождения более 2 мм. Наиболее современным обзором (2023) является книга Т.Грегори <<Метеориты. Космические камни, создавшие наш мир>> \cite{Gregory2023}.	Ввиду того, что метеориты прибывают на Землю регулярно (в сутки на Землю падает несколько тонн метеоритов, или около 2 тысячи тонн в год), их изучение даёт богатую информацию о Космосе. 

В частности, обнаружены метеориты более древние, чем Солнечная система. Свежий пример - \emph{метеорит Erg Chech 002}, \index{Erg Chech 002}   его возраст составляет 4.57 млрд лет \cite{Krestianinov2023}. 
% Evgenii Krestianinov, Yuri Amelin, Qing-Zhu Yin et al. Igneous meteorites suggest Aluminium-26 heterogeneity in the early Solar Nebula. Nature Communications ( 2023) 14:4940 p.1-11.

Очень древние включения обнаружены в метеорите Murchison CM2, \index{Murchison CM2} их возраст достигает 7 млрд лет, значительно старше Солнечной системы \cite{Heck2020}.
% Philipp R. Heck, Jennika Greer, Levke Kööpa et al.  Lifetimes-of-interstellar-dust-from-cosmic-ray-exposure-ages-of-presolar-silicon-carbide. PNAS, January 13, 2020. 117 (4) 1884-1889 https://doi.org/10.1073/pnas.1904573117

Орбита Земли имеет эллиптическую форму и претерпевает постоянные изменения. У эллипса есть главное направление, линия от перигелия до афелия, а характеристика отличия от круга называется эксцентриситет. В настоящее время $\varepsilon \approx 0.017$. И направление, и эксцентриситет орбит периодически меняются: эллипс проворачивается, а также становится то более вытянутым, то совсем круглым. 
Колебания эксцентриситета орбиты Земли. Быстрые колебания имеют период около 100 тыс. лет, медленные --- в несколько раз больше, около 300 тыс. лет. 

Угол наклона земной оси вращения к плоскости эклиптики меняется в диапазоне $22.1-24.5 ^{\circ}$ на масштабе в 40 т. лет, орбита Земли также имеет нутации. %Расчёт этих движений весьма сложен.
Достаточно популярное изложение можно найти в статье \cite{Decart2003} по ссылке https://www.astronet.ru/db/msg/1195760.
<<Ось вращения Земли испытывает смещения двух видов: движение с периодом примерно 26 тысяч лет называется прецессией, а более короткие движения, которые накладываются на прецессионное движение и периоды которых лежат в интервале от 18 с половиной лет до нескольких суток нутацией. Причиной прецессионно-нутационного движения оси вращения Земли является притяжение экваториального избытка масс Земли Луной и Солнцем. Силы притяжения пытаются совместить плоскость экватора Земли с плоскостью ее орбиты, однако из-за вращения Земли этого не происходит.>> 

	
	\subsubsection{Геохронологическая шкала истории Земли} \label{GeoTimeScales}

Геохронологическая (стратиграфическая) шкала  --- геологическая временная шкала истории Земли, применяемая в геологии и палеонтологии. \index{геохронологическая (стратиграфическая) шкала}
Шкала имеет масштабы времени в сотни тысяч и миллионы лет. 

Началом шкалы служит возраст \emph{метеорита Альенде}, который очень тщательно изучен. Его возраст $ 4568.5 \pm 0.5 $ млн лет \cite{Allende2007}. \index{метеорит Альенде}
% Amelin Y., Krot A. N. Pb isotopic age of the Allende chondrules.  Meteoritics & Planetary Science, Volume 42, Issue 1321, pp. 1321-1335. 2007 DOI: 10.1111/j.1945-5100.2007.tb00577.
Время формирования Земли как планеты может быть позже этой даты на миллионы и даже многие десятки миллионов лет.

\begin{itemize}
	\item 2.45 млрд лет — кислородная катастрофа — исчезли почти все анаэробы;
	\item 440 млн лет назад — ордовикско-силурийское вымирание — исчезло более 60 \% видов морских беспозвоночных;
	\item 364 млн лет назад — девонское вымирание — численность видов морских организмов сократилась на 50\%;
	\item 251.4 млн лет назад — «великое» пермское вымирание, самое массовое вымирание из всех, приведшее к исчезновению более 95 \% видов всех живых существ;
	\item 199.6 млн лет назад — триасовое вымирание — в результате которого вымерла, по меньшей мере, половина известных сейчас видов, живших на Земле в то время;
	\item 66 млн лет назад — мел-палеогеновое вымирание — массовое вымирание, уничтожившее шестую часть всех видов, в том числе и нептичьих динозавров.
	\item 33.9 млн лет назад — эоцен-олигоценовое вымирание — уступает в масштабах предыдущим 5 массовым вымираниям, значительно изменился состав морской и наземной флоры и фауны
	\item 130 тыс. лет назад — настоящее время: позднеплейстоцен — голоценовое вымирание[1] — массовое вымирание, вызванное в значительной степени человеческой деятельностью, наиболее пострадала мегафауна.
\end{itemize}

\begin{figure}[ht] 
	\centering\small
	\unitlength=1mm
	\begin{picture}(110, 50)
	\put(17,0){\includegraphics[width=80mm]{Figures/Extinction_intensity.png}}
	\put(-2,45){\mbox{Количество}}	
	\put(-2,42){\mbox{видов}}
	\put(40,-3){\mbox{Возраст, миллионов лет назад}}				
	\put(30,28){\mbox{Ор-сил}}	
	\put(42,24){\mbox{Дев}}	
	\put(57,41){\mbox{Пм-тр}}		
	\put(63,28){\mbox{Тр-юр}}
	\put(80,28){\mbox{Мел-пал}} 
	%		\put(50,0){\mbox{$A$, а.е.м.}} 
	\end{picture}	
	\caption{Массовые вымирания на Земле} 
	\label{f:ExtinctionIntensity}
\end{figure}
Последующее время в истории Земли было разделено на различные временные интервалы. Их границы проведены по важнейшим событиям, происходившим тогда.
Граница между эрами фанерозоя проходит по крупнейшим эволюционным событиям --- глобальным вымираниям биоты. Палеозой отделён от мезозоя крупнейшим за историю Земли пермо-триасовым вымиранием видов. Мезозой отделён от кайнозоя мел-палеогеновым вымиранием.


%	\subsubsection{Биологические масштабы времени} \label{BioTimeScales}
В статье \cite{18Orecords2005} представлен частотный анализ изотопа кислорода $^{18}O$ в бентосе.
\emph{Бентос} \index{бентос} (от греч. $\beta \varepsilon \nu \theta o \sigma$ — глубина) --- совокупность организмов, обитающих на грунте и в грунте дна водоёмов. 
\begin{figure}[ht] 
	\centering\small
	\unitlength=1mm
	\begin{picture}(110, 65)
	\put(15,0){\includegraphics[width=80mm]{Figures/18OSpectralDensity.png}}
	%		\put(5,25){\mbox{$lg T_{1/2}$, с}} 
	%		\put(50,0){\mbox{$A$, а.е.м.}} 
	\end{picture}	
	\caption{Спектральная плотность изотопа кислорода в бентосе} 
	\label{f:18Orecords}
\end{figure}
% 2005 Lisiecki Raymo A Pliocene-Pleistocene stack of 57 globally distributed benthic D18O records.pdf


\subsubsection{Тектоника континентов} \label{GeoTime}

Палеотектонические реконструкции на основе тектоники плит. --- https://edu.kpfu.ru/kek/geotektonika/4\_3.php

Проведение палеотектонических реконструкций представляет собой один из важных и сложных видов геологических исследований. Они необходимы для выявления особенностей формирования каких-либо структурных элементов или крупных территорий, истории их развития и взаимосвязи, прогнозирования новых месторождений полезных ископаемых и т.д. Суть данных реконструкций состоит в восстановлении строения и палеогеографических условий формирования отдельных структур или территорий в геологическом прошлом. Реконструкции могут проводиться на один или несколько последовательных отрезков времени, при этом они базируются на определенных представлениях теоретической геологии. До 60-70-х годов прошлого столетия палеотектонические карты (реконструкции) составлялись на основе геосинклинальной теории, по которой структурообразование протекает за счет проявления, в основном, вертикальных тектонических движений. 

С 1970-1980-х годов они проводятся уже на базе тектоники плит, постулирующей крупномасштабные горизонтальные перемещения отдельных блоков литосферы.

Самыми интересными и трудоемкими являются глобальные (в масштабе всей Земли) реконструкции основных структурных элементов литосферы – океанов и континентов – в течение длительного времени. Наиболее достоверными из них являются построения на венд-фанерозойский этап развития Земли (последние 650 млн. лет). Это связано с его более удовлетворительной изученностью, в сравнении с более ранним периодом развития планеты, за счет возможности использования биостратиграфического метода датирования горных пород, являющихся основным источником информации о палеотектонических условиях своего формирования. %Широко известными и используемыми в различных целях являются реконструкции Л. П. Зоненшайна [11, 13, 14, 16], Я. Голонке [24], К. Р. Скотиза [40], Р. Блеки [39]. В общих чертах построения этих авторов, особенно трех последних, довольно близки, при этом наиболее наглядными являются палеотектонические карты Р. Блеки.


Венд-фанерозойский этап развития Земли в тектоническом отношении можно подразделить на 3 временных подъэтапа. Первый из них, с возрастными границами 650-320 млн лет назад (венд - ранний карбон), представляет время формирования и развития океанов Япетус, Тетис и др., образовавшихся за счет распада 750-800 млн лет назад суперконтинента Родиния, предположительно сформированного на рубеже 1,0 млрд лет т.н.

\begin{figure}[h] 
	\centering\small
	\unitlength=1mm
	\begin{picture}(110, 50)
		\put(15,0){\includegraphics[width=80mm]{Figures/EarlyDevonian390Ma.jpg}}
		%		\put(5,25){\mbox{$lg T_{1/2}$, с}} 
		%		\put(50,0){\mbox{$A$, а.е.м.}} 
	\end{picture}	
	\caption{Реконструкция континентов в Девонском периоде} http://www.scotese.com/newpage3.htm
	\label{f:EarlyDevonian390Ma}
\end{figure}
Приведём пример реконструкции  К.Р. Скотиза. В \emph{Девонском периоде} \index{Девонский период} ($419 - 358$ млн лет назад) океаны раннего Палеозоя начали закрываться. Пресная вода могла проникать из южного полушария в Северную Америку и Европу. Климат на всей Земле был тропическим. В это время  рыбы достигают огромного разнообразия, поэтому период имеет неофициальное название  <<эпоха рыб>>.


Второй подъэтап – 320-200 млн лет т.н. (ранний карбон – поздний триас) – это период образования суперконтинента, который А. Вегенером в начале XX века был назван \emph{Пангеей} \index{Пангея} \cite{Wegener1925}.
% Вегенер А. Возникновение материков и океанов / перевод с 3 немецкого издания, редактор Г. Ф. Мирчинк. Л.: Госиздат, 1925. XIV, 147 с. (Современные проблемы естествознания; Кн. 24).

Этот позднепалеозойско-раннемезозойский суперконтинент в настоящее время никем серьезно не оспаривается, т.к. он хорошо доказан комплексом геолого-геофизических данных. Его реконструкции, при условии неизменности объема Земли за последние 200 млн лет, практически идентичны у большинства тектонистов. Третий временной интервал - 200-0 млн лет - это период раскола Пангеи-II, раскрытия и развития Атлантического, Индийского и Северного Ледовитого океанов, и обособления шести современных континентов (Рис. 4.31).


\subsection{Коэволюция природных явлений на Земле} 

В \S\ref{EarthTimeScales} рассмотрены сначала масштабы существования изотопов различных элементов.
Далее был совершён экскурс по масштабам различных природных явлений (тектоника, климат, жизнь атмосферы и океана, эволюция биоты) на нашей планете. В геологическую летопись эти явления входят во взаимном переплетении, и между ними существуют различные связи. 

В публикациях последних десятилетий различными авторами проводится мысль о \emph{коэволюции} природных явлений на Земле, \index{коэволюция природных явлений на Земле}  компонент живой и неживой природы.
На русском языке издано ряд популярных книг ведущих специалистов, внесших свой вклад в развитие этой идеи. Среди них книги Р.\,Хейзена 
<<Симфония №6: Углерод и эволюция почти всего>> \cite{2021Hazen}
и <<История Земли. От звездной пыли - к живой планете. Первые 4 500 000 000 лет>> \cite{2015Hazen}, 
П.\,Уорда и Д.\,Киршвинка <<Новая история происхождения жизни на Земле>> \cite{WardKirschvink}, А.Ю.\,Журавлёва <<Сотворение Земли. Как живые организмы создали наш мир>> \cite{Zhuravlev2019}.

Приведём некоторые сведения об эволюции Земли, которые ещё слабо отражены в учебной литературе.

\paragraph{Эволюция минералов} \label{MineralEvolution}

На сайте https://hazen.carnegiescience.edu данные роста численности минералов таблицы 1 приводятся в виде графика.  
\begin{figure}[h] 
	\centering\small
	\unitlength=1mm
	\begin{picture}(110, 45)
	\put(15,0){\includegraphics[width=80mm]{Figures/earths_mineralogy_is_unique_2_600.jpg}}
	\end{picture}	
	\caption{Ten stages of mineral evolution of terrestrial planets.} https://hazen.carnegiescience.edu
	\label{f:earths_mineralogy}
\end{figure}


Р.\,Хейзен выдвинул концепцию совместной эволюции минералов и биоты. В коллективном обзоре <<Эволюция минералов>> \cite{2008Hazen}  \index{эволюция минералов} представлен обширный материал, в котором произведено разбиение \emph{эволюции минералов} в истории Земли на 10 этапов. Наиболее радикальной являются идеи роли минералов в зарождении жизни на Земле как необходимой структуры для формирования сложной органика, и обратного влияния на минералы биоты. 
<<Основываясь на обзоре литературы, Хейзен и его соавторы подсчитали, что количество минералов в Солнечной системе выросло с примерно дюжины на момент ее формирования до более чем 4300 в настоящее время. (По состоянию на 2017 год последнее число выросло до 5300). Они предсказали, что с течением времени происходит систематическое увеличение числа видов минералов, и определили три основные эпохи изменений: формирование Солнечной системы и планет; переработка коры и мантии и начало тектоники плит; и появление жизни. После первой эры минералов было 250; после второго - 1500. Остальные стали возможными благодаря действию живых организмов, в частности добавлению кислорода в атмосферу. За этой статьей в течение следующих нескольких лет последовало несколько исследований, концентрирующихся на одном химическом элементе и намечающих первые появления минералов, включающих каждый элемент.>>
См. также материал \S\ref{IsotopePlanetology}.

В статье \cite{2014Grew} приводится пример для роста разнообразия минералов бериллия.
\begin{figure}[h] 
	\centering\small
	\unitlength=1mm
	\begin{picture}(110, 52)
	\put(15,0){\includegraphics[width=80mm]{Figures/BeMineralDivesity.png}}
	\end{picture}	
	\caption{Рост разнообразия минералов бериллия \cite{2014Grew}} 	\label{f:BeMineralDivesity}
\end{figure}

График рис.~\ref{f:BeMineralDivesity} имеет характерный вид с резкими ступеньками, связанными с важными событиями эволюции Земли.
	
	
%	\chapter{Атомные веса элементов} \label{PeridicTableElemIsotopes}
	
	
\chapter{Периодическая таблица элементов и изотопов} \label{PeriodicTableElemIsotopes}

%Эта глава, как и \S\ref{AtomicWeigthChapter}, носит информационный характер. 
	Эта глава носит информационный характер. Значительно более подробно материал представлен в  \cite{InteIsotopes2023}.

Международный союз теоретической \index{Периодическая таблица Менделеева} и прикладной химии IUPAC (International Union of Pure and Applied Chemistry, рус. аббр. ИЮПАК) занимается, в частности, стандартизацией атомного веса элементов через один из старейших постоянных комитетов --- Комиссии по изотопному и атомному весу (Commission on Isotopic Abundances and Atomic Weights, CIAAW) \cite{CIAAW}.
\index{The International Union of Pure and Applied Chemistry, IUPAC}   

% А.Н.Баженов. Интервальные арифметики и прослеживаемость изотопной подписи: учебное пособие. Санкт-Петербургский политехнический университет Петра Великого. С.-Петербург, 2023. https://doi.org/10.18720/SPBPU/5/tr23-167 	


\section{Таблица Менделеева с изотопными данными}


В целом таблица Менделеева выглядит как показано на рисунке \ref{f:PeriodicTable} \cite{IUPAC}.
Легенда цветового поля каждого элемента дана в таблице.\\
\begin{table}
	%{\small
	{\footnotesize 
		\begin{tabular}{ll}
			Цвет фона & Пояснение \\
			\hline	
			красный & элемент имеет два или более стабильных изотопов. \\
			~ &	Соотношения изотопов различны в различных распространённых  \\
			~ & материалах. Эти вариации надёжно определены, \\
			~ &  атомный вес указывается в виде интервала, в квадратных скобках; \\
			\hline
			жёлтый & элемент имеет два или более стабильных изотопов. \\
			~ & Соотношения изотопов различны в различных распространённых  \\
			~ & материалах. При этом невозможно дать надежные оценки нижних  \\
			~& и верхних границ изменений. \\
			~ & Атомный вес даётся с неопределённостью, которая включает  \\
			~ & ошибку измерений и неопределённость вариации изотопных отношений; \\
			\hline
			голубой& элемент имеет один стабильный изотоп.  \\
			~ & Атомный вес даётся с неопределённостью, которая включает\\ 
			~ & ошибку измерений. \\
			\hline	
			белый & элемент не имеет стабильных изотопов.	\\
			~ & в распространённых материалах не содержится в таких  \\
			~ & количествах, по которым можно дать оценку изотопных отношений.\\
			\hline
		\end{tabular} 
		\caption{Обозначения на рис. \ref{f:PeriodicTable}.}
	}
\end{table}

\begin{figure}[ht] 
	\centering\small
	\unitlength=1mm
	\includegraphics[width=120mm]{Figures/PeriodicTable2016all.png} 
	\caption{Таблица Менделеева элементов и изотопов \cite{IUPAC}} 
	\label{f:PeriodicTable}
\end{figure}


Перейдём к описанию изотопного состава элементов.

В периодической таблице Менделеева, поддерживаемой Международным союзом теоретической \index{Периодическая таблица Менделеева}
и прикладной химии IUPAC, приводятся интервальные границы стабильных изотопов химических \index{стабильные изотопы}
элементов. Например, для кислорода, имеющего 3 изотопа с атомными массами 16, 17 и 18 
на стр. 1858 статьи \cite{IUPAC} приводятся данные, часть которых представлена 
в Табл.~\ref{IUPACOxygen}. 

%%%%%%%%%%%%%%%%%%%%%%%%%%%%%%%%%%%%%%%%%%%%%%%%%%%%%%%%%%%%%%%%%%%%%%%%%%%%%%%%%%%%%%%%

\begin{table}[h!]
	\centering
	\caption{Стабильные изотопы кислорода.} 
	\medskip 
	\begin{tabular}{|c|c|}
		\hline
		Стабильный  изотоп & Молярная доля \\
		\hline 
		$^{16}O$ & [0.997 38, 0.997 76] \\
		$^{17}O$ & [0.000 367, 0.000 400] \\
		$^{18}O$ & [0.001 87, 0.002 22] \\			
		\hline
	\end{tabular}
	\label{IUPACOxygen}
\end{table} 

%%%%%%%%%%%%%%%%%%%%%%%%%%%%%%%%%%%%%%%%%%%%%%%%%%%%%%%%%%%%%%%%%%%%%%%%%%%%%%%%%%%%%%%%  

Компактное представление кислорода в таблице Менделеева выглядит следующим образом.

\begin{figure}[ht] 
	\centering\small
	%	\unitlength=1mm
	\includegraphics[width=20mm]{Figures/Oxygen.png}
	%	\includegraphics[width=30mm]{Figures\Oxygen.png}
	\caption{Представление кислорода в таблице Менделеева} 
	\label{f:Oxygen}
\end{figure}	
На рисунке \ref{f:Oxygen} дано наглядное представление о распространенности изотопов кислорода в природе в форме полярной диаграммы. В нижней строчке приведён интервал атомной массы.

Для каждого стабильного изотопа приведены границы, в пределах которых данный изотоп 
встречается в различных породах, атмосфере, водной среде в различных местах Земли. 
Подробные сведения приводятся на рисунках 4.8.1-4.8.3 из работы \cite{IUPAC}. 



%Описание того, как получаются там эти интервалы, представлено на %Рис.~\ref{f:HistAtom}. 
%%%%%%%%%%%%%%%%%%%%%%%%%%%%%%%%%%%%%%%%%%%%%%%%%%%%%%%%%%%%%%%%%%%%%%%%%%%%%%%%%%%%%%%%
%\begin{figure}[ht] \label{f:HistAtom}
%	\centering\small
%	\unitlength=1mm
%	\begin{picture}(80,68)
%	\put(0,0){\includegraphics[width=80mm]{Figures\DistrPlot.eps} 
%	\put(60,24){\mbox{гистограмма частот}} 
%	\end{picture}
%	\caption{Как образуется интервал атомных весов элемента.} 
%\end{figure} 
%%%%%%%%%%%%%%%%%%%%%%%%%%%%%%%%%%%%%%%%%%%%%%%%%%%%%%%%%%%%%%%%%%%%%%%%%%%%%%%%%%%%%%%%
%На Рис.\begin{figure}[ht] 	\centering\small
%	\unitlength=1mm
%	\includegraphics[width=0.3\textwidth]{Figures/Lead.png}
%	\caption{Представление свинца в таблице Менделеева.} 
%	\label{f:Lead2}
%\end{figure}
%\ref{f:HistAtom} по оси абсцисс отложены массы изотопов, по оси ординат %--- их распространённость в природе.

Например, в случае ртути, известны изотопы с массовыми числами от 170 до 216 (количество протонов 80, нейтронов от 90 до 136).
Природная ртуть состоит из смеси 7 стабильных изотопов:

\begin{table}[h!]
	\centering
	\caption{Стабильные изотопы ртути.} 
	\medskip 
	\begin{tabular}{cc}
		Изотоп & Распространённость \\
		\hline
		$^{196}$Hg &  0,155 \% \\
		$^{198}$Hg & 10,04 \% \\
		$^{199}$Hg  & 16,94 \% \\
		$^{200}$Hg  & 23,14 \% \\
		$^{201}$Hg  &  13,17 \% \\
		$^{202}$Hg  &  29,74 \% \\
		$^{204}$Hg  &  6,82 \% \\
		\hline
	\end{tabular} 
	\label{Sulfur}
\end{table}		

Приведённые в таблице \ref{Sulfur} величины распространённости служат исходными данными для построения гистограммы частот. %, схематично представленной на Рис.$\vdots$\ref{f:HistAtom}.
Конкретно для атомов ртути этот рисунок показан на Рис.~\ref{f:HistHg}.

\begin{figure}[ht] 
	\centering\small
	\begin{tikzpicture}
	%\draw[help lines] (0,0) grid (11,5);
	\draw[->] (0,0) -- (0,4);
	\draw[->] (0,0) -- (10,0);
	\draw (-0.5, 1) node {10 \%};
	\draw (-0.5, 2) node {20 \%};
	\draw (-0.5, 3) node {30 \%};
	\draw[red] (0,0.0155) -- (1,0.0155);
	\draw[red] (1,0.0155) -- (1,0);
	\draw[red] (1,0) -- (2,0);
	\draw[red] (2,0) -- (2,1.004);
	\draw[red] (2,1.004) -- (3,1.004);
	\draw[red] (3,1.004) -- (3,1.694);
	\draw[red] (3,1.694) -- (4,1.694);
	\draw[red] (4,1.694) -- (4,2.314);
	\draw[red] (4,2.314) -- (5,2.314);
	\draw[red] (5,2.314) -- (5,2.314);
	\draw[red] (5,2.314) -- (5,1.317);
	\draw[red] (5,1.317) -- (6,1.317);
	\draw[red] (6,1.317) -- (6,2.974);
	\draw[red] (6,2.974) -- (7,2.974);
	\draw[red] (7,2.974) -- (7,0);
	\draw[red] (7,0) -- (8,0);
	\draw[red] (8,0) -- (8,0.682);
	\draw[red] (8,0.682) -- (9,0.682);
	\draw[red] (9,0.682) -- (9,0);
	\draw (0.6,-0.5) node {196};
	\draw (2.6,-0.5) node {198};
	\draw (4.6,-0.5) node {200};
	\draw (6.6,-0.5) node {202};
	\draw (8.6,-0.5) node {204};
	\draw (10,-0.5) node {Масса изотопа};
	\draw (2,4) node {Распространённость};
	\end{tikzpicture}
	\caption{Распространённость изотопов ртути на Земле}
	\label{f:HistHg}
\end{figure} 

Относительно характера графика, представленного на Рис.~\ref{f:HistHg}, следует заметить следующее, см. \S\ref{s:NuclPhys}. Согласно современным представлениям, атомное ядро составляют протоны и нейтроны (нуклоны). Характер сил, действующих между ними, таков, что для лёгких ядер количества протонов и нейтронов примерно равны, с небольшим преобладанием последних. Число стабильных изотопов при этом невелико.  В ядрах тяжёлых элементов нейтронов существенно больше, чем протонов, и количество изотопов может достигать десятков, из которых стабильна небольшая часть. При этом количество стабильных  изотопов с чётным количеством нуклонов заметно превышает количество стабильных  изотопов с нечётным количеством нуклонов. Для энергетически выгодной конфигурации количества нуклонов существуют и другие закономерности, подобные принципу заполнению электронных оболочек атомов (см.\S\ref{s:NuclPhys}). 

В целом график распределения стабильных изотопов для данного химического элемента имеет неправильную форму с возможными <<пробелами>> внутри графика, как правило, для нечётных атомных масс, что в случае ртути имеет место для изотопов с массами 197 и 203. Процентное содержание изотопа с массой 196, весьма малое: 0.155 \%, но надёжно определяется.

\section{Таблица стандартных атомных весов} \label{s:TableStandardAtomicWeights}

Таблица стандартных атомных масс \cite{IUPAC2021} приводится в порядке возрастания атомного номера (Таблица \ref{t:TSAW2021}). Она предназначена для применения ко всем обычным земным материалам с небольшими исключениями 
согласно сноскам. % Стандартные атомные веса не применяются ни к внеземным материалам, ни к материалам 	с намеренно измененным изотопным составом, за исключением лития, для которого искусственно $^6Li$-обедненные вещества были включены в определение его значения $A_{r}{\circ}(E)$. 

Стандартные атомные массы в таблице даны в виде одного значения с неопределенностями или в виде интервала (таблица \ref{t:TSAW2021}, столбцы 4 и 5). 

\begin{table}[h!]
	{\scriptsize 
		\begin{tabular}{ll}
			4 & Атомный вес в интервальной форме или среднее значение \\
			5 & Неопределённость \\	
			6 & Комментарий \\
			g & Известны геологические и биологические материалы, в которых элемент имеет  \\
			~ & изотопный состав за пределами нормального материала. Разница между атомным  \\
			~ & весом элемента в таких материалах и указанное в таблице может превышать указанную  \\
			~ & погрешность.\\
			m & Модифицированные изотопные составы можно найти в коммерчески доступных  \\
			~ & материалах, потому что материал был подвергнут неизвестному или непреднамеренному \\
			~ & изотопному фракционированию. \\ 
			~ & Могут возникнуть существенные отклонения атомного веса элемента от \\
			~ & указанного в таблице. \\
			r & Диапазон изотопного состава обычного земного вещества не позволяет \\
			~ & получить более точный  стандартный атомный вес; \\
			~ & табличное значение и неопределенность должны быть применимы к обычному материалу.\\
			7 & Атомный вес --- среднее значение \\
			8 & Неопределённость --- округлено до $0.001$
		\end{tabular}
	}
	\caption{Таблица стандартных атомных весов 2021 --- Обозначения столбцов}
	\label{t:TSAW2021Notes}
\end{table}

Приведём некоторые пояснения к обозначениям в Табл.~\ref{t:TSAW2021}. Для использования более крупного шрифта опущен столбец (1), содержащий английское название элемента. Для согласованности с исходным текстом IUPAC, нумерация столбцов начинается с 2. 
В Табл.~\ref{t:TSAW2021} приведены данные для элементов с атомными номерами 1-26. Они включают в себя все биогенные элементы. В этом диапазоне сожержатся 11 элементов с интервальными значениями атомных весов.
Далее приведён бром с атомным номером 35 как еще один элемент с интервальным значением атомного веса.
С существенным разрывом добавлена группа элементов  с атомными номерами 77-82. В их числе два элемента таллий (81) и свинец (82) с интервальными значениями атомных весов. Кроме того, иридий (78) упоминается для пояснения обозначений неопределённости его атомного веса. Данные об изотопном составе ртути приводятся в качестве иллюстрации в таблице \ref{Sulfur}. 

%В ходе рассмотрения последнего отчета Комиссии было отмечено, что выражение неопределенности стандартные атомные веса не соответствовали доокументам  GUM <<Руководство по выражению неопределенности в	измерении>> \cite{GUM2011}.
% Например, стандартный атомный вес иридия, равный $192.217$, с погрешностью	$\pm 0.002$, будет указано в таблице как $192.217(2)$. Однако этот формат не соответствует выражению 	неопределенность в GUM, поскольку этот формат предполагает, что заявленная неопределенность является стандартной неопределенностью. Основанный на работе Подкомитета по естественной оценке фундаментального понимания изотопов \cite{IUPACMS2020},
%Комиссия выбрала формат, в котором значение неопределенности обозначается символом <<$\pm$>>. таким образом 	стандартный атомный вес иридия теперь выражается как $ 192.217 \pm 0.002 $. В таблице стандартных атомных 	весов 2021 года неопределенности заносятся в новую колонку или в виде интервалов. Кроме того, на основе	сотрудничество между Подкомитетом и Комиссией, была добавлена новая сноска с символом двойной крестик $\ddag$ 	(Таблица \ref{t:TSAW2021}), чтобы подчеркнуть, что неопределенность атомного веса 	является результатом договоренности.

%Подробности и количество значащих цифр, указанные в Таблице стандартных атомных весов (Таблица \ref{t:TSAW2021}) 	во многих случаях превышает потребности пользователей, а в некоторых случаях для дальнейших расчетов требуется одно значениес учетом интервальных элементов. Поэтому были опубликованы таблицы сокращенных стандартных атомных весов, 	которые  содержат четыре значащих цифр. Эти значения включены, начиная с  2015 года в  единую таблицу, содержащую	как стандартные атомные веса, так и сокращенные стандартные атомные веса.


В главе представлена современная версия таблицы Менделеева с интервальными значениями атомных весов, разработанная Международным союзом теоретической и прикладной химии IUPAC. \index{IUPAC, Международный союз теоретической и прикладной химии} \index{таблица Менделеева} \index{таблица элементов и изотопов}
В интерактивной форме таблица доступна на сайте {\tt https://www.isotopesmatter.com/} по адресу {\tt https://applets.kcvs.ca/IPTEI/IPTEI.html}.


\begin{table}[h!]
	{\scriptsize 
		\begin{tabular}{ccccccc}
			Символ  & Атомный &  Вес  & Неопреде- & Комм-й &  Вес  & Неопреде-  \\
			~ & номер &  ~  & лённость &  ~  &  ~  & лённость  \\
			\hline 
			2 & 3 & 4 & 5  & 6 & 7 & 8\\
			\hline 
			~ & ~ & ~ & ~ & ~  & ~ & ~\\
			H & 1 & [1.000784, 1.00811] & ~ & m & 1.0080 & 0.0002\\ [1mm]
			He & 2 & 4.002602 & 0.00002 &  g r & 4.0026 & 0.0001 \\ [1mm]
			Li & 3 & [6.938, 6.997] & ~ & m  & 6.94 & 0.06 \\ [1mm]
			Be & 4 & 9.0121831 & 0.0000005 & ~  & 9.0122 & 0.0001 \\ [1mm]
			B & 5 & [10.806, 10.821] & ~ & m  & 10.81 & 0.02 \\ [1mm]
			C & 6 & [12.0096, 12.0116] & ~ & ~  & 12.011 & 0.002 \\ [1mm]
			N & 7 & [14.00643 14.00728] & ~ & m & 14.007 & 0.001 \\ [1mm]
			O & 8 & [15.99903, 15.99977] & ~ & m  & 15.999 & 0.001 \\ [1mm]
			F & 9 & 18.998403162 & 0.000000005 & ~  & 18.998 & 0.0001 \\ [1mm]
			Ne & 10 & 20.1797 & 0.0006 & g m  & 20.180 & 0.0001 \\ [1mm]
			Na & 11 & 22.98976928 & 0.00000002 & ~  & 22.990 & 0.0001 \\ [1mm]
			Mg & 12 & [24.304, 24.307] & ~ & ~  & 24.305 & 0.002 \\ [1mm]
			Al & 13 & 26.9815384 & 0.0000003 & ~  & 26.982 & 0.0001 \\ [1mm]
			Si & 14 & [28.084, 28.086] & ~ & ~  & 28.085 & 0.001 \\ [1mm]
			P & 15 & 30.973761998 & 0.000000005 & ~  & 30.974 & 0.0001 \\ [1mm]
			S & 16 & [32.059, 32.076] & ~ & ~  & 32.06 & 0.02 \\ [1mm]
			Cl & 17 & [35.446, 35.457] & ~ & m  & 35.45 & 0.01 \\ [1mm]
			Ar & 18 & [39.792, 39.963] & ~ & ~  & 39.95 & 0.16 \\ [1mm]
			K & 19 & 39.0983 & 0.001 & g  & 39.098 & 0.001 \\ [1mm]
			Ca & 20 & 40.078 & 0.004 & g  & 40.078 & 0.004\\ [1mm]
			Sc & 21 & 44.955907 & 0.000004 & ~  & 44.956 & 0.001\\ [1mm]
			Ti & 22 & 47.867 & 0.001 & ~  & 47.867 & 0.001\\ [1mm]
			V & 23 & 50.9415 & 0.0001 & ~  & 50.942 & 0.001\\ [1mm]
			Cr & 24 & 51.9961 & 0.0006 & ~  & 51.996 & 0.001\\ [1mm]
			Mn & 25 & 54.938043 & 0.000002 & ~  & 54.938 & 0.001\\ [1mm]
			Fe & 26 & 55.845 & 0.002 & ~  & 55.845 & 0.002\\ [1mm]
			$\ldots$ & $\ldots$ & $\ldots$ & $\ldots$ & $\ldots$  & $\ldots$ & $\ldots$\\ [1mm]
			Br & 35 & [79.901, 79.907] & ~ & g  & 79.904 & 0.003\\ [1mm]
			$\ldots$ & $\ldots$ & $\ldots$ & $\ldots$ & $\ldots$  & $\ldots$ & $\ldots$\\ [1mm]
			Ir & 77 &  192.217 & 0.002 & ~  & 192.22 & 0.01\\ [1mm]
			Pt & 78 & 195.084 & 0.009 & ~  & 195.08 & 0.02\\ [1mm]
			Au & 79 & 196.966570 & 0.000004 & ~  & 196.97 & 0.01\\ [1mm]
			Hg & 80 & 200.952 & 0.003 & ~  & 200.59 & 0.01\\ [1mm]
			Tl & 81 & [204.382, 204.385] & ~ & ~  & 204.38 & 0.01\\ [1mm]
			Pb & 82 & [206.14, 207.94] & ~ & ~  & 207.2 & 1.1\\ [1mm]
			$\ldots$ & $\ldots$ & $\ldots$ & $\ldots$ & $\ldots$  & $\ldots$ & $\ldots$\\ [1mm]
			Th & 90 & 232.0377 & 0.0004 & g  & 232.04 & 0.01\\ [1mm]
			U & 92 & 238.02891 & 0.00003 & g~m  & 238.03 & 0.01\\ [1mm]
			$\ldots$ & $\ldots$ & $\ldots$ & $\ldots$ & $\ldots$  & $\ldots$ & $\ldots$\\ [1mm]
			\hline 
		\end{tabular}
	}
	\caption{Таблица стандартных атомных весов 2021}
	\label{t:TSAW2021}
\end{table} 


В публикации \cite{IUPAC} предложена расширенная версия периодической  системы химических элементов.
Авторы пишут:  \index{Периодическая таблица элементов и изотопов} \index{Periodic Table of the Elements and Isotopes --- IPTEI}
<<Периодическая таблица элементов и изотопов (Periodic Table of the Elements and Isotopes --- IPTEI) IUPAC была создана для ознакомления студентов, преподавателей и непрофессионалов с существованием и важностью изотопов химических элементов.>> Они также предлагают использовать её в качестве наглядного пособия, подобно таблице периодических элементов.


\begin{figure}[ht] 
	\centering\small
	\unitlength=1mm
	\includegraphics[width=40mm]{Figures/IPTEIabcd.png} 
	\caption{Способы иллюстрации атомных весов} 
	\label{f:IPTEIabcd}
\end{figure}

%		\begin{itemize}
%			\item a. Element (chlorine)	whose standard atomic weight is not a constant of nature and is an interval.
%			\item  b. Element (mercury) whose standard atomic weight is not a constant of nature and is not an interval.
%			\item  c. Element (arsenic) whose standard atomic weight is a constant of nature because it has a single stable isotope. 
%			\item d. Element (americium) that has no stable isotopes and thus no standard atomic weight.
%		\end{itemize}




%\end{example} 


\chapter{Данные IUPAC-2021. <<Особые элементы>> на Земле.} \label{IUPAC2021ArPb}

Уточнено понятие, \emph{нормального материала}. Это материал, происходящий из земного источника, \index{нормальный материал} \label{NormalSubstance}
за исключением \\
(1) материалов, которые подверглись существенной искусственной изотопной модификации, \\
(2) внеземные материалы,\\
(3) изотопно-аномальные образцы, такие как природные продукты ядерного реактора из Окло (Габон) или другие уникальные явления.

В отличие от предыдущего определения, это пересмотренное определение признает тот факт, что изменение
атомного веса некоторых элементов обусловлено процессами изотопного фракционирования, которые действуют на разных масштабах времени - см. \S\ref{TimeScales}.

Он также вновь вводит исключение внеземных материалов из определения Стандартные атомные веса. Новое определение более всеобъемлющее, чем некоторые более ранние версии, в отношении
встречающиеся в природе материалы, имеющие нуклеогенные и радиогенные изотопные вариации, на примере аргона
\cite{IUPACArgon} и свинца \cite{IUPACLead}.

Именно эти два элемента, аргон и свниец, имеют особое, исключительно для Земли, распределение изотопов. В определённом смысле, это <<визитная карточка>> нашей планеты.

\section{Аргон} \label{s:Argon}

Аргон является уникальным веществом. Удивительно, что газ, который составляет около 1\% атмосферы по массе, был открыт только в конце XIX в. Причиной такого <<равнодушия>> является химическая инертность этого газа. Собственно, это первый \emph{благородный} газ, открытый на Земле. \index{благородные (инертные) газы}

\subsection{Изотопы аргона} \label{s:ArgonIsotopes}

Согласно ядерным данным, \S\ref{s:NuclPhys} и табл.~\ref{t:AbudanceSolarP-V}, аргон имеет 3 стабильных изотопа.
\begin{figure}[ht] 
	\centering\small
	%	\unitlength=1mm
	\includegraphics[width=0.6\textwidth]{Figures/ArgonIsotopes.png}
	\caption{Период полураспада изотопов аргона и его стабильные изотопы} --- https://applets.kcvs.ca/IPTEI/IPTEI.html
	\label{f:ArgonIsotopes}
\end{figure}


Аргон представлен в земной атмосфере тремя стабильными изотопами: $^{36}Ar$ (0.337 \%), $^{38}Ar$ (0.063\%), $^{40}Ar$ (99.600 \%). При этом в солнечной фотосфере и в атмосферах планет-гигантов изотопное содержание $^{40}Ar$ составляет лишь $\approx 0.01$ \%. Это объясняется тем, что почти весь аргон, содержащийся в земной атмосфере и недрах, является радиогенным — образован в результате постепенного распада $^{40}K$.



\begin{figure}[ht] 
	\centering\small
	%	\unitlength=1mm
	\includegraphics[width=0.6\textwidth]{Figures/40K40Ar.png}
	\caption{Распространённость изотопов калия и аргона на Земле.} 
	--- по данным https://www.nndc.bnl.gov/nudat3/
	\label{f:40K40Ar}
\end{figure}

На Рис.~\ref{f:40K40Ar} графически представлены данные о распространённости изотопов калия (Z=19) и аргона (Z=18) на Земле в координатах (A, Z).
Легенда справа отвечает распространённости элементов для каждого элемента среди его изотопов. 

Калия на Земле относительно очень много. {\it Кларк} \index{кларк} (кларк элемента --- число, выражающие среднее содержание химических элементов в системе) калия в земной коре составляет 2.4\% (5-й по распространённости металл, 7-й по содержанию в земной коре элемент). Соответствено, даже небольшая доля радиоактивного изотопа $^{40}K$, равная 0.012\%, тоже велика. Период полураспада этого изотопа составляет $\approx 10^9$ лет, и на геологических масштабах времени (см. \S\ref{GeoTimeScales}) привело к образованию огромного количества изотопа $^{40}Ar$.

Образование аргона $^{40}Ar$ идёт путём захвата орбитального электрона \S\ref{StabilityNucleiChannel} (вероятность $10.72 \pm 0.13$\%) \index{электронный захват} \eqref{40K40Ar}
\begin{equation}\label{40K40Ar}
	^{40}_{19}K + e^{-} \longrightarrow \ ^{40}_{18}Ar + \ov{\nu}_e.
\end{equation}

На Рис.~\ref{f:40K40Ar} этому процессу соответствует переход по вертикали вниз $(40,19)-(40,18)$ в координатах (A, Z). Переход по вертикали вверх $(40,19)-(40,20)$ соответствует электронному бета-распаду \index{бета-распад}
\begin{equation}
	^{40}_{19}K \longrightarrow \ ^{40}_{20}Ca + e^{-} + \ov{\nu}_e
\end{equation}
и пополняет природу изотопом  $^{40}_{20}Ca$. Это самый распространённый изотоп кальция.

Вероятные источники происхождения изотопов $^{36}Ar$  и $^{38}Ar$  --- неустойчивые продукты спонтанного деления тяжёлых ядер, а также реакции захвата нейтронов и альфа-частиц ядрами лёгких элементов, содержащихся в урано-ториевых минералах:
\begin{align}
^{36}_{17}Cl & \longrightarrow \ ^{36}_{18}Ar + e^{-} + \ov{\nu}_e, \\
^{33}_{16}S + ^{4}_2{He} & \longrightarrow \ ^{36}_{18}Ar + ^1_0n, \\
^{35}_{17}Cl  + ^{4}_2{He} & \longrightarrow \ ^{36}_{18}Ar + ^1_1p.
\end{align}

\subsection{Изотопы аргона на Земле} \label{s:ArgonVars}

Общая картина распространённости изотопов аргона, калия и кальция в земной коре приведена на рис.~\ref{f:Ar_K_Ca_Abudance.png}. Актуальным источником данных является база данных МАГАТЭ NUBASE2020 \cite{NUBASE2020}.

% Element_Abudance_Earth_Crust Isotopes.png
\begin{figure}[ht] 
	\centering\small 
		\unitlength=1mm
		\begin{picture}(80,45)
		\put(20,0){\includegraphics[width=0.5  \textwidth]{Figures/Ar_K_Ca_Abudance.png}}
		\end{picture}
	\caption{Распространённость изотопов аргона, калия и кальция в земной коре}
	\label{f:Ar_K_Ca_Abudance.png}
\end{figure}

Изотопный состав аргона в земных материалах изменчив. Эти вариации являются источником неопределенности в определении стандартных свойств аргона, в то же время они дают полезную информацию во многих областях науки. Большой объём информации о весах аргона в различных природных средах дан в публикации \cite{IUPACArgon}. 


В официальной публикации IUPAC 2022 \cite{IUPAC2021} года приводится рисунок изменчивости массы аргона 
\begin{figure}[ht] 
	\centering\small
	%	\unitlength=1mm
	\includegraphics[width=0.6\textwidth]{Figures/36ArgonIUPAC2021.png}
	\caption{Вариации атомного веса (чёрные линии) аргона, $A_r(Ar)$, и значения долей	(розовые линии) изотопа $^{36}Ar$, $\chi(36Ar)$, для некоторых веществ}
	\label{f:ArgonVar}
\end{figure}

%Because argon has three isotopes whose variations are not mass-dependent, the changes in the Ar(Ar) and x(36Ar) values are not superimposed. Each horizontal line spans the minimum and maximum values observed for the corresponding class of materials.

Вариации состава стабильных изотопов и атомного веса аргона вызваны несколькими причинами, в том числе \\
(1) производство изотопов аргона из других элементов путем радиоактивного распада (радиогенные
изотопы) или другие ядерные превращения (нуклеогенные изотопы), и  \index{радиогенные изотопы}  \index{нуклеогенные изотопы} \\
(2) изотопное фракционирование с помощью физико-химических процессов, таких как диффузия или фазовое равновесие. \index{изотопное фракционирование} 



Физико-химические процессы вызывают
коррелированные зависящие от массы вариации отношений изотопов аргона $n(^{40}Ar)/n(^{36}Ar)$ и $n(^{38}Ar)/n(^{36}Ar)$,
где относительная вариация $n(^{40}Ar)/n(^{36}Ar)$ примерно вдвое превышает вариацию $n(^{38}Ar)/n(^{36}Ar)$ из-за
двухкратного различия масс изотопов. 

Напротив, процессы ядерной трансформации вызывают изменения, которые не следуют этой модели. Например, процесс рождения $^{40}Ar$, изменит отношение $n(^{40}Ar)/n(^{36}Ar)$, но не $n(^{38}Ar)/n(^{36}Ar)$; процесс рождения $^{36}Ar$ привел бы к одинаковым относительным изменениям как $n(^{40}Ar)/n(^{36}Ar)$, так и $n(^{38}Ar)/n(^{36}Ar)$.



В то время как атмосферный аргон может служить обильным и однородным изотопным эталоном, отклонения от
атмосферных изотопных отношений в других источниках аргона ограничивают точность. Опубликованные данные указывают на изменение атомного веса аргона в нормальных земных условиях
материалы 39.792 и 39.963 \cite{IUPACArgon}. Верхний конец этого интервала соответствует атомному весу
(относительная атомная масса) $^{40}Ar$, поскольку некоторые образцы минералов, богатых калием, содержат почти чистый радиогенный $^{40}Ar$. Атомный вес чистого аргона-40, $A_r(^{40}Ar) = 39.962 383$, был округлен до $39.963$ для соответствия стандартному атомному весу аргона. Нижняя граница стандартного атомного веса аргона даёт образец \emph{настурана} \index{настуран} (урановая руда из Саскачевана, Канада), содержащий большое количество нуклеогенных изотопов \index{настуран}
$^{36}Ar$ и $^{38}Ar$ \cite{Argon1993}. Эти измерения были откалиброваны по атмосферному аргону.

Предлагаемая ячейка в Таблице IUPAC Periodic Table of the Elements and Isotopes \cite{IPTEI} приведена на Рис.~\ref{f:Argon}.

\begin{figure}[ht] 
	\centering\small
	%	\unitlength=1mm
	\includegraphics[width=0.25\textwidth]{Figures/Argon.png}
	\caption{Представление аргона в таблице Менделеева} 
	\label{f:Argon}
\end{figure}

Для надёжности, если предположить
оба соотношения изотопов как независимые и имеющие расширенную относительную неопределенность на 0.5 \% для согласования с  рекомендованными в 2007 г. значениями отношения изотопов в атмосферном аргоне, получаем изотопные
отношения $R(^{38}Ar/^{36}Ar) = 2.09 \pm 0.01$ и $R(^{40}Ar/^{36}Ar) = 45.2 \pm 0.22$ и атомный вес
$A_r(Ar) = 39.7931 \pm 0.0009 \, (k = 2)$, что дает нижнюю границу стандартного атомного веса аргона $39.7931 - 0.0009 = 39.792$. 

В пределах стандартного интервала атомного веса аргона измерения различных изотопов
отношения $R(^{40}Ar/^{36}Ar)$ или $R(^{38}Ar/^{36}Ar)$ с различными уровнями точности широко используются для исследований в геохронологии,
взаимодействие воды с горными породами, эволюция атмосферы и другие области \cite{IUPACArgon}. Если требуется одно значение атомного веса,
Комиссия рекомендует использовать $39.95\pm 0.16$, что соответствует аргону в воздухе с неопределённостью, которая обеспечивает покрытие <<нормальных>> материалов (см. стр. \pageref{NormalSubstance}).

В более широком контексте, обзор измерений весов благородных газов в атмосфере дан в публикации \cite{NobleGasesTracers}.

% Reported historical values of the standard atomic weight of argon have been [31,39]: $1902, 39.9; 1911, 39.88; 1920, 39.9; 1925, 39.91; 1931, 39.944; 1961, 39.948; 1969, 39.948 \pm 0.003;$ and $1979, 39.948 \pm 0.001.$ 
В завершение представим таблицу IUPAC молярных интервалов изотопов аргона.

\begin{table}[h!]
	\begin{center}
		{\small
			\begin{tabular}{c|cc|c}
				\hline
				& & &   \\
				Изотоп & Отн. ат. масса & Молярная доля &  Солнце \cite{Lodders2003} \\
				& & &  \% \\
				\hline
				& & &  \\ [1mm]
				$^{36}Ar$	& 35.967 5451 & [0.0000, 0.0208] & 84.6 \\
				$^{38}Ar$ 	& 37.962 732 & [0.0000, 0.0433] & 15.4 \\
				$^{40}Ar$   	& 39.962 383 12 & [0.936, 1.000] & 0.02 \\ [1mm]
				& & &   \\
				\hline
			\end{tabular}
		}
		\caption{Молярные интервалы изотопов аргона}
		\label{t:ArIsotopes}
	\end{center}
\end{table}
% K.Lodders. SOLAR SYSTEM ABUNDANCES AND CONDENSATION TEMPERATURES OF THE ELEMENTS. The Astrophysical Journal, 591:1220–1247, 2003 July 10
Как видно из Табл. \ref{t:ArIsotopes} интервалы изменчивости изотопного состава на Земле очень велики. Так же видна колоссальная разница с данными по Солнцу. Содержание <<земного>> изотопа аргона $^{40}Ar$ на Солнце ничтожно.

\section{Свинец} \label{s:Lead}

Свинец, в отличие от аргона (\S\ref{s:Argon}), известен человеку многие тысячелетия. Этот металл широко распространён, легко добывается и обрабатывается. 

\begin{figure}[ht] 
	\centering\small
	%	\unitlength=1mm
	\includegraphics[width=0.25\textwidth]{Figures/Lead.png}
	\caption{Представление свинца в таблице Менделеева} 
	\label{f:Lead}
\end{figure}

Как видно из Рис.~\ref{f:Lead}, изотопный портрет свинца весьма богат. При этом плотность свинца в различных материалах имеет рекордную изменчивость. Она настолько велика, что превышает единицу величины атомной массы. Эта разница была обнаружена даже при использовании весьма скромных по точности средств измерения. 

\subsection{Происхождение изотопов свинца} \label{PbIsotopesOrigin}

Изотопный состав и атомный вес свинца в земных материалах различны, поскольку три его самых тяжелых стабильных изотопа являются стабильными конечными продуктами радиоактивного распада различных изотопов урана 
\begin{align}
& ^{238}U \longrightarrow \ldots \longrightarrow  ^{206}Pb,  \label{U238Pb} \\ 
& ^{235}U \longrightarrow \ldots \longrightarrow  ^{207}Pb, \label{U235Pb}
\end{align}
и тория 
\begin{equation} \label{Th208Pb}
^{232}Th \longrightarrow \ldots \longrightarrow ^{208}Pb.
\end{equation}

Представим графически цепочки процессов, соотвествующих образованию изотопов свинца согласно \eqref{U238Pb} --- \eqref{Th208Pb}.
Для этого воспользуемся данными с сайта https://www.nndc.bnl.gov/nudat3/.


\begin{figure}[ht] 
	\centering\small
	%	\unitlength=1mm
	\includegraphics[width=120mm]{Figures/NZDecayModePbU.png}
	\caption{Виды распадов от урана до свинца} 	--- по данным https://www.nndc.bnl.gov/nudat3/
	\label{f:NZDecayModePbU}
\end{figure}

На Рис.~\ref{f:NZDecayModePbU} представлена часть $NZ$-диаграммы c отображением моды распада ядра цветом. Пурпурный цвет отвечает \emph{электронному бета-распаду}, \index{бета-распад}
жёлтый цвет соответствует \emph{альфа-распаду}. \index{альфа-распад}
Область отображения в координатах (N, Z) выбрана таким образом, что в правом верхнем угле находятся исходные изотопы реакций \eqref{U238Pb}---\eqref{Th208Pb}.
В левом нижнем угле расположены конечные элементы этих цепочек ядерных реакций. 

Пользуясь правилами изменения числа протонов и нейтронов при указанных способах распада --- см. \S\ref{NucleiDecay}, можно построить весь путь прохождения процессов, данных формулами реакций \eqref{U238Pb}---\eqref{Th208Pb}.

\paragraph{Ряд тория.} \label{ThSeries}
Радиоактивный ряд нуклидов с массовым числом, представимым в виде $4n$, называется  \index{ряд тория, ряд 4n} \emph{рядом тория}. Ряд начинается с встречающегося в коре Земли $^{232}Th$ и завершается образованием стабильного $^{208}Pb$.
%Рассмотрим с некоторыми упрощениями серию тория. В таком случае стартовым элементом серии является изотоп  $^{232}Th$.

Можно заметить по Рис.~\ref{f:NZDecayModePbU}, начало и вся цепочка процессов не содержит стабильных изотопов, кроме завершающего элемента. Таким образом, непонятно, как она вообще могла быть инициирована.
Для пояснения этого вопроса вспомним (\S\ref{ExpNuclData}, \S\ref{NuclearTimeScales}), что стабильность ядер --- понятие относительное. При наличии на Земле большого количества исходного радиоактивного элемента и сравнимого с геологическаими временами периода полураспада, эти процессы будут протекать и поныне, а сам исходный элемент по-прежнему будет в значимых количествах присутствовать в природных элементах.

Поэтому, извлечём данные по относительной распространённости изотопов тяжелых элементов из https://www.nndc.bnl.gov/nudat3/ и построим Рис.~\ref{f:ThLead} в координатах $(A, Z)$. Здесь $A = N+Z$ --- полное число нуклонов в ядре. Это в данном случае более удобно, поскольку в научной печати для изотопов обычно приводится заряд и полное число нуклонов, а не количество нейтронов. Кроме того, следует заметить, что цветовая легенда на  Рис.~\ref{f:ThLead} отвечает относительной распространённости изотопов для данного \emph{элемента} и ничего не говорит об его \emph{абсолютном количестве} на Земле.

\begin{figure}[ht] 
	\centering\small
	%	\unitlength=1mm
	\includegraphics[width=120mm]{Figures/ThoriumSeriesAZ.png}
	\caption{Серия тория} 	--- по данным https://www.nndc.bnl.gov/nudat3/
	\label{f:ThLead}
\end{figure}
На  рис.~\ref{f:ThLead} вертикальные линии соответствуют $\beta^{-}$-распадам. В таких процессах атомный вес практически неизменен. Наклонные линии соответствуют $\alpha$-распадам с одновременным изменением атомного веса на 4 и на электрического заряда --- на 2. В конкретном случае можно символически записать переход от начального изотопа к конечному и баланс весов и электрических зарядов:
\begin{align} 
& _{90}^{232}Th \longrightarrow  _{82}^{208}Pb + 6 \alpha+  4\beta^{-}, \label{ThSeriesShort} \\
& A: \quad 232 = 208 + 4 \cdot 6, \label{ThSeriesShortA} \\
& Z: \quad 90 = 82 + 2 \cdot 6 - 4. \label{ThSeriesShortZ}
\end{align}
Здесь учтено, что $Z_{\beta}=-1, Z_{\alpha}=+2, A_{\beta} \approx 0, A_{\alpha} =4$.

Прокоментируем график цепочки радиоактивных распадов  \eqref{Th208Pb} 
\begin{equation*} 
	_{90}^{232}Th \longrightarrow \ldots \longrightarrow _{82}^{208}Pb.
\end{equation*}


В первую очередь, обратим внимание на стартовую позицию в цепочке \eqref{Th208Pb}. Это --- изотоп 
 $^{232}Th$ с периодом полураспада около 14 миллиардов лет. Таким образом, он обладает достаточно большим периодом полураспада по отношению к возрасту Земли (см. \S\ref{EarthTimeScales}), поэтому практически весь природный торий состоит только из этого нуклида. Можно также заметить, что общие запасы тория в 3—4 раза превышают запасы урана в земной коре.

\begin{table}
{\footnotesize 
\begin{tabular}{c|ccccccc}
	\hline
Z	& 90  & 88 &  89 &  90 &  88 &  86 &  84  \\
A	&  232  & 228 &  228  & 228  & 224 &  220 &  216  \\
Изотоп	&  $^{232}Th$  & $^{230}Ra$ & $^{227}Ac$ & $^{228}Th$ & $^{224}Ra$ & $^{223}Rn$ & $^{217}Po$   \\
\hline
Z	 &  ~ &  ~ &  ~ &  82 &  83 &  84  & 82 \\
A	  &  ~ &  ~ &  ~ &  212 &  212 &  212  & 208 \\
Изотоп	 &  ~ &  ~ &  ~ & $^{213}Pb$ & $^{212}Bi$ & $^{212}Po$ & $^{208}Pb$ \\
\hline
\end{tabular}
}
\caption{Изотопы в ряду тория}
\label{t:ThSeries}
\end{table}


\begin{figure}[ht] 
	\centering\small
	%	\unitlength=1mm
	\includegraphics[width=120mm]{Figures/ThoriumSeriesAZ3d.png}
	\caption{Серия тория, трёхмерное преставление} 	--- по данным https://www.nndc.bnl.gov/nudat3/
	\label{f:ThLeadAZ3d}
\end{figure}
Всего между изотопом тория $^{232}Th$ и изотопом свинца $^{208}Pb$ находится 9 изотопов различных элементов. Все они нестабильны и не накапливаются в значимых количествах на Земле. 


Таким образом, график на Рис.~\ref{f:ThLead} отражает современное состояние и в качественном смысле будет оставаться неизменным в течение очень длительного периода.

Как было замечено в начале обсуждения данной цепочки ядерных превращения на стр. \pageref{ThSeries}, были допущены некоторые упрощения. Они состоят в том, что игнорируются минорные (маловероятные) конкурирующие процессы распадов. Они приводят к ветвлению процессов, но в конечном счете так же завершаются в $^{208}Pb$.

\paragraph{Ряды радия и актиния.} Рассмотрим теперь совсем кратко цепочки ядерных превращений, приводящих к образованию изотопов свинца с атомными массами 206 и 207.

Радиоактивный ряд нуклидов с массовым числом, представимым в виде $4n + 2$, называется \emph{рядом радия} \index{ряд радия или урана-радия} (иногда называют рядом урана или урана-радия). Ряд начинается с урана-238 и завершается образованием стабильного свинца-206.

Для ряда радия можно записать упрощённую схему распада и балансовые отношения для атомного веса и электрического заряда, аналогично случаю для ряда тория \eqref{ThSeriesShort}, \eqref{ThSeriesShortA} и \eqref{ThSeriesShortZ} 
\begin{align*} 
& _{92}^{238}U \longrightarrow  _{82}^{206}Pb + 8 \alpha +  6\beta^{-}, \\
& A: \quad 238 = 206 + 4 \cdot 8, \\
& Z: \quad 92 = 82 + 2 \cdot 8 - 6.
\end{align*}

Радиоактивный ряд нуклидов с массовым числом, представимым в виде $4n + 3$, называется \emph{рядом актиния} \index{ряд актиния или урана-актиния} или урана-актиния. Ряд начинается с урана-235 и завершается образованием стабильного свинца-207.

Балансовые отношения для ряда актиния можно записать как
\begin{align*} 
& _{92}^{235}U \longrightarrow  _{82}^{207}Pb+ 7 \alpha +  4\beta^{-}, \\
& A: \quad 235 = 207 + 4 \cdot 7, \\
& Z: \quad 92 = 82 + 2 \cdot 7 - 4.
\end{align*}

\paragraph{Все ряды создания стабильных радиогенных изотопов свинца.} Приведём общую картину всех трёх цепочек создания стабильных радиогенных изотопов свинца.

\begin{figure}[ht] 
	\centering\small
	\unitlength=1mm
	\begin{picture}(130,58)
	\put(-10,0){\includegraphics[width=130mm]{Figures/Lead206207208SeriesAZ.png}}
	\end{picture}
	\caption{Ряды тория, радия и актиния --- основные ветви} 	--- по данным https://www.nndc.bnl.gov/nudat3/
	\label{f:Lead206207208SeriesAZ}
\end{figure}

На рис.~\ref{f:Lead206207208SeriesAZ} схематично представлены основные ветви рядов тория, радия и актиния. 
Например, из распадов $^{212}Bi$ \eqref{212BiPo} и  \eqref{212BiTl} в ряду тория учитывался только \eqref{212BiPo}, как наиболее вероятный. \index{моды распада изотопа висмута-212}
Распространённость исходных изотопов также не учитывалась. 

За исключением конечных продуктов, изотопов свинца, все транзитные изотопы в рядах нестабильны.
Если посмотреть на рис.~\ref{f:NZDecayModePbU}, то можно заметить, что  $^{208}Pb$ является самым тяжёлым стабильным изотопом.
В частности, все изотопы висмута 210-212, содержащиеся в рядах, неустойчивы к вета-распаду. То же относится к изотопам свинца 210 и 212.

На рис.~\ref{f:235U207Pball} показан ряд актиния с учётом минорных ветвей. Использована интерактивная программа  J.\, Williams and J.\,Donev,
которая, в свою очередь, использует табличные данные с ресурса МАГАТЭ  https://www-nds.iaea.org/relnsd/NdsEnsdf/QueryForm.html.
\begin{figure}[ht] 
	\centering\small
	\unitlength=1mm
	\begin{picture}(130,47)
	\put(20,0){\includegraphics[width=80mm]{Figures/235U207Pball.png}}
	\end{picture}
	\caption{Ряд актиния в координатах (N, Z)--- все ветви} 	--- https://energyeducation.ca/simulations/nuclear/nuclidechart.html
	\label{f:235U207Pball}
\end{figure}

Подобные графики можно построить для всех рядов.


\subsection{Изотопы свинца на Земле} 

Эти различия в соотношениях изотопов и атомных весах дают полезную информацию во многих областях науки, включая геохронологию, археологию, исследования окружающей среды и изучение окружающей среды, криминалистике.
%Хотя элементарный свинец может служить обильным и однородным изотопным эталоном, отклонения от соотношения изотопов в других случаях свинца ограничивают точность, с которой может быть указан стандартный атомный вес свинца.



В ходе всестороннего обзора нескольких сотен публикаций и анализов более 8000 образцов \cite{IUPACLead},
опубликованные изотопные данные показывают, что самый низкий атомный вес свинца среди обычных земных материалов составляет
$206.1462\pm 0.0028 \, (k = 2)$,
определен для роста фосфатного минерала монацита из люизианских
комплекс на северо-западе Шотландии, который содержит в основном $^{206}Pb$ и почти не содержит $^{204}Pb$. Самый высокий из опубликованных
атомный вес свинца $207.9351 \pm 0.0005 (k = 2)$ для монацита из микровключений 
из люизианского комплекса на северо-западе Шотландии, содержащего практически чистый радиогенный $^{208}Pb$. \index{монацит}

Используя вышеупомянутые атомные веса свинца как нижнюю и верхнюю границы интервала, стандартный атомный вес вес свинца принят как 
$A_r(Pb) = [206.14,\, 207.94]$. Если требуется одно значение атомного веса, Комиссия рекомендует
использовать интервал $207.2 \pm 1.1$, что соответствует внешней оценке неопределенности
для нормальных материалов.
\begin{equation}
A_r(Pb) = [206.14, \, 207.94]; \quad 207.2 \pm 1.1.
\end{equation}

Предлагаемая ячейка в Таблице IUPAC Periodic Table of the Elements and Isotopes \cite{IPTEI} приведена на Рис.~\ref{f:Lead} со значением середины интервала атомного веса $207.2$, данным белым цветом. 

%Reported historical values of the standard atomic weight of lead have been [31, 39]: $1902, 206.9; 1909, 207.10; 1916, 207.20; 1937, 207.21; 1961, 207.19; and 1969, 207.2 \pm 0.1$.

На Рис.~\ref{f:204LeadVariation} представлены данные по распределению изотопа $^{204}Pb$ в различных химических формах и разного происхождения.
\begin{figure}[ht] 
	\centering\small
	%	\unitlength=1mm
	\includegraphics[width=80mm]{Figures/Pb204.png}
	\caption{Вариации изотопа $^{204}Pb$ в различных объектах} 
	\label{f:204LeadVariation}
\end{figure}

Рассмотрим часть данных, относящихся к свинцу природных объектов. Данные Табл.\ref{t:204LeadVariation} взяты из \cite{IUPACLead}.
% infPbSRchi204 = [ 0.0115, 0.0078, 0.0118, 0.0113, 0.0096]
% supPbSRchi204 = [ 0.0136, 0.016, 0.0137, 0.0131, 0.0140]
\begin{table}[h!]
	\begin{center}
		{\small
			\begin{tabular}{ccc}
				\hline
				Категория & Нижняя & Верхняя \\
				~ & граница & граница \\ 
				\hline
				Доломит	& 0.0115 & 0.0136 \\
				Известняк	& 0.0078 & 0.0160 \\
				Мергель 	& 0.0118 & 0.0137 \\
				Алевриты 	& 0.0113 & 0.0131 \\
				Фосфориты & 0.0096 & 0.0140 \\
				\hline
			\end{tabular}
		}
		\caption{Вариации изотопа $^{204}Pb$ в осадочных породах,  $ \chi ^{204}Pb$.}
		\label{t:204LeadVariation}
	\end{center}
\end{table}

	
	
	
	\chapter{Описание изотопных данных} \label{DataInte}

Излагаемый материал значительно более подробно представлен в учебном пособии <<Интервальные  арифметики и прослеживаемость изотопной подписи>> \cite{InteIsotopes2023}.	


Для математического описания и обработки данных используется интервальный анализ. 		
Кратко представлены понятия, методы и инструменты анализа данных с интервальной неопределённостью применительно к тематике изотопов. Для вычислений применяется арифметика твинов в форме Нестерова.


Предложен способ описания изотопных данных в виде твинов с целью прослеживания биохимического происхождения сложных веществ непосредственно по их молярной массе. Установлен критерий возможности прослеживания изотопной подписи из исходных категорий веществ в различных классах сложных органических соединений.	

  \section{Интервальный анализ}\label{InteAnalysis}

Описание неопределённостей с помощью интервалов имеет ряд уникальных достоинств. %В целом, именно интервалы реализуют намеченные в \S\ref{GeneralIdeas} требования к объектам и арифметикам, востребованным в статистике.
В этом разделе представлена краткая информация о классической интервальной арифметике и анализе данных с интервальной неопределенностью (иначе, интервальной статистике).

Также понадобятся сведения о
расширениях классической интервальной арифметики $\mathbb{IR}$, которые можно использовать для  описания изотопных данных. Нам понадобится полная интервальная арифметика Каухера $\mathbb{KR}$, которая преодолевает ряд трудностей классической интервальной арифметики и позволяет естественным образом работать с непересекающимися интервалами и сиспользовать полную мощь минимаксных методов.

Для работы со структурами данных при обработке выборок понадобятся объекты, основанные на интервалах:
объединения интервалов --- мультиинтервалы, интервалы с интервальными вершинами --- твины. \index{интервальная арифметика классическая} \index{интервальная арифметика Каухера}

\subsection{Классическая интервальная арифметика $\mathbb{IR}$}
\label{PrimaryConceptChap} 

\paragraph{Вещественные интервалы.} 
Первичное понятие интервального анализа данных и интервальной статистики 
--- \emph{интервал}. Это простое подмножество множества всех вещественных 
(действительных) чисел, которое задаёт целый диапазон значений интересующей нас 
величины. С помощью интервалов можно описывать и моделировать неопределённости 
и неоднозначности. 

Интервалы могут определяться на вещественной оси, на комплексной плоскости, 
а также в многомерных пространствах. Кроме того, существуют различные определения 
интервалов, и некоторые их них не равносильны друг другу, задавая разные 
математические объекты. Далее нас будут интересовать, главным образом, вещественные 
интервалы, вещественные интервальные векторы и матрицы, так как именно они играют 
главную роль в измерениях и их обработке. 

\begin{definition} 
	\textsl{Интервалом} $[a,b]$ вещественной оси $\mbb{R}$ называется  
	множество всех чисел, расположенных между заданными числами $a$ и $b$, 
	включая их самих, т.\,е.                           \index{интервал} 
	\begin{equation*} 
	[a, b] := \{\, x\in\mbb{R} \mid a\leq x\leq b\,\}. 
	\end{equation*} 
	При этом $a$ и $b$ называются \textsl{концами} интервала $[a,b]$, \textsl{левым} 
	(или нижним) и \textsl{правым} (или верхним) соответственно. 
\end{definition}

\paragraph{Характеристики интервала.} 
Любой интервал полностью задаётся двумя числами --- своими концами, но на практике 
широко используются также другие характеристики интервалов и представления интервалов 
на их основе. 

Важнейшими характеристиками интервала являются его \emph{середина} (центр) 
\begin{equation*}
\textstyle\index{середина интервала} 
\m\mbf{a} = \frac{1}{2}(\ov{\mbf{a}} + \un{\mbf{a}}),
\end{equation*}
и его \emph{радиус} 
\begin{equation*}
\textstyle\index{радиус интервала} 
\r\mbf{a} = \frac{1}{2}(\ov{\mbf{a}} - \un{\mbf{a}}).
\end{equation*} 
Также используется понятие \emph{ширины} 
интервала  \index{ширина интервала} \label{InteWid}
\begin{equation*}
\w\mbf{a} = \ov{\mbf{a}} - \un{\mbf{a}}. 
\end{equation*}
%	В целом, $\mbf{a} = \m\mbf{a} + [-1, 1]\cdot\r\mbf{a}$, что равносильно представлению \begin{equation}	\label{MidRadRepres}\mbf{a} = \{\,x\in\mbb{R} \mid \,|x-\m\mbf{a}|\leq \r\mbf{a}\,\}.	\end{equation} 
Таким образом, задание середины и радиуса интервала также однозначно определяет его, 
чем часто пользуются и в теории, и на практике. 

Середина интервала --- это точка, которая <<представляет его>> наилучшим образом, 
так как наименее удалена от остальных точек этого интервала. 

Радиус и ширина характеризуют разброс (рассеяние) точек интервала. 
Интервалы нулевой ширины обычно называют  \textit{вырожденными}. Они отождествляются с вещественными числами, то есть, 
$[1, 1]$ --- это то же самое, что и $1$. 

%%%%%%%%%%%%%%%%%%%%%%%%%%%%%%%%%%%%%%%%%%%%%%%%%%%%%%%%%%%%%%%%%%%%%%%%%%%%%%%%%%%%%%% 

%	\paragraph{Расстояние между интервалами.} 
%	Для интервалов $\mbf{a}$ и $\mbf{b}$ \index{расстояние}\index{метрика} наиболее популярное и содержательное определение расстояния имеет следующий вид: 
%	\begin{equation} \label{IntvalDist}
%	\dist(\mbf{a}, \mbf{b}) \  = \  \max 	\bigl\{|\un{\mbf{a}} - \un{\mbf{b}}|, 	|\ov{\mbf{a}} - \ov{\mbf{b}}| \bigr\}.	\end{equation}

%%%%%%%%%%%%%%%%%%%%%%%%%%%%%%%%%%%%%%%%%%%%%%%%%%%%%%%%%%%%%%%%%%%%%%%%%%%%%%%%%%%%%%% 

\paragraph{Отношения между интервалами.} 
Интервалы являются множествами, и для них определены теоретико-множественные отношения и операции (объединение, пересечение и др.). Особенно важно отношение включения одного интервала 
в другой:  
\begin{equation} 
\label{InclInteOrder} 
\mbf{a}\subseteq\mbf{b} \  \text{ равносильно тому, что } \ 
\un{\mbf{a}}\geq\un{\mbf{b}}\;\text{ и }\;\ov{\mbf{a}}\leq\ov{\mbf{b}}.  
\end{equation} 
Отношение включения является частичным порядком и превращает множество интервалов 
в частично упорядоченное множество (см. \cite{Shreider}). 

Важную роль играет линейное упорядочение интервалов: 

\begin{definition} 
	Для интервалов $\mbf{a}$, $\mbf{b}\in\mbb{IR}$ условимся считать, что
	\textsl{$\mbf{a}$ не превосходит $\mbf{b}$} и писать <<$\,\mbf{a}\leq 
	\mbf{b}$>> тогда и только тогда, когда $\,\un{\mbf{a}}\leq\un{\mbf{b}}\,$
	и $\,\ov{\mbf{a}}\leq\ov{\mbf{b}}$. 
\end{definition} 


%%%%%%%%%%%%%%%%%%%%%%%%%%%%%%%%%%%%%%%%%%%%%%%%%%%%%%%%%%%%%%%%%%%%%%%%%%%%%%%%%%%%%%%  

\paragraph{Теоретико-множественные операции над интервалами.} 
Если интервалы $\mbf{a}$ и $\mbf{b}$ имеют непустое пересечение, т.\,е. $\mbf{a}
\cap \mbf{b} \neq\varnothing$, то можно дать простые выражения для результатов 
теоретико-множественных операций пересечения и объединения через концы этих интервалов 
\begin{align*} 
\mbf{a}\cap\mbf{b} &= 
\bigl[\max\{\un{\mbf{a}}, \un{\mbf{b}}\}, \min\{\ov{\mbf{a}}, \ov{\mbf{b}}\}\bigr], 
\\[5pt] 
\mbf{a}\cup\mbf{b} &= 
\bigl[\min\{\un{\mbf{a}}, \un{\mbf{b}}\}, \max\{\ov{\mbf{a}}, \ov{\mbf{b}}\}\bigr].  
\end{align*} 
Если же $\mbf{a}\cap\mbf{b} = \varnothing$, т.\,е. интервалы $\mbf{a}$ и $\mbf{b}$ 
не имеют общих точек, то эти равенства уже неверны. 

Обобщением операций пересечения и объединения являются операции взятия точной нижней 
грани и точной верхней грани относительно включения <<$\subseteq$>>:  
\begin{align} 
\mbf{a}\wedge\mbf{b} &= \label{InteMinExpr}
\bigl[\max\{\un{\mbf{a}}, \un{\mbf{b}}\}, \min\{\ov{\mbf{a}}, \ov{\mbf{b}}\}\bigr], 
\\[2mm]
\mbf{a}\vee\mbf{b} &= \label{InteMaxExpr}
\bigl[\min\{\un{\mbf{a}}, \un{\mbf{b}}\}, \max\{\ov{\mbf{a}}, \ov{\mbf{b}}\}\bigr].  
\end{align} 
Точная нижняя грань не обязательно присутствует во множестве, в отличие от минимума по множеству. 

Операции \eqref{InteMinExpr} и \eqref{InteMaxExpr} используются при обработке 
интервальных данных --- \S\ref{InternEstChoice}. 

\paragraph{Классическая интервальная арифметика} 

Определение операций между интервалами производится через результаты операций между их членами, т.\,е. <<по представителям>>. Именно, результат интервальной операции есть множество всевозможных результатов операции между числами из интервалов. Для двухместной операции <<$\star$>> имеем
\begin{equation} 
\label{IAMainPrinciple} 
\mbf{a}\star\mbf{b}\; = 
\;\bigl\{\, a\star b \mid a\in\mbf{b}, \,b\in\mbf{b}\, \bigr\}.   
\end{equation} 
Аналогично определяются интервальные одноместные операции. 

Если рассматриваются арифметические операции, т.\,е. $\star\in\{ +, -, \cdot, / \}$, 
то множества, задаваемые правилом \eqref{IAMainPrinciple}, тоже являются интервалами. 
Для конкретных арифметических операций имеем формулы: 
\begin{align}
& \mbf{a} + \mbf{b} = \left[\,\un{\mbf{a}} + \un{\mbf{b}},\,\ov{\mbf{a}}
+\ov{\mbf{b}}\,\right],  \label{Addition}\\[5pt]
& \mbf{a} - \mbf{b} = \left[\,\un{\mbf{a}} - \ov{\mbf{b}},\,\ov{\mbf{a}}
- \un{\mbf{b}}\,\right],  \label{Subtraction}\\[5pt]
& \mbf{a}\cdot\mbf{b} = \left[\,\min\{\un{\mbf{a}}\,\un{\mbf{b}},
\un{\mbf{a}}\,\ov{\mbf{b}},\ov{\mbf{a}}\,\un{\mbf{b}},\ov{\mbf{a}}\,
\ov{\mbf{b}}\}\right.,\left.\max\{\un{\mbf{a}}\,\un{\mbf{b}},
\un{\mbf{a}}\,\ov{\mbf{b}},\ov{\mbf{a}}\,\un{\mbf{b}},\ov{\mbf{a}}\,
\ov{\mbf{b}}\}\,\right],  \label{Multiplication}\\[5pt]
& \mbf{a}/\mbf{b} = \mbf{a}\cdot\left[\,1/\ov{\mbf{b}},\,1/\un{\mbf{b}},
\,\right]\qquad\mbox{ для } \ \mbf{b}\not\ni 0.  \label{Division}
\end{align}  
Множество всех интервалов вещественной оси с операциями сложения, вычитания, 
умножения и деления, определёнными формулами \eqref{Addition}--\eqref{Division}, 
называется \textit{классической интервальной арифметикой}, и его обозначают 
$\mbb{IR}$.                
% \index{классическая интервальная арифметика} 
\index{интервальная арифметика классическая}

%%%%%%%%%%%%%%%%%%%%%%%%%%%%%%%%%%%%%%%%%%%%%%%%%%%%%%%%%%%%%%%%%%%%%%%%%%%%%%%%%%%%%%%%

\subsection{Полная интервальная арифметика (Каухера) $\mathbb{KR}$} \label{KaucherArithmSect}

Элементами арифметики $\mbb{KR}$ являются пары чисел вида $[\alpha, \beta]$. 
Если 	$\alpha\leq \beta$, то $[\alpha, \beta]$ обозначает обычный интервал вещественной оси, 
и его называют \textit{правильным}. Если же $\alpha >  \beta$, то $[\alpha, \beta]$ 
--- \textit{неправильный интервал}. Таким образом, $\mbb{IR}\subset\mbb{KR}$. 
\index{интервальная арифметика Каухера} \index{правильный интервал}
\index{неправильный интервал} 

Правильные и неправильные интервалы, две <<половинки>> $\mbb{KR}$, переходят друг 
в друга в результате отображения \emph{дуализации}\index{дуализация}, которое обозначается 
символом {\tt dual} и меняет местами (переворачивает) концы интервала, т.\,е. 
\begin{equation}
\label{Dualization}
\dual\mbf{a}\, := \,[\;\ov{\mbf{a}},\,\un{\mbf{a}}\;].
\end{equation} 
\textit{Правильной проекцией} интервала $\mbf{a}$ из $\mbb{KR}$ называют интервал, 
обозначаемый $\pro\mbf{a}$:
\begin{equation*} 
\pro\mbf{a} = 
\left\{ \ 
\begin{array}{cl}
\mbf{a}, & \text{если $\mbf{a}$ --- правильный,} \\[1mm] 
\dual\mbf{a}, & \text{если $\mbf{a}$ --- неправильный.} 
\end{array} 
\right. 
\end{equation*} 
С помощью правильной проекции из произвольного интервала получается правильный. 

Арифметические операции между интервалами в $\mbb{KR}$ продолжают операции в $\mbb{IR}$, их подробное описание можно 
найти в \cite{SSharyBook}. Умножение интервала из $\mbb{KR}$ на число определяется 
совершенно так же, как и для обычных правильных интервалов.

Чрезвычайно важным в интервальной арифметике Каухера является обратимость 
арифметических операций. В частности, для любого интервала имеется противоположный ему, 
т.\,е. обратный по сложению. Для интервалов, не содержащих нуля, имеются обратные к ним 
по умножению. Для сложения \eqref{Addition} обратной операцией является не операция 
интервального вычитания \eqref{Subtraction}, а операция, которую называют <<алгебраическим вычитанием>> и обозначают знаком <<$\ominus$>>: 
\begin{equation}
\label{AlgebrMinus} 
\mbf{a}\ominus\mbf{b} 
= [\un{\mbf{a}} - \un{\mbf{b}}, \ov{\mbf{a}} - \ov{\mbf{b}}].
\end{equation} 
Для любых интервалов $\mbf{a}$, $\mbf{b}$ из $\mbb{KR}$ 
справедливы равенства 
\begin{equation*} 
\mbf{a}\ominus\mbf{a} = 0,  \hspace{15mm} 
(\mbf{a} + \mbf{b})\ominus\mbf{b} = \mbf{a}, \hspace{15mm} 
(\mbf{a}\ominus\mbf{b}) + \mbf{b} = \mbf{a}. 
\end{equation*} 

%%%%%%%%%%%%%%%%%%%%%%%%%%%%%%%%%%%%%%%%%%%%%%%%%%%%%%%%%%%%%%%%%%%%%%%%%%%%%%%%%% 

\begin{example} 
	\begin{equation*} 
	[1, 2] \ominus [1, 2] = [1, 2] + [-1, -2] = [1-1, 2-2] = [0, 0] = 0.  
	\end{equation*} 
\end{example} 

%	\emph{Абсолютное значение} интервалов из $\mbb{KR}$ определяется, как абсолютное 	значение их правильных проекций, т.\,е. 	
%\begin{equation*} 	|\mbf{a}|\  = \;\max\,\{\,|\un{\mbf{a}}|, |\ov{\mbf{a}}|\,\}.  	\end{equation*} 

Полная интервальная арифметика Каухера $\mbb{KR}$ пополняет классическую интервальную 
арифметику $\mbb{IR}$ не только в алгебраическом смысле, но также и относительно 
естественного порядка по включению <<$\subseteq$>>. 

\begin{definition} 
	\label{IncluDefi}  
	Для интервалов $\mbf{a}$, $\mbf{b}\in\mbb{KR}$ выполняется 
	включение $\mbf{a}\subseteq\mbf{b}$, если  
	\begin{equation*} 
	\un{\mbf{a}}\geq\un{\mbf{b}}\quad\text{ и }\quad\ov{\mbf{a}}\leq\ov{\mbf{b}}. 
	\end{equation*} 
\end{definition} 

Относительно введённого таким образом отношения включения в $\mbb{KR}$ для любых 
двух интервалов существуют интервалы точной нижней грани и точной верхней грани
по включению, т.\,е. результаты операций $\mbf{a}\wedge\mbf{b}$ и $\mbf{a}\vee\mbf{b}$  
всегда определены. 

\begin{example} 
	\begin{equation*} 
	[0, 1]\wedge [4, 5] = [4, 1], \quad [0, 1]\vee [4, 5] = [0, 5]. 
	\vspace{-4mm} 
	\end{equation*} 
\end{example} 

%%%%%%%%%%%%%%%%%%%%%%%%%%%%%%%%%%%%%%%%%%%%%%%%%%%%%%%%%%%%%%%%%%%%%%%%%%%%%%%%%%%%%%%% 
\subsection{Составные интервальные объекты} \label{CompoundObjects}	

Данные и результаты операций с ними можно описывать не только интервалами, но и более сложными объектами, которые поддерживают дополнительную функциональность.

\paragraph{Твины.}\label{twin}

В данных мы можем использовать  интервал с интервальными концами ---\emph{твин}. \index{твин}
Слово <<твин>> является акронимом английского выражения <<twice interval>>, т. е. <<двойной интервал>>. Впервые такие объекты были рассмотрены Э. Гарденьесом с коллегами в 80-х годах XX века \cite{Twins1981}. Они назвали развитое ими направление  Modal Interval Analysis --- модальный интервальный анализ. \index{модальный интервальный анализ} \index{Modal Interval Analysis}

Твин можно представить в виде
\begin{equation*}
\mbf{X} = [\mbf{a},\mbf{b}]= [ \ [\un{\mbf{a}},\ov{\mbf{a}}],[\un{\mbf{b}},\ov{\mbf{b}}] \ ],
\label{eq:deftwin}
\end{equation*}
и в зависимости от того, как мы определяем понятия <<больше или равно>> и <<меньше или равно>>, подразумевать под ним множество всех интервалов, больших или равных $[\un{\mbf{a}},\ov{\mbf{a}}]$ и меньших или равных $[\un{\mbf{b}},\ov{\mbf{b}}]$. Так как на множествах интервалов из $\mathbb{IR}$ и $\mathbb{KR}$ существуют частичные упорядочения <<$\subseteq$>> и <<$\leq$>>, то, соответственно, возможны два типа твинов: <<$\subseteq$>>-твины и <<$\leq$>>-твины.	

\begin{figure}[hbt]
	\centering\small 
	\setlength{\unitlength}{1mm}
	\begin{picture}(70,10)
	\put(0,0){\includegraphics[width=70mm]{Figures/twinfig.eps}}
	\put(-5,6.6){\vector(1,0){80}} \put(71.5,7.6){$\mbb{R}$} 
	\put(21,10){$\un{\mbf{a}}$} \put(30,10){$\ov{\mbf{a}}$} 
	\put(41,10){$\un{\mbf{b}}$} \put(50,10){$\ov{\mbf{b}}$} 
	\put(35,1){$\mbf{X}$}  
	\end{picture}
	\caption{Твины на вещественной оси} 
	\label{TwinsPic2} 
\end{figure}
На рис. \ref{TwinsPic2} твин $\mbf{X}$ представлен в графической форме. Концы твина, 
т.\,е. интервалы $\mbf{a}$ и $\mbf{b}$, представлены более темной заливкой, чем остальная часть 
твина. 

В.М.Нестеров развил идеи твинов \cite{Nesterov99}. Особенно значимы его идеи о твинах, как способе одновременного вычисления внутренних и внешних оценок. Детали вычислений с твинами обсуждаются в \cite{InteIsotopes2023}.

Как представление твинов, так и работа с ними сложнее, чем с обычными интервалами. В настоящее время развиты программные средства, реализующие арифметику твинов \cite{TwinLib}.  Вычисления с использованием твинов для изотопных данных обсуждаются в \cite{TwinIsotope}, \cite{Diana2023}, \cite{TwinMC}.

\paragraph{Мультиинтервалы.} 
\label{MultiIntervalSect} 

В ряде разделов науки и техники встречаются ситуации, когда исследуемая величина  
содержится в неодносвязной области. 

Согласно определению, приведённому в книге \cite{SSharyBook}, \emph{мультиинтервал} 
--- это объединение конечного числа несвязных интервалов числовой оси 
(Рис. \ref{MultiInterval}). \index{мультиинтервал}

\begin{figure}[ht]
	\centering
	\begin{picture}(70,30)
	\put(-90,-10){\includegraphics[width=0.7\textwidth]{Figures/Multifig.pdf}}
	\put(-100,12){\vector(1,0){260}} \put(155,15){$\mbb{R}$} 
	\end{picture}	
	\caption{Мультиинтервал в $\mathbb{R}$.} %Рис. 1.11 из \cite{SharyBook}.}
	\label{MultiInterval} 
\end{figure}

Между мультиинтервалами также могут быть определены арифметические операции 
<<по представителям>>, аналогично тому, как это делается на множестве интервалов. 

Мультиинтервалы возникают, в частности, при вычислении интервальной моды в случае мультимодальных распределениях изотопных данных см. Главу <<Численные данные по изотопам биогенных элементов>> \S\ref{IsotopeData}, подробно --- см. \cite{InteIsotopes2023}.







%\input{TablesIsotopesNatureChapter}
\chapter[Численные данные по нуклеосинтезу и \\распространённости изотопов элементов]
{Численные данные по нуклеосинтезу и распространённости изотопов элементов} \label{TablesIsotopesNature}

\section{Табличные данные по группам элементов}	 \label{TableNucleosynthesis}

Привёдём некоторые данные по нуклеосинтезу --- Табл.4.1 в \cite{Nucleosynthesis}. 
В дальнейшем будем сопоставлять эти данные с <<Таблицей стандартных атомных весов 2021>>
табл.~\ref{t:TSAW2021} \S\ref{s:TableStandardAtomicWeights}.

Ниже в таблицах \ref{t:AbudanceSolarHSi}---\ref{t:AbudanceSolarPt-U} приводятся данные по всем элементам таблицы Менделеева, в значимом количестве представленными в коре Земли и её атмосфере.
Данные по искуственно синтезированным элементам и чрезвычайно редким элементам (технеций и прометий, см. \S\ref{AbsentElements}) не приводятся.

\begin{table}[h!]
	{ 
		\begin{tabular}{ll}
			I & Элемент \\
			II & Массовое число A \\	
			III & Содержание нуклида в \% по отношению к общему количеству \\
			& элемента в настоящее время \\
			IV & Процесс нуклеосинтеза  \\
			V & Распространённость нуклида (нормировано к Si = $10^6$)  \\
			& в эпоху образования Солнечной системы		
		\end{tabular}
	}
	\caption{Распространённость нуклидов  --- Обозначения столбцов}
	\label{t:AbudancesSolarNotes}
\end{table}

Обозначения  процессов нуклеосинтеза:\\
{
	\begin{tabular}{lcl}
		U & --- & космологический синтез до образования звёзд \\
		H & --- & горение водорода \\
		CNO & --- & горение водорода при высоких температурах (CNO-цикл) \\
		He & --- & взрывное горение гелия \\	
		С & --- & взрывное горение углерода \\
		O & --- & взрывное горение кислорода \\
		Si & --- & взрывное горение кремния \\
		NSi & --- & обогащённое нейтронами горение кремния \\
		E & --- & статическое ядерное равновесие \\
		s & --- & s-процесс. Продукты медленного захвата нейтронов\\
		r & --- & r-процесс. Продукты быстрого захвата нейтронов\\
		p & --- & p-процесс. Процесы на обеднённой нейтронами стороне \\
		& & долины $\beta$-стабильности\\
		X & --- & Дробление космическими лучами \\
	\end{tabular}
} 



\begin{table}[h!]
	{\small
		\begin{tabular}{ccccc}
			Символ  & Атомный &  Содержание  & Процесс & Распростра   \\
			~ & Вес &  нуклида  &  нуклеосинтеза  & нённость  \\
			\hline 
			I & II &  III  & IV & V \\
			\hline 
			%			~ & ~ & ~ & ~ & ~  \\
			H & 1 & 99.9885 &  &  $2.59 \times 10^{10}$\\ [1mm]
			& 2 & 0.0115 & U & $5.03 \times 10^{15}$\\ [1mm]
			He & 3 & 0.000134 & U &  $1.03 \times 10^{6}$\\ [1mm]
			& 4 & 99.999866 & H, U & $2.51 \times 10^{9}$\\ [1mm]
			& 5 & &  & \\ [1mm]
			Li & 6 & 7.59 & X &  4.2 \\ [1mm]
			& 7 & 92.41 & X, H, U & 51.4\\ [1mm]
			& 8 & &  & \\ [1mm]
			Be & 9 & 100 & X  &  0.612 \\ [1mm]
			B & 10 & 19.9 & X &  3.7 \\ [1mm]
			& 11 & 80.1 & X & 15.1\\ [1mm]
			\hline 			
			C & 12 & 98.93 & He &  $7.11 \times 10^{6}$\ \\ [1mm]
			& 13 & 1.07 & H & $7.99 \times 10^{4}$ \\ [1mm]
			N & 14 & 99.636 & H &  $2.12 \times 10^{6}$\ \\ [1mm]
			& 15 & 0.364 & H & $7.78 \times 10^{3}$ \\ [1mm] 
			O & 16 & 99.757 & He &  $2.12 \times 10^{6}$\ \\ [1mm]
			& 17 & 0.038 & H & 5900 \\ [1mm] 			
			& 18 & 0.205 & He, CNO & $3.15 \times 10^{4}$ \\ [1mm] 
			F & 19 & 100 & CNO &  804 \\ [1mm] 
			Ne & 20 & 90.48 & C &  $3.06 \times 10^{6}$\ \\ [1mm]
			& 21 & 0.27 & He, CNO & 7330 \\ [1mm] 			
			& 22 & 9.25 & He, CNO & $2.25 \times 10^{5}$ \\ [1mm] 
			Na & 23 & 100 & C &   $5.77 \times 10^{4}$ \\ [1mm] 
			Mg & 24 & 78.99 & C &  $8.10 \times 10^{5}$ \\ [1mm]
			& 25 & 10.00 & C & $1.03 \times 10^{5}$ \\ [1mm] 			
			& 26 & 11.01 & C & $1.13 \times 10^{5}$ \\ [1mm] 
			Al & 27 & 100 & C &   $8.46 \times 10^{4}$ \\ [1mm] 
			Si & 28 & 92.223 & O, Si &  $9.22 \times 10^{5}$ \\ [1mm]
			& 29 & 4.685 & O & $4.68 \times 10^{4}$ \\ [1mm] 			
			& 30 & 3.092 & O & $3.09 \times 10^{4}$ \\ [1mm] 
			\hline 
		\end{tabular}
	}
	\caption{Pаспространённость нуклидов, H---Si}
	\label{t:AbudanceSolarHSi}
\end{table} 

\begin{table}[h!]
	{\small
		\begin{tabular}{ccccc}
			Символ  & Атомный &  Содержание  & Процесс & Распростра   \\
			~ & Вес &  нуклида  &  нуклеосинтеза  & нённость  \\
			\hline 
			I & II &  III  & IV & V \\
			\hline 
			~ & ~ & ~ & ~ & ~  \\
			P & 31 & 100 & O  &  8300 \\ [1mm]
			S & 32 & 94.99 & O, Si & 400258 \\ [1mm]
			& 33 & 0.75 & O, Si & 3160 \\ [1mm] 			
			& 34 & 4.25 & O, Si & 17800\\ [1mm] 
			& 36 & 0.01 & NSi (?) & 72 \\ [1mm]
			Cl & 35 & 75.76 & O, Si &  3920 \\ [1mm]
			& 37 & 24.24 & O, Si & 1250 \\ [1mm] 
			Ar & 36 & 0.3336 & O, Si &  78400 \\ [1mm]
			& 38 & 0.0629 & O, Si & 14300 \\ [1mm] 			
			& 40 & 99.6035 & s & 22\\ [1mm] 
			\hline
			K & 39 & 93.2581 & O, Si &  3500 \\ [1mm]
			& 40 & 0.0117 & O, Si & 6 \\ [1mm] 			
			& 41 & 6.7302 & O, Si  & 253\\ [1mm] 	
			Ca & 40 & 96.94 & O, Si &  58500 \\ [1mm]
			& 42 & 0.647 & Si, s & 391 \\ [1mm] 			
			& 43 & 0.135 & Si, s  & 82\\ [1mm] 
			& 44 & 2.09 & Si, s & 1260 \\ [1mm] 			
			& 46 & 0.004 & NSi (?)  & 2\\ [1mm] 	
			& 48 & 0.187 & NSi (?)  & 113\\ [1mm] 	 
			Sc & 45 & 100 & Si, E  &  34.4 \\ [1mm]	
			Ti & 46 & 8.25 & E &  204 \\ [1mm]
			& 47 & 7.44 & E & 184 \\ [1mm] 			
			& 48 & 73.72 & E  & 1820 \\ [1mm] 
			& 49 & 5.41 & E & 134 \\ [1mm] 			
			& 50 & 5.18 & E, NSi (?) & 128 \\ [1mm] 
			V & 50 & 0.250 & E &  0.7 \\ [1mm]
			& 51 & 99.750 & E & 285.7 \\ [1mm] 	
			\hline 
		\end{tabular}
	}
	\caption{Pаспространённость нуклидов, P---V}
	\label{t:AbudanceSolarP-V}
\end{table}

\begin{table}[h!]
	{\small
		\begin{tabular}{ccccc}
			Символ  & Атомный &  Содержание  & Процесс & Распростра   \\
			~ & Вес &  нуклида  &  нуклеосинтеза  & нённость  \\
			\hline 
			I & II &  III  & IV & V \\
			\hline 
			~ & ~ & ~ & ~ & ~  \\
			Cr & 50 & 4.35 & E & 569 \\ [1mm]
			& 52 & 83.789 & E & 11000 \\ [1mm] 			
			& 53 & 9.501 & E & 1240 \\ [1mm] 
			& 54 & 2.365 & E & 309 \\ [1mm]
			Mn & 55 & 75.76 & E &  9220 \\ [1mm]
			Fe & 54 & 5.845 & E &  49600 \\ [1mm]
			& 56 & 91.754 & E & $7.78 \times 10^5$ \\ [1mm] 			
			& 57 & 2.119 & E & 18000 \\ [1mm] 
			& 58 & 0.282 & E & 2390 \\ [1mm]			
			Co & 59 & 100 & E &  2350 \\ [1mm]
			Ni& 58 & 68.077 & E &  33400 \\ [1mm]
			& 60 & 26.233 & E & 12900 \\ [1mm] 			
			& 61 & 1.1399 & E & 559 \\ [1mm] 
			& 62 & 3.6346 & E & 1780 \\ [1mm]
			& 64 & 0.9255 & E & 454 \\ [1mm]
			Cu& 63 & 69.15 & E &  374 \\ [1mm]
			& 65 & 30.85 & E & 167 \\ [1mm]
			Zn & 64 & 49.17 & E &  630 \\ [1mm]
			& 66 & 27.13 & E, s & 362 \\ [1mm] 			
			& 67 & 4.04 & E, s & 53 \\ [1mm] 
			& 68 & 18.45 & E, s & 243 \\ [1mm] 
			& 70 & 0.61 & E, s & 8 \\ [1mm] 
			Ga & 69 & 60.108 & E, s &  22.0 \\ [1mm]
			& 71 & 39.892 & E, s & 14.6 \\ [1mm]
			Ge & 70 & 20.57 & E, s &  24.3 \\ [1mm]
			& 72 & 27.45 & E, s & 31.7 \\ [1mm] 
			& 73 & 7.75 & E, s & 8.8 \\ [1mm] 
			& 74 & 36.50 & E, s & 41.2 \\ [1mm] 
			& 76 & 7.73 & E, s & 8.5 \\ [1mm] 
			\hline 
		\end{tabular}
	}
	\caption{Pаспространённость нуклидов, Cr---Ge}
	\label{t:AbudanceSolarCrGe}
\end{table}

\begin{table}[h!]
	{\small
		\begin{tabular}{ccccc}
			Символ  & Атомный &  Содержание  & Процесс & Распростра   \\
			~ & Вес &  нуклида  &  нуклеосинтеза  & нённость  \\
			\hline 
			I & II &  III  & IV & V \\
			\hline 
			~ & ~ & ~ & ~ & ~  \\
			As & 75 & 100 & s, r &  6.10 \\ [1mm]
			Se & 74 & 0.89 & p & 0.60 \\ [1mm]
			& 76 & 9.37 & s & 6.32 \\ [1mm] 
			& 77 &  7.63 & s, r & 5.15 \\ [1mm]  			
			& 78 & 23.77 & s & 16.04\\ [1mm] 
			& 80 & 49.61 & s, r & 33.48 \\ [1mm]			
			& 82 & 8.73 & r & 5.89 \\ [1mm]			
			Br & 79 & 50.69 & s, r  & 5.43 \\ [1mm]
			& 81 & 49.31 & s, r  & 5.28 \\ [1mm] 
			Kr & 78 & 0.335 & p & 0.20 \\ [1mm]
			& 80 & 2.286 & s, p & 1.30 \\ [1mm] 
			& 82 &  11.593 & s & 6.51 \\ [1mm]  			
			& 83 & 11.500 & s, r & 6.45 \\ [1mm] 
			& 84 & 59.987 & s, r & 31.78 \\ [1mm]			
			& 86 & 17.729 & r & 9.61 \\ [1mm]			
			Rb & 85 & 72.17 & s, r  & 5.121 \\ [1mm]
			& 87 & 27.83 & r  & 2.108 \\ [1mm] 	
			Sr & 84 & 0.56 & p & 0.13 \\ [1mm]
			& 86 & 9.86 & s & 2.30 \\ [1mm] 
			& 87 &  7.00 & s & 1.60 \\ [1mm]  			
			& 88 & 82.58 & s, r & 19.2 \\ [1mm] 
			Y & 89 & 100 & s, r & 4.63 \\ [1mm]
			Zr & 90 & 0.56 & p & 0.13 \\ [1mm]
			& 91 & 9.86 & s & 2.30 \\ [1mm] 
			& 92 &  7.00 & s & 1.60 \\ [1mm]  			
			& 94 & 82.58 & s, r & 19.2 \\ [1mm] 
			& 96 & 2.80 & r & 0.302 \\ [1mm] 			
			\hline 
		\end{tabular}
	}
	\caption{Pаспространённость нуклидов, As---Zr}
	\label{t:AbudanceSolarAsZr}
\end{table}

\begin{table}[h!]
	{\small
		\begin{tabular}{ccccc}
			Символ  & Атомный &  Содержание  & Процесс & Распростра   \\
			~ & Вес &  нуклида  &  нуклеосинтеза  & нённость  \\
			\hline 
			I & II &  III  & IV & V \\
			\hline 
			~ & ~ & ~ & ~ & ~  \\
			Nb & 93 & 100 & s, r &  0.780 \\ [1mm]
			Mo & 92 & 14.53 & p & 0.370 \\ [1mm]
			& 94 & 9.15 & p & 0.233 \\ [1mm] 
			& 95 &  15.84 & s, r & 0.404 \\ [1mm]  			
			& 96 & 16.67 & s & 0.425 \\ [1mm]			
			& 97 & 9.60 & s, r & 0.245 \\ [1mm]			
			& 98 & 24.39 & s, r & 0.622 \\ [1mm]				
			& 100 & 9.82 & r & 0.250 \\ [1mm]			
			Ru & 96 & 5.55 & p & 0.099 \\ [1mm]
			& 98 & 1.87 & p & 0.033 \\ [1mm] 
			& 99 &  12.76 & s, r & 0.227 \\ [1mm]  			
			& 100 & 12.16 & s & 0.224 \\ [1mm]			
			& 101 & 17.06 & s, r & 0.304 \\ [1mm]			
			& 102 & 31.55 & s, r & 0.562 \\ [1mm]				
			& 104 & 18.62 & r & 0.332 \\ [1mm]			
			Rh & 103 & 100 & s, r &  0.370 \\ [1mm]		
			Pd & 102 & 1.02 & p & 0.00139 \\ [1mm]
			& 104 & 11.14 & s & 0.1513 \\ [1mm] 
			& 105 &  22.33 & s, r & 0.3032 \\ [1mm]  			
			& 106 & 27.33 & s & 0.371 \\ [1mm]			
			& 108 & 26.46 & s, r & 0.359 \\ [1mm]			
			& 110 & 11.72 & r & 0.159 \\ [1mm]				
			Ag & 107 & 51.839 & s, r & 0.254 \\ [1mm]		
			& 109 & 48.161 & s, r & 0.236 \\ [1mm]		
			\hline 
		\end{tabular}
	}
	\caption{Pаспространённость нуклидов, Nb---Ag}
	\label{t:AbudanceSolarNbAg}
\end{table}

\begin{table}[h!]
	{\small
		\begin{tabular}{ccccc}
			Символ  & Атомный &  Содержание  & Процесс & Распростра   \\
			~ & Вес &  нуклида  &  нуклеосинтеза  & нённость  \\
			\hline 
			I & II &  III  & IV & V \\
			\hline 
			~ & ~ & ~ & ~ & ~  \\
			Cd & 106 & 1.25 & p & 0.020 \\ [1mm]
			& 108 & 0.89 & p & 0.014 \\ [1mm] 
			& 110 &  12.49 & s & 0.197 \\ [1mm]  			
			& 111 & 12.80 & s,r & 0.201 \\ [1mm]			
			& 112 & 24.13 & s, r & 0.380 \\ [1mm]			
			& 113 & 12.22 & s, r & 0.192 \\ [1mm]				
			& 114 & 28.73 & s, r & XXX \\ [1mm]			
			& 116 & 7.49 & r & 0.118 \\ [1mm]
			In & 113 & 4.29 & p, s & 0.008 \\ [1mm]
			& 115 &  95.71 & s, r & 0.170 \\ [1mm]  			
			Sn & 112 & 0.97 & p & 0.035 \\ [1mm]
			& 114 & 0.66 & p & 0.024 \\ [1mm] 
			& 115 &  0.34 & p, s, r & 0.012 \\ [1mm]  			
			& 116 & 14.54 & s & 0.524 \\ [1mm]			
			& 117 & 7.68 & s, r & 0.277 \\ [1mm]			
			& 118 & 24.22 & s, r & 0.873 \\ [1mm]				
			& 119 & 8.59 & s, r & 0.309 \\ [1mm]			
			& 120 & 32.58 & s, r & 1.175 \\ [1mm]
			& 122 & 4.63 & r & 0.167 \\ [1mm]			
			& 124 & 5.79 & r & 0.209 \\ [1mm]
			Sb & 121 & 57.21 & s, r & 0.179 \\ [1mm]		
			& 123 & 42.79 & s, r & 0.134 \\ [1mm]		
			\hline 
		\end{tabular}
	}
	\caption{Pаспространённость нуклидов, Cd---Sb}
	\label{t:AbudanceSolarCdSb}
\end{table}

\begin{table}[h!]
	{\small
		\begin{tabular}{ccccc}
			Символ  & Атомный &  Содержание  & Процесс & Распростра   \\
			~ & Вес &  нуклида  &  нуклеосинтеза  & нённость  \\
			\hline 
			I & II &  III  & IV & V \\
			\hline 
			~ & ~ & ~ & ~ & ~  \\
			Te & 120 & 0.09 & p & 0.005 \\ [1mm]
			& 122 & 2.55 & s & 0.122 \\ [1mm] 
			& 123 &  0.89 & s & 0.043 \\ [1mm]  			
			& 124 & 4.74 & s & 0.226 \\ [1mm]			
			& 125 & 7.07 & s, r & 0.335 \\ [1mm]			
			& 126 & 18.84 & s, r & 0.889 \\ [1mm]				
			& 128 & 31.74 & r & 1.489 \\ [1mm]			
			& 130 & 34.08 & r & 1.585 \\ [1mm]
			I & 127 & 100 & s, r & 1.10 \\ [1mm]
			Xe & 124 & 0.0952 & p & 0.007 \\ [1mm]
			& 126 & 0.0890 & p & 0.006 \\ [1mm] 
			& 128 &  1.9102 & s & 0.122 \\ [1mm]  			
			& 129 & 26.4006 & s, r & 1.499 \\ [1mm]			
			& 130 & 4.0710 & s & 0.239 \\ [1mm]			
			& 131 & 21.232 & s, r & 1.190 \\ [1mm]				
			& 132 & 26.9086 & s, r & 1.438 \\ [1mm]			
			& 134 & 10.4537 & r & 0.527 \\ [1mm]
			& 136 & 8.8573 & r & 0.429 \\ [1mm]
			Cs & 133 & 100 & s, r & 0.371 \\ [1mm]
			Ba & 130 & 0.106 & p & 0.005 \\ [1mm]
			& 132 & 0.101 & p & 0.005 \\ [1mm] 
			& 134 &  2.417 & s & 0.108 \\ [1mm]  			
			& 135 &  6.592 & s, r & 0.295 \\ [1mm] 			
			& 136 & 7.854 & s & 0.351 \\ [1mm]			
			& 137 & 11.232 & s, r & 0.502\\ [1mm]			
			& 138 & 71.698 & s, r & 3.205 \\ [1mm]				
			\hline 
		\end{tabular}
	}
	\caption{Pаспространённость нуклидов, Te---Ba}
	\label{t:AbudanceSolarTeBa}
\end{table}

\begin{table}[h!]
	{\small
		\begin{tabular}{ccccc}
			Символ  & Атомный &  Содержание  & Процесс & Распростра   \\
			~ & Вес &  нуклида  &  нуклеосинтеза  & нённость  \\
			\hline 
			I & II &  III  & IV & V \\
			\hline 
			~ & ~ & ~ & ~ & ~  \\
			La & 138 & 0.0881 & p & 0.000 \\ [1mm]
			& 139 & 99.9119 & s, r & 0.457 \\ [1mm] 
			Ce & 136 & 0.185 & p & 0.002 \\ [1mm]
			& 138 & 0.251 & p & 0.003 \\ [1mm] 
			& 140 &  88.450 & s, r & 1.043 \\ [1mm]  			
			& 142 & 11.114 & r & 0.131 \\ [1mm]	
			Pr & 141 & 100 & s, r & 0.172 \\ [1mm]
			Nd & 142 & 27.152 & s & 0.231 \\ [1mm]
			& 143 & 12.174 & s, r & 0.103 \\ [1mm] 
			& 144 & 23.798 & s, r & 0.203 \\ [1mm]  			
			& 145 & 8.293 & s, r & 0.075 \\ [1mm]	
			& 146 & 17.189 & s, r & 0.147 \\ [1mm] 
			& 148 &  5.756 & r & 0.049 \\ [1mm]  			
			& 150 & 5.638 & r & 0.048 \\ [1mm]	
			Sm & 144 & 3.07 & p & 0.008 \\ [1mm]
			& 147 & 14.99 & s, r & 0.041 \\ [1mm] 
			& 148 &  11.24 & s & 0.030 \\ [1mm]  			
			& 149 & 13.82 & s, r & 0.037 \\ [1mm]	
			& 150 & 7.38 & s & 0.020 \\ [1mm] 
			& 152 &  26.75 & r & 0.071 \\ [1mm]  			
			& 154 & 22.75 & r & 0.60 \\ [1mm]	
			Eu & 151 & 47.81 & s, r & 0.0471 \\ [1mm]
			& 153 & 52.19 & s, r & 0.0514 \\ [1mm] 
			Gd & 152 & 0.20 & p & 0.0007 \\ [1mm]
			& 154 & 2.18 & s & 0.0078 \\ [1mm] 
			& 155 &  14.80 & s, r & 0.0533 \\ [1mm]  			
			& 156 & 20.47 & s, r & 0.0736 \\ [1mm]	
			& 158 & 24.84 & s, r & 0.0894 \\ [1mm] 
			& 160 &  21.86 & r & 0.0787 \\ [1mm]  			
			\hline 
		\end{tabular}
	}
	\caption{Pаспространённость нуклидов, La---Gd}
	\label{t:AbudanceSolarLaGd}
\end{table}

\begin{table}[h!]
	{\small
		\begin{tabular}{ccccc}
			Символ  & Атомный &  Содержание  & Процесс & Распростра   \\
			~ & Вес &  нуклида  &  нуклеосинтеза  & нённость  \\
			\hline 
			I & II &  III  & IV & V \\
			\hline 
			~ & ~ & ~ & ~ & ~  \\
			Tb & 159 & 100 & s, r & 0.0634 \\ [1mm]
			Dy & 156 & 0.056 & p & 0.0002 \\ [1mm]
			& 158 & 0.095 & p & 0.0004 \\ [1mm] 
			& 160 &  2.329 & s & 0.0094 \\ [1mm]  			
			& 161 & 18.889 & s, r & 0.0762 \\ [1mm]	
			& 162 & 25.475 & s, r & 0.1028 \\ [1mm] 
			& 163 &  24.896 & s, r & 0.1005 \\ [1mm]  	
			& 164 &  28.260 & s, r & 0.1141 \\ [1mm]  	
			Ho & 165 & 100 & s, r & 0.0910 \\ [1mm]
			Er & 162 & 0.139 & p & 0.0004 \\ [1mm]
			& 164 & 1.601 & p, s & 0.0042 \\ [1mm] 
			& 166 &  33.503 & s, r & 0.088 \\ [1mm]  			
			& 167 & 22.869 & s, r & 0.060 \\ [1mm]	
			& 168 &  26.978 & s, r & 0.071 \\ [1mm]  	
			& 170 &  14.910 & r & 0.039 \\ [1mm]  
			Tm & 169 & 100 & s, r & 0.0406 \\ [1mm]
			Yb & 168 & 0.123 & p & 0.0003 \\ [1mm]
			& 170 & 2.982 & s & 0.0076 \\ [1mm] 
			& 171 &  14.09 & s, r & 0.0361 \\ [1mm]  			
			& 172 & 21.68 & s, r & 0.0556 \\ [1mm]	
			& 173 &  16.103 & s, r & 0.0413 \\ [1mm]  	
			& 174 &  32.026 & s, r & 0.0821 \\ [1mm] 
			& 176 &  12.996 & r & 0.0333 \\ [1mm]  
			Lu & 175 & 0.123 & s, r & 0.0370 \\ [1mm]
			& 176 & 2.599 & s & 0.0011 \\ [1mm] 
			\hline 
		\end{tabular}
	}
	\caption{Pаспространённость нуклидов, Tb---Lu}
	\label{t:AbudanceSolarTbLu}
\end{table}

\begin{table}[h!]
	{\small
		\begin{tabular}{ccccc}
			Символ  & Атомный &  Содержание  & Процесс & Распростра   \\
			~ & Вес &  нуклида  &  нуклеосинтеза  & нённость  \\
			\hline 
			I & II &  III  & IV & V \\
			\hline 
			~ & ~ & ~ & ~ & ~  \\
			Hf & 174 & 0.16 & p & 0.0003 \\ [1mm]
			& 176 & 5.26 & s & 0.0081\\ [1mm] 
			& 177 &  18.60 & s, r & 0.0290 \\ [1mm]  			
			& 178 & 27.28 & s, r & 0.0425 \\ [1mm]	
			& 179 & 13.62 & s, r & 0.0212 \\ [1mm] 
			& 180 &  35.08 & s, r & 0.0547 \\ [1mm]  	
			Ta & 180 & 0.01201 & s, r & 0.0000026 \\ [1mm]
			& 181 & 99.98799 & s, r & 0.0210 \\ [1mm]
			W & 180 & 0.12 & p & 0.0002 \\ [1mm] 
			& 182 &  26.50 & s, r & 0.0363 \\ [1mm]  			
			& 183 & 14.31 & s, r & 0.0196 \\ [1mm]	
			& 184 &  30.64 & s, r & 0.0420 \\ [1mm]  	
			& 186 &  28.43 & r & 0.0390 \\ [1mm]  
			Re & 185 & 37.40 & s, r & 0.0207 \\ [1mm]
			& 187 & 62.60 & s, r & 0.0374 \\ [1mm]
			Os & 184 & 0.02 & p & 0.0001 \\ [1mm] 
			& 186 &  1.59 & s & 0.0108 \\ [1mm]  			
			& 187 & 1.96 & s & 0.0086 \\ [1mm]	
			& 188 &  13.24 & s, r & 0.0904 \\ [1mm]  	
			& 189 &  16.15 & s, r & 0.110 \\ [1mm] 
			& 190 &  26.36 & s, r & 0.0179 \\ [1mm]  
			& 192 & 40.78 & r & 0.278 \\ [1mm]
			Ir & 191 & 37.3 & s, r & 0.250 \\ [1mm] 
			& 193 &  62.7 & s, r & 0.421 \\ [1mm]
			\hline 
		\end{tabular}
	}
	\caption{Pаспространённость нуклидов, Hf---Ir}
	\label{t:AbudanceSolarHfIr}
\end{table}

\begin{table}[h!]
	{\small
		\begin{tabular}{ccccc}
			Символ  & Атомный &  Содержание  & Процесс & Распростра   \\
			~ & Вес &  нуклида  &  нуклеосинтеза  & нённость  \\
			\hline 
			I & II &  III  & IV & V \\
			\hline 
			~ & ~ & ~ & ~ & ~  \\
			Pt & 190 & 0.012 & p & 0.0002 \\ [1mm] 
			& 192 &  0.782 & s & 0.010 \\ [1mm]  			
			& 194 & 32.86 & s, r & 0.420 \\ [1mm]	
			& 195 &  33.78 & s, r & 0.431 \\ [1mm]  	
			& 196 &  25.21 & s, r & 0.322 \\ [1mm] 
			& 198 &  7.36 & r & 0.091 \\ [1mm]  
			Au  & 197 & 100 & s, r & 0.195 \\ [1mm]
			Hg & 196 & 0.15 & p & 0.001 \\ [1mm] 
			& 198 &  9.97 & s & 0.046 \\ [1mm]  			
			& 199 & 16.87 & s, r & 0.077 \\ [1mm]	
			& 200 &  23.10 & s, r & 0.106 \\ [1mm]  	
			& 201 &  13.18 & s, r & 0.060 \\ [1mm] 
			& 202 &  29.86 & s, r & 0.0137 \\ [1mm] 
			& 204 &  6.87 & r & 0.031 \\ [1mm]
			Tl & 203 & 29.524 & s, r & 0.054 \\ [1mm] 
			& 205 &  70.476 & s, r & 0.129 \\ [1mm] 
			Pb & 204 & 1.4 & s &  0.066 \\ [1mm]
			& 206 & 24.1 & s, r & 0.614 \\ [1mm] 			
			& 207 & 22.1 & s, r & 0.680 \\ [1mm] 				
			& 208 & 52.4 & s, r & 1.946 \\ [1mm] 	
			Bi & 209 & 100 &  s, r &  0.1382 \\ [1mm]
			\hline
			Th & 232 & 100 &  r &  0.0440 \\ [1mm]	
			U & 234 & 0.0054 &  r &  0.00000049 \\ [1mm]
			& 235 & 0.7204 &  r &  0.00058 \\ [1mm]	
			& 238 & 99.2742 &  r &  0.0180\\ [1mm]		
			\hline 
		\end{tabular}
	}
	\caption{Pаспространённость нуклидов, Pt---U}
	\label{t:AbudanceSolarPt-U}
\end{table}

\chapter{Численные данные по изотопам биогенных элементов} \label{IsotopeData}

\section{Измерения величины $\delta$ изотопов.} \label{DeltaIsotopes}

Обычно измерения изотопной дельты являются основой для определения атомного веса  \cite{IUPACMS2020}.
Величина $\delta$ изотопа получается из отношения числа изотопов $R(^{i/j}E)$ в веществе $P$:
\begin{equation} \label{REP}
R(^{i/j}E, P) = N(^iE, P)/N(^jE, P) % (3)
\end{equation}
где $N(^iE, P)$ и $N(^jE, P)$ — число атомов каждого изотопа, а $^iE$ в общем случае обозначает наибольшее
(верхний индекс $i$) и $jE$ наименьшее (верхний индекс $j$) атомные массовые числа изотопов химического элемента $E$ в вещество $P$. $^jE$ представляет эталонный изотоп, который не обязательно является изотопом с наименьшим атомным массовым числом. 

Дельта-значение изотопов (символ $\delta$), также называемое разностью относительных изотопных отношений, представляет собой дифференциальное измерение, полученное из соотношения изотопов вещества $P$ и шкалы, представленной опорным материалом \cite{IUPAC2014}. \index{параметр $\delta$ для изотопов} 
\begin{equation} \label{DeltaIsotopesDef} % (4)
\delta_{\tt Ref}(^{i/j}E, P)  = R(^{i/j}E, P)/R(^{i/j}E, {\tt Ref}) - 1.
\end{equation}

Дельта-значения изотопов являются небольшими числами и поэтому часто представляются кратными $10^{-3}$ или промилле (символ $\permil$). \index{промилле, $\permil$}
Чтобы согласовать дельта-шкалу изотопов элемента со шкалой количеств изотопов, необходимо вещество,
содержание изотопов и дельта-значения изотопов которых также хорошо известны.
Обычно это вещество является изотопным эталонным материалом, который служит материалом <<лучшего качества>> (эталона) для определения содержания изотопов. Например, для углерода $\chi(^{13}C)$
шкала содержания согласуется с  $\delta_{\tt VPDB}(^{13/12}C)$ посредством измерения изотопного эталона
материал NBS 19 (карбонат кальция), которому было присвоено согласованное значение $\delta_{\tt VPDB}(^{13/12}C), {\tt NBS 19}) = +1.95 \permil$. 
Отношение числа изотопов углерода NBS 19 составляет
$R(^{13/12}C, {\tt NBS 19}) = 0.011 202 \pm 0.000 028$. \index{NBS 19, карбонат кальция}
Это измерение служит <<наилучшим измерением одного наземного источника \cite{IUPAC2016}. Белемнит Vienna Peedee (VPDB) является нулевой точкой на шкале дельта-изотопов углерода и, следовательно,
$  \delta_{\tt VPDB}(^{13/12}C, {\tt VPDB}) = 0$. Поскольку $1 \permil = 0.001$, отсюда следует: \index{VPDB, венский белемнит}
\begin{equation} % (5)
R(^{13/12}C, {\tt VPDB}) = 0.011 202/(1 + 1.95 \times 0.001) = 0.011 180
\end{equation}

Таким образом, без учета неопределенности соотношение между значениями дельты изотопов углерода ($\delta$) и $^{13}C$ составляет доли ($\chi$) материала P
\begin{equation} %(6)
\chi(^{13}C, P) = 1/ \left[  1 + 1/ \left\lbrace  R (^{13/12}C, {\tt VPDB} ) \times \left[  1 + \delta_{\tt VPDB}(^{13}C, P) \right]  \right\rbrace \right]  
\end{equation}


Данные по изотопам конкретных элементов  рис. \ref{f:PeriodicTable}	доступны по ссылке https://www.sciencebase.gov/catalog/item/580e719ae4b0f497e794b7d8 \cite{IUPACTables} в виде рисунков и численных данных. В интерактивной форме эти данные доступны с сайта https://www.ciaaw.org/natural-variations.htm \cite{CIAAWnaturalVariation}. Рассмотрим, как эти данные представляются и как извлекать численные данные.



\section{Численные данные по изотопам водорода} \label{IsotopeDataH}
Для изотопов водорода графически данные представлены на Рис.~\ref{f:HydrogenDeltaFigure}.

\begin{figure}[ht] 
	\centering\small
	\unitlength=1mm
	\includegraphics[width=80mm]{Figures/HydrogenDeltaFigure.png} 
	\caption{Данные по изотопам водорода для различных веществ} и в разных природных или искусственных  окружениях \cite{IUPACTables}
	\label{f:HydrogenDeltaFigure}
\end{figure}


\begin{table}[h]
	\centering
%{\small
	{\footnotesize      
%	{\tiny 
		\begin{tabular}{|c|c|c|c|}
			\hline
			Category &	Subcategory  & $\un{\chi(^2H)}$	&  $\ov{\chi(^2H)}$\\
			\hline
			Standard  & ~& ~ & ~   \\
			atomic weight & ~ &  0,0000149 & 0,0002832  \\
			\hline
			VSMOW & ~ &  0,0001557  & 0,0001557  \\
			\hline
			Water & Sea water (deep)&   0,0001554  & 0,0001562 \\
			Water  & Other (nat. occurring) &   0,0000787 & 0,0001758 \\
			Water  & Fruit juice and wint  &  0,000149 & 0,0001631 \\ 
			\hline
			Methane  & Atmospheric &  0.0001196 & 0.0001447 \\
			Methane  & Other (nat. occurring)  & 0.0000731 & 0.0001350 \\ 
			\hline
		\end{tabular}
	}
	\caption{Данные по изотопным вариациям для водорода}
	\label{t:IsotopeDataH}
\end{table}


На Рис.~\ref{f:HydrogenDeltaFigure}  имеются три шкалы: атомные веса, изотопные распространённости и $\delta$-отклонения относительно выбранного стандарта VSMOW, умноженные на 1000.
(VSMOW - Vienna Standard Mean Ocean Water) \cite{VSMOW}. \index{VSMOW}

Численные данные извлекаются с сайта или из .csv файла с указанных выше ресурсов. Как можно видеть из графического представления, в данных содержатся атомные веса, изотопные распространённости и $\delta$-отклонения относительно выбранного стандарта VSMOW.

Данные отсортированы по категориям и подкатегориям. Рассмотрим категорию Water и три подкатегории: глубоководная морская вода, природная вода вообще и вода в напитках.

%Фактически при анализе можно выделить три типа данных, которые приведены в соотвествующих таблицах.

В табл.~\ref{t:AtomicWeightH} приведены данные по верхним и нижним границам атомного веса водорода.

\begin{table}[h]
	\centering
%	{\small	 
				{\footnotesize       
%	{\tiny 
		\begin{tabular}{|c|c|c|c|}
			\hline
			Category &	Subcategory & $\un{At. w.}$	&  $\ov{At. w.}$ \\
			\hline
			Standard  & ~& ~ & ~    \\
			atomic weight & ~& 1.00784 & 1.00811   \\
			\hline
			VSMOW & ~ & 1.0079817  & ~  \\
			\hline
			Water & Sea water (deep)& 1.0079814 & 1.0079823  \\
			Water  & Other (nat. occurring)& 1.0079042 & 1.0080020  \\
			Water  & Fruit juice and wint& 1.0079750 & 1.0079891   \\ 
			\hline
			Methane  & Atmospheric &  1.0079454 & 1.0079706 \\
			Methane  & Other (nat. occurring)  & 1.0078985 & 1.0079609 \\ 
			\hline
		\end{tabular}
	}
	\caption{Данные по  вариациям атомной массы для водорода}
	\label{t:AtomicWeightH}
\end{table}

В табл.~\ref{t:IsotopeDeltaH} приведены данные по $\delta$-отклонениям от стандарта VSMOW.	

\begin{table}[h]
	\centering
%		{\small	       
				{\footnotesize       
%	{\tiny 
		\begin{tabular}{|c|c|c|c|}
			\hline
			Category &	Subcategory  & $\un{\delta(^2H)}$	&  $\ov{\delta(^2H)}$\\
			\hline
			Standard  & ~& ~ & ~   \\
			atomic weight & ~ & -904.5 & 818.9  \\
			\hline
			VSMOW & ~ &  0 & ~  \\
			\hline
			Water & Sea water (deep)&   -2.5 & 3.2 \\
			Water  & Other (nat. occurring) &   -495 & 129\\
			Water  & Fruit juice and wint  & -43 & 47 \\ 
			\hline
			Methane  & Atmospheric & -232 & -71 \\
			Methane  & Other (nat. occurring)  & -531 & -133 \\ 
			\hline
		\end{tabular}
	}
	\caption{Данные по $\delta$-отклонениям от стандарта VSMOW для водорода}
	\label{t:IsotopeDeltaH}
\end{table}

Приведённые Табл.~\ref{t:IsotopeDataH}-\ref{t:IsotopeDeltaH} содержат необходимые сведения для расчёта масс молекул, содержащих водород в воде различного происхождения.

Для характеризации веществ наиболее выразительны данные Табл.~\ref{t:IsotopeDeltaH}. Именно такое представление отностельно выбранного стандарта используют специалисты предметных областей науки: геологи, биологи, археологи, палеонтологи и др.

\section{Численные данные по изотопам углерода} \label{IsotopeDataС}

В табл.~\ref{t:AtomicWeightС} приведены данные по верхним и нижним границам атомного веса углерода.
VPDB --- стандартный материал Vienna Pee Dee Belemnite \cite{VPDB}. 

\begin{table}[h]
	\centering
%	{\small	   
		{\footnotesize 		    
%	{\tiny 
		\begin{tabular}{|c|c|c|c|}
			\hline
			Category &	Subcategory & $\un{At. w.}$	&  $\ov{At. w.}$ \\
			\hline
			Standard  & ~& ~ & ~  \\  \index{VPDB, венский белемнит}
			atomic weight & ~& 12.0096 & 12.0116   \\
			\hline
			VPDB & ~ & 12.011115  & ~  \\
			\hline
			Carbonate and bicarbonate & Sea water & 1.0079814 & 1.0079823  \\
			Carbonate and bicarbonate  & Other water & 1.0079042 & 1.0080020  \\
			\hline
			Methane  & Air  &  12.010538 & 12.010665\\
			Methane  & Marine and other  & 12.009896 &12.011233 \\ 
			Methane  & Fresh water   &  12.010149 & 12.010545 \\
			Methane  & Commercial tank gas  & 12.010534 & 12.010676 \\ 
			\hline
		\end{tabular}
	}
	\caption{Данные по  вариациям атомной маcсы для  углерода}
	\label{t:AtomicWeightС}
\end{table}

У углерода имеется два стабильных изотопа, $^{12}C$ и $^{13}C$. Более тяжёлый изотоп менее распространён, поэтому в справочных материалах обычно приводятся данные именно по нему.

В табл.~\ref{t:IsotopeDeltaС} приведены данные по верхним и нижним границам  $\delta$-отклонениям от стандарта VPDB.
\begin{table}[h]
	\centering
%		{\small	 
					{\footnotesize       
%	{\tiny 
		\begin{tabular}{|c|c|c|c|}
			\hline
			Category &	Subcategory & $\un{\delta(^{13}C)}$	&  $\ov{\delta(^{13}C)}$ \\
			\hline
			Standard  & ~& ~ & ~    \\
			atomic weight & ~& -135.9 & 46.2  \\
			\hline
			VPDB & ~ & 0  & ~  \\
			\hline
			Carbonate and bicarbonate & Sea water & -0.8 & 2.2  \\
			Carbonate and bicarbonate  & Other water & -37.1 & 37.5  \\
			\hline
			Methane  & Air  &  -50.6 & -39\\
			Methane  & Marine and other  & -109 & 12.7 \\ 
			Methane  & Fresh water   &  -86 & -50 \\
			Methane  & Commercial tank gas  & -51 & -38 \\ 
			\hline
		\end{tabular}
	}
	\caption{Данные по  вариациям атомной массы для  углерода}
	\label{t:IsotopeDeltaС}
\end{table}


Для изотопов  углерода графически данные представлены на Рис.~\ref{f:CarbonDeltaFigure}.
\begin{figure}[ht] 
	\centering\small
	\unitlength=1mm
	\includegraphics[width=90mm]{Figures/CarbonDeltaFigure.png} 
	\caption{Данные по изотопам углерода для различных веществ} и в разных природных или искусственных  окружениях \cite{IUPACTables}
	\label{f:CarbonDeltaFigure}
\end{figure}

%В дальнейшем данные по изотопам водорода и углерода будут использованы для демонстрации расчётов --- см. \S\ref{CalcExamples}.

\section{Численные данные по изотопам кислорода} \label{IsotopeDataO}

Для изотопов кислорода графически данные представлены на Рис.~\ref{f:OxygenDeltaFigure}.
\begin{figure}[ht] 
	\centering\small
	\unitlength=1mm
	\includegraphics[width=90mm]{Figures/OxygenDeltaFigure.png} 
	\caption{Данные по изотопам кислорода для различных веществ} и в разных природных или искусственных  окружениях \cite{IUPACTables}
	\label{f:OxygenDeltaFigure}
\end{figure}


В табл.~\ref{t:AtomicWeightO} приведены данные по верхним и нижним границам атомного веса кислорода.
\begin{table}[h]
	\centering
%	{\small	 
	{\footnotesize       
%	{\tiny 
		\begin{tabular}{|c|c|c|c|}
			\hline
			Category &	Subcategory & $\un{At. w.}$	&  $\ov{At. w.}$ \\
			\hline
			Standard  & ~& ~ & ~    \\
			atomic weight & ~& 15.9990291 & 15.99977   \\
			\hline
			VSMOW & ~ & 15.9993045  & ~  \\
			\hline
			Water & Sea water (deep)& 15.9993003 & 1.0079823  \\
			Water  & Other (nat. occurring)& 15.9990406& 1.0080020  \\
			Water  & Fruit juice and wint& 15.9992860 & 1.0079891   \\ 
			\hline
		\end{tabular}
	}
	\caption{Данные по  вариациям атомной массы для кислорода}
	\label{t:AtomicWeightO}
\end{table}


В табл.~\ref{t:IsotopeDeltaO} приведены данные по $\delta$-отклонениям от стандарта VSMOW.	

\begin{table}[h]
	\centering
%	{\small	 
	{\footnotesize        
%	{\tiny 
		\begin{tabular}{|c|c|c|c|}
			\hline
			Category &	Subcategory  & $\un{\delta(^{18}H)}$	&  $\ov{\delta(^{18}H)}$\\
			\hline
			Standard  & ~& ~ & ~   \\
			atomic weight & ~ & -65.5 & 111.3  \\
			\hline
			VSMOW & ~ &  0 & ~  \\
			\hline
			Water & Sea water (deep)&   -1 & 0.6 \\
			Water  & Other (nat. occurring) &  -62.8 & 31.3 \\
			Water  & Fruit juice and wint  & -4.4 & 15.3\\ 
			\hline
		\end{tabular}
	}
	\caption{Данные по $\delta$-отклонениям от стандарта VSMOW для кислорода}
	\label{t:IsotopeDeltaO}
\end{table}

Водород, кислород и углерод являются важнейшими биогенными элементами.	Из атомов углерода образован <<скелет>> молекул живой природы --- биогенные аминокислоты, водородные связи цементируют крупные молекулы белков,  кислородное дыхание обеспечивает метаболизм на Земле, из водорода и кислорода состоит вода. 


\chapter*{Заключение}

%<<
%После открытия изотопов химических элементов прошло 100 лет. За это время исследователями накоплено колоссальное количество данных. 
%Специфическая изотопная подпись характеризует различные материалы и процессы. Например, она может служить термометром геологических процессов или характерным признаком нахождения особи в пищевой цепочке. 
В связи с расширением понимания свойств <<химических элементов>>, классификация элементов по электрическому заряду была дополнена Международным союзом теоретической и прикладной химии IUPAC описанием свойств изотопов для каждого элемента --- появилась и используется таблица Менделеева с интервальными значениями атомных весов.

Широкое использование изотопных данных мотивирует создание математических понятий, приёмов и программных инструментов для описания данных и проведения вычислений. 
Изотопные данные имеют свою специфику. IUPAC и Комиссией по изотопным и атомным весам CIAAW накоплен большой объём информации об изотопных подписях в виде интервалов.
%В настоящее время общего подхода к их представлению и обработки нет: наряду с интервальным анализом, взятым за основу IUPAC, в документах этой же организации предлагаются различные теоретико-вероятностные подходы.
%>>
В настоящей публикации даны сведения по ядерной физике и ядерному  нуклеосинтезу, необходимые для понимания происхождения изотопов и их различной распространённости в материи.

Что следует обсудить дальше?
На сегодняшний день просматриваются следующие направления.

В каждой научной дисциплине важен исторический аспект. Для теории атома, а далее --- атомного ядра помимо точных измерений весов, огромную роль сыграла спектроскопия. В XIX веке было собрано огромное количество спектральных данных элементов и веществ и произведены математические обобщения закономерностей в этих данных. Эта информация легла в основу создания квантовой механики через работы Планка и Бора. В дальнейшем спектроскопия осталась, наряду с масс-спектроскопией, основным методом изотопных исследований. Для пространственно недоступных объектов спектроскопия является сейчас и будет главным инструментом исследования. Эти вопросы необходимо осветить более подробно и доступно.

Физика элементарных частиц и ядерная физика дают описание фундамента материи: атомных ядер и их <<метаболизма>>: взаимодействий и превращений (распадов). Строгое описание этих дисциплин весьма сложно и требует специальных знаний. Есть желание представить идеи и сведения максимально простым образом, с тем, чтобы читатель понял, как можно получать нужную информацию и уметь ей пользоваться на качественном уровне и делать простые численные оценки.

Получению и использованию изотопных распределений на Земле посвящено огромное количество теоретических и прикладных публикаций. Необходимо популярное объяснение основных закономерностей нуклеосинтеза, которое даёт исходный изотопный материал. Также хочется очертить групп химических элементов таблицы Менделеева и их изотопов по характеру их происхождения. Далее необходимо осветить процессы изотопного фракционирования в различных природных процессах. Речь идёт в первую очередь о геологии, палеонтологии и биологии, в т.ч. неземной.

Для математического представления, обработки и анализа изотопных данных необходимо развивать математические методы. В описательной статистике необходимо развитие и пропаганда методов, дающих содержательные внутренние и внешние оценки выборок данных, в том числе многомодальных и несовместных. Требуется дальнейшее развитие оценок соотношений множеств и мер совместности. В анализе зависимостей также необходимо развитие методов оценок парамтеров моделей общего вида, мультизначных и многомодальных. На всех этапах необходимо развитие программных средств представления и вычислений, с тем, чтобы у исследователя, обращающегося к данной тематике, уже на старте работ имелись базовые программные инструменты. 

	
	%%%%%%%%%%%%%%%%%%%%%%%%%%%%%%%%%%%%%%%%%%%%%%%%%%%%%%%%%%%%%%%%%%%%%%%%%%%%%%%%%%%%%%%%%%%%%%%%
	\addcontentsline{toc}{chapter}{Литература} 
	\begin{thebibliography}{99}
		
%%%%%%%%%%%%%%%%%%%%%%%%%%%%%%%%%%%%%%%%%%%%% Введение

%		\bibitem{Mendeleev1893}
%		\textsc{Менделеев, Д.} Вес атомов или паи или веса элементов. / 		Энциклопедический словарь. т. VIIA Издатели Ф.А.Брокгауз и И.А.Ефрон. 1893 г. Стр.658-660.
		
		\bibitem{Mendeleev1869ru}
		\textsc{Менделеев, Д.} (1869). “Соотношение свойств с атомным весом элементов”. Журнал Русского Химического Общества. 1: 60—77. \\
		Д.И.Менделеев. Избранные сочинения. т.II АН СССР, ОНТИ - Гостхимтехиздат, Ленингардское отд. 1934, с.3-17.		
		
		\bibitem{Mendeleev1869}
		\textsc{Mendeleev, Dmitri} (1869). “Versuche eines Systems der Elemente nach ihren Atomgewichten und chemischen Functionen”. Journal für Praktische Chemie. 106: 251.
		
		\bibitem{Mendeleev1870}
		\textsc{Менделеев, Д.} (1870). “Опыт системы элементов, основанной на их атомном весе и химическом сходстве”. Выписка из протокола заседания Русского Химического Общества. 3 декабря 1870 г.
		
		
		\bibitem{Lisnevsky1984}
		\textsc{Лисневский Ю.И.} Атомные веса и возникновение ядерной физики. М., Наука, 1984, 255 с.
		
		\bibitem{Trifonov1974}
		\textsc{Трифонов Д.Н., Кривомазов А.Н., Лисневский Ю.И.} Учение о периодичности и учение о радиоактивности (комментированная хронология важнейших событий). М., Атомиздат, 1974, 248 с.

		\bibitem{Scerri2019}
		\textsc{E. Scerri}
		The Periodic Table. Its Story and Its Significance. 
		2nd edition.  New York, NY : Oxford University Press, 2019 

%%%%%%%%%%%%%%%%%%%%%%%%%%%%%%%%%%%%%%%%%%%%% Ядро атомное
		
				\bibitem{NuclPhys}
		В.В.Варламов, Б.С.Ишханов, С.Ю.Комаров Атомные ядра. Учебное пособие. ISBN 978-5-91304-122-72010. –М., Университетская книга, 2010.
		http://nuclphys.sinp.msu.ru/anuc/index.html
		
		\bibitem{Bekman}
\textsc{Бекман, И. Н.}  Атомная и ядерная физика: радиоактивность и ионизирующие излучения : учебник для вузов / И. Н. Бекман. — 2-е изд., испр. и доп. — Москва : Издательство Юрайт, 2022. — 493 с. — (Высшее образование). — ISBN 978-5-534-08692-8. 

%\bibitem{1913Bohr} \textsc{N. Bohr.} N. Bohr.	On the Constitution of Atoms and Molecules, 1913, Part I. Philos. Mag. 26, 1-25. 

%\bibitem{1922Bohr} \textsc{Бор Н.} О строении атомов. Доклад автора при вручении Нобелевской премии. 11.12.1922 // УФН. – 1985. – Т. 3, No 4. – С. 417–448. 

%\bibitem{2006Budkiziewicz} \textsc{ Budzikiewicz, H.; Grigsby, R.D.} Mass Spectrometry and Isotopes: A Century of Research and Discussion. Mass Spectrom. Rev. 2006, 25, 146–157.

%\bibitem{1925Aston} \textsc{F.W.Aston} F.W. Aston Sc.D. F.R.S. (1925) CXIX. The mass-spectra of chemical elements.—Part VI. Accelerated anode rays continued, Philosophical Magazine Series 6, 49:294, 1191-1201, DOI: 10.1080/14786442508634698

		\bibitem{IsotopeGeoChem}
Арбузов С.И. Геохимия изотопов. Лекции. Томский политехнический университет.
https://portal.tpu.ru/SHARED/s/SIARBUZOV/ucheb\_rabota/\\
Geochemistry/lecture/Lecture10.pdf

\bibitem{Mausolf2021}
Mausolf, E.J.; Johnstone, E.V.; Mayordomo, N.;Williams, D.L.; Guan, E.Y.Z.; Gary, C.K. Fusion-Based
Neutron Generator Production of Tc-99m and Tc-101: A Prospective Avenue to Technetium Theranostics.
Pharmaceuticals 2021, 14, 875. 
https://
doi.org/10.3390/ph14090875


\bibitem{Technetium2017}		
Johnstone, E.V.; Yates, M.A.; Poineau, F.P.; Sattelberger, A.P.; Czerwinski, K.R. Technetium: The first radioelement on the Periodic
Table. J. Chem. Educ. 2017, 94, 320–326.

		\bibitem{Weizsacker}
%C. von Weizsacker. Metastabile Zustände der Atomkerne (англ.) // Naturwissenschaften (англ.)русск. : journal. — 1936. — Vol. 24, no. 51. — P. 813—814.
C.F.V. Weizsacker
Zur Theorie der Kernmassen. Z.Phys. 96 (1935) 431-458 DOI: 10.1007/BF01337700

\bibitem{AME2016}
Wang Meng et al 2017 Chinese Phys. C 41 030003 The AME2016 atomic mass evaluation (II). Tables, graphs and references. DOI 10.1088/1674-1137/41/3/030003 

\bibitem{NUBASE2020}
F.G. Kondev et al The NUBASE2020 evaluation of nuclear physics properties 
2021 Chinese Phys. C 45 030001
DOI 10.1088/1674-1137/abddae

\bibitem{NUDAT3} National Nuclear Data Center at Brookhaven National Laboratory. NuDat application. 
https://www.nndc.bnl.gov/nudat3	

\bibitem{ENSDF}
 Описание формата данных {\tt ENSDF}. Автор перевода: Г.И.\,Шуляк. 	Издательство: ПИЯФ РАН, Гатчина, 2006. http://cdfe.sinp.msu.ru/services/ensdfr/ensdfhelp\_ru/ENSDF.html	
%%%%%%%%%%%%%%%%%%%%%%%%%%%%%%%%%%%%%%%%%%%% Нуклеосинтез		
		
		\bibitem{Smith2023}
		Michael S. Smith.	Nuclear data resources and initiatives for nuclear astrophysics 	November 2023 Frontiers in Astronomy and Space Sciences 10 	DOI:10.3389/fspas.2023.1243615	
		

		\bibitem{Nucleosynthesis}
		Чечев В.П., Иванчик А.В., Варшалович Д.А.
		Синтез элементов во Вселенной: От Большого взрыва до наших дней
		URSS. 2020. 304 с. ISBN 978-5-9710-7626-1.

		\bibitem{ElementsOrigin}
		Бедняков В. А. О происхождении химических элементов. Э. Ч. А. Я., Том 33 (2002), Часть 4 стр.914-963.

		\bibitem{MSU98}
		Б.C. Ишханов, И.М. Капитонов, И.А. Тутынь. Нуклеосинтез во вселенной.
		М., Изд-во Московского университета. 1998.
		http://nuclphys.sinp.msu.ru/nuclsynt/index.html
		
		\bibitem{r-process2007}
		M. Arnould, S. Goriely, K. Takahashi,
		The r-process of stellar nucleosynthesis: Astrophysics and nuclear physics achievements and mysteries, Physics Reports, Volume 450, Issues 4–6, 2007, Pages 97-213, ISSN 0370-1573,
		https://doi.org/10.1016/j.physrep.2007.06.002.
		https://www.sciencedirect.com/science/article/pii/S0370157307002438
		
		\bibitem{s-process2007}
		L.-S. The, Mounib F. El Eid and B.S. Meyer. s-PROCESS NUCLEOSYNTHESIS IN ADVANCED BURNING PHASES OF MASSIVE STARS. The Astrophysical Journal, 655:1058Y1078, 2007 February 1
		
		\bibitem{s-processTermination2004}
		U. Ratzel, C. Arlandini, F. Käppeler, A. Couture, M. Wiescher, R. Reifarth, R. Gallino, A. Mengoni, and C. Travaglio. Nucleosynthesis at the termination point of the s process Phys. Rev. C 70, 065803 – Published 10 December 2004
		
		\bibitem{NuclearAstrophysics2022}
		H. Schatz, A.D.B. Reyes, A Best et al. Horizons: nuclear astrophysics in the 2020s and beyond. Journal of Physics G Nuclear and Particle Physics. Vol 49, No 11, November 2022 pp.1-78. DOI:10.1088/1361-6471/ac8890
		
%\bibitem{rprocess2015} M. Eichler et al. THE ROLE OF FISSION IN NEUTRON STAR MERGERS AND ITS IMPACT ON THE r-PROCESS PEAKS. The Astrophysical Journal, 808:30 (13pp), 2015 July 20. doi:10.1088/0004-637X/808/1/30

%%%%%%%%%%%%%%%%%%%%%%%%%%%%%%%%%%%%%%%%%%%% /Нуклеосинтез		
		
		


%%%%%%%%%%%%%%%%%%%%%%%%%%%%%%%%%%%%%%%%%%%% Популярная	и науки о Земле

%		\bibitem{EarthEvolutin2022} 				Сорохтин О.Г., Ушаков С.А. Развитие Земли. М: Изд-во МГУ, 2002. - 506 с.
		
%		\bibitem{2017Kaminsky} Felix V. Kaminsky. The Earth's Lower Mantle: Composition and Structure (Springer Geology) 1st ed. 2017 Edition. 340 p.
		
		\bibitem{2021Hazen} 		Хейзен Роберт. Симфония №6: Углерод и эволюция почти всего. М.: Альпина нон-фикшн, 2021. — 288 c.
		
	\bibitem{2015Hazen} 		Хейзен Роберт.Роберт Хейзен: История Земли. От звездной пыли - к живой планете. Первые 4 500 000 000 лет.		The Story of Earth. The First 4.5 Billion Years, from Stardust to Living Planet. 		 М.: Альпина нон-фикшн, 2015. — 346 c.	

		\bibitem{2008Hazen} 		R.M.Hazen, D.Papineau, W.Bleeker, R.T.Downs, J.M.Ferry, T.J.McCoy, D.A.Sverjensky, and H.Yang. Review Paper. Mineral evolution. American Mineralogist, Volume 93, pages 1693–1720, 2008. DOI: 10.2138/am.2008.2955 1693

\bibitem{2014Grew} E.S. Grew, and R.M. Hazen.
Beryllium mineral evolution. American
Mineralogist, Volume 99, pages 999–1021, 2014

		 \bibitem{WardKirschvink} 		 \textsc{П.\,Уорд, Д.\,Киршвинк} Новая история происхождения жизни на Земле. -- Питер. 2016. --  464 с. 	
		 
%		 \bibitem{SnowballEarth1992} Kirschvink, J. Late Proterozoic low-latitude global glaciation: the Snowball Earth // The Proterozoic Biosphere: A Multidisciplinary Study (англ.) / J. W. Schopf; C. Klein. — Cambridge University Press, 1992.	
	
%		\bibitem{ClumpedIsotope} 		John M. Eiler, <<Clumped-isotope>> geochemistry—The study of naturally-occurring, multiply-substituted isotopologues,Earth and Planetary Science Letters,Volume 262, Issues 3–4,2007,Pages 309-327, ISSN 0012-821X,https://doi.org/10.1016/j.epsl.2007.08.020. (https://www.sciencedirect.com/science/article/pii/S0012821X07005109)
		
%		\bibitem{IsotopologyMethane} T. Giunta, J. Labidi, I.E. Kohl3, L. Ruffine, J.P. Donval, L. Géli, M. N. Çağatay, H. Lu, E.D. Young. Evidence for methane isotopic bond re-ordering in gas reservoirs sourcing cold seeps from the Sea of Marmara
		
%		\bibitem{IsotopologyIRMS}  Edward D. Young, Douglas Rumble, Philip Freedman, Mark Mills. A large-radius high-mass-resolution multiple-collector isotope ratio mass spectrometer for analysis of rare isotopologues of O2, N2, CH4 and other gases
		
%		\bibitem{Nature2023} Spilker, J.S., Phadke, K.A., Aravena, M. et al. Spatial variations in aromatic hydrocarbon emission in a dust-rich galaxy. Nature (2023). https://doi.org/10.1038/s41586-023-05998-6
		
%		\bibitem{SolarIrradiance2023} Valentina V. Zharkova, Irina Vasilieva, Simon Shepherd, Elena Popova. Periodicities of solar activity and solar radiation derived from observations and their links with the terrestrial environment. https://doi.org/10.48550/arXiv.2301.07480
		
%		\bibitem{IsotopeSignatureWiki} 		https://ru.wikipedia.org/wiki/Изотопная\_подпись
		
		\bibitem{Zhuravlev2019} Журавлев А.Ю. Сотворение Земли. Как живые организмы создали наш мир.
		М.: Альпина Паблишер. ISBN 978-5-91671-902-4. 514 стр.



\bibitem{Fry2006}
 Brian Fry. Stable Isotope Ecology. 2006 Springer Science+Business Media, LLC. 390 pages. ISBN: 0387305130 ISBN-13(EAN): 9780387305134
 
 \bibitem{HobsonWassenaar2018}
 Keith A. Hobson PhD (Editor), Leonard I. Wassenaar (Editor). Tracking Animal Migration with Stable Isotopes. Academic Press; 2nd edition (September 17, 2018), 268 pages

		\bibitem{Chizevsky2004} Чижевский А.Л. Земля в объятиях Солнца. — М.: Эксмо, 2004. — 928 с.	
		
		\bibitem{Gregory2023} 
		Т.Грегори. Метеориты. Космические камни, создавшие наш мир. - М.: Эксмо, 2023. - 288 с.	
		
		\bibitem{Krestianinov2023}
E.Krestianinov, Yu.Amelin, Q.-Z.Yin et al. Igneous meteorites suggest Aluminium-26 heterogeneity in the early Solar Nebula. Nature Communications ( 2023) 14:4940 p.1-11.

		\bibitem{Heck2020}
Ph.R.Heck, J.Greer, L.Kööpa et al.  Lifetimes-of-interstellar-dust-from-cosmic-ray-exposure-ages-of-presolar-silicon-carbide. PNAS, January 13, 2020. 117 (4) 1884-1889 https://doi.org/10.1073/pnas.1904573117

		\bibitem{Decart2003}
В.Е.Жаров. Нутация неупругой Земли. О работе международного коллектива, удостоенной премии Декарта 2003 г. https://www.astronet.ru/db/msg/1195760

%		\bibitem{Oxygen2014} T. W. Lyons, C. T. Reinhard, N. J. Planavsky. The rise of oxygen in Earth’s early ocean and atmosphere // Nature. 2014. V. 506. P. 307–315. doi:10.1038/nature13068 

	\bibitem{18Orecords2005}
L.E. Lisiecki, M.E. Raymo. A Pliocene-Pleistocene stack of 57 globally distributed benthic D18O records.
PALEOCEANOGRAPHY, VOL. 20, PA1003, doi:10.1029/2004PA001071, 2005
		
%		\bibitem{He2019} 	He, T. et al. Possible links between extreme oxygen perturbations and the Cambrian radiation of animals. Nat. Geosci. 12, 		468–474 (2019).

\bibitem{Allende2007}
Amelin Y., Krot A. N. Pb isotopic age of the Allende chondrules.  Meteoritics \& Planetary Science, Volume 42, Issue 1321, pp. 1321-1335. 2007 DOI: 10.1111/j.1945-5100.2007.tb00577
		
	\bibitem{Wegener1925}		
Вегенер А. Возникновение материков и океанов / перевод с 3 немецкого издания, редактор Г. Ф. Мирчинк. Л.: Госиздат, 1925. XIV, 147 с. (Современные проблемы естествознания; Кн. 24).		
		
%		\bibitem{Rader2017} 		\textsc{J.A. Rader, J. A., Newsome, S. D., Sabat, P., Chesser, R. T., Dillon, M. E., and Martínez del Rio, C.} (2017). Isotopic niches support the resource breadth hypothesis. J. Anim. Ecol. 86, 405–413. doi:10.1111/1365-2656.12629
		
%		\bibitem{OpaevIsotope} 		\textsc{Опаев А.} Изотопная подпись. \\			https://elementy.ru/problems/1523/Izotopnaya\_podpis
		
%		\bibitem{Bowen2010} 		\textsc{G.J. Bowen} Isoscapes: Spatial Pattern in Isotopic Biogeochemistry.  Annu. Rev. Earth Planet. Sci. 2010. 38:161–187 \\ 		http://www.iai.int/admin/site/sites/default/files/uploads/ 2010\_Bowen\_Isoscapes\_Spatial-Pattern-in-Isotopic-Biogeochemistrypdf.pdf
		
%		\bibitem{Newsome2007} \textsc{S.D. Newsome, C.Martinez del Rio, S. Bearhop, and D.L. Phillips.}		A niche for isotopic ecology. Front Ecol Environ 2007; 5(8): 429–436, doi:10.1890/060150.01
		
		
	\bibitem{Iakovlev2013}  \textsc{И.Яковлев.} Изучение трофической структуры сообществ с помощью анализа стабильных изотопов. Дискуссионные лекции-семинары по эволюционной экологии, 08.11.2013 \\		http://www.eco.nsc.ru/lectures/Iakovlev\_Stable\_Isotopes.pdf
	
	\bibitem{Grichuk2022} \textsc{Гричук Д.В.} Геохимия стабильных изотопов. МГУ. 2022.
	http://www1.geol.msu.ru/deps/geochems/rus/distance/gr\_course\_gch\_t10-2.pdf
		
%		\bibitem{Sheath2016}		Danny Sheath. Ecological consequences of indigenous and non-indigenous freshwater fish parasites. Thesis World Health Organization WHO· November 2016. DOI: 10.13140/RG.2.2.11682.68806
		
		\bibitem{IsotopeTraceEarlySun2022} 
		Васильев Г.И., Мелихова Е.С., Павлов А.К. ИЗОТОПНЫЕ СЛЕДЫ АКТИВНОСТИ РАННЕГО СОЛНЦА. XXVI Всероссийская ежегодная конференция по физике Солнца. Солнечная и солнечно-земная физика. г.Санкт-Петербург. 3-7 октября 2022 \\
		http://www.gaoran.ru/russian/publ-s/conf\_2022/conf\_2022.pdf
		
		\bibitem{2015Planetology}		
		Планетология: учебное пособие / под ред. Н. И. Цидаевой;
		Сев.-Осет. гос. ун-т. Владикавказ: Изд-во СОГУ, 2015. – 232 с.


%		\bibitem{Kerogen} M. Vandenbroucke, C. Largeau. Kerogen origin, evolution and structure // Organic Geochemistry. — 2007-05-01. — Vol. 38, iss. 5. — P. 719–833. — ISSN 0146-6380	

%		\bibitem{WhiticarPetroleum} Michael J. Whiticar. Carbon Isotopes in Petroleum Science.		Springer Nature Switzerland AG 2021. R. Sorkhabi (ed.), Encyclopedia of Petroleum Geoscience,
%https://doi.org/10.1007/978-3-319-02330-4_310-1

%		\bibitem{Whiticar2020} Whiticar MJ. The biogeochemical methane cycle. In: Wilkes		H (ed) Hydrocarbons, oils and lipids: diversity, origin, chemistry and fate. Springer International Publishing, 2020, pp 669–746

%		\bibitem{Semenov2020}		Semenov P.B., Pismeniuk A.A., Malyshev S.A., Leibman M.O., Streletskaya I.D., Shatrova E.V., Kizyakov A.I., Vanshtein B.G. Methane and Dissolved Organic Matter in the Ground Ice Samples from Central Yamal: Implications to Biogeochemical Cycling and Greenhouse Gas Emission. 	Geosciences, MDPI Publishing (Basel, Switzerland), vol. 10, No. 11, p. 450-450 


%%%%%%%%%%%%%%%%%%%%%%%%%%%%%% IUPAC CIIAW
		
%		\bibitem{SLAP} 	Lin Y, Clayton RN, Gröning M. Calibration of $\delta^{17}O$ and $\delta^{18}O$ of international measurement standards–VSMOW, VSMOW2, SLAP, and SLAP2. Rapid Communications in Mass Spectrometry. 2010;24:773–776. doi: 10.1002/rcm.4449.
		
		
%		\bibitem{2018rainsnowfall}				Tian C, Wang L, Kaseke KF, Bird BW. Stable isotope compositions ($\delta^2H$, $\delta{18}O$ and $\delta{17}O$) of rainfall and snowfall in the central United States. Sci Rep. 2018 Apr 30;8(1):6712. doi: 10.1038/s41598-018-25102-7. PMID: 29712983; PMCID: PMC5928101.
		
%		\bibitem{2008DeuteriumRiver}			J. Palige, S. Ptaszek, R. Zimnicki, A.G. Chmielewski, R. Wierzchnicki. 		Stable isotope deuterium as a natural tracer of mixing processes in rivers. 		NUKLEONIKA  2008; 53(2) p.63-67
		
%		\bibitem{1998IsotopeTracers}		Isotope Tracers in Catchment Hydrology (1998), C. Kendall and J.J. McDonnell (Eds.) Elsevier Science B.V., Amsterdam. 839 p.
		
%		\bibitem{IsotopeMSinBiology}		А.М. Зякун. Теоретические основы изотопной масс-спектрометрии в биологии: учебное пособие для студентов высших учебных заведений, обучающихся по направлению 020200.68 "Биология" / А. М. Зякун ; Пущинский гос. ун-т, Учреждение Российской акад. наук Ин-т биохимии и физиологии микроорганизмов им. Г. К. Скрябина. - Пущино : Фотон-век, 2010. - 224 с. : ил., табл.; 22 см.; ISBN 978-5-903789-25-2 
		
%		\bibitem{2011IsotopeGas} 		Проблемы аналитической химии // Отделение химии и наук  о материалах РАН. — М.: ФИЗМАТЛИТ, 2011.		Т. 15: Изотопная масс-спектрометрия легких газообразующих элементов / Под ред. В. С. Севастьянова; Ин-т геохимии и аналити ческой химии им. В. И. Вернадского РАН. — 2011. — 240 с. — ISBN 978-5-9221-1344-1.
		
%		\bibitem{SPbSTU2021}		{\bf Баженов\,А.\,Н.} Естественнонаучные и технические применения интервального анализа. 	: учеб. пособие / А.\,Н.\,Баженов. --- СПб., 2021. 		URL: https://elib.spbstu.ru/dl/5/tr/2021/tr21-169.pdf/info.
		
		
%		\bibitem{WaterIsotope}		https://ru.wikipedia.org/wiki/Изотопный\_состав\_воды
		
%		\bibitem{LightWater}		Патент RU 2295493. «Способ и установка для производства лёгкой воды». Соловьев С. П.
		
%		\bibitem{MonoisotopicMass}		A.B. Attygalle, Ju. Pavlov, and J.Ruzicka.  Monoisotopic Mass? https://www.researchgate.net/publication/356631437
		

		%%%%%%%%%%%%%%%%%%%%%%%%%%%%%%%%%%%%%%%%%%%%%%%%%%%%%%%%%%%%%%%%%%%%%%%%%%%%%%%%%%%%%%%%%%%%%%%%%%%%%%%%%%
		
%		\bibitem{MetodikaBook} \textsc{А.Н.\,Баженов, С.И.\,Жилин, С.И.\,Кумков, С.П.\,Шарый.} <<Обработка и анализ данных с интервальной неопределённостью>>. (готовится к печати)

		\bibitem{InteIsotopes2023} 
 А.Н.Баженов. Интервальные арифметики и прослеживаемость изотопной подписи: учебное пособие. Санкт-Петербургский политехнический университет Петра Великого. С.-Петербург, 2023. https://doi.org/10.18720/SPBPU/5/tr23-167 
		
		\bibitem{IUPAC} 
		\textsc{Meija, J., Coplen, T.B., Berglund, M., Brand, W.A., De Bièvre, P., Gröning, M., Holden, N.E., Irrgeher, J., Loss, R.D., Walczyk, T., Prohaska, T.} 
		Atomic weights of the elements 2013 (IUPAC Technical Report) // Pure and Applied Chemistry. -- 2016. -- Vol.~88, Issue~3. -- P.~265--291. \   DOI: 10.1515/pac-2015-0305 
		
		
%		\bibitem{AtomicWeights1969} Atomic weights of the elements 1969 // Pure Appl. Chem., 1970, Vol. 21, No. 1, pp. 91-108 \\ http://dx.doi.org/10.1351/pac197021010091
		
		\bibitem{IAEA} AMDC --- Atomic Mass Data Center
		https://www-nds.iaea.org/amdc/
		
		%%%%%%%%%%%%%%%%%%%%%%%%%%%%%%%%%%%%%%%%%%%%%%%%%%%%%%%%%%%%%%%%%%%%%%%%%%%%%%%%%%%%%%%%%%%%%%%%
		
%		\bibitem{IUPACIntervalHistory} 		T.B. Coplen and N. E. Holden	Atomic Weights—No Longer Constants of Nature //		Chemistry International -- Newsmagazine for IUPAC, vol. 33, no. 2, 2011, pp. 10-15.// https://doi.org/10.1515/ci.2011.33.2.10
		
		\bibitem{CIAAW}
		Commission on Isotopic Abundances and Atomic Weights, CIAAW. https://www.ciaaw.org/
		
%		\bibitem{IUPAC2018} Norman E. Holden, Tyler B. Coplen, John K. B\"{o}hlke, Lauren V. Tarbox, Jacqueline Benefield, John R. de Laetera, Peter G. Mahaffy, Glenda O’Connorb, Etienne Rotha, 		Dorothy H. Tepper, Thomas Walczyk, Michael E. Wieser and Shigekazu Yoneda. \ 		IUPAC Periodic Table of the Elements and Isotopes (IPTEI) for the Education Community 		(IUPAC Technical Report) \ Pure Appl. Chem. 2018; 90(12): 1833–2092 		https://doi.org/10.1515/pac-2015-0703 		Received August 3, 2015; accepted July 23, 2018
		
%		\bibitem{IUPACGoldBook} 		IUPAC. Compendium of Chemical Terminology, Blackwell Scientific Publications, Oxford, 2nd ed. (1997), (the <<Gold Book>>). Compiled by A. D. McNaught and A. Wilkinson, 0-9678550-9-8, Online version (2019-) created by S. J. Chalk. URL, https://doi.org/10.1351/goldbook 
		
		\bibitem{IUPAC2021}
		Thomas Prohaska, Johanna Irrgeher, Jacqueline Benefield, John K. Böhlke,
		Lesley A. Chesson, Tyler B. Coplen, Tiping Ding, Philip J. H. Dunn, Manfred Gröning,
		Norman E. Holden, Harro A. J. Meijer, Heiko Moossen, Antonio Possolo,
		Yoshio Takahashi, Jochen Vogl, Thomas Walczyk, Jun Wang, Michael E. Wieser,
		Shigekazu Yoneda, Xiang-Kun Zhu and Juris Meija. \\
		Standard atomic weights of the elements 2021 (IUPAC Technical Report)	Pure Appl. Chem. 2022; 94(5): 573–600
		
%		\bibitem{IUPACUncertainty}	Antonio Possolo, Adriaan M. H. van der Veen, Juris Meija and D. Brynn Hibbert	Interpreting and propagating the uncertainty of the standard atomic weights (IUPAC Technical Report). \\		Pure Appl. Chem. 2018; 90(2): 395–424
		
%		\bibitem{IUPACCosensus} 		T.B. Coplen, N.E. Holden, T. Ding, H.A.J. Meijer, J. Vogl, and X. Zhu, The Table of Standard Atomic Weights-An exercise in consensus, Rapid Commun Mass Spectrom. 2022; 36:e8864.		https://doi.org/10.1002/rcm.8864
		
%		\bibitem{IUPAC2007} 		M. E. Wieser, M. Berglund. Atomic weights of the elements 2007 (IUPAC Technical Report). Pure Appl. Chem. 81, 2131 (2009).
		
		\bibitem{IUPAC2014}
		Brand, Willi A., Coplen, Tyler B., Vogl, Jochen, Rosner, Martin and Prohaska, Thomas. "Assessment of international reference materials for isotope-ratio analysis (IUPAC Technical Report)" Pure and Applied Chemistry, vol. 86, no. 3, 2014, pp. 425-467. https://doi.org/10.1515/pac-2013-1023
		
%		\bibitem{IUPAC2015} J. Meija. Chem. Int. 37(5), 26 (2015), https://doi.org/10.1515/ci-2015-0512, https://iupac.org/standard-atomic-weight-ofytterbium-revised/, (accessed Jan 21, 2021).
		
		\bibitem{IUPAC2016}
		J. Meija, T. B. Coplen, M. Berglund, W. A. Brand, P. De Bi`evre, M. Gröning, N. E. Holden, J. Irrgeher, R. D. Loss, T. Walczyk, T. Prohaska. Pure Appl. Chem. 88, 293 (2016).
		
%		\bibitem{IUPAC2016data} T. B. Coplen, Y. Shrestha. Pure Appl. Chem. 88, 1203 (2016).
		
%		\bibitem{IUPAC2018} J. Meija. Chem. Int. 40(4), 23 (2018), https://doi.org/10.1515/ci-2018-0409, https://iupac.org/standard-atomic-weights-of-14-chemical-elements-revised/, (accessed Jan 23, 2021).
		
%		\bibitem{IUPAC2019data}  T. B. Coplen, Y. Shrestha. Pure Appl. Chem. 91, 173 (2019).
		
%		\bibitem{IUPAC2020} J. Meija. Chem. Int. 42(2), 31 (2020), https://doi.org/10.1515/ci-2020-0222, https://iupac.org/standard-atomic-weight-ofhafnium-revised/, (accessed Jan 23, 2021).
		
		\bibitem{IUPACTables}  T. B. Coplen, Y. Shrestha. Tables and Charts for Isotope-Abundance Variations and Atomic Weights of Selected Elements: 2016
		(Ver. 1.1, May 2018), U.S. Geological Survey Data Release (2018).
		
		\bibitem{CIAAWnaturalVariation} CIAAW. Commission on Isotopic Abundances and Atomic Weights. https://ciaaw.org/natural-variations.htm
		
%		\bibitem{TablesCharts}		https://www.sciencebase.gov/catalog/item/580e719ae4b0f497e794b7d8
		
		\bibitem{IPTEI}  Isotopes Matter,  electronic version of the IUPAC Periodic Table of the Elements and Isotopes.
		https://www.isotopesmatter.com
		
%		\bibitem{GUM2011} 		Joint Committee for Guides in Metrology. Evaluation of measurement data – Supplement 2 to the <<Guide to the expression of		uncertainty in measurement>> --- Extension to any number of output quantities, International Bureau of Weights and Measures (BIPM), S\`{e}vres, France (2011), URLBIPM, IEC, IFCC, ILAC, ISO, IUPAC, IUPAP and OIML, JCGM 102:2011, 		https://www.bipm.org/en/committees/jc/jcgm/publications (accessed Jan 25, 2021).
		
		\bibitem{IUPACArgon} J. K. Böhlke. Pure Appl. Chem. 86, 1421 (2014).
		
		\bibitem{Lodders2003}
		 K. Lodders. SOLAR SYSTEM ABUNDANCES AND CONDENSATION TEMPERATURES OF THE ELEMENTS. The Astrophysical Journal, 591:1220–1247, 2003 July 10
		
		\bibitem{Argon1993} J. Eikenberg, P. Signer, R. Wieler. Geochem. Cosmochim. Acta. 57, 1053 (1993).
		
		\bibitem{IUPACLead} X. K. Zhu, J. Benefield, T. B. Coplen, Z. Gao, N. Holden. Pure Appl. Chem. 93, 155 (2021).
		
		\bibitem{IUPACMS2020}
		T. B. Coplen, N. E. Holden, T. Ding, H. A. J. Meijer, J. Vogl, X. Zhu. Rapid Commun. Mass Spectrom., e8864 (2020).
		
%		\bibitem{Crocetane}		M. Elvert, E. Suess, J. Greinert, M. J. Whiticar. Org. Geochem. 31, 1175 (2000).
		
		\bibitem{NobleGasesTracers}
		P. Burnard (ed.), The Noble Gases as Geochemical Tracers, Advances in Isotope Geochemistry,
		DOI: 10.1007/978-3-642-28836-4\_2,  Springer-Verlag Berlin Heidelberg 2013
		

		
		\bibitem{VSMOW} Vienna Standard Mean Ocean Water --- VSMOW standard. 
		https://en.wikipedia.org/wiki/Vienna\_Standard\_Mean\_Ocean\_Water
		
		\bibitem{VPDB} Vienna Pee Dee Belemnite --- VPDB standard.
		https://en.wikipedia.org/wiki/$\Delta$13C

		%%%%%%%%%%%%%%%%%%%%%%%%%%%%%%%%%% Математика
		
		\bibitem{Shreider} 
		\textsc{Шрейдер Ю.А.} Равенство, сходство, порядок. -- Москва: Наука, 1971. 		
		
%		\bibitem{InteNotation} Вычислительные технологии
		
		\bibitem{Twins1981}
		\textsc{Garde\~{n}es E., Trepat A., Janer J.M.} Approaches to simulation and to the linear 
		problem in the SIGLA system // Freiburger Intervall-Berichte. -- 1981. -- No.~8. 
		-- S.~1--28. 
		
		\bibitem{Nesterov99} 
		\textsc{Нестеров В.М.} Твинные арифметики и их применение в методах и алгоритмах 
		двустороннего интервального оценивания. дисс. \ldots д.ф.-м.н. Санкт-Петербург, 
		Санкт-Петербургский институт информатики и автоматизации РАН, 1999, 234 с. 
		
		\bibitem{TwinLib}	А.Г.\,Жаворонкова. Арифметика твинов Нестерова.
		{\tt twin} ---  https://github.com/Zhavoronkova-Alina/twin
		
%		\bibitem{Jaccard2022} 		\textsc{Баженов А.Н., Тельнова А.Ю.} 		Обобщение коэффициента Жаккара для анализа данных с интервальной неопределённостью //		Измерительная техника. -- 2022. №~12, С.~12--15. 
		
		\bibitem{SSharyBook} 
		\textsc{Шарый С.П.} Конечномерный интервальный анализ. -- ФИЦ ИВТ: 
		Новосибирск, 2023. \     Электронная книга, доступная 
		на \url{http://www.nsc.ru/interval/Library/InteBooks/SharyBook.pdf} 
		
%		\bibitem{Pgamma1992} 		Circular polarization of $\gamma$-quanta in the $np \rightarrow d\gamma$		reactions with polarized neutrons / A.\,N.\,Bazhenov [et~al.] //		Physics Letters B.~3. --- September 1992.  Vol.~289. --- No. 1–2. --- P.~17-21. 
		
%		\bibitem{HuCHuZH}  		\textsc{Hu C., Hu Z.H.} On statistics, probability, and entropy of interval-valued 		datasets // Lesot MJ. et al. (eds) Information Processing and Management of Uncertainty 		in Knowledge-Based Systems. IPMU 2020. Communications in Computer and Information 		Science, vol 1239. -- Cham: Springer, 2020. 
		
%		\bibitem{MassSpectrumNuclearFission} Спектр масс осколков ядра при захвате нейтронов. 		URL:~https://ru.wikipedia.org/Деление\_ядра.
		
%		\bibitem{Tukey1962}		\textsc{Tukey J.W.} The Future of Data Analysis // 		Annals of Mathematical Statistics -- 1962. -- Vol.~33, Issue~1. -- P.~1--67 \  		\doi{10.1214/aoms/1177704711} 
		
%		\bibitem{Tukey1972}		\textsc{Tukey J.W.} Data Analysis, Computation and Mathematics //		Quarterly of Applied Mathematics. -- 1972. -- Vol.~30, -- P. 51-–65.
		
%		\bibitem{BazhenovAppInte}			А.Н. Баженов. Естественнонаучные и технические применения интервального анализа. 2022
		
%		\bibitem{Webinar2022}		А.Н. Баженов. Интервальная таблица Менделеева элементов и изотопов. Всероссийский веб-семинар <<Интервальный анализ и его приложения>>. Заседание 28 ноября 2022 года.		http://www.nsc.ru/interval/WebSeminar/ANBazhenov-28.XI.2022.pdf
		
%		\bibitem{OnlineOctave} С.И.Жилин. \		Примеры анализа интервальных данных в Octave. Сборник jupyter-блокнотов с примерами анализа интервальных данных.\\		\url{https://github.com/szhilin/octave-interval-examples}
		

		
		\bibitem{TwinIsotope}	Т.О.\,Яворук.	Вычисления с изотопами		{\tt MendeleevTwin}. https://github.com/Tatiana655/MendeleevTwin
		
%		\bibitem{TwinAlt} А.Н. Баженов, Т.О.Яворук. Альтернативная форма формул твинной арифметики. 2023	
		
		
%		\bibitem{Tensor2021} 	Баженов А.Н., Яворук Т.О., 2021. Тензорные разложения и их применение в флуориметрии: учебное пособие. DOI: 10.18720/SPBPU/5/tr21-170.
		
		\bibitem{Diana2023} \textsc{Д.Г.\,Кожевникова.} Применение твинов для вычислений с изотопными распределениями.  Выпускная квалификационная работа бакалавра. С.-Петербург. 2023. 	https://elib.spbstu.ru/dl/3/2023/vr/vr23-4186.pdf/info
		
\bibitem{TwinMC} Tatiana Iavoruk and Alexander Bazhenov. The Use of Twins in Isotopic Analysis.
IC-MSQUARE 2023 12th Int'l Conference on Mathematical Modeling in the Physical Sciences August 28-31, 2023. DOI:10.1007/978-3-031-52965-8\_9
		
	\end{thebibliography}
	%%%%%%%%%%%%%%%%%%%%%%%%%%%%%%%%%%%%%%%%%%%%%%%%%%%%%%%%%%%%%%%%%%%%%%%%%%%%%%%%%%%%%%%%%%%%%%%%		
	\addcontentsline{toc}{chapter}{Предметный указатель} 
	\raggedright\small\printindex 
	
\end{document}